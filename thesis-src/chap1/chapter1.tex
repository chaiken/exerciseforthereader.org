\chapter{Introduction}
\setcounter{chapter}{1}
\pagestyle{headings}
\label{intro}

        Anisotropic     superconductivity    is     the     fastest-growing
condensed-matter physics subfield of the late 1980's.  The primary boom has
been in the high-temperature superconductivity  industry, which was created
seemingly  overnight  by Bednorz and   M\"uller's astonishing discovery  of
over-30          K         superconductivity         in                $\rm
La_{5-x}Ba_xCu_5O_y$.\cite{muller87} The  subsequent observation of 90
K superconductivity in  $\rm YBa_2Cu_3O_7$\cite{wu87} changed the character
of superconductivity research forever.

        Even before the attention of the physics community  was focussed on
the  new high-$\rm  T_c$ materials, the  field of superconductivity was far
from dormant.  There  were contributions from superconductivity research to
many    different   disciplines,   such  as  the    influence  of  granular
superconductivity research on percolation theory,\cite{orr87,lobb87} or the
influence of the  study  of  Josephson-junction arrays   on the  theory  of
two-dimensional phase  transitions.\cite{nelson79,forrester88} Despite  the
many  aspects of superconductivity  research in   the '80's,  though,   one
paradigm   has been   central  throughout:   the  concept  of   anisotropic
superconductivity.  This theme has been manifested in all sorts of systems,
from     quasi-one-dimensional      superconductors   like     the  polymer
SN$_x$,\cite{greene84}  to   heavy-fermion  superconductors   with possibly
anisotropic   pair-states,\cite{stewart84}    to      artificially  layered
superconducting superlattices  which  show   a  3D-2D crossover  in   their
critical fields.\cite{ruggiero85} The  unexpected  advent of high-$\rm T_c$
materials  is in harmony  with the theme of  anisotropic  superconductivity
since the oxide  superconductors  show orientation-dependent  properties in
many ways similar to those  of  the superconductors studied earlier in  the
decade.\cite{orlando87,worthington87,iye87a,moodera88}

        The principal topic of this thesis is anisotropic superconductivity
of   graphite  intercalation compounds     (GIC's).  Graphite intercalation
compounds   are superlattices  formed  by the  chemical insertion  into the
graphite     matrix     of          a     species         called        the
intercalant.\cite{I94,whittingham82}  GIC's are called donor  compounds  if
the intercalant gives electrons to  the carbon layers, and  acceptor compounds
if  the  intercalant  takes electrons from the  carbon  layers.  None   of the
acceptor GIC's is known  to be superconducting, despite  the fact that many
of them are metallic.\cite{I94,erice:scy}

        Superconducting  GIC's do  not  have   an  exotic  electron-pairing
mechanism,  do not  show  a  3D-2D crossover,  and   have  no   foreseeable
technological applications.  Yet there are many interesting questions about
graphite-based superconductors which remain to be answered, especially with
regard to their anisotropy.  The  most obvious of  these questions concerns
the    very      existence   of  superconductivity    in     the   GIC $\rm
C_8K$,\cite{hannay65} a compound made  from two  constituents which are not
separately superconducting.   $\rm  C_8K$  is  a binary GIC  formed by  the
insertion of potassium atoms between the layers  of graphite.  The accepted
structure of $\rm C_8K$ is shown in Figure~\ref{c8kstruct}.

\begin{figure}
\vspace{15cm}
\caption[Structure of $\rm C_8K$]{Structure of $\rm C_8K$, a typical stage 1
intercalation  compound.   Stage n means that  n  layers  of  graphite  are
present between each  pair  of  intercalant layers.\cite{I94} a)  The layer
structure.  The lattice constant along the  direction  perpendicular to the
planes is called $\rm I_c$ in GIC's.  b) Three  common in-plane structures.
$\rm C_8K$ has the $(2 \times 2)$R0$^{\circ}$ structure.}
\label{c8kstruct}
\end{figure}

        The   superconductivity     of    $\rm   C_8K$    is still    under
investigation,\cite{aljishi88} but  in recent   years attention  has turned
more  to  the ternary   GIC superconductors,  which have   two intercalated
components, an alkali metal (denoted M) and a heavy metal.  The ternary GIC
superconductors are highly anisotropic,  with critical  fields that vary by
as much  as a factor  of 47 as  a function of orientation.\cite{iye83a} The
binary  GIC  superconductors, on   the   other hand,  have  critical  field
anisotropies only on the  order of 4.\cite{koike80}  Despite  an  extensive
study of the critical fields of the ternary GIC  superconductors by Iye and
Tanuma,\cite{iye82}  there  are still some  fundamental questions about the
angular and temperature dependence of the critical fields to be answered.

        Unlike    the   binary GIC   superconductors,   the MHg-GIC's   are
synthesized from an  intercalant,  the MHg  amalgam,   which is  in  itself
superconducting.  The intercalation compounds  formed from the MHg amalgams
have higher  superconducting    transition temperatures  than  the amalgams
themselves.  In addition, the second-stage MHg-GIC's, which have two carbon
layers between the graphite planes, have higher $\rm  T_c$'s than the stage
1 GIC's, with  one carbon layer between  the graphite planes.   These facts
run counter     to    usual   expectations  based     on   proximity-effect
theories,\cite{werthamer63,cooper61} which predict the monotonic depression
of  $\rm    T_c$ when  a  superconducting material     is  layered  with  a
non-superconducting  one   like     graphite.   Therefore    the    ternary
graphite-based  superconductors display  qualitatively different phenomena
than the artificially structured superlattices, which do show behavior
qualitatively in agreement with proximity theories.\cite{ruggiero85}

        The ternary GIC superconductors are in many respects similar to the
transition metal di\-chal\-cog\-enides (TMDC's) and their in\-ter\-ca\-la\-tion com\-pounds
(TMDCIC's).   The  TMDC's are  different from graphite  in that they are by
themselves   superconducting,  but  alike  in   their divergence from   the
predictions of proximity-effect theories.  Intercalation of the TMDC's with
organic   molecules    increases     their  transition    temperature  $\rm
T_c$,\cite{gamble70,prober80} despite  the  fact that the proximity  effect
should cause the opposite behavior.  In graphite-based superconductors, the
$\rm T_c$ of  the intercalant is enhanced  by intercalation, whereas in the
TMDC-based superconductors,  the $\rm T_c$   of the host  is   enhanced  by
intercalation.

        The enhancement of  $\rm T_c$ in  the TMDCIC's that  is  associated
with intercalation  is attributed to the  suppression  of  a charge-density
wave (CDW) state.\cite{murphy75} Besides suppression through intercalation,
CDW's in TMDC's can also be inhibited by hydrogen absorption\cite{murphy75}
and the  application of  small hydrostatic  pressures.\cite{monceau77} Both
hydrogen sorption  and  hydrostatic pressure in   suppressing the CDW raise
$\rm T_c$ in  the TMDC's.  Application  of a small pressure can  also raise
$\rm T_c$  in  the ternary GIC  superconductor $\rm C_4KHg$.\cite{delong83}
Recent experiments have  also shown  that  hydrogen sorption increases  the
transition temperature  of $\rm C_4KHg$.\cite{H242}   These experiments  in
combination with other  similarities  between the TMDC's and  GIC's suggest
the possibility of a CDW in $\rm C_4KHg$.\cite{delong83}
                
        The fundamental questions in the field of GIC superconductivity are
therefore  the following: Why  is $\rm C_8K$  superconducting?   Why do the
MHg-GIC's  have higher $\rm T_c$'s than  their intercalants  alone?  Why do
stage 2 MHg-GIC's have higher $\rm T_c$'s than  the  stage 1?  How  can one
account for the unusual angular and  temperature dependence of the critical
fields in GIC   superconductors?   and Why do  hydrogenation  and  pressure
dramatically increase  the $\rm T_c$ of $\rm  C_4KHg$?  Why aren't acceptor
GIC's superconducting?

        These questions  will be addressed   in  the chapters that  follow.
Chapter~\ref{othersys}     is  a     discussion    of non-GIC   anisotropic
superconductors  and the   ideas they  give regarding the  GIC experiments.
Chapter~\ref{samprep} explains how the samples used in the  new experiments
were  prepared, and discusses issues  in  GIC synthesis as  they  relate to
superconductivity.  Chapter~\ref{critf}, the  heart  of   this  work,  is a
report of detailed critical field studies  on $\rm C_4KHg$,  and an account
of what these  studies reveal about GIC  superconductivity.   Hydrogenation
experiments  on $\rm C_4KHg$ and the  possibility  of a CDW  suppression of
$\rm T_c$   are discussed  in    Chapter~\ref{hydrog}.   Chapter~\ref{csbi}
details     attempts    to find superconductivity    in    CsBi-GIC's,  and
Chapter~\ref{concl} is a summary.

\bigskip
\noindent {\bf A Word about Notation:}  In the GIC literature, it is
customary to measure all orientation angles  from the  graphite c-axis.  On
the  other  hand,  in the   anisotropic superconductivity literature it  is
customary to measure all angles from the layer  planes.  The GIC convention
has been adopted here,  and all formulae  taken from  papers  that used the
other   convention   have  been  converted.   Thus    the  direction called
$\parallel$ in the superconductivity literature is now called $\rm \perp
\hat{c}$, and the direction called $\perp$ in the superconductivity
literature is now called $\rm \parallel \hat{c}$.  While on  the subject of
confusion, it is worth mentioning that cgs units have been used throughout,
the  quantity N(0) is  the density-of-states for  both spin directions, and
$e$ is the electronic charge, a positive quantity.

        
