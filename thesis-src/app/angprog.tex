\chapter[Computer Programs used to Fit $\rm H_{c2}(\theta)$ Data]{Computer Programs used to Fit Angular Dependence of the Critical
Field Data}
\label{angprog}
\pagestyle{headings}
\markright{$\rm H_{c2}(\theta)$ Programs}

        This appendix contains the  source  code for the programs typeI and
newtink.  typeI performs a least-squares fit  of Eqn.~\ref{ldtheor} to $\rm
H_{c2}(\theta)$ data.  The program also allows for  the possibility of type
I behavior (as suggested in Equation~\ref{typeItheta}),  sample tilt (as in
Equation~\ref{tilthc2}), or mosaic spread (as in Equation~\ref{mostheta}).
The derivation of Equation~\ref{tilthc2} is given in Appendix~\ref{tiltderiv}.

        newtink  is     similar    to typeI,   only   it     is  based   on
Equation~\ref{tinkham} rather  than Equation~\ref{ldtheor}.  newtink allows
for mosaic spread and type I superconductivity, but not for sample tilt.

        Both typeI and newtink make use of some of the utility subroutines
listed in Appendix~\ref{utility}.

%typeI.c min.c fpeerr.c doublemod.c moshc2.c
%newtink.c  min.c fpeerr.c doublemod.c moshc2.c
%Makefile
