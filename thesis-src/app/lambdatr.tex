\chapter[Est\-imation of the Di\-men\-sion\-less Pa\-ram\-eter $\rm \lambda_{tr}$]{Estimation of the Di\-men\-sion\-less ``Clean\-li\-ness'' Pa\-ram\-eter}
\label{lambda_tr}
\pagestyle{headings}
\markright{Cleanliness Parameter $\rm \lambda_{tr}$}

        In Section~\ref{models},   the dimensionless  cleanliness parameter
$\rm
\lambda_{tr}$ is discussed.  $\rm \lambda_{tr} \gg 1$ indicates dirty-limit
superconducting behavior typical    of  disordered materials,   while  $\rm
\lambda_{tr} \ll 1$ is  indicative of clean-limit superconducting  behavior
typical  of single-crystals.  Below  $\lambda_{tr}$ is estimated   for $\rm
C_4KHg$.

        The standard definition of  $\rm \lambda_{tr}$ is\cite{orlando79}

\[ 0.882  \xi_0 / \ell \]

\noindent  where $\ell$ is the mean-free path, and

\[ \rm \xi_0 \equiv    0.18   \hbar  v_F  /   k_B   T_c   \]

\noindent is     the    BCS coherence
length.\cite{tinkham80}  Since for  $\rm C_4KHg$ the only experimental data
related to $\rm v_F$ is for in-plane transport, only the in-plane $\rm
\lambda_{tr}$ can be calculated.

        The Fermi velocity can be  estimated for $\rm  C_4KHg$ by using the
Shubnikov-deHaas data of Timp {\em et al.\/}\cite{W179} in conjunction with
the rigid-band  model for  the  graphitic $\pi$ carriers.  In this context,
the rigid-band model implies the use of graphitic band parameters for the shape of
the $\pi$-carrier  piece of the GIC's  Fermi  surface.   In  the rigid-band
picture, only the size of the $\pi$ piece of the Fermi surface changes 
upon intercalation, thus changing $\rm k_F$ and $\rm v_F$.  There
is   ample   precedent for  using   the     rigid-band  model to estimate
electronic parameters of the   donor
GIC's,\cite{W179,Z260}  although the  approximations  made are not as valid
for stage 1 as for higher stages.  Since c-axis transport is believed to be
through   the    intercalant    bands      rather   than   the graphitic  $\pi$
bands,\cite{M143,E343} the rigid-band analysis is inapplicable.  Therefore,
there is no good way of estimating $\rm v_F$  for the c-axis direction, and
no reasonable method to calculate $\xi_0$ or $\rm \lambda_{tr}$.

        Timp and colleagues\cite{W179} measured a  maximum in-plane Shubnikov-deHaas
frequency of $\rm \nu_{max}$ = 2490 T for $\rm C_4KHg$.  Since  $\rm k_F \,
= \, \sqrt{2e  \nu_{max} / \hbar  c }$, this gives $\rm  k_F$ = 2.75$\times
10^7$ cm$^{-1}$.\cite{Z260}  For graphitic $\pi$ bands, $\rm m^* \, = \, \hbar^2 k_F /
p_0$,         where    $\rm p_0$          =         1.08$\times       10^{-19}$
erg-cm.\cite{sugihara87} This formula  gives  $\rm  m^*  \, = \,
0.31 \: m_{e}$ for the in-plane effective mass of $\rm C_4KHg$, in excellent  agreement with the  $\rm m^*
\, = \, 0.32 m_{e}$ found using reflectivity.\cite{yang88}  Using
$\rm v_F \, = \, \hbar k_F / m^*$,\cite{sugihara87} the  result is $\rm v_F
\, = \, 1.0\times 10^8$ cm/s.  Then $\xi_0$ = 9000 \AA.

        The main obstacle  in estimating $\ell_a$,  the  mean-free-path for
in-plane transport, is that the in-plane  resistivity has not been reported
below 100 K  (see Section~\ref{chardisc}).\cite{elmakrini80b} However, $\rm
\rho_c$ measurements     have  been   published   down   to   liquid-helium
temperature.\cite{fischer83}  In order to make further progress, a
reasonable procedure is to estimate 

\[\rm \rho_a(4.2K) \, \approx \, \rho_c(4.2K) \frac{\rho_a(100K)}{\rho_c(100K)} \; .\] 

\noindent This procedure gives  a rough estimate of $\rm \rho_a(4.2K) \approx$  0.7 $\mu
\Omega$-cm, which compares well with the measured $\rm \rho_a(4.2K) \, = \,
0.6 \: \mu\Omega$-cm for $\rm C_8KHg$.\cite{pendrys81}

        For the mean-free path, the relationship is

\[\rm \ell_a \,  = \, \hbar  k_F / n e^2 \rho_a \; . \]

\noindent  For an ellipsoidal Fermi surface, the carrier density $n$ is
given by

\[ \rm n \; = \; \frac{4k_F^2}{3 \pi I_c}
\]

\noindent where $\rm I_c$ is the c-axis repeat distance.\cite{Z260}  Then
$\ell$ is 

\[ \rm \ell \; = \; \frac{4 \pi \hbar I_c}{3 e^2 \rho_a k_F} \; .
\]

\noindent  Plugging in the numbers gives $\ell_a \: \approx$ 9100 \AA\ at
liquid helium temperature.

        Finally now it is possible to calculate  $\rm \lambda_{tr} \approx$
0.86.   This is  indicative of  fairly   clean  superconductivity for  $\rm
C_4KHg$ for in-plane transport ($\rm \vec{H} \parallel  \hat{c}$), although
not really the ``clean limit''.

        All these formulae can be reduced to the expression 

\[ \rm \lambda_{tr} \; = \; 0.72 \frac{\rho_a k_F}{T_c I_c}
\]

\noindent where $\rm \rho_a$ is in $\Omega$-cm, k$\rm _F$ is in cm$^{-1}$,
$\rm T_c$  is in  K, and $\rm  I_c$ is in  \AA.  Getting $\rm k_F$ from the
$\rm  C_8KHg$  maximum SdH   frequency of  1490   T,\cite{W179}   a similar
calculation for  $\rm C_8KHg$ shows  $\rm \lambda_{tr}$   = 0.36.  For $\rm
C_8K$, using $\rm k_F$  =  4.7$\times 10^{7}$  cm$^{-1}$\cite{takada82} and
using $\rm \rho_a$  =  0.08 $\mu \Omega$-cm at  4  K (this  is actually the
value  for $\rm  C_8Rb$\cite{guerard77}),  $\rm  \lambda_{tr}$  is 22.   The
``dirty'' value of  $\rm \lambda_{tr}$ for $\rm  C_8K$ is  a result of  its
lower $\rm T_c$ and $\rm I_c$ that go into the equation above.
