\chapter{Estimation of the Hydrogen Debye-Waller Factor}
\label{hydrogdw}
\pagestyle{headings}
\markright{Hydrogen Debye-Waller Factor}

        In order to fit neu\-tron diff\-rac\-tion da\-ta on hy\-dro\-gen\-ated GIC's (see
Section~\ref{neutrons}),   it is   necessary  to have  an estimate   of the
hydrogen  Debye-Waller    factor.   The $\rm   C_4KH_x$   in\-el\-ast\-ic neu\-tron
scat\-ter\-ing data  of  Kamitakahara,   Doll, and Eklund,\cite{solin88}   were
employed here to make an estimate of  $\rm B_H$ using a procedure suggested
by Kamitakahara.\cite{kamitakahara88}

        
        Kamitakahara, Doll,  and    Eklund\cite{solin88} report  that   the
hydrogen vibrational  mode in $\rm C_4KH_{0.8}$  has a frequency  of 93 meV
in-plane and  71   meV  out-of-plane.  The Debye-Waller  factor corresponds
approximately to the mean-square vibrational amplitude of the
atom.\cite{ashcroft76}  Semiclassically, a harmonic oscillator has  energy

\[ \rm E \; \approx \; m \omega^2 \langle x^2 \rangle  \; \approx \; \frac{\hbar}{m \omega}
\]

\noindent where m is the mass of the oscillator,  $\omega$ its characteristic
 frequency, and  $\rm \langle x^2  \rangle$ its mean-square  displacement,
the quantity  of interest.  Plugging  in the numbers for  the  out-of-plane
vibration, the one relevant to $(00\ell)$ diffraction patterns,  gives $\rm
\langle x^2 \rangle$ = 0.058\AA$^2$.

        The Debye-Waller factor is  actually 8$\rm \pi^2 \langle  x \rangle
^2$     due  to  corrections   that   come   into    play    in  an   exact
calculation.\cite{ashcroft76} From the value of $\rm \langle x \rangle ^2$,
the final result is $\rm B_H$ = 4.58 \AA$^2$.
