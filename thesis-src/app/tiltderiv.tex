\chapter{Effect of Tilt on An\-gul\-ar De\-pen\-dence of the Crit\-ical Field}
\label{tiltderiv}
\pagestyle{headings}
\markright{Effect of Tilt}

        In this section, a detailed derivation of Eqn.~\ref{tilthc2} is
presented.   In order to understand the geometry of the sample tilt, the
reader is referred to Figure~\ref{misalign}.

        The underlying assumption of  the derivation presented here is that
the material in question has  uniaxial symmetry.   In the current  context,
uniaxial symmetry means that the material has a plane in which $\rm H_{c2}$
is constant.  The  angle between  the perpendicular  to this plane  and the
magnetic field is called $\theta$.  

        For graphite-based  superconductors,  the planes  of  constant $\rm
H_{c2}$ are  those parallel to the  graphite layers, and  the perpendicular
axis which defines the  angle $\theta$ is  called the  c-axis.   (The angle
$\theta$ is defined in  Figure~\ref{hc2def}.)    In-plane isotropy has  not
actually been  experimentally confirmed in  GIC superconductors  because of
the difficulty of  preparing single-crystal samples.  However, any in-plane
anisotropy  is expected to  be  small   because the in-plane band structure
should be relativelty  free-electron-like.\cite{holzwarth88} In addition, the
critical  field experiments discussed  in   this thesis were  performed  on
HOPG-based samples.   Since  HOPG    is  composed of  randomly     oriented
crystallites  in-plane, but  has a well-defined c-axis direction,\cite{I94}
GIC's   prepared from  HOPG   should meet the  definition  of   a  uniaxial
superconductor even if in-plane anisotropy is present in single crystals.

        The derivation of Eqn.~\ref{tilthc2} follows  the simple derivation
of  Eqn.~\ref{ldtheor2} that  was    published by   Morris,   Coleman,  and
Bhandari.\cite{morris72}  Let's  first  review   their  derivation.   These
authors started with the  idea  that  a the coherence  length of a uniaxial
superconductor in  any  direction is equivalent to  the  length of a vector
joining the center of a biaxial ellipsoid with its edge.  The ellipsoid has
a circular  cross-section of radius $\rm \xi_a$,   which corresponds to the
circular cross-section of a flux quantum formed when $\rm \vec{H} \parallel
\hat{c}$.  The elliptical cross-section of the ellipsoid corresponds to the
case where $\rm \vec{H}
\perp  \hat{c}$,  where the flux  quantum  has radii $\rm  \xi_a$ and  $\rm
\epsilon \xi_a$.  Here  $\epsilon$ is  the AGL  model anisotropy  parameter
introduced in Ref.~\cite{morris72}.  The biaxial ellipsoid  is shown in
Figure~\ref{fquant}a), and its two cross-sections in Figure~\ref{xsection}a).

        For  any  orientation of the applied  field, one  of the  coherence
lengths  in the plane perpendicular  to the  field will be $\rm \xi_a$, the
in-plane coherence  length.  The other  coherence length in this plane will
vary continuously with $\theta$ from $\rm  \xi_a$ to $\rm  \epsilon \xi_a$.
Thus

\begin{equation}
\rm H_{c2}(\theta) =  \Phi_0/2 \pi  \xi_a  \xi(\theta)
\label{xitheta}
\end{equation}

\noindent using   Eqn.~\ref{cohlen}.  Figure~\ref{fquant}a)   and    some     simple
geometrical manipulations show that $\rm \xi(\theta) \, = \, \xi_a
\sqrt{\cos^2 \theta
\, + \, \epsilon^2 \sin^2 \theta }$.  Plugging the $\xi (\theta)$
expression   into  Eqn.~\ref{xitheta}  above   gives
Eqn.~\ref{ldtheor2}:

\begin{equation}
\label{ldtheor2}
\rm H_{c2}(\theta) \; = \; \frac{H_{c2}(0^{\circ})}{\sqrt{\cos^2\theta \,
+ \, \epsilon^2 \sin^2\theta}}
\end{equation}

\noindent as expected.  This is the result obained by Morris, Coleman, and Bhandari.\cite{morris72}

\begin{figure}
\vspace{15cm}
\caption[Coherence lengths of a uniaxial superconductor.]{Coherence lengths
of  a  uniaxial  superconductor.    As  pointed out  by   Morris  {\em   et
al.\/},\cite{morris72} the coherence length of a uniaxial superconductor is
the length of a vector from the center of a biaxial ellipsoid to  its edge.
a) The case of an aligned sample, which is described by Eqn.~\ref{ldtheor2}.
(See Figure~\ref{misalign}a).)  The ellipsoid has two  radii of length $\rm
\xi_a$ and one of length $\rm
\epsilon \xi_a$.  b) The case where the sample is tilted by an angle
$\phi$, which  is described by  Eqn.~\ref{tilthc21}.  Now  the  ellipsoid is
triaxial, with one of  the  coherence  lengths of size $\rm \xi_a$  from a)
being replaced by one of size $\rm \xi_a
\sqrt{\cos^2 \phi \, + \, \epsilon^2 \sin^2 \phi}$.}
\label{fquant}
\end{figure}

\begin{figure}
\vspace{18cm}
\caption[Cross-section of the flux quantum in a uniaxial
superconductor.]{Cross-section   of   the  flux   quantum   in   a uniaxial
superconductor.  In all cases,  the magnetic field is directed  out  of the
paper. a) The  aligned  case.   For $\rm  \vec{H}  \parallel  \hat{c}$, the
cross-section of  the flux quantum  along the  field direction is circular.
b) The tilted case.  Now that  the ellipsoid that  determines $\xi(\theta)$
is triaxial, the cross-section of the flux quantum  is non-circular for all
field orientations.}
\label{xsection}
\end{figure}

        Now let's turn to the case  where the sample is  tilted so that its
c-axis is rotated by an angle $\phi$ away  from the horizontal plane, which is
the plane in which the applied field is rotated.  From examination of
Eqn.~\ref{ldtheor2}, it is obvious that the measured magnitude of $\rm
H_{c2, \parallel \hat{c}}$ will be $\rm H_{c2, \parallel \hat{c}} ^{eff}
\equiv H_{c2,\parallel \hat{c}}/ \sqrt{\cos^2 \phi \, + \, \epsilon ^2
\sin^2 \phi}$.  Note that this gives $\rm H_{c2, \parallel \hat{c}}^{eff} =
H_{c2,   \parallel  \hat{c}}$ when     $\phi$  =   0$^{\circ}$.   If   $\rm
H_{c2}(\theta)$ still is given by Eqn.~\ref{xitheta}, then it  follows that
$\xi(0^{\circ})$ = $\rm \xi_a \sqrt{\cos^2 \phi \, + \, \epsilon^2 \sin^2
\phi}$.   

        Now that $\xi(\theta)$ has been  found  for $\theta$ = 0$^{\circ}$,
the   full form  of   the function   can  be   determined   by  noting that
$\xi(90^{\circ})$ still is  $\rm \epsilon  \xi_a$.  The reason is  that the
tilt of the c-axis from  the horizontal  corresponds to a rotation about an
in-plane direction (an $\hat{a}$  direction), so the  in-plane $\rm H_{c2}$
is not affected by   the  tilt. This statement is   no longer true  if $\rm
H_{c2}$ is not constant in  the layer plane  (non-uniaxial superconductor),
although  everything  said  up  to this   point holds for  that case. Using
standard formulae for an ellipse gives for general $\theta$

\[
\rm  \xi(\theta) \, = \,  \xi_a
\sqrt{\cos^2 \phi \, + \,   \epsilon^2 \sin^2  \phi} \:  \sqrt{\epsilon_{eff}^2 \sin^2\theta \, +  \,
\cos^2 \theta} \;   
\]

\noindent where $\rm \epsilon_{eff}$ = $\epsilon / \sqrt{\epsilon^2 \sin^2
\phi \, + \, \cos ^2 \phi}$.  Plugging this formula for $\xi(\theta)$ into
Eqn.~\ref{xitheta} now gives Eqn.~\ref{tilthc21}:

\begin{equation}
\label{tilthc21}
\rm H_{c2}^{eff}(\theta,\, \phi) \; = \; \frac{H_{c2, \parallel
\hat{c}}/\sqrt{\cos^2\phi \: + \: \epsilon^2\sin^2\phi}}{\sqrt{\cos^2\theta
\: + \: \epsilon^2\sin^2\theta / \sqrt{\cos^2\phi \: + \: \epsilon^2\sin^2\phi}}} 
\end{equation}

\noindent as desired.   
