\section{Upper Critical Field Studies of Transition Metal Dichalcogenides 
            and Their Intercalation Compounds}
\label{tmdc}

	The transition metal dichalcogenides (TMDC) were the  first layered
superconductors to be intensively studied.   The observation of anisotropic
superconductivity in these  materials  is  not surprising considering  that
they contain some   of the transition   metal elements (Nb, Ta)  that  show
anisotropic   superconductivity   in  bulk form.   As  it   turns out,  the
properties  of the  TMDC's  are   qualitatively  similar  to those  of  the
transition metals on which they are based, only with a larger anisotropy.

\begin{table}
\caption[Values of the KLB parameter $r$ for a number of
transition metal dichalcogenides.]{Values of the KLB parameter $r$ (defined
in Eqn.~\ref{klbparam1})for a  number of  transition metal dichalcogenides.
$r \,  < \,  1.7$ means that   a 3D-2D  crossover  will  occur at  a finite
temperature.  s is the layer repeat distance,  which is equivalent  to $\rm
I_c$   for GIC's. NR  means that  this parameter   was not reported  in the
indicated reference.}
\label{tmdcklb}
\begin{center}
\begin{tabular}{|c|ccccc|}
\hline
Compound & $\rm T_c$ (K) & $r$ & 1/$\epsilon$ & $\xi_{\parallel \hat{c}}(0)$ & s (\AA) \\
\hline
& & & & & \\
TaS$_{1.6}$Se$_{0.4}$\cite{prober80}& 4.1 & 4.53& 5.5& 18.2& 6.1\\
& & & & & \\
4H(a)-Nb$_{0.8}$Ta$_{0.2}$Se$_2$\cite{ikebe80}& 5.37 &  20& 6.7& 13& NR \\
& & & & & \\
4H(a)-Nb$_{0.7}$Ta$_{0.3}$Se$_2$\cite{ikebe80}& 4.61 &  27& 6.4& 15& NR \\
& & & & & \\
Fe$_{0.05}$TaS$_2$\cite{coleman83}& 3.0 & 25.6 & $\geq$20 & 13.7 & 6.0 \\
& & & & & \\
2H-NbSe$_2$\cite{prober80}& 7.1 & 87& 2.9& 26 & 6.3 \\
& & & & & \\
2H-NbSe$_2$\cite{toyota76}& 7.2-7.3 & 100& 3.2& 27 & 6.3\\
& & & & & \\
2H-TaS$_2$\cite{muto78,coleman83}& 0.8-1.2 & 2500& 6.0 & NR  &6.0 \\
\hline
\end{tabular}
\end{center}
\end{table}

        Table~\ref{tmdcklb} shows selected properties of some  of the TMDC
superconductors.  Interest in  the TMDC's derives  partly from their highly
anisotropic critical fields ($\epsilon$  as large as  13.7.\cite{coleman83})
$\rm  H_{c2}(\theta)$ data for  the prototypical TMDC superconductor,  $\rm
NbSe_2$,  are shown  in  Figure~\ref{nbse2theta} along  with  a fit  to the
square of Eqn.~\ref{ldtheor1}.

\begin{figure}
\vspace{12cm}
\caption[$\rm H_{c2}(\theta)$ for NbSe$_2$, a TMDC superconductor]{$\rm 
(H_{c2}(\theta)/H_{c2}(90^{\circ}))^2$ versus angle   from   the   NbSe$_2$
planes.\cite{muto73} Here  $\theta \, =  \, 90^{\circ}$  is  defined to  be
along  the  crystalline  c-axis.    The   fit  is   with the    square   of
Eqn.~\ref{ldtheor1}.}
\label{nbse2theta}
\end{figure}

        Besides their  anisotropy, the  other prominent feature of the TMDC
results is the  unusual temperature dependence of  their  critical  fields.
Nearly all of the TMDC's and their intercalation compounds show an extended
linearity or  positive curvature  of their critical fields.\cite{woollam74}
That  is, $\rm  H_{c2}(T)$  does not   show the  low-temperature saturation
expected  from the standard   WHHM theory.   This  deviation from the usual
theory  occurs despite  the fact  that the TMDC's   are 3D superconductors,
albeit highly anisotropic ones.  The unusual shape of the TMDC data are best
displayed by plotting $\rm h^*
\, \equiv \,  H_{c2}  / (dH_{c2}/dT \:  *  \: T_c)$  versus t ($\equiv T/T_c$)
since   this  plot  should  yield a   universal  curve for  superconductors
describable by the WHHM theory.\cite{helfand66} An $\rm  h^*$ versus t plot
for Nb$_{1-x}$Ta$_{x}$Se$_2$ samples  is   shown in Figure~\ref{nbse2temp}.
The linear behavior of the data  causes it to  deviate above the dashed and
solid         curves    that      show         typical      low-temperature
saturation.\cite{saintjames69,orlando79} $\rm H_{c2}$ also   deviates above
the WHHM prediction in the transition metals, although  the effect there is
smaller.\cite{kerchner81}

\begin{figure}
\vspace{12cm}
\caption[Extended linearity of $\rm H_{c2}(t)$ in  Nb$_{1-x}$Ta$_x$Se$_2$]{Reduced 
field     $\rm  h^*$    versus  reduced   temperature    t  for     several
Nb$_{1-x}$Ta$_x$Se$_2$  samples.\cite{ikebe80}  The dashed and solid curves
are   the WHHM  theory   for   dirty  and  clean  limit    superconductors,
respectively.  The third curve is the best linear fit.}
\label{nbse2temp}
\end{figure}
%Figure 4  from Ref.~\cite{ikebe80}  

        Two  types of   superconducting TMDCIC's have   been  studied:  one
variety  intercalated  with  alkali  metal and   alkaline  earth atoms, and
another class  intercalated with large  organic molecules.  The first group
are qualitatively  similar to the  pristine TMDC superconductors, only with
larger   angular   anisotropy  and  greater  positive   curvature  of  $\rm
H_{c2}(T)$.\cite{woollam76} The alkali-metal  intercalation compounds based
on the TMDC's appear to be 3D  superconductors, just  like the alkali-metal
GIC's.\cite{tanuma81}   As       Table~\ref{tmdcicklb}     shows,       the
three-dimensionality   of  the    alkali   intercalates  of the  TMDC's  is
demonstrated by  their  value of $r \, \geq  3.0$.  Therefore in going from
the  transition  metals to  the TMDC's  to  the alkali-intercalated TMDC's,
qualitatively   the   same behavior     is seen.  As       the   number  of
non-superconducting  layers  between   the   transition  metal    layers is
increased, the main trend is  increasing positive curvature of the critical
fields.

% The $\rm H_{c2}(T)$ data for Cs$_{0.3}$MoS$_2$
%are  shown   in  Figure~\ref{mos2fig}.    $\rm H_{c2}(\theta)$   for  these
%materials is fairly well fit by Eqn.~\ref{ldtheor1}.

        The alkali intercalation   compounds of  MoS$_2$ are  of particular
interest to those working with GIC's  since MoS$_2$ is a semiconductor, but
its intercalation compounds are metals with  $\rm  T_c$ on the order of 4-6
K.\cite{somoano73}  Thus  the alkali-metal    intercalates of   MoS$_2$ are
superconducting despite the fact that  neither of the starting materials is
superconducting,   a   situation  much   like  that  of  the   alkali-metal
GIC's.\cite{hannay65}   As in the  alkali-metal   GIC's,  the occurrence of
superconductivity in the alkali-metal/MoS$_2$  compounds  is  attributed to
charge transfer  from   the  alkali $s$-bands to   the  bands  of  the host
material.\cite{somoano73}

\begin{table}
\caption[Values of the KLB parameter $r$ for a number of
transition metal dichalcogenide intercalation compounds.]{Values of the KLB
parameter $r$ for a number of transition metal dichalcogenide intercalation
compounds.  s  is the layer repeat  distance, which  is equivalent  to $\rm
I_c$  for GIC's.  \dag\  means that the   value  of  $r$ was calculated from
parameters in the cited references. NR means the parameter was not reported
in the cited reference.}
\label{tmdcicklb}
\begin{center}
\begin{tabular}{|c|ccccc|}
\hline
Compound & $\rm T_c$ (K) & $r$ & 1/$\epsilon$ & $\xi_{\parallel \hat{c}}(0)$ & s (\AA) \\
\hline
& & & & & \\
TaS$_{1.6}$Se$_{0.4}$/pyridine\cite{prober80} & 2.1 & 0.32& 75& 3.0 & 12.0  \\
& & & & & \\
TaS$_{1.6}$Se$_{0.4}$/collidine\cite{prober80}& 2.6 & 0.41& 30& 2.8 & 9.9 \\
& & & & & \\
2H-TaS$_2$/aniline\cite{prober80} & 2.9 & 0.5& 37  & 5.6& 17.9 \\
& & & & & \\
2H-TaS$_2$/pyridine\cite{muto78}& 3.16  & 0.8& NR & 5.0 & 11.85 \\
& & & & & \\
2H-TaS$_2$/pyridine\cite{coleman83} & 3.6 & 1.3& $\geq$50 & 6.0 & 11.85 \\
& & & & & \\
2H-TaS$_2$/pyridine\cite{prober80} & 3.47 &  1.9& 25 & 7.3 & 12.0 \\
& & & & & \\
2H-TaS$_2$/collidine\cite{prober80}& 3.2 & 1.95& 30& 6.0 & 9.7 \\
& & & & & \\
2H-TaS$_2$/methylamine\cite{coleman83}& 4.0 &0.91-2.2 & $\geq$60 & 4.0-6.1 & 9.24 \\
& & & & & \\
Cs$ _x$MoS$_2$\cite{woollam76,somoano73}& 6.9 & 3.0$^{\dagger}$ & 6.1 & 15.0 & 19.6 \\
& & & & & \\
Na$ _x$MoS$_2$\cite{woollam76,somoano75} & 3.6 & 22.7$^{\dagger}$ & 6.7 & 31.7 & 15.0 \\
& & & & & \\
Sr$ _x$MoS$_2$\cite{woollam76,somoano75}& 5.6 & 31.2$^{\dagger}$ & 4 & 46.0 & 18.6 \\
& & & & & \\
Ca$ _x$MoS$_2$\cite{woollam76,somoano75} & 4.0 & 71.5$^{\dagger}$ & 5 & 69.7 & 18.6 \\
& & & & & \\
2H-TaS$_2$/ethylenediamine\cite{coleman83} & 4.0 & 24.8 & $\geq$33 & 21.0 & 9.53 \\
& & & & & \\
2H-TaS$_2$/dimethylamine\cite{coleman83}& 4.6 & 32.4 & $\geq$28 & 24.0 & 9.59 \\
& & & & & \\
\hline
\end{tabular}
\end{center}
\end{table}


        The tran\-si\-tion  metals which   are them\-selves su\-per\-con\-duct\-ing  also
show anom\-a\-lous  effects upon in\-ter\-cal\-ation.  The  in\-ter\-cal\-ation  of organic
molecules   into    several     TMDC's,  such  as   NbS$_2$,\cite{gamble70}
TaSe$_{1.6}$Se$_{0.4}$,\cite{prober80} and TaSeS\cite{morris73}   depresses
$\rm T_c$.  This  depression  is  expected  according  to  proximity-effect
theories\cite{degennes66,cooper61} which say that layering a superconductor
with an insulator should lower $\rm T_c$.  On the other hand, in\-ter\-cal\-ation
of organic  molecules  into other TMDC's  such  as TaS$_2$,  TaSe$_2$,  and
PdTe$_2$  actually increases $\rm  T_c$.\cite{gamble70} The compounds whose
$\rm  T_c$   was   enhanced by  in\-ter\-cal\-ation   were  later found   to   be
charge-density  wave  (CDW)   materials.\cite{wilson74,dalrymple83}   Since
superconductivity  and CDW's both   get their condensation  energy from the
opening of a gap on the Fermi surface, they  are competing ground states of
the system.  Therefore  any modification (such  as  in\-ter\-cal\-ation) that can
suppress the  CDW enhances superconductivity.  CDW's   are quite common  in
layered   materials.\cite{friend79} The    possibility  of a  CDW  in   the
KHg-GIC's\cite{delong83} will be discussed further in Chapter~\ref{hydrog}.

        Unlike         the        alkali-intercalated      TMDC's,      the
organic-molecule-intercalated TMDC's show qualitatively different  behavior
from their parent  compounds in the  form of a   transition from 3D   to 2D
coupling.\cite{coleman83,prober80} The  3D-2D  crossover is manifested as a
kink  in  the  $\rm   H_{c2,  \perp  \hat{c}}(T)$    curves,  as  shown  in
Figure~\ref{tmdcic}a).  
Examination of the  data shows  that  the shape  of the curve  does in fact
change  from approximately  linear with  temperature to  approximately  the
square-root of temperature behavior expected in the 2D  limit.  The data is
well-described by the KLB model of dimensionality change.\cite{klemm75}

\begin{figure}
\vspace{16cm}
%Fig. 6a from coleman83
\caption[Dimensionality crossover in methylamine-intercalated TaS$_2$]{Dimensionality crossover in methylamine-intercalated TaS$_2$.\cite{coleman83}
a) $\rm H_{c2,  \perp \hat{c}}$.  Note the  deviation from  linear behavior
near 3 K (= T$^*$) which indicates the dimensionality crossover.  The solid
line     is  a fit   to    the     KLB  theory.\cite{klemm75}     b)   $\rm
10^{-3}(H_{c2}(\theta)/H_{c2}(90^{\circ}))^2$ versus angle from the TaS$_2$
planes.   The fit to  Eqn.~\ref{ldtheor1} (the  solid  line) is much better
above  than  below  the dimensionality  crossover  at about  3  K.  The fit
parameters are also shown.}
\label{tmdcic}
\end{figure}
%a) is Figure 6a from Ref.~\cite{coleman83}.  b) is Figure18b from Ref.~\cite{coleman83}.

        As discussed  in the previous  section,  the two  signatures of  2D
coupling are a square-root of   T temperature   dependence and a  Tinkham's
formula   angular    dependence for $\rm H_{c2}$.      $\rm    H_{c2}(\theta)$    data   for
methylamine-intercalated  TaS$_2$ are shown in  Figure~\ref{tmdcic}b) along
with a fit  to Eqn.~\ref{ldtheor1}.\cite{coleman83}  The fit is much better
in the 3D regime (T $>$ T$^*$) than in the 2D regime (T $<$ T$^*$),  as one
would expect.   Unfortunately none  of the  research groups  who  have done
extensive research on the TMDCIC's seem to  have fit their  data in  the 2D
regime to Tinkham's formula.\cite{coleman83,prober80}

	The  primary result of  the critical field experiments on  the TMDC's
and  their   intercalation compounds   was  the   first observation   of  a
dimensionality crossover.  The other  notable findings are measurement for
the first time of a large critical field anisotropy, and the observation of
positive curvature in  $\rm H_{c2}(T)$  both with and without an associated
dimensionality crossover.  The TMDCIC experiments are, however, limited by
several factors.  One   is  crystal   quality: the  samples   tend to   be
``wrinkled''      and   may       include     several   of    the    stable
polytypes.\cite{prober80}  These inhomogeneities    will   obviously impact
negatively upon any anisotropy measurements.  The other problem with the
TMDCIC experiments is that the critical field  anisotropy is so  large that
accurate measurements of $\rm H_{c2}(\theta)$ become quite difficult for
$\theta$ near 90 $^{\circ}$.  Both
of  these difficulties are  lessened with   the advent of the artificially
structured superlattices described in the next section.

