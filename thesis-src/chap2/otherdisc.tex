\section{Discussion and Conclusions}
\label{otherdisc}

        Comparison  of critical field  experiments  on the  many   types of
anisotropic   superconductors  shows  that  these disparate  materials look
surprisingly similar.  In particular positive  curvature of $\rm H_{c2}(T)$
and temperature-dependent  anisotropy appear to  be  common  to most of the
members of the class.  Somewhat different features occur in those materials
in which the  superconducting layers decouple  at  low temperatures.  These
materials undergo a transition  which is marked by a  discontinuity in $\rm
H_{c2}(T)$ and a change from an angular dependence of  $\rm H_{c2}$ from 3D
anisotropic (Eqn.~\ref{ldtheor1})  to  thin-filmlike (Eqn.~\ref{tinkham1}).
Except  for    those of  the   TMDC's  in   which  a   CDW   competes  with
superconductivity,       diluting     the superconducting    layers    with
non-superconducting layers   depresses $\rm   T_c$,  in    accordance  with
proximity-effect theories.  As discussed further later on, the KHg-GIC's do
not  follow the proximity effect  $\rm T_c$-dependence,  with the $\rm T_c$
even of stage 3 higher than that  of the pristine KHg alloy.\cite{J140} The
findings from the  other anisotropic superconductors suggest  that the lack
of $\rm T_c$ depression with increasing stage may be a crucial clue.

        Most   of  the findings from    previous  studies  on   anisotropic
superconductors can   be  summarized  by examining  the progression from Nb
metal   to    NbSe$_2$      to      alkali-intercalated      TMDC's      to
organic-molecule-intercalated NbSe$_2$.   Niobium metal in bulk form  is an
anisotropic  3D   superconductor described  by  Fermi   surface  anisotropy
theories.\cite{butler80,kerchner81} NbSe$_2$ shows the same  features, only
with a greater degree of positive curvature (or extended linearity) of $\rm
H_{c2}(T)$  and   larger  anisotropy.\cite{dalrymple84,muto73,ikebe80}  The
alkali-metal and alkaline-earth TMDCIC's are even more anisotropic and have
even  larger  anomalies in   $\rm H_{c2}(T)$.\cite{somoano75,woollam76} The
organic TMDCIC's are 2D-coupled superconductors, like Nb thin films, and so
are qualitatively different from the others, which are 3D-coupled.  Yet the
organic TMDCIC's are also clearly the endpoint  of a continuum which has bulk Nb
as its other limit.

        The materials  in this chapter  also have  many features in  common
with the superconducting  GIC's which are the subject  of the  rest of this
work.   Many of  the theories  developed   for  application to  these other
systems will  be useful in the study  of  the  superconducting GIC's, so  a
familiarity  with the previous  work is mandatory.   The similarity between
the alkali-metal GIC's  and the alkali-metal  TMDCIC's is already apparent,
but an  attempt will    also  be made  to    show  the likeness    of   the
organic-intercalated TMDCIC's and the non-superconducting acceptor GIC's.

        The consideration  of  the properties of the  superconducting GIC's
will begin with the preliminaries, sample preparation and characterization.
