\section{Upper Critical Field Studies of High Temperature 
Superconductors}
\label{hitc}

        The study of high-$\rm  T_c$ superconductors is  still very new, so
the critical field properties are still  being sorted out.  Even though the
number  of papers published  on the copper  oxides  must already be several
times the number published on the  TMDC's  and the synthetic superlattices,
there  are several difficulties   standing  in   the  way  of a  consistent
understanding   of  the   properties  of  these  materials.   The principal
experimental   difficulty is  the  (currently  unobtainable) high  magnetic
fields needed to measure $\rm H_{c2}$ in an  extensive temperature range, a
problem which is   exacerbated  by the quite broad  transitions  seen in  a
magnetic  field.   Even pulsed  magnetic  field   experiments  do not allow
measurement   of  $\rm   H_{c2,   \perp   \hat{c}}$  below  about    t    =
0.85,\cite{orlando87a,sakakibara87} which means  that any conclusions drawn
must be considered tentative because of the limited magnetic field range of
the data.

        The other major difficulty with the high-temperature critical field
measurements  so  far is  the uncertain  theoretical picture of  these  new
materials.  Until  a theory  of   high-$\rm  T_c$  superconductivity  earns
widespread  acceptance,  experimentalists must   rely  on  the  formulae of
low-$\rm  T_c$   superconductivity,  such  as  the   WHHM  theory   and the
anisotropic     Ginzburg-Landau    theory   (see     Section~\ref{models}).
Interpretation of the data in terms of these models may well be misleading,
but they are the best anyone can do under the present circumstances.

        Even  with all  these warnings  in mind, the  data  that have  been
published  so far on the  high-$\rm T_c$ materials  are intriguing to those
familiar  with previous  research  on anisotropic superconductors.  Some of
the    parameters  from   recent    experiments  on   the   high-$\rm  T_c$
superconductors are   gathered  in   Table~\ref{hitctable}.    The  angular
dependence of $\rm H_{c2}$  measurements  are the  most straighforward   to
interpret since they are the least affected by magnetic  field limitations.
$\rm  H_{c2}(\theta)$  data taken      so    far  shows  that  the     90 K
rare-earth-barium-copper-oxide compounds  are 3D superconductors,  well fit
by     Eqn.~\ref{ldtheor1}.      $\rm       H_{c2}(\theta)$    results   on
HoBa$_2$Cu$_3$O$_x$ from the work  of   Iye {\em et  al.\/}  are  shown  in
Figure~\ref{hitctheta}.

\begin{figure}
\vspace{12cm}
\caption[$\rm H_{c2}(\theta)$ for a high-$\rm T_c$ superconductor]{$\rm
H_{c2}(\theta)$   for     a    high-$\rm   T_c$    superconductor.     From
Ref.~\cite{iye87a}.   Each  set of  symbols  corresponds  to   a  different
definition of $\rm H_{c2}$. For example, the curve labeled 0.7 R$\rm _N$ was
obtained  by  plotting versus  $\theta$   the  values  of  H  that  satisfy
R(H,$\theta$)  = 0.7 $\rm R_N$.   Each curve is  also labeled in parentheses
with the magnitude of the anisotropy  parameter  1/$\epsilon$ that was used
for the fit to Eqn.~\ref{ldtheor1}.}
\label{hitctheta}
\end{figure}

        A  series of  $\rm  H_{c2}(\theta)$   curves  taken   at  different
temperatures indicates    that   the   anisotropy parameter   $\epsilon$ is
temperature-dependendent    in    these     materials.\cite{iye87a}       A
temperature-dependent $\epsilon$   has   been widely  observed  in  layered
materials    such as Nb$_{(1    -  x)}$Ta$_x$Se$_2$,\cite{dalrymple84} $\rm
C_8K$\cite{koike80},        $\rm      C_8KHg$,\cite{B340}       and   Nb/Cu
superlattices\cite{chun84},  where it  has been  associated  with  positive
curvature of $\rm H_{c2}(T)$ .  As  noted previously, positive curvature of
the critical field parallel to the layer planes ($\perp$ to the c-axis) may
be associated with a  dimensionality  crossover, but the positive curvature
often observed in $\rm H_{c2}$ perpendicular to  the layer planes must have
a different    cause.   In the   high-$\rm  T_c$  superconductors, positive
curvature  has been  observed for both  field orientations by some research
groups,\cite{iye87,iye87a,moodera88,sakakibara87}    while   a       linear
temperature  dependence  has been reported for   both field orientations by
others.\cite{orlando87,orlando87a,uchida87b,vanbentum87}  Worthington  {\em
et  al.\/}, on  the   other hand, see no sign   of  positive curvature, but
instead  a change in the linear  slope of $\rm H_{c2}$.\cite{worthington87}
These disagreements about the  positive curvature and temperature-dependent
$\epsilon$ are reminiscent of  a similar controversy  about NbSe$_2$ in the
'70's.\cite{foner73,muto73}  Whether  the   disparaties among  the  various
research groups are  due to  differences in   sample quality  or  slightly
different methods of data analysis remains to be seen.

\begin{table}
\begin{center}
\caption[Selected properties of some of the high-$\rm T_c$
superconductors]{Selected  properties  of  some   of  the   high-$\rm  T_c$
superconductors.  $\dagger$  indicates a  single-crystal  measurement.  The
coherence  length quoted  for ceramic sample  is an average  one.  NR means
that the parameter was not reported in the cited reference.}
\label{hitctable}
\begin{tabular}{|ccccc|}
\hline
& & & & \\
Compound& $\rm T_c$ (K)& 1/$\epsilon$ & $\xi(0)$ (\AA)& Ref. \\
& & & & \\
\hline
%La$_{2-x}$Ba$_x$CuO$_{4-y}$ &  & 5 & & \cite{hidaka87}\\
La$_{2-x}$Sr$_x$CuO$_{4-y}$ & 38.1 & NR & 13 & \cite{orlando87b}\\
La$_{1.92}$Sr$_{0.08}$CuO$_{4-y}$$^{\dagger}$ & 20  & 5 & 37($\rm \perp \hat{c}$); 7($\rm \parallel \hat{c}$) & \cite{iye88}\\
La$_{2-x}$Sr$_x$CuO$_{4-y}$ & $\approx$30 & 13 & NR  & \cite{uchida87}\\
La$_{2-x}$Sr$_x$CuO$_{4-y}$ & 36 & 4-5 & 26.6 & \cite{vanbentum87}\\
YBa$_2$Cu$_3$O$_{(7-y)}$ & 81-90 & 20 & NR & \cite{mcguire87}\\
YBa$_2$Cu$_3$O$_{(7-y)}$ & 91.4 & $<$12 & 14.1 & \cite{orlando87}\\
YBa$_2$Cu$_3$O$_{(7-y)}$ & $\approx$90 & NR & 15.2 & \cite{rettori87}\\
YBa$_2$Cu$_3$O$_{(7-y)}$ & 89.5 & NR & 17 & \cite{orlando87a}\\
YBa$_2$Cu$_3$O$_{(7-y)}$$^{\dagger}$ & 88.8 &10  & NR & \cite{dinger87}\\
YBa$_2$Cu$_3$O$_{(7-y)}$ & 92 & 4-5 & 12.6 & \cite{vanbentum87}\\
YBa$_2$Cu$_3$O$_{(7-y)}$$^{\dagger}$ & 88.8 & 4.8 &34 ($\rm \perp \hat{c}$); 7 ($\rm \parallel \hat{c}$)  & \cite{worthington87}\\ 
YBa$_2$Cu$_3$O$_{(7-y)}$ & 93 & 25-50 & NR & \cite{welch87}\\
YBa$_2$Cu$_3$O$_{(7-y)}$$^{\dagger}$ & 89.7 & 3.6 & 23 ($\rm \perp \hat{c}$); 6.3 ($\rm \parallel \hat{c}$)  & \cite{moodera88}\\ 
YBa$_2$Cu$_3$O$_{(7-y)}$$^{\dagger}$ & 91 & 2 (90 K); 5 (86.5 K) & 33.9 ($\rm \perp \hat{c}$); 10 ($\rm \parallel \hat{c}$)  & \cite{iye87}\\ 
YBa$_2$Cu$_3$O$_{(7-y)}$$^{\dagger}$ & 94 & 6 & 22 ($\rm \perp \hat{c}$); 6 ($\rm \parallel \hat{c}$)  & \cite{sakakibara87}\\ 
HoBa$_2$Cu$_3$O$_{(7-y)}$$^{\dagger}$ & 88.1 & 3-7 & 22-24 ($\rm \perp \hat{c}$); 6-7 ($\rm \parallel \hat{c}$)  & \cite{iye87a}\\ 
Bi$_2$Sr$_2$CaCu$_2$O$_{(8+x)}$$^{\dagger}$ & 80 & 6 &30-40 ($\rm \perp \hat{c}$); 1-3 ($\rm \parallel \hat{c}$)  & \cite{iye88}\\ 
%EuBa$_2$Cu$_3$O$_{(7-y)}$ & & 4.3 & & \cite{hikita87}\\
TmBa$_2$Cu$_3$O$_{(7-y)}$$^{\dagger}$ & 86 & 8 &NR & \cite{noel87}\\
\hline
\end{tabular}
\end{center}
\end{table}

        Clearly a lot more work has  to be done  before the $\rm H_{c2}(T)$
data in  the copper-oxides can be understood.   One  of the major questions
still to be answered  is whether the  superconductivity in  these materials
becomes    two-dimensional  at   low temperatures.    The  good  fit     of
Eqn.~\ref{ldtheor1} (rather than  Tinkham's formula)  to the  data suggests
that the copper-oxides are 3D superconductors  at least  in the temperature
range accessible  with currently obtainable fields.  The  coherence lengths
which  have been measured    so   far support  the   identification   of 3D
coupling.\cite{worthington87}  However, a decoupling  of the  copper-oxygen
planes is suggested by the  fact that the  presence of  magnetic rare earth
ions between   the  planes  does    not seem to   suppress  the  transition
temperature.\cite{tamegai87,orlando87}  Also,  Table~\ref{hitctable}  shows
that the coherence length  $\parallel$ to the  c-axis is expected to  be on
the order of the c-axis lattice constant at low-temperature, which suggests
that a coupling-dimensionality  change might occur.  For  example, for $\rm
HoBa_2Cu_3O_{(7-y)}$ the c-axis lattice parameter is 11.67
\AA\cite{tamegai87} and $\rm
\xi_{\parallel \hat{c}}$ is about 6.5 \AA.  Plugging these numbers into
Eqn.~\ref{klbparam1} gives r = 1.58, slightly less  than the critical value
of 1.7.  Obviously estimates of this type are not to be taken too seriously
considering the large extrapolation.

        Another  interesting   question  is   whether  the   copper-oxides'
transition temperatures follow the predictions of the proximity effect.  If
the copper-oxygen planes are truly the superconducting layers,  then putting
several of  them together without separation by  an insulating layer should
increase $\rm T_c$,  according to  proximity-effect  models.   Exactly this
type        of      enhancement     has         been      seen      in  the
thallium-barium-calcium-copper-oxygen   superconductors,  where   $\rm T_c$
increases as  a  func\-tion of the  num\-ber of copp\-er-ox\-y\-gen   planes that are
ad\-ja\-cent.\cite{pool88} Of  course this $\rm  T_c$ behavior is  not proof of
proximity-effect  behavior    since   another mechanism    could  easily be
responsible if the pairing in these materials is non-conventional.

        The picture from the currently  available papers  on high-$\rm T_c$
superconductivity is obviously quite preliminary.  So far  they appear most
like  the   transition metal  dichalcogenides  in    their  critical  field
properties: they have an   anisotropy  of about  5,  $\rm   H_{c2}(\theta)$
well-fit  by Eqn.~\ref{ldtheor1}, and possible  positive  curvature of $\rm
H_{c2}(T)$    and    temperature-dependent   anisotropy.       Even if  the
superconductivity of the high-$\rm T_c$  materials turns out to be mediated
by an  exotic interaction,  the comparison with the  superconducting TMDC's
and GIC's will still be instructive.
