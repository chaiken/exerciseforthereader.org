\section{Introduction}
\label{otherintro}

        Anisotropy effects  are fundamental  to superconductivity.  This is
true because   any non-cubic material   that can be grown with   sufficient
structural perfection should show some anisotropy in its superconductivity.
Even  cubic   materials    can show deviation     from  isotropic models of
superconductivity if they have non-spherical Fermi surfaces.\cite{butler80}
Thus,   as  pointed out by   Dalrymple,\cite{dalrymple83}   just about  all
crystalline  superconductors  are    in   principle expected   to show some
anisotropy  effects.  Clearly   these effects are   more important in  some
materials than in others,  and the materials at the  high-anisotropy end of
the continuum are more interesting for comparison with GIC's.

        At the weakly anisotropic end of the continuum  are bulk transition
metals  like Nb,   V, and   Tc.    These  materials are   three-dimensional
anisotropic superconductors   characterized  by  different  superconducting
coherence  lengths in  different  crystallographic directions.  The Pippard
coherence length $\xi_0$, is defined  to  be the approximate spatial extent
of  the  Cooper pairs.    From  an  uncertainty-principle-type  argument by
Pippard,\cite{tinkham80} this extent must be approximately $\rm \hbar /
\Delta  p \approx \hbar  v_F  / k_B   T_c$.  This  relationship  shows  the
connection between the Fermi surface and superconductivity.

        In  the  Ginzburg-Landau theory   of  superconductivity  (discussed
further in Section~\ref{critf:intro} and ~\ref{models}), a more generalized
coherence length $\xi$ is  defined.  This Ginzburg-Landau  coherence length
is related to the upper critical field for a bulk type II superconductor by

\begin{equation}
\rm   H_{c2} \;  =   \;  \Phi_0/   2\pi \xi^2 \; ,
\label{xi_def}
\end{equation}

\noindent  where  $\Phi_0$   is   the
superconducting flux quantum.\cite{tinkham80} An angle-dependent  coherence
length      therefore gives  an  anisotropic      critical  field.  When  a
superconductor has uniaxial symmetry, two of  its  three pricipal coherence
lengths  are the   same.  (In the   most general case, all three  coherence
lengths may be different.)  Then the angular dependence  of $\rm H_{c2}$ is
given by the simple expression:\cite{kats69,morris72}

\begin{equation}
\rm H_{c2}(\theta) \; = \; \frac{H_{c2}(0^{\circ})}{\sqrt{\cos^2\theta \,
+ \, \epsilon^2 \sin^2\theta}} \; .
\label{ldtheor1}
\end{equation}

\noindent  where $\epsilon$ is the critical field anisotropy parameter of Morris, Coleman
and Bhandari,\cite{morris72}, defined by:\\

\[ \rm \epsilon \; \equiv \; H_{c2,\parallel \hat{c}}/ H_{c2,\perp \hat{c}} \; \; .
\]

\noindent Except for this    angular dependence  of  $\rm   H_{c2}$,  these  
weakly anisotropic superconductors behave for  the most part like isotropic
type II superconductors.  For  example, the temperature dependence of their
critical  fields   is given by   the usual Werthamer-Helfand-Hohenberg-Maki
theory.\cite{degennes66} This  theory gives a linear temperature dependence
near  $\rm T_c$ and saturation at   lower temperatures.  Linear behavior of
$\rm H_{c2}$ implies because of Eqn.~\ref{xi_def} that $\xi$ goes as $\rm (1
\: - \: t)^{-1/2}$.

        Besides    bulk anisotropic   superconductors    like  some  of the
transition  metals,  there  is another  familiar  class  of materials  with
anisotropic     superconducting  properties.     This   is    the  class of
superconducting  thin   films,    which    were the    first    anisotropic
superconductors to  be studied.\cite{tinkham63,burger65} When the thickness
of a film  is  less than  the coherence length,  the Cooper pairs  can only
interact with their neighbors in the plane of  the film.  In this case, the
film  is  commonly referred to  as a two-dimensional superconductor because
the Cooper pairs only interact in two directions.

        As might be expected, the lowering of  the effective dimensionality
of  a  superconductor  from three  to   two   dimensions has important  and
measurable consequences.  These consequences  stem  from the  fact that the
length scale for superconductivity  in  the direction perpendicular  to the
film is now the film thickness  rather than the  coherence length.  Most of
the  expressions that  describe thin-film superconductors   can  be derived
simply by  replacing one of the  coherence lengths in  Eqn.~\ref{xi_def} by
the film thickness, plus some numerical factors.  For example, consider the
critical fields of a thin film.  When  the external  field is applied along
the perpendicular to the film, Eqn.~\ref{xi_def} still holds.  However, for
a field applied in the plane of the film, the critical field for a thin
film is now $\rm H_{c2,
\perp \hat{c}}\, = \, \sqrt{3}\Phi_0 /
\pi  \xi d$,  where   d  is  the  film thickness.\cite{tinkham80}   From the
$\rm (1 \: - \: t)^{-1/2}$ temperature dependence of $\xi$ given above, one
can immediately  see that   $\rm    H_{c2,   \perp  \hat{c}}$  will have  a
square-root temperature  dependence: $\rm H_{c2} \, \propto  \, (1 \:  - \:
t)^{1/2}$.\cite{tinkham63,harper68}

        The square-root temperature dependence is one of the hallmarks of
two-dimensional superconductivity.  The other is an angular dependence of
the critical field given by Tinkham's formula:\cite{tinkham63}

\begin{equation}
\label{tinkham1}
\rm \left| \frac{H_{c2}(\theta) \: sin(\theta)}{H_{c2}(90^{\circ})} \right| \; + \;
\left( \frac{H_{c2}(\theta) \: cos(\theta)}{H_{c2}(0^{\circ})} \right)^2 \;
= \; 1 \; .
\end{equation}

\noindent  where  $\rm H_{c2}(0^{\circ}) \equiv H_{c2, \parallel \hat{c}}$
and  $\rm H_{c2}(90^{\circ})  \equiv    H_{c2, \perp \hat{c}}$.   The angle
$\theta$ is defined in Figure~\ref{hc2def}.

        The  square-root temperature dependence  and Tinkham's  formula for
the angular dependence have been observed many  times in thin  films.  What
is more surprising is that  these two-dimensional properties have also been
observed in bulk superconductors which  have a layered structure.  Rigorous
calculations for a   multilayer structure composed  of  superconducting and
insulating  planes     have   been    performed by  Klemm,      Luther  and
Beasley.\cite{klemm75}  The KLB  calculations   show that  a dimensionality
crossover is expected to  occur at a  finite temperature when the parameter
$r$ is less than 1.7,\cite{klemm75} where $r$ is defined by

\begin{equation}
\label{klbparam1}
r \equiv \frac{16}{\pi} \left( \frac{\xi_{\parallel \hat{c}}(0)}{s}
\right)^2
\end{equation}

\noindent  The temperature at which the decoupling of the superconducting 
planes occurs occurs is  called T$^*$.  T$^*$ is  approximately  defined by
$\rm \xi_{\parallel  \hat{c}}(T^*)$  = d.   Usually layered superconductors
show 3D anistropic superconductivity  like the bulk  transition metals, but
sometimes  they show 2D  superconductivity  like thin films, and  sometimes
they even show entirely new effects.  Below, the portion of the work on the
upper critical fields of layered  superconductors which is most relevant to
the GIC experiments is briefly reviewed.
