\section{Upper Critical Field Studies of Artificially Structured 
Superlattices}
\label{assl}

	Work on artificially structured superconducting superlattices up to
1984 was reviewed by Ruggiero and Beasley.\cite{ruggiero85}  Results in the
superconductivity  of artificial superlattices  are much broader in variety
than those in the TMDC and their intercalation compounds, encompassing such
species  as    superconductor/superconductor (S/S'),  superconductor/normal
metal  (S/N), superconductor/insulator (S/I), and   superconductor/magnetic
(S/M) compounds.  A greater variety  of materials combinations is  possible
with high-vacuum   vapor-phase  deposition   techniques because  the sample
grower is no longer limited to phases which  represent a global free-energy
minimum.  These new synthesis methods are just beginning to play  a role in
some areas of materials research, but  they have already made possible real
advances in the field of superconductivity.
        
        Some   parameters from  a  few of  the   many  recent  superlattice
experiments are gathered in  Table~\ref{summtable}.  It  is not possible to
make a neat summary of the superconductivity of the artificially structured
superlattices the way it is possible for the TMDC's.  The reason clearly is
that for any two materials that might be chosen for use  in a superlattice,
a  potentially unlimited   range of  thickness ratios can    be used.  This
situation  corresponds  to   having  stages  1   through $\infty$   in   an
intercalation compound.  With modern preparation techniques, even the
``stage 1'' limit of alternate monolayers of two different materials may have
been achieved in the Mo/Ta system.\cite{makous87}


\begin{table}
\vspace{18cm}
\caption[Properties of some of the artificially structured superlattices.]
{Properties of some of the artificially structured superlattices.  Adapted
from Ref.\cite{ruggiero85}, where references are given.}
\label{summtable}
\end{table}


	The category of S/I superlattices was the first to be investigated.
Pioneering   work was    done on   the    Al/Ge    system  by   Haywood and
Ast.\cite{haywood78}  The dimensionality  crossover model  of Klemm, Luther
and  Beasley\cite{klemm75} was developed  specifically  for the  S/I  case,
where the interlayer coupling  occurs through  tunnelling  of Cooper  pairs
through  the insulating layers.  The  excellent agreement of the KLB
model with $\rm H_{c2}(T)$ data for an S/I superlattice (Nb/Ge) is shown in
Figure~\ref{nbgetemp}.   The dimensional crossover  here is more impressive
than it was in the case of the TMDCIC's, where there were fewer data points
on the 2D side of the curve.


\begin{figure}
\vspace{12cm}
\caption[Evidence for dimensionality crossover in a Nb/Ge
superlattice]{Evidence   for  dimensionality    crossover  in    a    Nb/Ge
superlattice.\cite{ruggiero82}  Solid  lines    are    fits  to   the   KLB
theory.\cite{klemm75} The direction marked ``$\parallel$''   means  $\rm \perp
\hat{c}$ in the vernacular of GIC's and  the direction marked ``$\perp$''
is $\rm  \parallel \hat{c}$ in   GIC  terms.  The $\rm H_{c2}$  data marked
$\perp$ is almost  unaffected  by the  dimensionality crossover.   The $\rm
H_{c2}$ data marked $\parallel$ covers the range from the fully 3D regime
(45 \AA/ 7 \AA\ specimen) to the fully 2D regime (45 \AA/ 50
\AA\ specimen).  The intermediate 65 \AA/35 \AA\ specimen shows a crossover from 3D to
2D character as the temperature is lowered.}
\label{nbgetemp}
\end{figure}


        More recent studies have concentrated  on  S/S' and S/N multilayers
with niobium as one element.  The  Josephson-tunnelling theory does
not  apply quantitatively  to these other  types of superlattices,  because
they are coupled through the superconducting proximity effect.  The S/N and
S/S'       multilayers    are   described      by   new    proximity-effect
theories.\cite{biagi85,takahashi86b}  Although these new  models  differ in
detail from the KLB model, the basic physics  of the progressive decoupling
of superconducting layers at low  temperatures remains the  same.  The root
cause  of the low-temperature decoupling of  superconducting layers is 
the  decrease of the  coherence   length in  each case.    A dimensionality
crossover has been observed   in  the critical  field  behavior  in several
different  systems,  including the   S/N systems  Nb/Cu,\cite{chun84},  and
V/Ag,\cite{kanoda86} and the S/S' system Nb/Ta.\cite{broussard87}  In these
materials,  as  in  the  TMDCIC's and  S/I  multilayers, the   crossover is
manifested by a discontinuity in the temperature dependence of the critical
field.  The identification of the kink in $\rm H_{c2}(T)$ with an effective
dimensionality change  is  confirmed by the    good  agreement  obtained by
several groups\cite{chun84,broussard87}  with    the  different approximate
temperature  dependences  expected in  the   two  regimes, namely the  $\rm
(1-t)^{1/2}$  dependence   in  the 2D temperature  region   and  the linear
dependence  near   $\rm  T_c$.  The  linear-to-square-root  change  in  the
temperature    dependence   in     the  Nb/Ta  multilayers  is    shown  in
Figure~\ref{assltemp}.

\begin{figure}
\vspace{12cm}
\caption[$\rm H_{c2}(T)$ data on Nb/Ta superlattices showing 2
low-temperature transitions.]{$\rm  H_{c2}(t)$ data on Nb/Ta  superlattices
showing   two   low-temperature    critical    field discontinuities.    From
Ref.~\cite{broussard87}.  The discontinuity in slope at t $\approx$ 0.8-0.9
is the 3D-2D coupling change.  The identity of the lower transition at  t = 0.49 has
not been definitely determined.}
\label{assltemp}
\end{figure}
%Figure\ref{assltemp} is  from broussard87, Figure6.

        Figure~\ref{assltemp} also  demonstrates  that  the  S/S'  and  S/N
systems display   additional phenomena  besides  the coupling-dimensionality
change.  The Nb/Ta multilayers show a second low-temperature critical field
discontinuity after they  have already become 22  D-coupled.   The origin
of this second discontinuity  is currently uncertain.   It may be  due to a
shift   of  the   flux-line-lattice   from   one   set  of  layers  to  the
other.\cite{takahashi86c,broussard87b}   There    is  obviously  a  lot  of
interesting work left to be done on the S/N and S/S' multilayers.

	In  addition  to   the  strong experimental     confirmation  of  a
dimensionality crossover from the  $\rm  H_{c2}(T)$  measurements, there is
also positive corroboration from $\rm  H_{c2}(\theta)$ data.  The Nb/Cu and
Nb/Ta superconductors not  only  are fairly well-fit by Eqn.~\ref{ldtheor1}
in  the temperature range where their  behavior is   three-dimensional, but
they also are well-fit by  Tinkham's formula (Eqn.~\ref{tinkham1}) in their
2D-coupled   range.\cite{chun84,broussard87}  The  agreement   of  the  two
different  $\rm H_{c2}(\theta)$   formulae  with Nb/Ta  data   in  the  two
different temperature regions is displayed in  Figure~\ref{assltheta}.  The
observation  of a change in  the coupling dimensionality  in  both the $\rm
H_{c2}(T)$  and $\rm H_{c2}(\theta)$  measurements is strong  evidence that
the KLB model contains the right physics.

%Figure~\ref{assltheta} is from  broussard88, figure9
\begin{figure}
\vspace{15cm}
\caption[$\rm H_{c2}(\theta)$ data on Nb/Ta superlattices.]{$\rm H_{c2}(\theta)$ data on Nb/Ta superlattices from 
Ref.~\cite{broussard88}.  The data is for three  samples at a
reduced temperature t  =  0.9.   $\Lambda$  is  the  bilayer  period, which
corresponds to $\rm I_c$ in the GIC case.  Data in  the trace marked 3D are
fit with Eqn.~\ref{ldtheor1}, while data in the  traces  marked  2D are fit
with Eqn.~\ref{tinkham1}.}
\label{assltheta}
\end{figure}

        Interpretation of   the angular  dependence of the  superconducting
superlattices is not as clear-cut as the temperature dependence, where more
universal  behavior is seen.   As  shown in Figure~\ref{assltheta}, in  the
crossover region  neither the 3D nor  2D formulae fits well.   Chun {\em et
al.} found  a continuous variation   between the   2D-type and 3D-type $\rm
H_{c2}(\theta)$   behavior.    Ruggiero  {\em  et   al.\/} observed  better
agreement with Tinkham's   formula in  both  the 2D    and  3D regions   of
temperature.\cite{ruggiero82} While  the agreement of theory  and   $\rm
H_{c2}(\theta)$ experiments for some  of the S/I  and S/N systems  at  most
temperatures  is   gratifying, it   is  clear that   the behavior  of  $\rm
H_{c2}(\theta)$   in  anisotropic   superconductors   is   not   completely
understood.\cite{broussard87,ruggiero82}  The  agreement  between the  $\rm
H_{c2}(\theta)$  data and  the two available  formulae  is usually   at the
semiquantitative level in the artificially  structured superlattices, while
the  agreement     in    $\rm  H_{c2}(T)$    is    usually   quite     good
quantitatively.\cite{ruggiero82}   Quite  possibly   a calculation of   the
angular dependence  of $\rm H_{c2}$ near  the  dimensionality crossover  in
terms  of the  KLB   and  the proximity  models   would  solve    all these
discrepancies. Such calculations have not been performed up to this point.

        Another    somewhat puzzling    aspect      of the  superconducting
superlattice experiments is  the positive curvature  often observed in $\rm
H_{c2, \parallel \hat{c}}$.  Such positive curvature has  been seen in both
the     Nb/Cu       and   Nb/Ta    systems.\cite{chun84,broussard87}    The
coupling-dimensionality change should  have no effect on $\rm  H_{c2}$ when
the field is applied perpendicular to the layer  planes since this critical
field is determined solely by the in-plane transport properties.  Broussard
and  Geballe\cite{broussard87} have attributed   this positive curvature to
the Fermi  surface anisotropy of the Nb  layers since it  is similar to the
positive curvature of Nb thin films.  Biagi, Kogan, and Clem, on the other
hand, say that the positive curvature seen in $\rm H_{c2, \parallel
\hat{c}}$  is due to the proximity effect.\cite{biagi85}

        Many   interesting    experiments   have  been   performed  on  the
superconducting  multilayers  besides the  critical  field  experiments, as
Table~\ref{summtable} suggests.  The  most relevant  of  these for  the GIC
work is the measurement of $\rm T_c$  for  different layer thicknesses.  An
example is shown in Figure~\ref{assltc}, where $\rm T_c$  is plotted versus
layer  thickness for the  Nb/Cu multilayers.  For large layer  thicknesses,
the increase of $\rm T_c$ with increasing  thickness of the superconducting
component is in  accord with the  predictions of  standard proximity-effect
theories.\cite{cooper61,degennes66}   The      standard    proximity-effect
prediction is indicated by  the solid line.   For small layer  thicknesses,
the  proximity theory needs some  modifications to  explain the data.   The
dashed  line is also a proximity-effect  calculation, but it incorporates a
thickness-dependent $\rm T_c$ for the individual Nb layers.\cite{banerjee84}

        Note that the  proximity   effect is  a  feature  of all interfaces
formed by superconductors.  It  should not  be confused with proximity {\em
coupling\/}, which is  a manifestation of the  proximity effect that occurs
only in  the  S/N and  S/S'   superlattices.   Proximity coupling  is   the
interlayer interaction   in  S/N superlattices  which  is   mediated by the
diffusion of  Cooper pairs across  the  N  layer.   S/I interfaces can also
exhibit the proximity effect, but proximity-effect depression  of $\rm T_c$
is quite small there.  The size of the proximity-effect depression  of $\rm
T_c$ goes as the  factor $\eta$,  which  is the ratio of   the normal-state
conductivities of the normal and superconducting layers ($\eta$ = $\rm
\sigma_N   /  \sigma_S$).\cite{ruggiero82,degennes66}   Thus proximity
effects  are almost negligible   at S/I interfaces.   Because the proximity
effect is too weak  to couple the  layers in S/I  superlattices, the layers
are coupled instead   by Josephson  tunnelling.\cite{ruggiero85} The   good
agreement    of   the   artificially     structured    superlattices   with
proximity-effect theories  is  in  contrast   with the mixed results    for
TDMCIC's (and GIC's).

\begin{figure}
\vspace{15cm}
\caption[$\rm T_c$ versus layer thickness in Nb/Cu superlattices]{$\rm T_c$
vs        layer     thickness    in   Nb/Cu      superlattices.     From
Ref.~\cite{banerjee84}.  The decrease of $\rm T_c$ with decreasing
thickness of the superconducting component is predicted by
proximity-effect theories.\cite{cooper61,degennes66}  The predictions of
the standard proximity-effect theories is indicated by the solid line.  The
proximity effect calculation modified to allow a thickness-dependent $\rm
T_c$ for the Nb layers is indicated by the dashed line.}
\label{assltc}
\end{figure}

        The study of metallic multilayers has been a growth area in physics
and materials science.\cite{mrs87} Already there have been some significant
results from superconducting multilayers.  The most impressive  of these is
the unambiguous observation of a  coupling-dimensionality crossover in both
$\rm                H_{c2}(T)$              and                        $\rm
H_{c2}(\theta)$.\cite{ruggiero82,chun84,kanoda86,broussard87}     The  real
promise  of the synthetic superlattices lies  in the studies of new effects
not easily observable in GIC's or TMDCIC's, such as the com\-pe\-ti\-tion be\-tween
su\-per\-con\-duct\-ivity and mag\-net\-ism in Mo/Ni su\-per\-latt\-ices,\cite{uher86} or the
two     low-temperature    transitions    observed     in    the      Nb/Ta
superlattices.\cite{broussard87}  The superconducting multilayer work   has
been   somewhat   eclipsed    by  the   excitement  over   high-temperature
superconductors, but hopefully it will continue to be pursued, both because
of its intrinsic interest and its relevance to the high-$\rm T_c$ field.

