\section{Discussion and Conclusions}
\label{critfdisc}

        Critical field  studies on graphite intercalation compounds involve
several  experiments  on  a number  of  materials,  are afflicted  by  many
experimental  difficulties, and are qualitatively described   by  a host of
difficult   microscopic theories.  A   critical  review  of  the  facts and
speculations of the previous sections is in order to draw some conclusions.

        From the  experimental standpoint, there are  several ways in which
this  project could  have been improved.   Better cryogenic instrumentation
would have permitted  the collection  of more  accurate data  at additional
temperatures.   Better  structural  characterization as  far as crystalline
quality  and in-plane homogeneity would  have taken some of the uncertainty
out of the interpretation of the data.   The preparation of single crystals
of GIC superconductors  would allow anisotropy studies  to be extended to a
search for angular dependence of $\rm H_{c2}$ in the layer planes.

        The $\rm H_{c2}(\theta)$ experiments,  which were affected  more by
the  quality of the  crystals  than by  instrumental  problems, raised some
intriguing questions about $\rm C_4KHg$.  The most  interesting question is
whether $\rm C_4KHg$ has  type I character  for small  angles $\theta$, and
the answer seems to be yes, both from the critical  field and specific heat
studies.  The other unexpected finding is that the angular dependence seems
to be  better   described by   Tinkham's  formula  than  by the anisotropic
Ginzburg-Landau  model.   The  reason  for    the  unusual  shape  of  $\rm
H_{c2}(\theta)$ may be  understood only if  angle-dependent demagnetization
effects can be accounted for in  a quantitative  way, or perhaps if  one of
the microscopic models  of anisotropic superconductivity can  be fit to the
data.

        The most important finding of the angular dependence experiments is
that the magnitude of the linear specific heat coefficient, $\gamma$, found
by  Alexander {\em  et al.\/}  appears  to be  too  high.   One of the most
longstanding mysteries about  GIC's has been why $\rm  C_4KHg$  with N(0) =
0.41 states / (eV$\cdot$atom) has a lower $\rm  T_c$ than $\rm C_8KHg$ with
N(0)  =   0.058    states /(eV$\cdot$atom).\cite{alexander81}  However, the
$\gamma$  determined from  these specific  heat  measurements predicts such
high  values of  $\rm H_c$(t)  that $\rm C_4KHg$  would  be type I for most
angles  $\theta$ (see Table~\ref{htherm_table}),  in sharp contrast  to the
experimental observations.  The  magnitude of $\rm  H_c(t)$ determined from
the  $\rm H_{c2}(\theta)$  fits is about  a factor   of 1.8  lower than the
magnitude of $\rm H_{c}(T)$ determined from Alexander's specific heat data.

        Since    in   weak-coupling  superconductors    $\rm   H_c  \propto
\sqrt{\gamma}$,\cite{tinkham80} and $\gamma \propto N(0)$, this finding implies that the 
density-of-states measured for $\rm C_4KHg$ by Alexander and colleagues may
have been about a factor of 3.2 too high.  Then the corrected N(0) for $\rm
C_4KHg$  = 0.13 states/(eV$\cdot$atom).  If the  further assumption is made
that the specimen measured by Alexander {\em et al.\/} had  a  $\rm T_c$ of
0.7 K (they measured only down to 0.8 K,  and saw no transition), then this
N(0) is the phonon-dressed DOS, which requires a further division by $\rm (1 \,
+ \,
\lambda_{ep})$ = 1.38.  (See  Table~\ref{lambdatable}.)  Then the final value of
N(0)   estimated  from  the $\rm    H_{c2}(\theta)$  experiments   is 0.094
states/(eV $\cdot$atom).  This  is still higher than the  latest number for the
DOS of $\rm C_8KHg$ = 0.058 states /(eV$\cdot$atom) reported by Alexander,\cite{alexander81} but not higher
 than their  original estimate of N(0)  for $\rm C_8KHg$ = 0.18  states
/(eV$\cdot$atom).\cite{alexander80} Because $\rm C_8KHg$ is more  strongly type
II than $\rm C_4KHg$,\cite{iye82} using the higher  value of N(0) would  not 
imply type I behavior.

        This  extended discussion  of the  density-of-states  issue is  not
meant to produce hard numbers for $\rm C_4KHg$ or $\rm C_8KHg$.  Rather the
point of this inquiry is merely to suggest that the ``mystery'' of why $\rm
C_4KHg$ has a lower  $\rm T_c$ may  not be  so mysterious after all.  A new
specific heat measurement on a $\rm C_4KHg$ specimen with a known $\rm T_c$
would be valuable in clearing up these issues.

        The $\rm H_{c2}(T)$ measurements were more affected by instrumental
difficulties than   the angular dependence experiments.   Nonetheless,  two
interesting anomalies were found in the  data: evidence for the temperature
dependence of $\epsilon$, the  anisotropy parameter; and enhanced linearity
of   the  critical  fields.   (There  is  no   conflict   between these two
observations because $\epsilon$ was determined from  the more accurate $\rm
H_{c2}(\theta)$ experiments, and  the amount of  change in d$\rm H_{c2}$/dT
implied by the variation of $\epsilon$ is within the error bars of the $\rm
H_{c2}(T)$ measurements.)  These two features of the temperature dependence
contradict the predictions of  the  AGL model, which otherwise describe the
data rather well.

        The enhanced linearity of $\rm H_{c2}(T)$, temperature variation of
$\epsilon$, and unexpected fits to Tinkham's formula provide motivation for
interpreting the data using  one of the many  microscopic models available.
There  is ample justification  for doing so  considering that the anomalies
seen in  $\rm  C_4KHg$ were   observed on  a   larger  scale  in the  other
ternaries\cite{iye82} and the  binaries.\cite{koike80,kobayashi85} The most
sensible  models to consider are  those that fit  the data  of  the closely
related transition  metal dichalcogenide superconductors well.  Of  these,
the models  dealing with Fermi   surface   and   energy gap anisotropy   or
multiband effects seem most promising. The other enhancement mechanism that
seems   relevant to      GIC's   is  one     developed   by  Carter     and
coworkers\cite{carter81}   which incorporates  the   effects  of microscopic
inhomogeneity.  Unfortunately a more complete Fermi surface  computation is
needed for a ternary GIC before these theories can directly tested.

        To summarize the findings discussed in this chapter, let's see what
general statements can be made about ternary GIC superconductors.  First of
all, they are weak-coupling superconductors with $\rm \lambda_{ep}$ about 0.4, which
is   typical  for   materials  with   $\rm    T_c$ on   the   order    of 1 K.\cite{mcmillan68} Orbital pairbreaking is the principal influence on
the critical fields of GIC's since they have $\alpha  \approx 10^{-3}$, and
so are not  affected by Pauli-limiting effects.   Several GIC's ($\rm   C_8K$, $\rm
C_8Rb$,  and  $\rm C_4KHg$) show signs  of  type  I superconductivity for a
small angular region around $\rm  \vec{H}  \parallel \hat{c}$.  Ternary GIC
superconductors have anisotropy ratios 1/$\epsilon$ in the range from about
9   to 40, and    they  are  fairly    well  described by  the  anisotropic
Ginzburg-Landau model, despite  some  deviations.  These deviations,  which
are  outside the  scope of the  AGL model, include a  temperature-dependent
anisotropy and  either extended  linearity  or  positive curvature  of $\rm
H_{c2}(T)$.    All  in   all,  the ternary GIC superconductors   show great
similarity to  the TMDC's, the  principal difference being that  the TMDC's
are more strongly coupled, with higher $\rm T_c$'s and larger critical fields.

        Figure~\ref{c6k}  is offered as  a   final reminder of  how  much  is
unknown about superconductivity in graphite  intercalation compounds.  This
figure shows  the  critical  fields as a  function  of temperature  for the
binary GIC's $\rm C_8K$ and $\rm C_6K$.  $\rm C_8K$ is  the simplest of all
the superconducting  GIC's, and  hence  should be  the  first  test of  any
proposed model.  As has been  pointed out many times,  the very presence of
superconductivity in $\rm C_8K$ is something of a mystery since  neither of
its starting  constituents   is superconducting  alone.\cite{takada82,M143}
$\rm C_8K$  has large  positive curvature   of its critical  fields, as the
figure  shows.  $\rm  C_6K$ is  a high-pressure  phase of  $\rm C_8K$ which
shows  enhanced  linearity  rather    than   positive   curvature  in  $\rm
H_{c2}(T)$.\cite{avdeev87}   In  this,   and in   its  $\rm     T_c$ =  1.5
K,\cite{avdeev87} $\rm  C_6K$ is strikingly similar   to $\rm C_4KHg$.  The
challenge to critical-field theorists here is quite  clear: explain why the
application of pressure at  the same time increases  $\rm T_c$  tenfold and
suppresses  positive curvature.   Until basic questions about the prototype
GIC superconductor, $\rm C_8K$, are answered, one can hardly hope to have a
complete understanding of the much more complex ternary compounds.

\begin{figure}
\vspace{16cm}
\caption[Comparison of $\rm H_{c2}(T)$ in $\rm C_8K$ and $\rm
C_6K$.]{Comparison of $\rm H_{c2}(T)$ in $\rm C_8K$ and  $\rm C_6K$, one of
its high-pressure phases.   a)  Data  on a $\rm  T_c$ = 134  mK $\rm  C_8K$
sample taken by  Koike  and Tanuma.\cite{koike80} Note the  marked positive
curvature of the critical fields.  $\rm  H_{sc\parallel}$ is a supercooling
field.   b) Data on  a $\rm  T_c$  = 1.5    K  sample of  $\rm  C_6K$  from
Ref.~\cite{avdeev87}.    ($\circ$),    $\rm    H_{c2,   \perp    \hat{c}}$;
($\bigtriangleup$), $\rm H_{c2, \parallel   \hat{c}}$.  Note the   enhanced
linearity of the critical fields.}
\label{c6k}
\end{figure}

        It would be a great omis\-sion to end a dis\-cus\-sion of crit\-i\-cal fields
in an\-is\-o\-trop\-ic su\-per\-con\-duct\-ors  without touching on the  central problem of
the   underlying cause   of critical   field   enhancement  in  anisotropic
materials.   Anyone    who  follows    the    literature   on   anisotropic
superconductors  can plainly see  that   nearly all layered  materials have
extended  linearity   or positive  curvature  of  $\rm H_{c2}(T)$,  be they
artificially  structured superlattices, high-$\rm   T_c$   superconductors,
heavy-fermion or organic materials, transition   metal dichalcogenides  and
their intercalation compounds, or even   GIC's.   The ubiquity of  critical
field deviation  above  the WHH  theory  was  previously noted by  Woollam,
Somoano,  and O'Connor.\cite{woollam74}  The mass   of accumulated evidence
forces one   to look  for a  common  origin of   the  enhancement in  these
materials,   even  though   they are each    described by their own special
microscopic  models: the artificially  structured superlattice by Takahashi
and Tachiki,\cite{takahashi86b} the heavy-fermion materials by  Delong {\em
et  al.\/},\cite{delong87}, the organic  compounds and TMDCIC's  by the KLB
model,\cite{klemm75}   the GIC's  and   TMDC  and  Chevrel  phases  by  the
anisotropic gap/FS   or  multiband models,\cite{butler80,entel77}   and the
high-$\rm T_c$ superconductors by who-knows-what.\cite{marsiglio87}

        Happily a  common origin to  the enhancement in all these materials
can  be   deduced.  The key  is  the   juxtaposition of two superconducting
components, either  layers  or bands,  one of which  has a   high intrinsic
critical field,  and the other  of  which has  a lower intrinsic   critical
field.  Near $\rm T_c$, where the coherence lengths are large and non-local
effects are small, the two components are coupled, and  d$\rm H_{c2}$/dT is
determined by both  jointly.  That is, an average  sort  of critical  field
results  because the system is forced  to distribute field-induced vortices
equally over both components.   At  lower temperatures, the  two components
are more weakly coupled,  so  that the  superconductor can concentrate  the
vortices in  the intrinsically  low-$\rm H_{c2}$ component.  This component
is  now effectively   normal even when  H  $<$   $\rm H_{c2}$, so  that the
intrinsically  high-$\rm H_{c2}$ portion now  determines the upper critical
field.  Once the temperature is low enough that  the high-$\rm H_{c2}$ part
decouples from the low-$\rm H_{c2}$ part,   d$\rm  H_{c2}$/dT  will
increase, causing positive curvature.

        As an example, consider  S/N  superlattices where one type of layer
(the  N layer)  has a   low intrinsic $\rm  H_{c2}$.   Near  $\rm T_c$  the
vortices, which  of characteristic size  $\xi$(T),  are too  big to squeeze
into the  N layer,  so some average  of the diffusivities of the   N  and S
materials determines the slope of  $\rm  H_{c2}$.   At  lower  temperatures
where vortices can fit into  the N layers, those  layers effectively become
normal at  fields  significantly  below  $\rm H_{c2}$.  When they  do,  the
lower-diffusivity  S   layers determine   d$\rm  H_{c2}$/dT, and  the  $\rm
H_{c2}(T)$ curve turns up.  A detailed  analysis of this situation has been
worked   out by   Takahashi and Tachiki, who   predict  a first-order phase
transition when the vortices fit into  the N layer.\cite{takahashi86c} This
transition  may already    have  been   observed  experimentally on   Nb/Ta
multilayers.\cite{broussard87b,kanoda86}

        The application of the idea of the decoupling of high- and low-$\rm
H_{c2}$  components  to  anisotropic  gap/FS or  multiband  models  is less
obvious, but similar.  The explanation  has  been given by Entel and Peter:
``For $\rm T \rightarrow T_c$ and $\rm H_{c2}(T) \rightarrow$  0 one should
look upon the condensation  phenomenon  as considering all  electrons  in a
small energy  shell  around the Fermi level   even if local regions on  the
Fermi surface  contribute  with a different  weight to  $\rm T_c$.  But for
$\rm T \ll T_c$, the external field tends to  suppress superconductivity in
those regions of  the Fermi surface where  the  electrons are  more  weakly
coupled\ldots  the $\rm H_{c2}(0)$  enhancement is merely determined by the
strength  of       the    off-diagonal      electron-phonon        coupling
constants\ldots''\cite{entel76}  This interpretation  is written in general
enough language that  it   can also  be  seen as   applying  to  a   single
anisotropic band.  The similarity between the decoupling between bands in a
multiband   superconductor  and   layers    in  an  artificially structured
superlattice is evident in comparing this description and that of Takahashi
and Tachiki.\cite{takahashi86c}

        A  unified theoretical picture for the  critical fields of  layered
materials would  be   a   great achievement.   Such  a unification  appears
inevitable  as layer  thicknesses in artificially structured materials  get
smaller and smaller, approaching the  scale where Fermi  surface anisotropy
and  multiband   effects begin  to  take  over.   In    fact,  the possible
observation of Fermi surface anisotropy effects in  artificially structured
superlattices has  already been reported.\cite{broussard88} Undoubtedly new
experiments will keep theorists busy for many years to come as they attempt
to catch up with the experimental discoveries made possible by  advances in
fabrication.\cite{ruggiero85}  It   seems likely that    the  smaller layer
thicknesses in artificial structures  get,  the more   they will look  like
transition metal dichalcogenides  and GIC's  in their  qualitative critical
field behavior.

        In  this   Chapter  there has   been mention  several  times of the
different critical fields of the gold  low-$\rm T_c$ ($\rm T_c  \, \approx$
0.8K) and pink higher-$\rm T_c$  ($\rm T_c  \, \approx$ 1.5 K) $\rm C_4KHg$
samples.  Yet the intrinsic cause of the differences between these types of
$\rm  C_4KHg$ has   not been  discussed.  The   possible   origin of these
differences and the insight that  hydrogenation experiments can  give  into
this problem are the topic of the next Chapter.
        


