\section{Possible Interpretations of Enhanced Linearity and Positive
Curvature in GIC's}
\label{models}

        This section is  not  intended  to   be a complete  review of  $\rm
H_{c2}(T)$   theories.  Instead, the  following is  a  discussion  of a few
specific models which predict  enhanced critical fields  and  the prospects
for the application  of these models to  GIC's.  A recent complete overview  of $\rm
H_{c2}(T)$    theories   has     been     written   by      Decroux     and
Fischer.\cite{decroux82}

        Before reviewing the  models which help  to explain  the deviations
from the  anisotropic Ginzburg-Landau theory,  it makes sense to review the
basic AGL model  itself.  The anisotropic  Ginzburg-Landau model,  like the
Ginzburg-Landau theory\cite{tinkham80} from which it is descended, is based
on  an expansion of the  free-energy difference between the superconducting
and  normal phases into  a series of powers  of the order  parameter.   For
superconductivity, the order parameter is variously taken as the energy gap
(in theories based on  the Gor'kov equations\cite{gorkov59,gorkov60}),  the
density of superfluid particles (in the spirit of  the two-fluid model), or
the Ginzburg-Landau wave  function  $\psi$.  In the  simplest case   of the
isotropic GL theory, the free-energy difference of the  superconducting and
normal states is written:

\begin{equation}
\label{deltaf}
\rm \Delta F \;  =  \; \alpha \left| \psi \right|^2 \, + \,
\frac{1}{2m^*}\left| \left( \frac{\hbar}{i}\vec{\nabla} \: - \:
\frac{e^*}{c}\vec{A} \right) \right|^2 \, + \, \frac{H}{8\pi}^2
\end{equation}

\noindent Here $\alpha$ is the kinetic energy of the Cooper pair, and
$\rm \vec{A}$ is   the vector  potential.  The GL   differential  equations are
derived  from   the free-energy  variationally.    Their solution   for the
zero-field case shows that $\rm \mid \alpha \mid \, = \, \hbar^2 / 2m^* \xi^2$, where
$\xi$    is   the   superconducting     coherence  length    discussed   in
Section~\ref{critf:intro}.  Using the relation $\rm H_{c2} \, = \, \Phi_0 /
2\pi
\xi ^2$ and rearranging factors shows that  $\alpha$ is equal to the cyclotron 
energy  of an electron at H  = $\rm H_{c2}$.   This equality  points to the
connection between vortices in type II superconductors and the Landau-level
problem which is mentioned in Tinkham's book.\cite[Section 4-8]{tinkham80}

        The only elaboration  necessary  to go from  the  isotropic to  the
anisotropic case is to let the mass $\rm m^*$ become a tensor rather than a
scalar.\cite{tilley65,kats69}  All the  rest  of the results of the AGL
model follow directly  from this replacement.  The spirit  of the AGL model
is to let all the orientation dependence be lumped into the  masses, and to
ignore  any variation  with  angle  of other  microscopic parameters  ({\em
e.g.,\/} mean free path and phonon  dispersion).  Therefore, all anisotropy
effects are   described  in  this    model  by one  temperature-independent
parameter, $\epsilon$.  This parameter  was introduced by  Morris, Coleman,
and Bhandari\cite{morris72} who defined $\epsilon \equiv \sqrt{m/M}$, where
m is the in-plane mass  and M is the c-axis  mass.  With  the assumption of
in-plane isotropy, further   calculation shows that  $\rm H_{c2,  \parallel
\hat{c}}/H_{c2, \perp \hat{c}}$ = $\rm \xi_{\perp \hat{c}}/\xi_{\parallel
\hat{c}}$.  Because $\epsilon$ is the  sum of   all the  contributions to  
the anisotropy of  a layered  superconductor, it should not  necessarily be
thought of as the ratio of band masses.  Similarly, the  fact that the flux
quantum in the AGL model is ellipsoidal  does not imply  the presence of an
ellipsoidal Fermi surface.

        The simplicity of the AGL model is both its greatest virtue and its
greatest weakness.   On  one hand, the small  number  of parameters  in the
model  allow calculation of  quite complex quantities, such as  the angular
dependence of the lower critical field, $\rm H_{c1}$.\cite{klemm80} No sane
person would seriously contemplate the calculation of  $\rm H_{c1}(\theta)$
in     the  context     of   a   first-principles   model    of anisotropic
superconductivity,  where closed-form final  expressions are   rare.
On the  other hand, the AGL  model is so simple   that  it is inadequate to
describe  all the details  of superconductivity in  GIC's,  despite the fact
that it gives a satisfactory picture of the main features.   Both the poor
fits    to   $\rm  H_{c2}(\theta)$  (see    Section~\ref{angdata})  and the
temperature dependence of $\epsilon$ (see Section~\ref{hvstdata}) cannot be
explained in  the context  of  the AGL  model.  Furthermore,  the  enhanced
linearity of $\rm H_{c2}(T)$  cannot  be explained  within the  WHH theory,
which    reduces    to     the   AGL     model     for   $\rm    T  \approx
T_c$.\cite{gorkov59,fetter69}   Considering   the  many aspects   of
anisotropy in GIC's\cite{I94}, it is not surprising that the  one parameter
$\epsilon$ is   insufficient to summarize all  their manifestations.  All in
all,  the  anisotropic  Ginzburg-Landau   model is  successful  within  the
confines of its applicability, and useful as far as the physical insight it
provides, but it is not the last word on the subject by any means.

        Before going into  which models  are   the most appropriate   for a
detailed application to $\rm C_4KHg$, it is useful to quickly discuss those
possibilities  which produce similar effects   in  the  data, but  are  not
applicable here.  Most of these  models are  extensions  of the WHH theory,
but one, the Klemm-Luther-Beasley  (KLB) dimensionality-crossover model, is
based more directly on the AGL model.

\subsection{Dimensional Crossover Models}
\label{2d3d}

        The direct ancestor   of the KLB  papers,  and one  of the  primary
contributions to  the AGL  model,   is   the Josephson-coupling  theory  of
Lawrence and Doniach\cite{lawrence71}.   Lawrence and Doniach  started with
the idea of  a superconductor-insulator (S/I)  superlattice whose layers are
coupled by Josephson tunnelling  of quasiparticles.  They modified  the  GL
free-energy expression accordingly, and found  an effective mass along  the
c-direction in terms of  the tunnelling parameters.   From this,  following
the   work of  Kats\cite{kats69},  they  showed  that the  critical   field
anisotropy was   equal to the square  root  of the  ratio  of the effective
masses.

        Klemm,  Luther and Beasley\cite{klemm75}  studied  the GL equations
with Josephson-coupling further, and found that they predicted a divergence
of $\rm H_{c2, \perp \hat{c}}(T)$ at a temperature $\rm  T^*$ given by $\rm
\xi_{\parallel \hat{c}}(T^*)$ = s/$\sqrt{2}$.  In  this equation,  s is the
interlayer spacing corresponding  to $\rm I_c$  in GIC's.  KLB  then did  a
first-principles calculation of $\rm H_{c2}(\theta, T)$ and discovered that
T$^*$  is actually  the  temperature  at which  a  dimensionality crossover
occurs.  Dimensionality crossover in the context of layered superconductors
means a change in character  from  bulklike behavior to film-like behavior.
This change is  caused by  a decoupling of  the superconducting layers from
one  another when  $\xi_{\parallel\hat{c}}$   becomes  less  than   s.  The
symptoms  of 2D  behavior are that the  $\rm H_{c2}(\theta)$ curves are fit
with Tinkham's formula rather than the AGL  angular dependence formula, and
that $\rm H_{c2}(T)$ goes as $(1 \: - \: t)^{1/2}$ rather than the $(1
\: - \: t)$ shape found near $\rm T_c$ in 3D superconductors.  This type of
dimensionality  crossover   has  been   experimentally   observed  both  in
TMDCIC's\cite{prober80}   and  artificially structured superlattices (which
are   discussed   in    Chapter~\ref{othersys}).\cite{ruggiero82}  As    an
explanation  for anomalies    in   superconducting  GIC's,   dimensionality
crossover has a lot of appeal, since so many of the normal state properties
of  GIC's are  quasi-two-dimensional.  For  example,  the anisotropy of the
conductivity    in    $\rm    C_4KHg$   is   $\rm \sigma_a/\sigma_c  \approx
300$.\cite{fischer83}

\begin{figure} 
\vspace{6.5in}
\caption[Theoretical  demon\-stra\-tion  of
di\-men\-sional      cross\-over            in            Josephson-coupled
superlattices.]{Theoretical        demon\-stra\-tion  of    di\-men\-sional
cross\-over  in  Josephson-coupled  superlattices from the   work of Klemm,
Luther and Beasley.\cite{klemm75}.  r  is the parameter which characterizes
the dimensionality of coupling.   $\alpha$, $\rm \tau_{SO}$, and  H$\rm _P$
$\equiv$ 4$\rm k_B  T_c/ \pi \mu$  are  parameters which  characterize  the
degree   of    Pauli-limiting   (Pauli-limiting      is     discussed    in
Section~\ref{spin-orbit}).  The  inset shows a   plot of  T$^*$/$\rm   T_c$
(where T$^*$ is the dimensionality crossover temperature) versus r.}
\label{klbfig} 
\end{figure}

        According   to  KLB,   a layered   superconductor   will have   the
possibility of a coupling-dimensionality change when  T$^*$ is greater than
zero, which occurs  when  their parameter  $r$  is close  to 1.  The  exact
condition is\\

\begin{equation}
\label{klbparam}
\rm T^* \, > \, 0 \; when \; r \; \equiv \; \frac{16}{\pi} \left( \frac{\xi_{\parallel \hat{c}}(0)}{s}
\right)^2  \; < \; \frac{\pi}{1.78} \; .
\end{equation}

\noindent Calculated $r$ values for a number of layered superconductors are 
shown in Tables~\ref{tmdcklb} and  \ref{tmdcicklb}.  The r's  for  the  GIC
superconductors are  shown in  Fig.~\ref{klbgics}.  As  can easily be seen,
many  of the  transition  metal  dichalcogenide superconductors  and their
intercalation compounds  are  expected to show the dimensionality crossover
effect,  and      indeed     many  do,      as     was      discussed    in
Section~\ref{tmdc}.\cite{coleman83,prober80} On  the other  hand, the known
GIC superconductors are  more than an  order  of magnitude   away  from the
critical  value of r.    The KLB paper  found  no anomalies for $r$  $\geq$
10,\cite{klemm75} as  Fig.~\ref{klbfig}  shows.  Therefore,  dimensionality
effects can safely be ruled out as having any impact on the critical fields
of  the known  GIC superconductors.   This is  particularly  interesting in
light of the fact that many GIC's have anisotropies 1/$\epsilon$ comparable
to  those of   the  TMDCIC's.    For   example, both   $\rm   C_8KHg$   and
TaS$_{1.6}$Se$_{0.4}$  intercalated  with   collidine have  anisotropies of
about 30, yet  the  TMDCIC is expected to  show a dimensionality crossover,
and $\rm C_8KHg$ is strongly three-dimensional even at 0 K.

\begin{table}
\caption[Values  of the   KLB  parameter $r$   for the GIC
superconductors.]{Values  of the   KLB  parameter $r$   for the
GIC superconductors.  \dag means that the value of $r$ was calculated from
parameters in the cited references.}
\label{klbgics}
\begin{center}
\begin{tabular}{|c|ccccc|}
\hline
& & & & & \\
Compound & $r$ & 1/$\epsilon$ & $\xi_{\parallel \hat{c}}(0)$ & $\rm I_c$ (\AA) & Ref.\\
& & & & & \\
\hline
& & & & & \\
$\rm C_8KHg$ & 64-76& 30-31& 52 & 13.6  & \cite{koike81,tanuma81} \\
& & & & & \\
$\rm C_8KHg$ & 91& 25& 57 & 13.56  & \cite{pendrys83} \\
& & & & & \\
$\rm C_8RbHg$ & 86-176& 20-41& 58-83 & 13.56  & \cite{iye82} \\
& & & & & \\
$\rm C_4KTl_{1.5}$ & 514& 5 &57 & 12.1 & \cite{pendrys83,vogel81} \\
& & & & & \\
$\rm C_4KHg$ & 2000& 6-16& 200-300 & 10.24  & this work \\
& & & & & \\
$\rm C_4KHg$ & 2000-2800$^{\dagger}$& 10-11& 200-240 & 10.24  & \cite{iye83} \\
& & & & & \\
$\rm C_4RbHg$ & 3000-4500$^{\dagger}$& 10& 261-320 & 10.76  & \cite{iye82} \\
& & & & & \\
$\rm C_8K$ & 6.7e5& 4.7-6.2 & 1500-2900& 5.4  & \cite{koike81,koike80} \\
& & & & & \\
\hline
\end{tabular}
\end{center}
\end{table}

        
        At   this   juncture  it    is   probably   wise  to   discuss  the
proximity-coupled           superlattice            models               of
superconductivity\cite{biagi85,takahashi86b}, which  have some similarities
to  the  KLB  model.  These  models    are also   capable of quantitatively
explaining  dimensional   crossover, but  in  slightly   different systems.
Proximity-coupled superlattices are  typically  superconductor-normal-metal
(SN)  multilayers, while  the  Josephson-coupled superlattices described by
the KLB theory  are typically superconductor-insulator   (S/I) multilayers.

        The motivation for making this distinction was originally explained
by     Werthamer,     and        separately       by     Saint-James    and
deGennes.\cite{werthamer63,degennes66}  At an S/I  interface (the insulator
may be vacuum), the pairs  are confined in the superconductor  so that they
are almost unaffected by the presence of the boundary.  In this case  it is
appropriate to ignore   any variation of   the  order  parameter within the
superconducting  layers, and  to treat the interlayer  coupling  as  due to
Josephson tunnelling.\cite{lawrence71} On  the   other  hand, at   an   S/N
interface, the pairs  can easily diffuse into the  normal metal, where they
will  be destroyed.\cite{degennes66} The   destruction of  pairs  at an S/N
contact results in the suppression of the energy gap at the  interface, and
is one  aspect  of  the  proximity  effect.   Another  aspect  is that  the
superconductor can to some extent induce superconductivity  in a thin layer
of  the normal  metal.\cite{cooper61}   The proximity-coupled   theories of
Takahashi   and    Tachiki\cite{takahashi86b}   and  Biagi,    Kogan,   and
Clem\cite{biagi85} (BKC) were developed to treat the S/N case.

        In  these  models,  the  authors parameterize  the  S and  N layers
separately, unlike the KLB theory, where the properties  of  the insulating
layers are  ignored.\cite{klemm75} Takahashi and Tachiki\cite{takahashi86b}
assign different densities   of    states N(0), diffusivities D,   and  BCS
interaction  energies V to the S  and N layers.    BKC, on  the other hand,
attribute to each  type of layer  a mean-free-path  $\ell$, superconducting
transition  temperature      $\rm   T_c$,    and      Fermi  velocity  $\rm
v_F$.\cite{biagi85} Because v$\rm _F$ and  $\ell$ determine D, while N(0) and
V   determine    (with $\omega_D$)   $\rm    T_c$,  it  seems  that   these
parameterizations are more or less equivalent.
        
        The   full   theories  both predict   positive  curvature   of $\rm
H_{c2}(T)$.  Takahashi and Tachiki found signs  of a change in the coupling
dimensionality for    $\rm  H_{c2,  \perp  \hat{c}}$(T).\cite{takahashi86b}
Qualitatively these results  are quite similar to those  of KLB.  The exact
criteria for dimensional crossover are  different,  though: the authors say
that the necessary condition for decoupling of  the layers  (at T $>$ 0) is
$\rm  s/ \xi_{\parallel \hat{c}}(0) \,    \geq$  0.4.   This condition   is
equivalent, in terms of KLB's r-parameter, to $r$ $\leq \: 31.8$.  Thus the
proximity-coupled  multilayers  decouple   at   longer  coherence   lengths
($\xi_{\parallel    \hat{c}}$  = 2.5s)  than   the Josephson-coupled   ones
($\xi_{\parallel \hat{c}}$ = 1.7s).   Nonetheless, in the proximity-coupled
superlattice   models, no dimensionality     change   is  anticipated   for
superconducting  GIC's, which all exceed the  limit  in $r$  by at least  a
factor of 2.  (The possibility  of dimensionality crossover in higher-stage
GIC's is discussed in Section~\ref{concl}.)  The source of this accelerated
decoupling  is not   discussed  in Ref.~\cite{takahashi86b},  but  one  can
speculate that  it  comes from  the suppression of  the gap  near   the S/N
interface by the proximity effect.

        What is  even more intriguing for  those interested in GIC's is the
enhanced  critical  field  found for   $\rm   \vec{H}  \parallel  \hat{c}$.
BKC\cite{biagi85}  studied  only this case, and  found that ``the proximity
effect alone can produce the positive curvature [PC] in $\rm H_{c2}(T) \: \ldots
\:$ [despite the fact that] no  provisions were made  for anisotropy or any
other effects commonly thought to produce PC in $\rm  H_{c2}(T)$.''   These
remarks are  a little bit deceptive  in that anisotropy has been implicitly
introduced into the BKC model  through the layer thickness parameters; that
is, thin films are known to be anisotropic,  and the individual layers have
been chosen to be thin films.  Nonetheless, it  is impressive that positive
curvature has been found in a superlattice formalism which is not linked to
a dimensionality crossover.  Takahashi  and Tachiki also  found PC for $\rm
\vec{H} \perp \hat{c}$.\cite{takahashi86b}  These findings would seem to
have important implications for GIC's, for whom  a possible  explanation of
PC using dimensionality crossover has already been eliminated.  

        Unfortunately there   are sev\-er\-al prob\-lems  with ap\-ply\-ing
the  prox\-im\-ity-coupled    su\-per\-lat\-tice  models to   GIC's.    One
objection is that assigning  different mean-free-paths, diffusivities, $\rm
T_c$'s and  BCS interaction energies to the  S and  N layers  seems   a bit
questionable  when the  layers are only  a few atomic  layers thick.  These
models were intended to be applied to S/N  superlattices with layers on the
order  of   a few-hundred \AA \   thick,  where each of   the  constituents
individually  has transport properties  which are little  modified from the
bulk.  For layers on the order of 10 \AA \  thick, the transport properties
are so modified from the pristine materials that use of the bulk parameters
would  be highly erroneous.   The basic assumptions of the proximity-effect
models are  violated in GIC's, where  the interaction between the  host and
intercalant is  not a weak  perturbation.  Perhaps  in the  limit   of very
high-stage superconducting   GIC's (if they  exist!)   agreement   with the
proximity-coupled description might be found.

        A second reason  for  discomfort with  the KBC   and  Takahashi and
Tachiki models is the idea that superconducting  GIC's can be  described as
S/N superlattices.  While Iye and Tanuma felt that superconductivity in the
ternary GIC's was  probably  due  to superconductivity in  the  intercalant
layers,\cite{iye82} the idea that one  component is superconducting and the
other not appears somewhat shaky when $\rm \xi_{\parallel \hat{c}} \approx $
20 $\rm I_c$, as in $\rm C_4KHg$.  Also, as Al-Jishi pointed out, there is
good  reason to expect    that  the  graphitic $\pi$-bands contribute   to
superconductivity  in GIC's,   as  otherwise  their  large critical   field
anisotropy is difficult to explain.\cite{M143}

        All these discussions are actually somewhat academic since the idea
that GIC's are S/N superlattices turns out not to  be self-consistent.  The
reason  is  that all  proximity-effect theories\cite{cooper61,takahashi86b}
predict that the $\rm  T_c$'s of S/N  multilayers must always be  less than
the bulk transition temperature of the superconducting component.  In fact,
these models  predict that,   for small layer  thicknesses, $\rm  T_c$ must
decline  monotonically  with increasing thickness  of the normal layer, all
other parameters being held  equal.   The monotonic  decrease of  $\rm T_c$
with increasing normal metal thickness is illustrated in Fig.~\ref{proxeff}
for  Pb/Cu bilayers.\cite{werthamer63} Apparently no  studies of $\rm  T_c$
versus normal  metal       thickness  have been    performed   in  metallic
superlattices; the thicknesses of  both the S  and N components are usually
varied together.  (See   Figure~\ref{assltc}.)  However, the  depression of
$\rm T_c$ with increasing normal  metal  thickness is also  expected in the
superlattices.   If   one  believes that  only  the  intercalant layer   is
superconducting in GIC's,  and thinks  of the carbon  (graphene) layer as a
normal metal, then one would expect a decreasing $\rm T_c$  with increasing
stage.  Yet, as is well-known, $\rm T_c$ is higher in stage II than stage I
for   both  KHg-   and  RbHg-GIC's,\cite{iye83}  in  contravention   to the
expectation of the model.

\begin{figure}
\vspace{5in}
\caption[$\rm T_c$ versus layer thickness  for S/N bilayers.]{$\rm T_c$ versus normal-layer thickness  for S/N bilayers.
Figure taken from Ref.~\cite{werthamer63} Here D$\rm  _N$ and D$\rm _S$ are the
thicknesses of the normal and  superconducting layers, and T$\rm  _{cS}$ is
the bulk $\rm T_c$  of the superconducting component.  Approximately T/$\rm
T_{cS}$  =  $\rm  (1  \: -   \:  t(D_N \rightarrow    \infty))(1  \:  -  \:
\exp{-2D_N/\xi})$, where $\xi$ is the dirty-limit Pippard coherence length.}
\label{proxeff}
\end{figure}

        The  conclusion  is that,  while the  proximity-coupled models have
been extremely  successful   in  fitting  data on  artificially  structured
superlattices (for example, those of  Nb/Ta\cite{broussard88}), they do not
seem to  be well-suited  to  GIC's.  Perhaps if  high-stage superconducting
GIC's are ever discovered, it might be worthwhile  to  take another look at
proximity-coupling.   For  the known  GIC's, though,   it is  imperative to
consider alternative theories.

        Even though  intuition says that  the enhancement of  the  critical
fields   in superconducting GIC's  is   linked to their  anisotropy, it  is
important  to consider whether mechanisms not  related  to anisotropy could
also  play   a role.   The major causes  of  critical field enhancement  in
isotropic   superconductors are spin-orbit   scattering and strong-coupling
behavior.

\subsection{Critical Field Enhancement in Isotropic Superconductors}
\label{spin-orbit}

        One  often hears  the  statement that large  spin-orbit  scattering
increases $\rm   H_{c2}(T)$  in a   high-field superconductor.   Since  the
spin-orbit interaction   increases  rapidly  as  a  function of  the atomic
number\cite{merzbacher70} Z,  and since   Hg is a   high-Z element,  it  is
logical to ask whether spin-orbit scattering could enhance  $\rm H_{c2}$ in
$\rm C_4KHg$.  The  answer is  no, since  it  turns out that the spin-orbit
``enhancement'' only affects  materials whose critical  fields have already
been depressed    by   Pauli  limiting effects.\cite{werthamer66,orlando79}
``Pauli  limiting''  refers to the   upper   bound on $\rm  H_{c2}(T)$ from
spin-susceptibility                   (Pauli                 paramagnetism)
effects.\cite{clogston62,chandrasekhar62}   Superconductors  close to their
Pauli limit have critical fields which are depressed from the Maki-deGennes
temperature dependence.  Paramagnetic  limiting  also affects  the  angular
dependence  of the    critical field.\cite{aoi73,decroux82}  High  rates of
spin-orbit scattering   can   increase   the    susceptibility  of      the
superconducting  state to that  of  the normal state,   and  thus eliminate
Pauli-limiting effects.   Thus  spin-orbit  scattering  and  Pauli limiting
effectively cancel one another out, and the final result of both effects is
the     same       Maki-deGennes     curve        already   discussed    in
Section~\ref{hvstdata}.\cite{orlando79}   Since  the two mechanisms cancel,
and since spin-orbit scattering has no major impact for a superconductivity
far   from the   paramagnetic limit,  spin-orbit  scattering  cannot give a
reduced field  $\rm   h^*(0)$ $>$  0.7,  and cannot   explain  the enhanced
linearity of $\rm H_{c2}(T)$ observed in GIC's.

        While on the subject of electron spin, it is  worth mentioning that
GIC's appear to be far from the paramagnetic limit, as would be expected
for low-critical field materials.  The reason is that orbital
pair-breaking effects are  strong enough in  GIC superconductors that they
are far  from the spin-susceptibility  ceiling on $\rm H_{c2}$.   The small
values  of the  Maki   $\alpha$  parameter\cite{maki64}   for GIC's clearly
demonstrate the validity  of this  statement, since spin contributions  to the
energy   of  a superconductor   become   important   when $\alpha  \geq$  1.
WHH\cite{werthamer66} provided two ways of calculating $\alpha$:

\begin{equation}
\label{alphadef}
\begin{array}{l}
\rm \alpha \; = \; 3e^2\hbar \gamma \rho_n/2 m \pi^2 k_B^2 \\
\\
\rm \alpha \; = \; 5.2758\times10^{-5} \left. \frac{dH_{c2}}{dT}
\right|_{T_c} \; .
\end{array}
\end{equation}

\noindent Here $\gamma$ is the linear specific heat coefficient, and $\rho_n$
is the  normal-state dc resistivity.  In   Eqn.~\ref{alphadef}, m is the bare
electron mass, not the effective  mass, since it comes from  e$\hbar$/2$\rm
\mu_B$c.\cite{decroux82} For $\rm  C_4KHg$, the best method  is to use $\rm
\rho_a$ to calculate $\alpha$ for $\rm \vec{H} \parallel \hat{c}$, and $\rm
\sqrt{\rho_a
\rho_c}$ to   calculate  $\alpha$ for   $\rm \vec{H} \perp  \hat{c}$.   The
parameters    used  were  $\rm   \rho_c$    =  0.2  milliohm-cm     at  4.2
K,\cite{fischer83}    and    $\gamma$   =    0.95   millijoules    /   (mol
K$^2$)\cite{alexander81}.  $\rm  \rho_a$ at 4.2 K  for $\rm  C_4KHg$ (which
has not been reported) was estimated  by assuming a  resistivity anisotropy
of 280,  the  same  as the measured  resistivity anisotropy at  100  K, the
lowest   temperature    at  which    published     $\rm    \rho_a$     data
exist.\cite{elmakrini80}  Values  of $\alpha$ obtained  from both halves of
Eqn.~\ref{alphadef}    are      given       in      Table~\ref{alphatable},
which  demonstrates   that $\rm C_4KHg$  is  two to  three
orders of  magnitude  away from  Pauli-limiting regime.   With like  assumptions,
similar calculations for other GIC's give $\alpha$'s  of the same  order of
magnitude.  Therefore spin-orbit  scattering has no impact  on the critical
fields of $\rm C_4KHg$.

\begin{table}
\begin{center}
\begin{tabular}{|c|cccc|}
\hline
& & & &  \\
$\rm T_c$  & $\rm \alpha_{\parallel \hat{c}}$ (from $\rho$) & $\rm \alpha_{\parallel \hat{c}}$ (from slope) & $\rm \alpha_{\perp \hat{c}}$ (from $\rho$)
& $\rm \alpha_{\perp}$ (from slope) \\
& & & &  \\
\hline
& & & &  \\
0.72\cite{iye82} & 1.48$\times 10^{-3}$ & 4.24$\times 10^{-3}$  & 2.48$\times 10^{-2}$ & 4.14$\times 10^{-2}$\\
& & & &  \\
0.95 & 1.48$\times 10^{-3}$ & 2.32$\times 10^{-3}$  & 2.48$\times 10^{-2}$ & 1.98$\times 10^{-2}$\\
& & & &  \\
1.53 & 1.48$\times 10^{-3}$ & 3.78$\times 10^{-3}$  & 2.48$\times 10^{-2}$ & 2.58$\times 10^{-2}$\\
& & & &  \\
1.54 & 1.48$\times 10^{-3}$ & 2.78$\times 10^{-3}$ & 2.48$\times 10^{-2}$ & 2.60$\times 10^{-2}$\\
& & & &  \\
\hline
\end{tabular}
\end{center}
\caption[Comparison for   $\rm
C_4KHg$  of  two  different  methods for  determination of Maki's  $\alpha$
parameter.]{Comparison for   $\rm  C_4KHg$  of   two different methods  for
determination    of   Maki's   $\alpha$  parameter.\cite{werthamer66}   The
orientation indicated is that of the applied magnetic field. In parentheses
it is noted which of the two halves of Eqn.~\ref{alphadef} was used.}
\label{alphatable}
\end{table}

        WHH  say  about Eqn.~\ref{alphadef}  that  ``It  is a  test of  the
applicability of   our model    for  the   superconductor  that  these  two
determinations  of $\alpha$ should agree.''\cite{werthamer66}   Considering
the crudeness  of  the  assumptions made  in the  estimation of $\rho$, the
agreement in  Table~\ref{alphatable}  between   the two  determinations  of
$\alpha$   is quite   good.   This  is an   indication that   even   though
superconductivity   in GIC's has  some  anomalous  aspects, it    still  is
explained by the same basic electron-phonon coupling mechanism that applies
to  isotropic  superconductors.  Therefore, in  our  search  for models  to
explain the enhanced  critical fields of GIC's,  models  with  truly exotic
coupling  schemes  ({\em  e.g.,\/} plasmons  and  excitons) can   safely be
ignored.

         Another factor that contributes  to  critical field enhancement in
isotropic superconductors besides  spin-orbit scattering is strong-coupling
effects.  ``Strong-coupling'' refers to the case of a large electron-phonon
interaction.  The strength of electron-phonon  coupling is measured  by the
size  of the   dimensionless   parameter  $\rm \lambda_{ep}$,    the   same
electron-phonon parameter that appears   in  the dressed  density-of-states
found from   specific-heat measurements.\cite{ashcroft76}  Note   that $\rm
\lambda_{ep}$ is not  related   to the magnetic-field   penetration   depth
$\lambda$ (see  Section~\ref{angdata} for  a  discussion of the penetration
depth).    McMillan\cite{mcmillan68} calculated    $\rm T_c$  in  terms  of
$\rm \lambda_{ep}$  and    $\mu ^*$,  the   Coulomb  pseudopotential  of  Morel  and
Anderson\cite{morel62}.  He found:\\

\[
\rm T_c \; = \; \frac{\Theta_D}{1.45} \: \exp{-\left[ \frac{1.04 \left( 1
\, + \, \lambda_{ep} \right)}{\lambda_{ep} \; - \; \mu^* \left(1 \; + \; 0.62\lambda_{ep} \right)}  \right]}
\]

\noindent where $\rm \Theta_D$ is the Debye temperature.   Using a typical
number of  $\mu^*$ = 0.1,\cite{mcmillan68},  one can solve for
$\lambda_{ep}$:

\begin{equation}
\label{mcmillan}
\rm \lambda_{ep} \; = \; \frac{1.04 \; + \; 0.1 \ln{\left( \Theta_D/1.45 T_c
\right)}}{0.938 \ln{\left( \Theta_D/1.45 T_c \right)} - 1.04}
\end{equation}

\noindent  The results of this calculation for the superconducting GIC's
whose    Debye   temperature    has   been   measured   are  collected   in
Table~\ref{lambdatable}.  $\lambda_{ep}$  in  GIC superconductors appears  to be
$\approx$0.4,     about   the    same   as   in  prototypical weak-coupling
superconductors like aluminum and zinc.\cite{mcmillan68} This is in keeping
with one's expectations for a material with a rather low $\rm T_c$ of about
1-2 K.

\begin{table}
\caption[Values of $\lambda_{ep}$, the electron-phonon coupling parameter, for
GIC  superconductors.]{Values  of $\lambda_{ep}$, the  electron-phonon  coupling
parameter, for GIC superconductors. $\rm T_c$ = 0.73 K\cite{iye82} is used
for $\rm C_4KHg$ since no transition was observed down to 0.8 K during the
specific-heat measurement.\cite{alexander81}  Values of $\rm \lambda_{ep}$
for the KH-GIC's are gathered in Table~\ref{khcvtable}.}
\label{lambdatable}
\begin{center}
\begin{tabular}{|c|ccc|}
\hline
& & & \\
Compound & $\rm T_c$ (K) & $\rm \Theta_D$ (K) & $\lambda_{ep}$ \\
& & & \\
\hline
& & & \\
$\rm C_4KHg$ & 0.73\cite{iye82} & 269\cite{alexander81} & 0.38\cite{iye82} \\
& & & \\
$\rm C_8RbHg$ & 1.44\cite{alexander81} & 235\cite{alexander81} & 0.44\cite{iye82,alexander81} \\
& & & \\
$\rm C_8KHg$ & 1.93\cite{alexander81} & 260\cite{alexander81}, 301\cite{alexander80} & 0.46\cite{tanuma81,alexander80} \\
& & & \\
\hline
\end{tabular}
\end{center}
\end{table}

        In  amorphous superconductors, strong-coupling  effects   can cause
extended linearity at   low temperatures.\cite{bergmann74} However,   the
values  of   $\lambda_{ep}$ quoted  above  eliminate    the possibility that GIC
superconductors are subject  to  any critical field field  enhancement from
strong-coupling effects, since   these   effects are   important  only  for
$\lambda_{ep} \geq$ 1.  Furthermore,
since    the strong-coupling  enhancement   is  larger   near    $\rm   T_c$  than  at  low
t,\cite{decroux82,dalrymple83} strong-coupling effects tend  to actually {\em decrease}
$\rm h^*(0)$, the reduced field at zero temperature.

        Recently  some theories of  ultra-strong coupling superconductivity
have been pub\-lished,  in\-spired by the  ad\-vent of high-tem\-per\-ature
su\-per\-con\-duct\-ivity.\cite{bulaevskii87,marsiglio87} Some of these models find
pos\-i\-tive     cur\-va\-ture  of    $\rm  H_{c2}(t)$.    Bu\-laev\-ski\u{\i}      and
Dol\-gov\cite{bulaevskii87}      find        that       $\rm    h^*(0)$     =
$0.45\sqrt{\lambda_{ep}}$ for $\lambda_{ep} \gg $1.  Marsiglio and Carbotte
find   $\rm  h^*(0)$  about 1.6, but  only   when  $\rm T_c   \: \approx \:
\Theta_D$.\cite{marsiglio87} These   models are  clearly not applicable  to
superconductivity in known GIC's.

        For  GIC's, a more relevant  consideration  than strong-coupling is
inhomogeneity.     From     both  structural\cite{kamitakahara84,K167}  and
superconducting\cite{Z234,delong83} studies, there is abundant  evidence for
the coexistence of  multiple phases   in $\rm  C_4KHg$.  This  evidence  is
discussed  in detail in Chapter~\ref{hydrog}.    As far  as  critical field
experiments  go, this  multiphase behavior  is important  because  of   the
possibility  that inhomogeneity is the  cause of the  enhanced linearity of
the critical  fields of $\rm C_4KHg$.  Carter  and coauthors\cite{carter81}
developed a model for  the case of multiphase materials  which contain both
an  equilibrium  and   higher  free-energy phase.    The model  treats  the
inhomogeneity by allowing  the   superconductor  to have a distribution  of
diffusivities described by a function P(D).  Then, instead of the
Maki-deGennes equation (Eqn.~\ref{makidegennes}) for the critical fields of
dirty superconductors, one uses:

\begin{equation}
\label{carter}
\rm \ln \: t \; = \; \psi\left(\frac{1}{2}\right) \; - \; \int dD \, \psi\left(
\frac{1}{2} \; + \; \frac{D e H_{c2}}{2 \pi k_B T c}  \right) \, P(D) 
\end{equation}

\noindent where all the symbols are the same as before, and $\psi$ is the
digamma function.  Carter  {\em et  al.\/}  found    that by widening   the
distribution P(D) from  a $\delta$-function  (implied by the   choice of  a
single D value)   to a broad  hump that  they could produce  both  positive
curvature and $\rm h^*(0) \approx$ 0.85.  By skewing  the distribution P(D)
to  low D, they  could even get  $\rm  h^*(0)$  $>$ 0.9.\cite{carter81} The
results of their calculations are   shown in Fig.~\ref{inhomo}, where  $\rm
h^*$ (called h in the axis label) is plotted versus  t as a function of the
normalized diffusivity  distribution function,  Q(y).  Q(y) $\equiv$  D$\rm
_{ave}$  P(D), where $\rm D_{ave}$ is  the mean diffusivity, and y $\equiv$
D/$\rm D_{ave}$.

\begin{figure}
\vspace{7.5in}
\caption[Extended critical field linearity due to  small-scale sample
homogeneity (calculated).]{Extended critical   field   linearity  due    to
small-scale  sample  inhomogeneity.  From  a   calculation  by Carter   and
colleagues.\cite{carter81} The plots are  of reduced  field  versus reduced
temperature  for   several  different  normalized diffusivity  distribution
functions  Q(y).    Q(y)  $\equiv \rm  D_{ave}$P(D),  where   P(D)   is the
distribution function  for   diffusivity,  $\rm D_{ave}$   is  the   average
diffusivity, and y  $\equiv$ D/D$\rm _{ave}$.   In  the lower plot,  a P(D)
skewed  to lower  diffusivities  produces  an even greater  critical  field
enhancement at low temperatures.  The index n refers to the power of the
linear factor used to skew the symmetric distribution.}
\label{inhomo}
\end{figure}

        The  physical cause of  the inhomogeneity-related enhancement has to
do with the  temperature  dependence   of  $\xi$, which  is the approximate
radius of a normal  vortex in  a  superconductor.\cite{tinkham80} At $\rm T
\approx T_c$, $\xi$ is large, so that vortices must extend over both high-D
and low-D   regions  in the  material.  At  low t,   where $\xi$  has grown
considerably smaller, the  material can  save some  condensation energy  by
preferentially    packing  the vortices into  the   low-D  regions with higher
critical fields.  As  a result, when  $\xi$(T)  becomes on the order of the
domain size, $\rm H_{c2}$ will turn upward.\cite{carter81}

         Does this   model offer an explanation   of positive curvature and
enhanced   linearity in  the  critical   fields   of  GIC's?   Clearly  the
inhomogeneity interpretation has some appealing  features for $\rm C_4KHg$,
but it also has some problems.  One is that  among the GIC superconductors,
multiphase  behavior   has been  observed  only in   $\rm C_4KHg$.   As  is
discussed in  Chapter~\ref{hydrog},  $\rm C_4KHg$ is  remarkable  among the
superconducting GIC's for the wide range of $\rm T_c$'s  it  exhibits (from
0.7 to  1.6 K), and because it   undergoes  what is  apparently an ordering
transition     under     the    influence     of     small    perturbations
(hydrogenation\cite{Z234} and  small  hydrostatic pressure\cite{delong83}).
These features are  not observed  for other  GIC's, which have well-defined
transition      temperatures   and  show   no     unusual   behavior  under
pressure.\cite{iye83,delong83} It does not seem to make  sense to attribute
the anomalies in $\rm C_4KHg$ to a different cause than the deviations seen
in other GIC's, especially  considering that the other  GIC's  show  larger
anomalies (see Fig.~\ref{rbhct}).

        Even if one were willing to assume separate causes for the enhanced
critical fields  of the  various GIC's, it is not  clear that the  model of
Carter and colleagues would be applicable.  The problem is that their model
makes the (reasonable)  assumption   that   microscopically   inhomogeneous
superconductors  will be   in  the dirty  limit,  where  the  Maki-deGennes
equation is   applicable.   However,  $\rm  C_4KHg$  appears  to  be fairly
``clean,'' at  least for in-plane transport.  The  standard way to quantify
cleanliness in   a  superconductor  is  to calculate   the  parameter  $\rm
\lambda_{tr}  \, \equiv$   0.882 $\xi_0 / \ell$.\cite{helfand66}  Here $\rm
\xi_0 \equiv 0.18 \hbar v_F/ k_B T_c$  is the Pippard  coherence length, and
$\ell$ is  the mean-free path.  For $\rm  C_4KHg$, $\xi_0$ is roughly  9000
\AA, and $\rm
\ell_a$, the in-plane  mean-free path, is about 9100 \AA.  (These  numbers
were  obtained   from   Shubnikov-deHaas  data\cite{W179}   using  standard
rigid-band analysis,  as demonstrated  in Appendix~\ref{lambda_tr}.)  Since
$\rm   \ell_a$ $\approx$   $\xi_0$,   the    dimensionless  parameter  $\rm \lambda_{tr} \equiv$ 0.88 $\xi_0/\ell
\approx$  0.86 for in-plane transport.   $\rm \lambda_{tr}$ $<$ 1 is
indicative  of fairly   clean  behavior,\cite{helfand66}   so this  is   an
indication that inhomogeneities are not likely to be the cause  of enhanced
linearity, at least for $\rm
\vec{H}  \parallel  \hat{c}$.    However,    this calculation   does   not  rule   out   a
diffusivity-variation influence on $\rm  \vec{H} \perp \hat{c}$.  Transport
is expected to be much dirtier along $\rm  \hat{c}$,  where the resistivity
is about 300 times higher than in-plane.\cite{fischer83} $\rm
\lambda_{tr}$ is estimated very roughly to be about 50 for c-axis transport
by assuming a spherical band.  (No well-bounded  number  is available since
the   Shubnikov-deHaas  data   give no information   about  the intercalant
bands.\cite{W179}  c-axis transport   in    GIC's     is  discussed      in
Section~\ref{csbidisc}.)

        In summary, use   of   the inhomogeneity  model  for enhanced  $\rm
H_{c2}$ can be justified for $\rm C_4KHg$  for $\rm \vec{H} \perp \hat{c}$,
but it is hard to justify for $\rm \vec{H} \parallel \hat{c}$, or for other
superconducting GIC's.   Therefore   the most sensible   conclusion is that,
while the factors  discussed by Carter {\em  et al.\/} may play a  role  in
$\rm  C_4KHg$, they  probably  do  not   have a   dominant effect  on  $\rm
H_{c2}$(T).

        Finally, to round out the discussion  of critical field enhancement
in isotropic superconductors, mention should be made  of the anomalous $\rm
H_{c2}(T)$ behavior found in heavy-fermion  superconductors.\cite{delong87}
These  unusual materials  exhibit both   positive curvature and  large
values of $\rm  h^*(0)$.  Any explanation  of  these phenomena must  remain
tentative  since   the  basic   physics    of these compounds   is    still
controversial,\cite{lee86}  but  recently  an  interesting   model has been
proposed by  DeLong {\em et al.\/}\cite{delong87} The  explanation is based
on the observation  that dH$_{c2}$/dT $\propto  \: \rho$  in the  usual WHH
model, as shown in Eqn.~\ref{alphadef}. If  the normal-state resistance  $\rho$  is  assumed
independent of field up to $\rm H_{c2}$, then one gets the usual WHH result
$\rm h^*(0)   \:  \approx  0.7$.  However,   if  there   is a  very  strong
magnetoresistance    (the   manifestation   of  a  field-dependent  pairing
interaction), then the slope formula shows that $\rm h^*(0)$ can exceed 0.7
by a factor of $\rm \Delta
\rho(H_{c2}) / \rho$.\cite{delong87} Here $\Delta \rho(H_{c2}) / \rho$ is
the magnetoresistance  at  the   field $\rm  H_{c2}$;  the    normal  state
resistivity at  zero-field  is   extrapolated from above  $\rm T_c$.   This
formalism  may have wide applicability   to superconductors with  large  or
anomalous magnetoresistance; however, it appears  unlikely to help  much in
the case  of GIC's, since the critical  fields  are  small  enough that the
magnitude of $\rm \Delta
\rho(H_{c2})/\rho$ is anticipated to be $\approx$ 0.  The magnetoresistance
of the KHg-GIC's was  observed by Timp  and coworkers, who reported nothing
extraordinary.\cite{W179}

        Four    causes  of    critical  field   enhancement    in isotropic
superconductors have been discussed: spin-orbit scattering, strong-coupling
effects, inhomogeneities,  and magnetoresistance.    Of  these, only  the
multiphase-superconductor model of  Carter {\em et al.\/}  is thought to be
relevant to  GIC superconductors.  Inhomogeneity  effects could play a role
for $\rm \vec{H} \perp \hat{c}$,  but except for $\rm C_4KHg$,  there is no
hard evidence for multiphase behavior in other superconducting GIC's except
for $\rm C_4KH_x$.\cite{Z260,suzuki85c}  

        The extremely anisotropic  nature  of superconductivity  in   GIC's
(1/$\epsilon$  as  high as 47\cite{iye83})  provides motivation to consider
models   of   positive  curvature  and    enhanced  linearity which   treat
orientation-dependent effects as central.  Some of these models are
considered in the next section.

\subsection{Multiband and Anisotropic Gap Models of Superconductivity}
\label{multiband}

        There are  two-types of models which  take anisotropy as a point of
departure,  rather  than  treating   it as  a  small  perturbation,  as the
pioneering  Hohenberg and Werthamer  model did.\cite{hohenberg67} One class
consists  of models   that consider a   single non-spherical  Fermi surface
and/or    energy     gap.     This      category    includes     work    by
Takanaka,\cite{takanaka75} Pohl  and    Teichler,\cite{pohl76} Youngner and
Klemm,\cite{youngner80}    Butler,\cite{butler80}     and   Prohammer   and
Schachinger.\cite{prohammer87}  The  other   class  of theories  study  the
implications of  two different  Fermi   surface pieces that  contribute  to
superconductivity.  These models    have  been  developed   by  Entel   and
Peter,\cite{entel76,entel77}          Decroux,\cite{decroux80},         and
Al-Jishi\cite{M143}.

        Theories   in  the   first class, consisting   of  the  anisotropic
gap/Fermi surface  models,  are essentially extensions   of the  AGL  model
discussed   previously.   That   is, these theories  contain  the essential
physics  of  the AGL model  plus  modifications  describing what are called
non-local  effects.    ``Non-local''   is an   adjective that     refers to
clean-limit  phenomena outside  the the   scope of Ginzburg-Landau theories
which  are  related  to the    finite   spatial extent   of   the    Cooper
pair.\cite{tinkham80} That the more complex models reduce to  the AGL model
in   the      local     limit   is   demonstrated   by      Hohenberg   and
Werthamer.\cite{hohenberg67}  What Hohenberg and    Werthamer  did  was  to
rewrite  the  eigenvalue   equation   for  the  energy   gap  (which is  an
intermediate step in the  derivation of the  Helfand-Werthamer equation for
$\rm  H_{c2}$) in terms  of an infinite sum.   The sum has terms which  are
matrix elements of even powers  of a quantity  proportional to $\rm \vec{v}
\cdot \vec{\Pi}$, where $\vec{v}$ is  the Fermi velocity and $\rm \vec{\Pi}
\; \equiv \; \hbar/i \vec{\nabla} - e/c
\vec{A}$  is  the  gauge-invariant   momentum   operator.  Because  of  the
relationship between  the vector potential  $\rm \vec{A}$ and  the magnetic
field,  it   turns out   that  the component  of $\rm \vec{v}$  parallel to
$\vec{\Pi}$  is   orthogonal to  the applied  field,   in  keeping with the
expectation from the GL  theory that transport in  a plane perpendicular to
the applied field determines $\rm H_{c2}$.  The important  quantity in this
formalism is

\[ \rm \oint d\vec{q} \:  N(\vec{q}) \: v_{\perp \vec{H}}(\vec{q})
\]

\noindent where the integral is to be taken over the FS.   Here $\rm \vec{q}$ 
is a point on the Fermi surface, $\rm N(\vec{q})$ is the  density of states
at that point,  and $\rm v_{\perp  \hat{H}} (\vec{q})$  is the component of
$\rm \vec{v_F}(\vec{q})$   perpendicular    to  $\rm \vec{H}$.  Obviously   the
critical field anisotropy is uniquely specified  by one parameter $\epsilon$
only if  the  FS  is   ellipsoidal.  Otherwise  knowledge  of the entire FS
geometry is needed.  Notice that $\rm v_{\perp  \vec{H}} (\vec{q})$ depends
not only on the magnitude of $\rm \vec{v_F}$ at the point $\rm
\vec{q}$, but also on the angle between the angle between $\rm \vec{H}$ and
$\rm  \vec{v_F}(\vec{q})$.  This point   is illustrated by Fig.~\ref{perpexplain}
from Ref.~\cite{dalrymple83}.

\begin{figure}
\vspace{8in}
%Fig. 7-4, p. 263, from Dalrymple's thesis
\caption[Illustration of how $\rm v_{\perp  \vec{H}}  (\vec{q})$ changes as
a     function   of  $\rm      \vec{q}$   for   an     ellispsoidal   Fermi
surface.]{Illustration of how $\rm v_{\perp  \vec{H}} (\vec{q})$ changes as
a function  of wavevector $\rm \vec{q}$  for an ellipsoidal Fermi surface.
$\rm
\vec{q}$ is the coordinate of a point on the  Fermi surface.  B is Dalrymple's
anisotropy  parameter, which  is   equivalent  to $\epsilon$  in   the  AGL
model.\cite{dalrymple83}}
\label{perpexplain}
\end{figure}

        Hohenberg and Werthamer   went on to  show that  keeping  only  the
zeroth-order term  in the infinite  sum gives  the isotropic theory,  while
keeping terms  up  to the second  order in the  dot  product  gives the AGL
model,          including      Eqn.~\ref{ldtheor}           for        $\rm
H_{c2}(\theta)$.\cite{decroux82}  What   all  subsequent models of   FS/gap
anisotropy do is to  include more terms in  the   sum over powers  of ($\rm
\vec{v} \cdot
\vec{\Pi}$).  Increasingly higher-order terms correspond both to higher degrees of
non-locality,   and  because   of symmetry,  also  to  more complex  FS/gap
geometries  ({\em i.e.,\/}  higher  angular momentum  spherical harmonics).
Hohenberg   and  Werthamer         included     fourth-order  contributions
perturbatively,\cite{hohenberg67} which corresponds to  the weak-anisotropy
limit.   The weak-anisotropy limit  is probably  less appropriate for GIC's
than for superconductors like niobium  or vanadium, which have 1/$\epsilon$
on the  order of 2 or  3.  The Hohenberg  and Werthamer model only modifies
$\rm H_{c2}(t)$ near  $\rm  T_c$, and cannot  produce the low-t enhancement
observed in GIC's.

        Takanaka  included  terms up to  order four  in  a non-perturbative
manner,   but his  theory   is limited   to  t $\approx$   1 because of  an
error.\cite{youngner80,ikebe80} Youngner and Klemm, who fit their equations
to  NbSe$_2$ data, found a way  to  sum the  series  exactly up to infinite
order.\cite{youngner80}   They remark   that    their  model  is  basically
equivalent  to  that  of   Pohl   and   Teichler,    who    fit   data   on
vanadium.\cite{pohl76} These models  are also very  similar to  the work of
Butler,\cite{butler80} who applied    his theory  to  niobium, except  that
Butler dropped  all  terms due   to gap anisotropy  in  his final equations
because he thought were quite small.  Butler's equations have also been used
by  Dalrymple  and  Prober\cite{dalrymple84}       to   fit    data      on
Nb$_{1-x}$Ta$_x$Se$_2$.

        From examination of the  $\rm h^*$ versus t curves  that  appear in
either  Butler's    paper\cite{butler80}     or   Youngner   and    Klemm's
paper,\cite{youngner80}  it is clear   that either   one  can produce  both
extended  linearity  using an  ellipsoidal Fermi surface   model.   This is
encouraging  because the  $\pi$-bands of GIC's  are  generally  believed to
produce an ellipsoidal FS piece,\cite{I94} but it  is  not the whole story.
A   calculation  using     Butler's  equations      taken from  Dalrymple's
thesis\cite{dalrymple83} is shown in Fig.~\ref{butlerellipse}.  Notice that
$\rm h^*(t)$ is enhanced only  when  Dalrymple's parameter B  (which is the
same for an ellipsoidal Fermi surface as $\epsilon$ from the  AGL model) is
less than one,   and that $\rm h^*(0)$  is  actually suppressed  from   the
isotropic  value when B is greater  than one.   The   significance  of this
conclusion is that for a simple ellipsoidal  FS,  the Butler model extended
linearity gives only for one field orientation.\cite{dalrymple83} Therefore
one cannot describe critical field enhancement for both  field orientations
using a simple ellipsoidal FS.

\begin{figure}
\vspace{5in}
%Fig. 7-2 from Dalrymple's thesis
\caption[Enhanced linearity of $\rm h^*(t)$ calculated from Butler's
equations  using  an  ellipsoidal FS   model.]{Enhanced  linearity of  $\rm
h^*(t)$ calculated    from  Butler's  equations\cite{butler80}    using  an
ellipsoidal  FS  model.  Taken   from Ref.~\cite{dalrymple83}.  Dalrymple's
parameter B is  equivalent to 1/$\epsilon$ in the  AGL model.  The B  = 1.0
curve is for a  spherical Fermi Surface, and  so is equivalent  to  the WHH
theory.}
\label{butlerellipse}
\end{figure}

        Because Dalrymple and Prober observed  enhanced critical fields for
NbSe$_2$ for both $\rm  \vec{H}  \parallel \hat{c}$  and $\rm
\vec{H}  \perp \hat{c}$,\cite{dalrymple84}  they tried fitting  their  data
with  more complex   Fermi   surface  geometries.  When   they  combined an
ellipsoid     with  a cylindrical   FS    model calculated   by Wexler  and
Woolley\cite{wexler76}, they found excellent quantitative agreement between
theory and  experiment  for    $\rm \vec{H} \parallel  \hat{c}$,   and good
qualitative agreement  for $\rm  \vec{H} \perp  \hat{c}$.\cite{dalrymple83}
These fits are shown in  Fig.~\ref{nbse2fit}.  In order to get quantitative
agreement for $\rm \vec{H} \perp \hat{c}$, the authors had to  multiply all
the data through by a factor of  2.1.  Dalrymple  and Prober explained that
this factor was necessary  because  of mean-free-path anisotropy, which  is
not taken into  account in the Butler   model.   The mean-free  path $\ell$
comes  in  because of  the dependence of  $\rm H_{c2}$ on the diffusivity D
$\equiv$   1/3$\rm v_F \ell$.    This type of  mean-free-path anisotropy is
quite believable  since    much larger ratios   have often    been seen  in
GIC's.\cite{mcrae88} Essentially Dalrymple  and Prober find that the Butler
model describes $\rm  H_{c2,  \parallel \hat{c}}$ data without  correction.
This agreement occurs  since in-plane transport  in  NbSe$_2$ is clean, and
Butler's model is for clean-limit superconductors.  The  Butler model needs
a correction to fit $\rm H_{c2, \perp
\hat{c}}$ data because transport along $\rm \hat{c}$ is dirty.

\begin{figure}
\vspace{16cm}
%Figs. 7-5 and 7-6 of Dalrymple's thesis
\caption[Butler-model fit of NbSe$_2$ $\rm H_{c2}(t)$
data.]{Butler-model\cite{butler80} fit of NbSe$_2$ $\rm H_{c2}(t)$ data.
Figures taken from Dalrymple's thesis.\cite{dalrymple83} a) $\rm H_{c2,
\perp}$ = $\rm H_{c2, \parallel \hat{c}}$.  An excellent fit is obtained by
using the Wexler-Woolley Fermi Surface model\cite{wexler76} plus an additional ellipsoid.
b) $\rm H_{c2, \parallel}$ = $\rm H_{c2, \perp \hat{c}}$. The
Wexler-Woolley-plus-ellipsoid model produces the correct shape, but needs
to be multiplied by an additional factor of 2.1 to account for mean-free-path
anisotropy.}
\label{nbse2fit}
\end{figure}

        Youngner and Klemm's model of $\rm H_{c2}(T)$ seems to be  the same
as Butler's, but with the added feature of gap anisotropy.\cite{youngner80}
In  this  case, $\rm \vec{v}(\vec{q}) \cdot  \vec{\Pi}$ may still depend on
$\rm  \vec{q}$.  In addition,  though, the  states between which the matrix
element  of $\rm \vec{v} \cdot \vec{\Pi}$  is evaluated may also themselves
be non-spherical.  Youngner's model  produces curves quite similar to those
of  Butler, including  extended linearity and  positive  curvature for some
choices   of   parameters.     The  recent    theory   of    Prohammer  and
Schachinger\cite{prohammer87} is nearly  the same as  Youngner and  Klemm's
model, only   these  authors  consider electron-phonon  coupling anisotropy
rather than explicit gap anisotropy.

        As if these weren't already enough  models, there are also theories
which     consider the  effect   of   multiple    bands     contributing to
superconductivity.  The most extensive of these was developed by  Entel and
Peter.\cite{entel76} A two-band model  fit by Entel and  Peter to  data  on
Cs$_{0.1}$WO$_{2.9}$F$_{0.1}$,  a tungsten  fluoroxide  bronze,  is shown  in
Fig.~\ref{entelfit}.    Decroux and   Fischer   advocate   the use  of  the
Entel-Peter  model   for fits       to  ternary  molybdenum    chalcogenide
data.\cite{decroux82}  The main  parameters  which   determine the critical
field enhancement in this theory are the interband and intraband scattering
times.   Al-Jishi\cite{M143}  has proposed a  theory quite similar to Entel
and Peter's, but his critical field calculations are still in a preliminary
phase.   In GIC terms, these   models  propose  that  intercalant  and
graphitic bands   both contribute  to  superconductivity.   This   idea was
originally suggested by Al-Jishi,\cite{M143} and  it is supported by a large
amount of experimental evidence, as is described in Section~\ref{csbidisc}.

\begin{figure}
\vspace{5in}
% Fig. 2 of entel77
\caption[Two-band model fit to anomalous $\rm H_{c2}(t)$ of
Cs$_{0.1}$WO$_{2.9}$F$_{0.1}$.]{Two-band  model  fit   to  anomalous   $\rm
H_{c2}(t)$ of Cs$_{0.1}$WO$_{2.9}$F$_{0.1}$  from  Ref.~\cite{entel77}. The
plot    is of  $\rm   h^*$  versus t.    The   curve labeled   (4)  is  the
Helfand-Werthamer  isotropic theory.  The crosses,  circles and squares are
experimental data  for three  different crystallographic  orientations (the
orientations are not specified).   Curve (1) is the  two-band model with no
interband-scattering,  whereas  (2)   and   (3) correspond  to   increasing
interband-scattering.   The parameters of  these  fits  are too numerous to
list here, but may be found in Ref.~\cite{entel77}.}
\label{entelfit}
\end{figure}

        The profusion of models which  predict an enhanced  $\rm H_{c2}(T)$
is quite confusing,  especially  since both the multiband  and  anisotropic
gap/FS theories seem  to have features  which are very  sensible for GIC's.
Prohammer and Schachinger say in  their recent paper\cite{prohammer87}  that
the   Entel-Peter model  and the  anisotropic gap/FS   models are  actually
equivalent, really amounting to different   parameterizations of  the  same
phenomena.    Considering  the   similarities    of    the   curves      in
Figs.~\ref{nbse2fit} and \ref{entelfit}, this is not too surprising.  

        The question of whether these models  will explain the anomalies in
GIC  $\rm H_{c2}(T)$  data, ranging  from  the extended linearity  of  $\rm
C_4KHg$ to  the  positive  curvature of $\rm  C_8RbHg$, still  has not been
directly addressed.  Clearly  all of the multiband and   anisotropic gap/FS
theories  are capable of   producing  curves of the  right  shape, but this
doesn't mean that they would  give good  fits to  the data using reasonable
FS/gap parameters.  Unfortunately,  no complete  theoretical Fermi  surface
computation  for the ternary GIC's is   available, and  there  is only
limited   knowledge  about    the   normal-state    transport   properties.
Holzwarth\cite{holzwarth88} has  calculated  the   band structure  of  $\rm
C_4KHg$,  but published  only  a qualitative sketch  of the  Fermi surface.
This sketch  is reproduced in  Fig.~\ref{holzfs}.   If Holzwarth  publishes
quantitative information about the FS of $\rm C_4KHg$, or,  even better, FS
information for  $\rm C_8RbHg$, then a  fit could be made  to  the critical
field data.  Obviously this hypothetical fit would be the decisive  test of
the applicability of these models.

\begin{figure}
\vspace{5in}
\caption[Fermi surface com\-pu\-ted for $\rm C_4KHg$ by Holz\-warth and
col\-leagues.]{Fermi  surface  com\-pu\-ted  for $\rm  C_4KHg$  by  Holz\-warth and
colleagues.\cite{holzwarth88} The  basic structure of  the Fermi surface is
similar to that of NbSe$_2$\cite{dalrymple84}  in  that both have pieces of
nearly  cylindrical symmetry at the  corner of  a hexagonal Brillouin zone,
and both have higher masses for transport along $\rm k_z$ than in the layer
planes.   The  hexagonal solid line   is   the Brillouin zone;  the roughly
triangular pieces drawn with a solid line at the  corners of the BZ are the
graphitic $\pi$ bands.    The pieces  drawn with a  dotted line  are due  to
mercury bands.  The small circular zone-center part is from  Hg 6$s$ holes;
the  hexagonal  portion  is  from  Hg 6$p\pi$   electron carriers; and  the
trigonal pieces at the zone corner are derived from H 6$p\sigma$ bands.}
\label{holzfs}
\end{figure}

        Lamentably an attempt to invert the critical  field data to predict
the geometry  of the Fermi  surface is not justified  by the critical field
data  which  is available.  In principle  at  least, if $\rm H_{c2}(T)$ had
been measured at a large number of angles $\theta$, this inversion could be
performed.  

        What conclusions can  be drawn  about  the appropriateness  of  the
anisotropic gap/FS or multiband models for  GIC's in the  absence of a fit?
On  the plus side, these  models contain features  compatible  with what is
already  known about GIC's: GIC's  have highly anisotropic  Fermi surfaces;
they are more disordered along the c-axis than in-plane; they have multiple
bands present at the Fermi surface; and their  $\rm h^*(t)$ curves are much
like those  of the transition metal dichalcogenides  for whom most  of  the
models  were intended.   Therefore the  tentative  conclusion is  that  GIC
superconductivity is  described by  one  of these  theories,   although for
positive confirmation, fits to the data are still needed.




