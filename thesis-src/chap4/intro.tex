% -*- mode: latex -*-
\section{Introduction}
\label{critf:intro}
	Ever since   Meissner    and  Ochsenfeld  discovered the    perfect
diamagnetism of superconductors in 1933,\cite{meissner33}  the study of the
behavior of  superconductors   in  an applied magnetic   field has   been a
principal  activity  of superconductivity experimentalists.   One reason is
that perfect diamagnetism is a phenomenon  peculiar to superconductors, and
thus is often used as a diagnostic for the occurrence of superconductivity,
as indeed was done in the experiments described  in this thesis.  A second
reason    is that critical   field  measurements   provide a great  deal of
information about  the material being  studied,  as  is  illustrated by the
idealized results shown in Figure~\ref{fig:meiss}.

%Figure 1a is Figure 5-2 of Tinkham; and Figure 1b is sketched in the margin next
%to Figure 5-2 of Tinkham. (p. 157)

	Figure~\ref{fig:meiss}a  illustrates the  difference between type I
and type II superconductors.  The  group of type I superconductors  is made
up of almost all the elemental superconductors.   These materials exhibit a
first-order  transition  in  an applied   magnetic    field.   Due to   the
supercooling effects associated  with a first-order transition, experiments
on  type I  superconductors are often   complicated by hysteresis.  Type II
superconductors,  on the other  hand,  show a second-order,  non-hysteretic
transition in a magnetic field.  Type I superconductivity can be identified
experimentally  by  noting the presence  or absence of  hysteresis and  the
differential  paramagnetic  effect (DPE) in the  field  sweeps.  The DPE is
characterized by  the occurrence of a ``bump''   in the susceptibility just
above $\rm H_{c2}$ which is a property of the  intermediate state of type I
superconductors.\cite{hein61}  Available evidence suggests  that   most GIC
superconductors are of  the type  II  variety, although  some GIC's do have
type I transitions for a range of applied-field orientations.

\begin{figure}
\vspace{14cm}
%\beginpicture
%\setcoordinatesystem units <115mm,26.0mm>
%\setplotarea x from 0 to 1, y from -1 to 1
%\axis bottom shiftedto y=0 
%ticks withvalues $\rm H_{c1}$ $\rm H_{c2}$ /  at 0.1 0.344 / /
%\axis left ticks
%length <0pt> withvalues 0 $\chi$ / at 0 0.9 / /
%\put {H} [t] at 0.8 -0.05
%\endpicture
\caption[Comparison of type I and type II superconductors.]{a) dc magnetization
versus field for ideal type I and type II superconductors.  $\rm H_{c1}$ is
the lower critical field, $\rm  H_c$ is the  thermodynamic critical  field,
and $\rm H_{c2}$  is the upper  critical field.  $\kappa$  $<$ 1/$\sqrt{2}$
indicates type I  superconductivity; $\kappa \, \approx \,  0.8$  indicates
weak type II behavior; $\kappa \, \approx \, 2$  indicates strongly type II
behavior.   b)   ac   susceptibility   versus  field  for   ideal  type  II
superconductor with $\kappa \,
\approx \, 0.8$.    Adapted  from
Ref.~\cite{tinkham80}.}
\label{fig:meiss}
\end{figure}


	As    Figure~\ref{fig:meiss}a  shows, there  are two characteristic
fields of importance for a type  II superconductor.  These are respectively
termed   the lower critical  field,  $\rm   H_{c1}$, and the upper critical
field,  $\rm H_{c2}$.  $\rm  H_{c1}$ is  the field    at  which flux  first
penetrates a long, thin cylindrical  sample, the only  shape of sample  for
which ``demagnetization''  effects  are  not important.  Unfortunately  the
sensitive dependence of $\rm  H_{c1}$ on sample  shape makes its extraction
from actual experimental data difficult.  The same comments apply to trying
to extract $\rm H_c$, the thermodynamic critical  field, from magnetization
curves  of type I superconductors.   Sample shape does not impact  upon the
measurement of  the upper critical field,  $\rm  H_{c2}$,  making it a much
more  accessible quantity  experimentally.   Most  of the  rest    of  this
discussion  concerns   the    upper  critical    field,  which is   defined
theoretically as that highest  applied field at which superconductivity can
nucleate in a bulk superconductor.

	The history of   upper critical field  experiments   is a long  and
fruitful one, dating  back to the recognition  of the existence  of the two
types     of    superconductivity     by   Abrikosov    in     the     late
Fifties\cite{abrikosov57}.  $\rm     H_{c2}$ measurements  give      direct
information about the Ginzburg-Landau coherence length $\xi$,  a parameter
defined for isotropic bulk superconductors by the equation:

\begin{equation}
\rm H_{c2}\; = \; \frac{\Phi_0}{2\pi \xi^2} \; ,
\label{cohlen}
\end{equation}

\noindent where  $\Phi_0$   is   the  superconducting    flux   
quantum.\cite{tinkham80} The coherence length  is one of  several important
length scales  in a superconductor.  For example,  there  is also the field
penetration  depth, $\lambda$,  whose  size  relative to  $\xi$  determines
whether  the    superconductivity   is   of type  I     or  type  II   (see
Section~\ref{typeI}):

\[ \begin{array}{ll}
\lambda \, < \, \xi, & \rm type \;  I \\
\lambda \, > \, \xi, & \rm type \; II
   \end{array} \; .
\]

\noindent In addition,  there is the  transport mean free  path
$\ell$,  whose    size   relative   to    $\xi$   determines  whether   the
superconductivity  is   in    the   clean  or   dirty  limit:

\[ \begin{array}{ll}
\ell \, \gg \, \xi, & \rm clean \; limit \\
\ell \, \ll \, \xi, & \rm dirty \; limit 
   \end{array} \; .
\]

\noindent In  layered superconductors,   such   as  the  superconducting   
graphite intercalation compounds, another natural  length scale to consider
is the lattice constant perpendicular to the  carbon (graphene) planes.  In
the  GIC's, this lattice  constant is  termed $\rm I_c$.   The most studied
phenomena in the field of layered superconductors are due to the dependence
of $\xi$ on temperature and crystallographic direction.  The consequence of
this variability  is  that in  layered  systems  there  is  an  additional,
experimentally controllable, distinction between 3D-coupled  and 2D-coupled
superconductors:

\[ \begin{array}{ll}
\xi \; > \; s, & \rm 3D \; coupling\\
\xi \; < \; s, & \rm 2D \; coupling
   \end{array}
\]

\noindent where s is the layer spacing. The consequences of   the interplay of  the  various
length  scales    for other layered   superconductors  are   discussed   in
Chapter~\ref{othersys}.  Before turning to  the  results of critical  field
measurements  on   $\rm  C_4KHg$, it is  appropriate  to  review  how the
measurements were made.



