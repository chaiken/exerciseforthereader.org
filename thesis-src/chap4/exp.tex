\section{Experimental Methods: $\rm H_{c2}$ Data}
\label{critf:exp}

	This section  describes  techniques used   in  this thesis  for the
gathering  of $\rm  H_{c2}$  and other low--temperature  data,  except  for
information  relating  to    the  Shubnikov-deHaas  and   magnetoresistance
experiments, which are  described in Chapter~\ref{csbi}.   The  preparation
and characterization of the samples used in these experiments are described
in Chapter~\ref{samprep}.

\subsection{Mounting the GIC's for Critical Field Measurements}
\label{mounting}

	The   first   consideration    when   undertaking   critical  field
measurements on  an\-iso\-trop\-ic samples is  how to mount  the specimens.  High
anisotropy,  the property of  central  interest, unfortunately presents the
investigator with special  problems.    In   a nutshell, the  larger    the
anisotropy, the greater  are the  errors   induced by small  misalignments.
Since most of the experiments in the present work were done on HOPG-derived
polycrystals, the  only  concern relevant to   alignment  was  the  careful
orientation of the crystalline c-axis  with respect to the applied magnetic
field.  Due  to  the samples'  air  sensitivity, they  had to  be preserved
inside sealed glass  tubes.  Therefore alignment in practice  meant  fixing
the orientation of the GIC with respect to the glass  tube.  Early critical
field results were found to be unreliable due to the  fact that the samples
were mobile inside the glass tubes.

	The obvious tack to  take  in mounting the samples  was to  use the
technique  employed by Iye  and Tanuma.\cite{iye82}  These workers  wrapped
their GIC's in Parafilm, a kind of wax paper manufactured by Union Carbide,
and took  the  sample thus wrapped  out  of  their   glovebox directly to a
coldfinger,  where    it    was   immediately   mounted and   cooled  down.
Unfortunately in the present case the available  cryostat and glovebox were
separated by  a great distance, so  this transfer method  was thought to be
infeasible.  Also, this  method presents difficulties  for  preserving  the
sample upon warming  to room temperature,  and it is  desirable to save the
samples for  repetition of experiments.   Therefore it  was decided that the
specimens would have to be always encapsulated in glass  tubes for handling
and transfer.  The idea of wrapping the sample in Parafilm  in order to fix
its position within the glass tube did not seem to  be practical.

	Several methods were tried in an effort to fix the  samples without
degrading them.  First, it was  attempted  to hold the  samples in place by
using a gas torch  to make an  indentation of the encapsulating tube around
the sample's  position, trapping it  in one  orientation.  This method  was
unsuccessful due to the deintercalation that the heat of  the torch caused.
The   next idea,   that of gluing  the samples  to a  holder, was discarded
because of the extreme chemical reactivity of alkali metal compounds.

	 Success  was finally   achieved  by   mounting the specimens    on
specially made metal holders whose general features are  sketched in Figure
~\ref{fig:holder}.  In this method, sheets of copper or brass were cut into
pieces of such a width that they fit tightly  into the  appropriate size of
glass tube.  Copper and  brass   were   chosen because  their high  thermal
conductivity  should  prevent  the  development of  a  temperature gradient
across  the     samples at  the  very   low    temperatures  used  in   the
superconductivity studies.  Also, these  metals are neither superconducting
nor ferromagnetic, and so produce no temperature-dependent inductive signal
to compete with that from the superconducting  transition.  After the metal
pieces were  cut, they were cleaned  with  a  weak solution of hydrochloric
acid, dipped in a bicarbonate solution to stop the  acid reaction, and then
rinsed in de-ionized  water.  The metal holders were  then heated under  to
vacuum to remove any residual water, and  taken  into the inert environment
of the glovebox.  (See general comments on handling samples in the glovebox
in Chapter~\ref{samprep}.)

\begin{figure}
\vspace{5in}
\caption[Sample holder used in critical field measurements.]{A sketch of the sample holder used in the critical field
measurements.   The dimension  $d$ of the metal piece  was chosen to be the
inner diameter of the sample tube so that the holder would  be centered and
fixed  inside the tube.   A careful effort  was  made  to orient the carbon
(graphene) planes parallel to the holder's surface.  The GIC's were affixed
to the metal pieces with Apiezon N grease.}
\label{fig:holder}
\end{figure}

	When the metal holders  were first  used, the samples were attached
to them by pressure  applied with a folded-over  flap of metal.   This flap
was squeezed over the sample with  tweezers after the  sample was placed on
top of the  metal.  Half the time this  technique was  very successful: the
sample after mounting   showed  no  degradation   of  its   superconducting
transition    or    of    its  (00$\ell$)    x-ray    diffractogram.   (See
Chapter~\ref{samprep} for a discussion of staging characterization of GIC's
by (00$\ell$) x-ray diffraction.)  The other half of the time, the sample's
transition temperature  was reduced,  and  its x-ray  spectrum   was either
attenuated or completely  suppressed,  presumably  due to  fracture of  the
carbon (graphene) planes resulting from excess pressure.

	Because the sample attrition rate from this  method of mounting was
unacceptably high, it was decided to try attaching the samples to the metal
holders with vacuum grease.  Vacuum grease is considerably  more inert than
commercially available glues, and Apiezon N grease was used successfully in
mounting $\rm C_8K$ by   Sano and coworkers.\cite{sano80}  In the  end, the
best method of mounting the samples turned out to  be dabbing grease on one
side with a piece of wire, and then using the grease to  adhere the samples
to the metal holder.  The grease was left overnight in an open container in
the  glovebox before use so that  it  would outgas.    Mounting with vacuum
grease allowed specimens to be fixed in such a way that their $\rm T_c$ and
x-ray  spectrum  were unaffected,  and in addition   satisfied  the goal of
keeping the GIC stationary even when the sample tube was shaken.  The GIC's
should have been firmly   fixed at  low temperatures  where the  grease can
freeze them to the holder.

	Once the GIC's  were  fixed in their glass   tubes, the tubes  were
fixed inside the  inductance  bridge's primary coil  with GE  7031 varnish,
taking care  that the   tube's axis was parallel  to  the modulation  field
direction.  This alignment  is thought to be good  to within a degree or so
since the walls of the  tubing  and  the inside of the  coil are both flat.
The  result was  that the modulation  field was  applied in  the  direction
perpendicular to the graphite c-axis, where it had the  least effect on the
critical field measurements.

\subsection{Data Acquisition System}
\label{electronics}

	The  superconducting  transitions  were  measured by monitoring the
ac susceptibility  of the sample  as  the temperature or  magnetic
field was swept.  An ac inductance bridge  was used to record  the sample's
susceptibility.   The homemade inductance  bridge,   which is  sketched  in
Figure~\ref{ibridge}, was composed of a  primary  solenoid, around which
two counterwound secondary solenoids about  half the length  of the primary
were placed.  The  secondaries were balanced  to within about  2 microvolts
under the typical operating conditions of the experiment.   The off-balance
voltage included contributions from the sample holder and small unavoidable
differences between the secondaries.  This voltage was nulled  by using the
zero adjust knob on  the  lock-in amplifier (see   Figure~\ref{schematic})
before  the field  and  temperature  sweeps    began.   Several   different
versions of the  inductance bridge were  used in the early zero-field
experiments,  but  all critical  field measurements  reported here were done
with the same unit.

\begin{figure}
\setlength{\unitlength}{0.0125in}
\begin{picture}(520,455)(120,330)
\thicklines
\put( 20,580){\circle{0}}
\put( 40,680){\circle{14}}
\put( 38,360){\circle{14}}
\put(460,640){\circle{14}}
\put(460,400){\circle{14}}
\put(160,642){\circle{6}}
\put(160,642){\circle{16}}
\put(162,482){\circle{16}}
\put(317,642){\circle{16}}
\put(320,482){\circle{16}}
\put(320,482){\circle{6}}
\put(510,400){\vector( 0,-1){ 60}}
\put(510,640){\vector( 0, 1){ 60}}
\put(180,750){\vector(-1, 0){  0}}
\put(180,750){\vector( 1, 0){120}}
\put(480,340){\line( 1, 0){ 60}}
\put(480,700){\line( 1, 0){ 60}}
\put(300,780){\line( 0,-1){ 60}}
\put(180,780){\line( 0,-1){ 60}}
\put(140,540){\framebox(200,120){}}
\put(140,380){\framebox(200,120){}}
\put(180,340){\framebox(120,360){}}
\put(340,560){\line( 1, 0){ 40}}
\put(380,560){\line( 0,-1){ 80}}
\put(380,480){\line(-1, 0){ 40}}
\put(340,480){\line( 0, 1){  1}}
\put(340,400){\line( 1, 0){120}}
\put(340,640){\line( 1, 0){120}}
\put(180,680){\line(-1, 0){140}}
\put(180,360){\line(-1, 0){140}}
\put(156,488){\line( 1,-1){ 12}}
\put(168,488){\line(-1,-1){ 12}}
\put(323,648){\line(-1,-1){ 12}}
\put(311,648){\line( 1,-1){ 12}}
\put (500,600) {\makebox(0,0) [lb] {\raisebox{0pt}[0pt][0pt]{\twlrm 30 mm}}}
\put (25,545) {\makebox(0,0) [lb] {\raisebox{0pt}[0pt][0pt]{\twlrm Primary }}}
\put (30,520) {\makebox(0,0) [lb] {\raisebox{0pt}[0pt][0pt]{\twlrm Coil}}}
\put (440,545) {\makebox(0,0) [lb] {\raisebox{0pt}[0pt][0pt]{\twlrm Secondary }}}
\put (450,520) {\makebox(0,0) [lb] {\raisebox{0pt}[0pt][0pt]{\twlrm Coil}}}
\put (220,765) {\makebox(0,0) [lb] {\raisebox{0pt}[0pt][0pt]{\twlrm 4.0 mm}}}
\put (205,725) {\makebox(0,0) [lb] {\raisebox{0pt}[0pt][0pt]{\twlrm inner diameter}}}
\end{picture}
\caption[Sketch of the inductance bridge]{A schematic drawing of the inductance bridge.
The sample capsule was placed inside the  primary coil.   The windings were
made from 38 gauge magnet wire.  There were 20 complete  layers  of winding
in the secondary coils and 2 complete layers of winding in the primary.}
\label{ibridge}
\end{figure}

\begin{figure}
\setlength{\unitlength}{0.0125in}
\begin{picture}(535,415)(125,310)
\thicklines
\put(520,642){\oval(30,30)}
\put( 40,640){\oval(30,30)}
\put(200,480){\line( 0, 1){200}}
\put(200,680){\line(-1, 0){ 80}}
\put(175,480){\line( 1, 0){ 25}}
\put(120,480){\line( 1, 0){ 25}}
\put(120,600){\line( 0,-1){120}}
\put(465,665){\framebox(30,30){}}
\put(145,465){\framebox(30,30){}}
\put(520,600){\line( 0,-1){105}}
\put(520,495){\line(-1, 0){ 40}}
\put(120,625){\line( 0,-1){ 25}}
\put(120,600){\line(-1, 0){ 80}}
\put( 40,600){\line( 0, 1){ 25}}
\put( 40,655){\line( 0, 1){ 25}}
\put( 40,680){\line( 1, 0){ 80}}
\put(120,680){\line( 0,-1){ 20}}
\put(360,440){\line( 0,-1){ 40}}
\put(380,440){\line( 0,-1){ 40}}
\put(320,320){\framebox(160,80){}}
\put(360,670){\line( 0, 1){ 10}}
\put(360,680){\line(-1, 0){ 20}}
\put(340,680){\line( 0,-1){160}}
\put(360,610){\line( 0,-1){ 90}}
\put(360,650){\line( 0,-1){ 20}}
\put(480,465){\line( 1, 0){ 80}}
\put(560,465){\line( 0, 1){215}}
\put(560,680){\line(-1, 0){ 40}}
\put(520,625){\line( 0,-1){ 25}}
\put(520,600){\line(-1, 0){120}}
\put(400,600){\line( 0, 1){ 10}}
\put(495,680){\line( 1, 0){ 25}}
\put(520,680){\line( 0,-1){ 20}}
\put(400,670){\line( 0, 1){ 10}}
\put(400,680){\line( 1, 0){ 65}}
\put(320,440){\framebox(160,80){}}
\put(115,625){\framebox(10,35){}}
\put( 80,560){\dashbox{4}(360,160){}}
\put(385,680){\line( 0,-1){ 80}}
\put(375,680){\line( 0,-1){ 80}}
\put(395,610){\framebox(10,60){}}
\put(355,610){\framebox(10,20){}}
\put(355,650){\framebox(10,20){}}
\put (460,705) {\makebox(0,0) [lb] {\raisebox{0pt}[0pt][0pt]{\twlrm Ammeter}}}
\put (135,450) {\makebox(0,0) [lb] {\raisebox{0pt}[0pt][0pt]{\twlrm Voltmeter}}}
\put (320,690) {\makebox(0,0) [lb] {\raisebox{0pt}[0pt][0pt]{\twlrm Inductance Bridge}}}
\put (135,640) {\makebox(0,0) [lb] {\raisebox{0pt}[0pt][0pt]{\twlrm Ge Thermometer}}}
\put (345,480) {\makebox(0,0) [lb] {\raisebox{0pt}[0pt][0pt]{\twlrm Lock In Amplifier}}}
\put (340,445) {\makebox(0,0) [lb] {\raisebox{0pt}[0pt][0pt]{\twlrm Output}}}
\put (340,360) {\makebox(0,0) [lb] {\raisebox{0pt}[0pt][0pt]{\twlrm Flatbed Plormer}}}
\put (490,480) {\makebox(0,0) [lb] {\raisebox{0pt}[0pt][0pt]{\twlrm Reference}}}
\put (335,510) {\makebox(0,0) [lb] {\raisebox{0pt}[0pt][0pt]{\twlrm Signal Input}}}
\put (475,675) {\makebox(0,0) [lb] {\raisebox{0pt}[0pt][0pt]{\twlrm A}}}
\put (510,635) {\makebox(0,0) [lb] {\raisebox{0pt}[0pt][0pt]{\twlrm Osc.}}}
\put (35,640) {\makebox(0,0) [lb] {\raisebox{0pt}[0pt][0pt]{\twlrm I}}}
\put (175,700) {\makebox(0,0) [lb] {\raisebox{0pt}[0pt][0pt]{\twlrm Cryostat}}}
\end{picture}

\caption[A schematic of  the data  acquisition  system.]{A schematic of  the data  acquisition  system.  For zero-field  temperature
sweeps, the thermometer voltage was attached  to the x-input of the flatbed
plotter.  For fixed-temperature magnetic  field sweeps, the  dc output of a
stepping motor on the  magnet power supply  was attached to the x-input  of
the plotter.}
\label{schematic}
\end{figure}

	The fraction of a sample which is  superconducting can be estimated
in the inductive measurements because the transition height is proportional
to the fraction  of the primary coil which  is  filled with superconducting
material.  The exact expression, from Faraday's Law in cgs units, is:\\

\begin{equation}
\rm V_{secondary} \; = \; -\frac{\alpha A B_{mod} \omega_{mod}}{c}
\label{vbridge}
\end{equation}

\noindent where $\rm V_{secondary}$ is the superconducting transition
height,  A is the  effective area of   the  superconducting material in the
direction perpendicular to the  primary  modulation field, $\rm B_{mod}$ is
the  modulation field, and  $\rm \omega_{mod}$ is the modulation frequency.
The dependence on the area perpendicular to the modulation field comes from
the fact that this area determines the time-rate of change of magnetic flux
in Faraday's Law.   Here $\alpha$ is a  dimensionless constant of the order
of unity which takes into account the  coupling of the  primary coil to the
secondary.  This constant can in principle  be calculated\cite{abel64}, but
in practice it is preferable to simply measure  $\rm alpha$  by filling the
primary coil completely with a superconducting material, so that  A  = $\rm
A_{coil}$.  Once $\alpha$  is determined in this  way, a measurement of the
transition   height for a    given   specimen  gives   its  superconducting
cross-sectional area.  From   knowledge of  the  sample's total    area, an
estimate   of the   fraction which  is   superconducting  (fraction =  $\rm
A_{superconducting}$/$\rm A_{total}$) can be made, as discussed below.

	It is important to note that this  areal fraction is an upper bound
on  the {\em volume} fraction  of superconductivity.  To  see why, consider
the case where the modulation field is applied along $\hat{z}$, so that the
contribution to the signal comes from planes of constant $z$ from $z$=0 to $z$=L.
[See Figure~\ref{scingfract}.]  Portions  of the  sample  at  all positions
along the coil axis can contribute  equally  to the effective  area.  For a
given position  $(x,y)$  in the  cross-section,  though,  only  one  volume
element can contribute.  This volume element  will screen all others at the
same $(x,y)$ position, so that A is really the area of the projection along
$\hat{z}$ of   all the superconducting   volume  elements  onto a  plane of
constant $z$. Therefore  the cases where 100\%  of the volume superconducts
and just one full sheet (say  that  at  $z$=L)  superconducts cannot be
distinguished by this technique.


        The inductance  bridge used  in the critical field measurements was
calibrated by filling its end with aluminum foil.  The foil had a $\rm T_c$
of    1.1 K    (in      surprisingly   good   agreement    with   tabulated
values\cite{ashcroft76})  with  a  transition height   of 38 $\mu$V.    The
thermal drift of the inductance bridge off-balance  is about 50 nV/K, so it
is estimated that the minimum observable transition height is about 100 nV.
From the known area  of  the primary coil  of 50.3 mm$^2$, it was estimated
that the transition height of a  typical  GIC should be  about 5 $\mu$V, as
was indeed the case.  Thus the  inductive experiment should be sensitive to
superconductivity in  about   2\%  of  a  typical sample.    Unfortunately,
repositioning a  large  GIC inside the  primary coil changed its transition
height  by as  much as a  factor of 5.  This  variation occurs because  the
secondary coil  is not much longer than  the sample,  and the  secondary is
much  more  sensitive to inductance changes near  its  center than near its
end.  A   larger  inductance bridge  could not  be accomodated  inside  the
available $^3$He refrigerators.


\begin{figure}
\setlength{\unitlength}{0.0125in}
\begin{picture}(325,395)(169,350)
\thicklines
\put(340,662){\circle{16}}
\put(340,662){\circle{6}}
\put(340,622){\circle{6}}
\put(340,622){\circle{16}}
\put(340,582){\circle{16}}
\put(340,582){\circle{6}}
\put(340,542){\circle{6}}
\put(340,542){\circle{16}}
\put(340,502){\circle{16}}
\put(340,502){\circle{6}}
\put(340,462){\circle{6}}
\put(340,462){\circle{16}}
\put(340,422){\circle{16}}
\put(340,422){\circle{6}}
\put(177,662){\circle{16}}
\put(171,668){\line( 1,-1){ 12}}
\put(183,668){\line(-1,-1){ 12}}
\put(177,622){\circle{16}}
\put(183,628){\line(-1,-1){ 12}}
\put(171,628){\line( 1,-1){ 12}}
\put(177,582){\circle{16}}
\put(171,588){\line( 1,-1){ 12}}
\put(183,588){\line(-1,-1){ 12}}
\put(177,542){\circle{16}}
\put(183,548){\line(-1,-1){ 12}}
\put(171,548){\line( 1,-1){ 12}}
\put(177,502){\circle{16}}
\put(171,508){\line( 1,-1){ 12}}
\put(183,508){\line(-1,-1){ 12}}
\put(177,462){\circle{16}}
\put(183,468){\line(-1,-1){ 12}}
\put(171,468){\line( 1,-1){ 12}}
\put(177,422){\circle{16}}
\put(171,428){\line( 1,-1){ 12}}
\put(183,428){\line(-1,-1){ 12}}
\put(177,382){\circle{16}}
\put(183,388){\line(-1,-1){ 12}}
\put(171,388){\line( 1,-1){ 12}}
\put(340,382){\circle{6}}
\put(340,382){\circle{16}}
\multiput(260,680)(0.00000,8.00000){8}{\line( 0, 1){  4.000}}
\put(260,740){\vector( 0, 1){0}}
\multiput(360,520)(8.00000,0.00000){8}{\line( 1, 0){  4.000}}
\put(420,520){\vector( 1, 0){0}}
\multiput(280,400)(8.20513,0.00000){20}{\line( 1, 0){  4.103}}
\multiput(280,640)(8.20513,0.00000){20}{\line( 1, 0){  4.103}}
\put(240,400){\framebox(40,240){}}
\put(320,680){\line( 0,-1){320}}
\put(200,680){\line( 0,-1){320}}
\put (280,720) {\makebox(0,0) [lb] {\raisebox{0pt}[0pt][0pt]{\twltt $\rm H_{mod}$}}}
\put (440,510) {\makebox(0,0) [lb] {\raisebox{0pt}[0pt][0pt]{\twltt c-axis}}}
\put (440,535) {\makebox(0,0) [lb] {\raisebox{0pt}[0pt][0pt]{\twltt Graphite }}}
\put (450,400) {\makebox(0,0) [lb] {\raisebox{0pt}[0pt][0pt]{\twltt z = 0}}}
\put (450,640) {\makebox(0,0) [lb] {\raisebox{0pt}[0pt][0pt]{\twltt z = L}}}
\end{picture}
\caption[Illustration of how a partially superconducting  sample can mimic    
full   superconductivity.]{Illustration of  how a    sample  which  is only
partially superconducting can mimic full superconductivity in  an inductive
transition.}
\label{scingfract}
\end{figure}

	As  a practical  matter, the  GIC's used in   these experiments are
thought to  be fairly homogeneous  on a  macroscopic  scale.   Therefore, in
general   the  effective   areal fraction of   superconductivity  should be
approximately  the volume one  as well.  In the cases  where the GIC's were
multiphased, the matter of  superconducting volume  and  screening deserves
more thought, as discussed in Section~\ref{khghyd:exp}.

	The   full    data       acquisition system    is     sketched   in
Figure~\ref{schematic}.  The primary excitation current  was supplied by an
oscillator,   either one inside the lock-in   amplifier, or an external one
attached to the lock-in's reference port.  The usual excitation current was
1 milliampere at 490 Hz. This frequency was chosen because  it seemed to be
particularly   free of electrical noise.   Neither  the critical fields nor
$\rm T_c$ obtained from this apparatus were found to be sensitive to either
the magnitude or frequency  of the  primary current.  The  primary magnetic
field  produced under  this excitation  was calculated  to be 0.17 oersteds
(compared to the 0.2  oersteds at 27 Hz  used in Ref.~\cite{iye82}).  These
conditions were sufficient to give a typical signal-to-noise ratio of about
1000, with the superconducting transition height usually  between  5 and 10
microvolts.   The  detailed theory  of operation of  an   inductance bridge
circuit is described in Ref.~\cite{abel64}.

	The  output signal of  the inductance bridge   was filtered  by the
lock-in,  and the lock-in's  output was  fed to the  y-input  of  a flatbed
plotter.  The x-input for temperature sweeps  was either the voltage across
a calibrated germanium resistor (Lake Shore  GR-200A-100), or the dc output
of  a capacitive pressure sensor (MKS  Baratron) which was  attached to the
$^3$He refrigerator.  The temperature derived  from the pressure sensor was
in   excellent    agreement  (the disagreement  about    the   same as  the
interpolation error) with that obtained from the germanium resistor down to
about 1.0 K, below  which deviations increased with decreasing temperature.
At  the lowest  temperatures (T  $\approx$  0.4 K), the germanium  resistor
consistently showed  readings about 30\%  higher than those of  the  $^3$He
vapor pressure under hard pumping, with the  thermometer reaching agreement
with the vapor pressure  to within a  few percent  after  about  one hour's
wait.  The  implication   here  is that   a well-calibrated thermometer  is
essential for work below 1.0 K.  At the lowest temperatures, the reading of
the thermometer   was    always    recorded   in     preference    to   the
vapor-pressure-derived    temperature.  The    magnetoresistance   of   the
thermometer was negligible throughout the magnetic field range of interest,
and was not corrected for.
	
	Besides the question of agreement between the  temperature sensors,
there is also the problem  of thermal equilibrium  between  the temperature
sensors and the sample.  The sample tube contained about 200 torr of helium
gas  intended to provide  thermal contact of  the GIC  to the  bath  and to
minimize any temperature lag.   Furthermore, the sample  was allowed to sit
at constant temperature for about five minutes prior to each field sweep if
the temperature had been changed.   The actual degree of equilibration that
was  achieved with  these  precautions can  be estimated  by comparing  the
values of $\rm T_c$ obtained in heating and cooling temperature sweeps.  In
general, the offset between the values of $\rm T_c$ measured in heating and
cooling  was on the  order  of  10 mK,  indicating that equilibrium between
temperature sensors and sample was nearly achieved.  The offset between the
sensors   and the  GIC is probably   the  largest  source of  error in  the
determination   of  $\rm   T_c  $  for reasonably  narrow   superconducting
transitions.  This offset is also   one of the most   important sources  of
error in $\rm H_{c2}(T)$ measurements at low temperatures.

	The magnetic field was supplied  by a  set of homemade water-cooled
copper Helmholtz coils about 2 feet in diameter and separated by  about the
same distance.  No  difficulties with hysteresis were  encountered in using
this magnet.  With the largest  available  power supply, a  pair of Hewlett
Packard Model 8012A supplies connected in series, the magnet could be swept
up to a field of about 1000 oersteds at about  30  amperes, which was adequate
for  the   $\rm  C_4KHg$  critical  field   measurements.   The field was
calibrated with a Hall  probe against the output of  a potentiometer on the
sweep control of the power supply.  This  sweep control was  built by Bruce
Brandt of the National Magnet Laboratory.  No effort was made to screen out
the  earth's field,  as its contribution to   the  experimental  errors was
thought to be inconsequential overall.  The largest source of inaccuracy in
the critical field measurements was in the  field calibration, as the exact
value of  the  field  inside  the  Helmholtz  coils was position-sensitive,
varying by  as much as  10\% over the distance of  an inch.  However, since
the field versus potentiometer  calibration is linear,  any inaccuracies in
sample position or calibration should change only the magnitude of the data
curves, not their shape.  Because of the possible errors in sample mounting
and field calibration, it is estimated  that the critical  field at a given
orientation and  temperature  could  be  off by as  much  as 15\%   in either
direction.  However,  the  ratio of the  values  of $\rm   H_{c2}$  at  two
different  angles or temperatures  is   known considerably more accurately,
perhaps to within about $\pm 5$\% of the measured value.

	Most of the  data described in  this chapter were obtained  using a
$^3$He closed-cycle refrigeration system   built  by  Mike  Blaho   of  the
National  Magnet Laboratory.   The inductance  bridge   was attached to   a
standard $^3$He probe loaded into a $^3$He refrigerator  which was immersed
into a  $^4$He  bath.  The   germanium resistor was  glued  directly to the
bridge in  order  to  insure   good  thermal contact  to  the  sample.  The
recirculation system and refrigerator are of standard design.\cite{betts76}
Some of the data above 0.9 K were taken in a single-shot  $^4$He evaporator
cryostat.  No  discrepancy was    found  between   these  and  the   $^3$He
experiments.

\subsection{Procedure for Obtaining Data and Data Analysis}
\label{procedure}

	When a full set of critical field data was desired on a sample, its
zero-field transition was first  obtained to make sure  that the particular
specimen was  in fact superconducting.  Then the  sample  was cooled to the
lowest obtainable temperature, usually about 0.44 K,  at which a  series of
field sweeps was performed as  a function of the  angle between the  sample
c-axis  and the applied field.  The   definition of this angle, hereinafter
called $\theta$, is  illustrated in Figure~\ref{hc2def}a.  The samples were
mounted with  their a-axis  vertical, and   the  field of   the  magnet was
parallel to  the ground, so  that the rotation  was accomplished by turning
the probe by hand about its vertical axis.  The relative amount of rotation
was determined by fixing  a pointer to   the  probe and noting its  angular
displacement  with   respect to  a metal  compass  bolted  onto  the $^4$He
cryostat.  Orientation readings made  in  this fashion were thought   to be
reproducible to about $\pm 1^{\circ}$, an estimate  based on the scatter in
the $\rm H_{c2}(\theta)$ curves.  The exact  direction corresponding to the
samples' c-axis could not be determined {\em a priori}  and so was found by
a fit to the angular dependence after the data were entered into a computer
at  a later date.  The  approximate orientation of  the $\perp \hat{c}$ and
$\parallel  \hat{c}$   directions  was   estimated  at runtime   by  visual
inspection of the data, and this information  was used  for the measurement
of the temperature dependence of $\rm H_{c2}$ at constant orientation.

	After   a  complete angular   dependence of  $\rm  H_{c2}$ had been
measured at the lowest obtainable temperature, the sample was rotated until
the  applied  field was aligned  parallel to the carbon  (graphene) planes.
This  alignment was not  perfect for several reasons.   The first reason is
the inaccuracy in the $\theta$ reading. Secondly, the specimens' a-axis may
not have been exactly vertical due to small inaccuracy in  sample mounting.
The  misalignment  angle between  the   carbon  (graphene)  planes  and the
vertical will be called $\phi$ to  distinguish it  from $\theta$, the angle
in a horizontal plane between the graphite c-axis and the  applied magnetic
field.  (The angle $\phi$ is defined pictorially in Figure~\ref{misalign}.)
Misalignment due to this cause is estimated to be at  most 5$^{\circ}$, and
probably  less.  A third  cause of  imperfect alignment  is that the carbon
(graphene) planes of the $\rm C_4KHg$ specimens used  in these measurements
are  not exactly   flat due to   imperfect alignment of  the c-axis  of the
graphite host.  The amount  of misalignment of  the  c-axis  is called  the
mosaic spread.  Also, a small amount  of exfoliation (non-uniform spreading
of  the  graphite layers) tends  to  occur  during intercalation.   Rocking
curves     taken   using     elastic  neutron   $(00\ell)$     scans   (see
Chapter~\ref{samprep}) showed the mosaic spread  of a few $\rm C_4KHg$ HOPG
samples to  be 2$^{\circ}$ to 3$^{\circ}$.   Some  of the HOPG-based  GIC's
likely  had larger mosaic  spreads.  Mosaic  spreads   as  large as  8 or 9
degrees are not uncommon after  intercalation in ternary and metal-chloride
compounds.\cite{speck88}  The  flattest samples from  a   given  batch were
always chosen for the  critical field measurements, but nonetheless effects
due to   sample  warping  could   not   be  completely eliminated.  Overall
inaccuracy  due to all  these causes  could  perhaps have been as   much as
7$^{\circ}$,  and is   estimated  to have  averaged  about  3$^{\circ}$  in
practice.  Because these alignment and flatness problems tend to reduce the
observed value of $\rm H_{c2\perp\hat{c}}$, the anisotropy values quoted in
this work should be thought of as representing lower bounds.

	With the applied field parallel to  the carbon (graphene)  planes, a series of
field  sweeps  at  increasing  temperatures   was taken  to  determine $\rm
H_{c2\perp\hat{c}}(T)$.  The temperature was stabilized during these sweeps
by   rapid adjustment of  the various  valves in  the  pumping system.  The
temperature range in which the sample could  be stabilized  during the time
necessary to complete the field sweep was about $\pm 3$\%, according to the
thermometer.  Of course, the GIC's temperature varied more  slowly than the
thermometer's  due  to its  insulation from  the bath  by its encapsulation
tube.

	When the temperature series with  $\rm  \vec{H} \perp \hat{c}$  was
completed,  a   similar series with  $\rm  \vec{H}  \parallel  \hat{c}$ was
similarly performed.  These   measurements were  much  less  sensitive   to
misalignment, but were susceptible  (because of the   smaller size of  $\rm
H_{c2\parallel \hat{c}}$) to electrical noise that  generated stray fields.
When the  two  temperature series were  finished, two   more  sets  of $\rm
H_{c2}(\theta)$ at constant temperature data could be collected: one at 1.2
K,  which  corresponds to  the lowest obtainable temperature  of the $^4$He
bath with a roughing pump; and  one set at  0.9 K, which corresponds to the
lowest $^4$He temperature  obtainable by using a booster  pump in addition.
Reliable $\rm  H_{c2}(\theta)$ data  sets could  not be  obtained  at other
temperatures  because  of the impossibility  of stabilizing the temperature
for  the necessary period of  time  (2 to 3   hours) without an  electronic
feedback system.

	When  a  full set of   data had  been   obtained,  it was  analyzed
graphically.  $\rm H_{c2}$ was defined as the  intersection of a line drawn
tangent to the transition with  the level upper  portion of the  sweep, the
same definition as was used in Ref.~\cite{iye82}.  The  application of this
criterion for $\rm H_{c2}$  is illustrated in  Figure~\ref{hc2def}b.  Other
definitions of  $\rm H_{c2}$ were tried in  analyzing the  data; while they
slightly changed the results quantitatively, they had no effect on the {\em
shape} of any of the curves described here.  (See Figure~\ref{ninety-tangent}
for  a demonstration of  this.)  The  possibility of bias introduced in the
data analysis is discussed  further in  Section~\ref{extrinsic}.  The  data
were then typed in  by  hand  to a  VAX 11/750  computer   and  fit by  the
appropriate formulae. 

	Electrical    noise   was  not a problem   in   the  critical field
determination since   the signal-to-noise  ratio for the  field  sweeps was
about   1000.   However, the width  and  shape  of the transitions  changed
drastically as the angle $\theta$  was  varied.  This  lack of constancy in
the  shape  of  the transition made   a consistent definition of  the upper
critical field a tricky matter.  The different shapes of the transition are
illustrated  in Figure~\ref{transwidth}.    The change in  breadth  of  the
superconducting transition with angle  is common to polycrystalline layered
superconductors\cite{prober80},  and   is due    to the   contribution   of
misaligned grains as $\theta$ is varied.  Misaligned  grains have almost no
effect on the field  sweep when $\rm  \vec{H}$  is applied parallel to  the
c-axis, but will contribute noticeably when $\rm \vec{H}$ is applied in the
layer planes, perpendicular  to the c-axis.   The reason for  this behavior
comes from the form of $\rm H_{c2}(\theta)$, which  is strongly peaked near
$\rm \vec{H} \perp \hat{c}$.  In essence, near the $\rm
\parallel \hat{c}$ orientation, since $\rm   H_{c2}(\theta)$ is  fairly  flat, 
slightly misoriented grains have almost the same critical field as the bulk
of the sample.     On  the  other hand,  near   the  $\rm  \perp   \hat{c}$
orientation, where   $\rm    H_{c2}(\theta)$ is  strongly  angle-dependent,
slightly misoriented grains have much lower critical fields  than the bulk,
and  thus contribute to the  foot of the transition, making  it appear much
broader than it would in a perfect crystal.

        To some extent, the greater breadth of the $\rm \vec{H} \perp
\hat{c}$ transition may be an intrinsic effect.  The reason is that the
anisotropy of $\rm H_{c1}$ in a superconductor described by the anisotropic
Ginzburg-Landau theory is  expected to be  approximately the reciprocal  of
the $\rm H_{c2}$  anisotropy.\cite{lawrence71}  That is, theoretically $\rm
H_{c1}$ should be lower in the  direction where $\rm  H_{c2}$ is higher, so
that the transition should be  broader in  the high-$\rm H_{c2}$ direction.
It is not clear whether the $\rm H_{c1}$ anisotropy is also contributing to
the orientation dependence of the transition width.

\begin{figure}
\setlength{\unitlength}{0.0125in}
\begin{picture}(324,335)(160,350)
\thicklines
\put(260,540){\vector( 1,-2){ 80}}
\put(260,540){\vector( 1, 0){155}}
\put(320,680){\line( 0,-1){220}}
\put(320,460){\line(-6,-5){120}}
\put(160,600){\line( 3, 2){120}}
\put(280,680){\line( 1, 0){ 40}}
\put(320,680){\line(-3,-2){120}}
\put(200,600){\line( 1, 0){  1}}
\put(160,360){\framebox(40,240){}}
\put (430,540) {\makebox(0,0) [lb] {\raisebox{0pt}[0pt][0pt]{\twltt $\hat{c}$}}}
\put (370,460) {\makebox(0,0) [lb] {\raisebox{0pt}[0pt][0pt]{\twltt $\theta$}}}
\put (360,380) {\makebox(0,0) [lb] {\raisebox{0pt}[0pt][0pt]{\twltt $\vec{H}$}}}
\put (345,645) {\makebox(0,0) [lb] {\raisebox{0pt}[0pt][0pt]{\twltt Graphite Planes}}}
\end{picture}
%\vspace{4in}
%\beginpicture
%\setcoordinatesystem units <4in,5in>
%\setplotarea x from 0 to 1, y from 0 to 1
%\axis bottom ticks withvalues $\rm H_{c2}$ / at 0.6 / /
%\axis left label {\lines {inductive\cr voltage\cr}} /
%\put {0} [tl] at 0 -0.02
%\put {S} at  0.1 0.1
%\put {N} [r] at 0.9 1.0
%\put {H} [t] at 0.9 -0.02
%\linethickness=1pt
%\setdashes
%\putrule from 0.6 0 to 0.6 1
%\putrule from 0.3 0.9 to 0.9 0.9
%\setsolid
%\endpicture
\caption[Definition of the angle $\theta$ for $\rm H_{c2}(\theta)$ measurements
.]{a) Definition of  the  angle $\theta$,  the   angle  between the applied
magnetic field and the  graphite  c-axis.  This angle is  the complement to
that  usually used   in the thin-film     superconductivity literature, but
corresponds to customary usage in the GIC literature.  b) A  sketch showing
how $\rm H_{c2}$  is determined graphically  from raw susceptibility versus
magnetic field  data.    Note     the similarity    of  this    trace    to
Figure~\ref{fig:meiss}b).}
\label{hc2def}
\end{figure}

\begin{figure}
\vspace{8in}
\caption[Typical field sweeps used to measure the critical field.]{a) 
Superconducting transitions with  the magnetic field applied  parallel  and
perpendicular to the graphite  c-axis for a  typical $\rm C_4KHg$ sample.
Notice how much broader the transition is in the $\rm \vec{H}\perp \hat{c}$
case.  b) Similar data from Iye and Tanuma, Ref.\cite{iye82}, Figure 2.}
\label{transwidth}
\end{figure}

	There  are   several ways in  which  the critical field experiments
described  in this  chapter    could  have  been  improved.   Firstly,    a
servo-controlled gearing  system for  rotating   the sample  would probably
reduce   the errors  in  the $\rm  H_{c2}(\theta)$  measurements, and would
certainly make these measurements easier to perform.  Probes were available
that have gearing systems for rotating the sample around a horizontal axis,
but not for rotating the sample about a  vertical axis, as was necessary in
this experiment.  Secondly, an electronic temperature controller would have
made it possible to obtain $\rm H_{c2}(\theta)$  scans at much more closely
spaced intervals below $\rm  T_c$, and thus would have  allowed taking many
points in an anisotropy  versus temperature curve.  Thirdly, performing the
data collection with a computer rather than a chart recorder  would greatly
reduce the labor involved in reducing  the data, and also would  allow more
sophisticated real-time  data analysis.  Lastly,  if  a more powerful x-ray
apparatus were  available for doing  diffraction on samples in glass tubes,
it would be possible to measure the mosaic spread of the crystals directly.
This measurement would eliminate some of the  uncertainty in the fitting of
$\rm  H_{c2}(\theta)$ described  in  the next  section.   All in all, it is
expected that  improvement of the  instrumentation would have a real impact
on  the  $\rm H_{c2}(T)$   experiment,  but  would not   affect   the  $\rm
H_{c2}(\theta)$ data  much because  of  the limitations due to  crystalline
quality and misalignment.
