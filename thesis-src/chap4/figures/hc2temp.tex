\documentstyle[12pt]{report}
\input{/usr/local/lib/tex/macros/prepictex.tex}
\input{/usr/local/lib/tex/macros/pictex.tex}
\input{/usr/local/lib/tex/macros/postpictex.tex}
\newdimen\xposition
\newdimen\yposition
\newdimen\dxposition
\newdimen\crossbarlength
\def\putxerrorbar at #1 #2 with fuzz #3 {%
\xposition=\Xdistance{#1}
\yposition=\Ydistance{#2}
\dxposition=\Xdistance{#3}
\setdimensionmode
%\put {.} at {\xposition} {\yposition} %
%horizontal line
\dimen0 = \xposition 
  \advance \dimen0 by -\dxposition 
\dimen2 = \xposition 
  \advance \dimen2 by \dxposition 
\putrule from {\dimen0} {\yposition} to {\dimen2} {\yposition}
%endcaps
\dimen4 = \yposition
  \advance \dimen4 by -.5\crossbarlength
\dimen6 = \yposition
  \advance \dimen6 by .5\crossbarlength
\putrule from {\dimen0} {\dimen4} to {\dimen0} {\dimen6}
\putrule from {\dimen2} {\dimen4} to {\dimen2} {\dimen6}
\setcoordinatemode}


\newdimen\xposition
\newdimen\yposition
\newdimen\dyposition
\newdimen\crossbarlength
\def\putyerrorbar at #1 #2 with fuzz #3 {%
\xposition=\Xdistance{#1}
\yposition=\Ydistance{#2}
\dyposition=\Ydistance{#3}
\setdimensionmode
%\put {.} at {\xposition} {\yposition} %
%vertical line
\dimen0 = \yposition 
  \advance \dimen0 by -\dyposition 
\dimen2 = \yposition 
  \advance \dimen2 by \dyposition 
\putrule from {\xposition} {\dimen0} to {\xposition} {\dimen2}
%endcaps
\dimen4 = \xposition
  \advance \dimen4 by -.5\crossbarlength
\dimen6 = \xposition
  \advance \dimen6 by .5\crossbarlength
\putrule from {\dimen4} {\dimen0} to {\dimen6} {\dimen0}
\putrule from {\dimen4} {\dimen2} to {\dimen6} {\dimen2}
\setcoordinatemode}


\pagestyle{empty}
\begin{document}
\begin{figure}
\label{hc2temp}
\beginpicture
%Figure~\ref{hc2temp}a
\crossbarlength=5pt
\setcoordinatesystem units <100mm,0.1333mm>
\setplotarea x from 0 to 1, y from 0 to 600
\axis bottom label {Reduced temperature} ticks 
	numbered from 0 to 1 by 0.2
	unlabeled short quantity 11 /
\axis left label {\lines {$\rm H_{c2}$\cr (gauss)\cr}} ticks
	numbered from 0 to 600 by 100
	unlabeled short quantity 13 /
\put {a)} at 0.8 500
\put {$\rm \perp \hat{c}$} at 0.2 380
\put {$\rm \parallel \hat{c}$} at 0.2 80
\putyerrorbar at 0.397248 244.3 with fuzz 23.6 %13c perp. c
\putxerrorbar at 0.397248 244.3 with fuzz 0.08 %40 mK uncertainty for T_c=0.95
%\putyerrorbar at with fuzz 1.5 %8b par. c
%\multiput {} at "/users/alison/hctdata/stI/13c/redtpertang.13c"
\multiput {$\circ$} at "redtpertang.13c"
%\multiput {} at "/users/alison/hctdata/stI/iye/iyec4khg_per.redt"
\multiput {$\diamond$} at "iyec4khg_per.redt"
%\multiput {} at "/users/alison/hctdata/stI/13c/redtlpertangtheor.13c"
\multiput {.} at "redtlpertangtheor.13c"
%\multiput {} at "/users/alison/hctdata/stI/iye/ltheorc4khg_per.redt"
\multiput {.} at "ltheorc4khg_per.redt"
\multiput {$\bullet$} at "redtpartang.13c"
\multiput {$\times$} at "iyec4khg_par.redt"
\multiput {.} at "redtlpartangtheor.13c"
\multiput {.} at "ltheorc4khg_par.redt"
%Figure~\ref{hc2temp}b
\setcoordinatesystem units <100mm,0.1333mm> point at 0 750
\setplotarea x from 0 to 1, y from 0 to 600
\axis bottom label {Reduced Temperature} ticks 
	numbered from 0 to 1 by 0.2
	unlabeled short quantity 11 /
\axis left label {\lines {$\rm H_{c2}$\cr (gauss)\cr}} ticks
	numbered from 0 to 600 by 100
	unlabeled short quantity 13 /
\put {b)} at 0.8 500
\put {$\rm \perp \hat{c}$} at 0.3 400
\put {$\rm \parallel \hat{c}$} at 0.3 120
%\putyerrorbar at 0.287582 525.5 with fuzz 31.72 %7a perp. c
\putxerrorbar at 0.311184 538.7 with fuzz 0.06 %40 mK error for 1.54 K Tc
\putxerrorbar at 0.854605 116.199997 with fuzz 0.012 %10 mK error for 1.54 K Tc
%\putyerrorbar at with fuzz 5.26 %7a par. c
\putyerrorbar at 0.311184 538.7 with fuzz 35.36 %8b perp. c
%\putyerrorbar at with fuzz 2.8 %8b par. c
%\multiput {} at "/users/alison/hctdata/stI/7a/redtshiftsumper.7a"
\multiput {$\circ$} at "redtshiftsumper.7a"
%\multiput {} at /users/alison/hctdata/stI/8b/tangent/pertang_8b.redt
\multiput {$\diamond$} at "pertang_8b.redt"
%\multiput {} at /users/alison/hctdata/stI/7a/redtshiftsumpar.7a
\multiput {$\bullet$} at "redtshiftsumpar.7a"
%\multiput {} at "/users/alison/hctdata/stI/8b/tangent/partang_8b.redt"
\multiput {$\times$} at "partang_8b.redt"
\multiput {.} at "redtheorshiftper.7a"
\multiput {.} at "redtheorshiftpar.7a"
%\multiput {.} at "lpartangtheor_8b.redt"
%\multiput {.} at "lpertangtheor_8b.redt"
\endpicture

%\caption[Critical field $\rm H_{c2}$ as a function of reduced temperature
%for $\rm  C_4KHg$.]{Critical field $\rm  H_{c2}$  as a function  of reduced
%temperature for $\rm C_4KHg$.  Dotted curves are least-squares line fits to
%the data.   Fit parameters  are given in Table~\ref{linearparams}.  a) Data
%for a $\rm C_4KHg$ with $\rm T_c$  = 0.95 K:  ($\circ$), $\rm \vec{H} \perp
%\hat{c}$.   ($\bullet$), $\rm  \vec{H} \parallel  \hat{c}$.
%Data  for    a $\rm  T_c$ =    0.73   K sample  from  Ref.~\cite{tanuma81}:
%($\diamond$), $\rm \vec{H} \perp \hat{c}$.  ($\times$), $\rm \vec{H}
%\parallel \hat{c}$.  b) Data for two $\rm C_4KHg$-GIC's with $\rm  T_c \,
%\approx$ 1.5 K.  $\rm T_c$  =  1.53 K sample:  ($\circ$), $\rm \vec{H}
%\perp \hat{c}$.  ($\bullet$), $\rm \vec{H} \parallel \hat{c}$. $\rm T_c$ = 1.54
%K sample: ($\diamond$), $\rm \vec{H} \perp \hat{c}$.  ($\times$), $\rm \vec{H} \parallel \hat{c}$.}
\end{figure}
\end{document}
