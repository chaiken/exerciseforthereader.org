\documentstyle[12pt]{report}
\headheight 8pt
\input{/usr/local/lib/tex/macros/prepictex.tex}
\input{/usr/local/lib/tex/macros/pictex.tex}
\input{/usr/local/lib/tex/macros/postpictex.tex}
\pagestyle{empty}
\begin{document}
\begin{figure}
\label{TINKvsAGL_noHc}
\beginpicture
\setcoordinatesystem units <0.8333mm,0.2667mm> 
\setplotarea x from 0 to 120, y from 0 to 300
\axis bottom label {$\theta$} ticks 
 numbered from 0 to 120 by 15
 unlabeled short quantity 25 /
\axis left label {\lines {$\rm H_{c2}$\cr (gauss)\cr}} ticks
 numbered from 0 to 300 by 50 /
\put {$\rm H \perp \hat{c}$} at 90 50
%\multiput {$\bullet$} at /users/alison/angdata/C_4KHg/8b/medt/folded.8b.strip
\multiput {$\bullet$} at "folded.8b.strip"
%\multiput {$\circ$} at /users/alison/angdata/C_4KHg/8b/medt/ldfit.06
\multiput {$\circ$} at "ldfit.06"
%\multiput {$\diamond$} at /users/alison/angdata/C_4KHg/8b/medt/tinkh33a7.out
\multiput {$\diamond$} at "tinkh33a7.out"
\endpicture
\caption[Comparison of Tinkham's formula and AGL theory fits with no type
I superconductivity to $\rm H_{c2}(\theta)$  data on a  $\rm T_c$  =  1.5 K
$\rm C_4KHg$-GIC.]{Comparison of Tinkham's formula and AGL theory fits with
no type I superconductivity  to $\rm H_{c2}(\theta)$ data on  a $\rm T_c$ =
1.5 K  $\rm C_4KHg$-GIC.  $\bullet$,  data at t  =  0.55.  $\circ$, AGL fit
with $\rm H_{c2}(0^{\circ})$  = 33 g, 1/$\epsilon$ =  5.5,  and  a residual
$\cal R$ =  1.25.  $\diamond$, TF fit with  $\rm H_{c2}(0^{\circ})$ = 33 g,
1/$\epsilon$ = 7, and $\cal R$ = 0.76.}
\end{figure}
\end{document}
