\documentstyle[12pt]{report}
\input{/usr/local/lib/tex/macros/prepictex.tex}
\input{/usr/local/lib/tex/macros/pictex.tex}
\input{/usr/local/lib/tex/macros/postpictex.tex}
\newdimen\xposition
\newdimen\yposition
\newdimen\dxposition
\newdimen\crossbarlength
\def\putxerrorbar at #1 #2 with fuzz #3 {%
\xposition=\Xdistance{#1}
\yposition=\Ydistance{#2}
\dxposition=\Xdistance{#3}
\setdimensionmode
%\put {.} at {\xposition} {\yposition} %
%horizontal line
\dimen0 = \xposition 
  \advance \dimen0 by -\dxposition 
\dimen2 = \xposition 
  \advance \dimen2 by \dxposition 
\putrule from {\dimen0} {\yposition} to {\dimen2} {\yposition}
%endcaps
\dimen4 = \yposition
  \advance \dimen4 by -.5\crossbarlength
\dimen6 = \yposition
  \advance \dimen6 by .5\crossbarlength
\putrule from {\dimen0} {\dimen4} to {\dimen0} {\dimen6}
\putrule from {\dimen2} {\dimen4} to {\dimen2} {\dimen6}
\setcoordinatemode}


\newdimen\xposition
\newdimen\yposition
\newdimen\dyposition
\newdimen\crossbarlength
\def\putyerrorbar at #1 #2 with fuzz #3 {%
\xposition=\Xdistance{#1}
\yposition=\Ydistance{#2}
\dyposition=\Ydistance{#3}
\setdimensionmode
%\put {.} at {\xposition} {\yposition} %
%vertical line
\dimen0 = \yposition 
  \advance \dimen0 by -\dyposition 
\dimen2 = \yposition 
  \advance \dimen2 by \dyposition 
\putrule from {\xposition} {\dimen0} to {\xposition} {\dimen2}
%endcaps
\dimen4 = \xposition
  \advance \dimen4 by -.5\crossbarlength
\dimen6 = \xposition
  \advance \dimen6 by .5\crossbarlength
\putrule from {\dimen4} {\dimen0} to {\dimen6} {\dimen0}
\putrule from {\dimen4} {\dimen2} to {\dimen6} {\dimen2}
\setcoordinatemode}


\pagestyle{empty}
\begin{document}
\begin{figure}
\label{whhfit}
\beginpicture
\crossbarlength=5pt
\setcoordinatesystem units <100mm,0.1164mm> 
\setplotarea x from 0 to 1, y from 0 to 600
\axis bottom label {Reduced Temperature} ticks 
	numbered from 0 to 1 by 0.2
	unlabeled short quantity 11 /
\axis left label {\lines {$\rm H_{c2, \perp \hat{c}}$\cr (gauss)\cr}} ticks
	numbered from 0 to 600 by 100
	unlabeled short quantity 13 /
\put {a)} at 0.9 500
\putxerrorbar at 0.307237 501.200012  with fuzz 0.06 %40 mK error for 1.54 K Tc
\putxerrorbar at 0.854605 116.199997 with fuzz 0.012 %10 mK error for 1.54 K Tc
\putyerrorbar at 0.307237 501.200012 with fuzz 35.36 %8b perp. c
%\multiput {} at /users/alison/hctdata/stI/8b/tangent/pertang_8b.redt
\multiput {$\bullet$} at "pertang_8b.redt"
\multiput {.} at "lpertangtheor_8b.redt"
%\multiput  {$\circ$} at "/users/alison/hctdata/stI/8b/tangent/newpertang.fit"
\multiput {$\circ$} at "newpertang.fit"
\setcoordinatesystem units <100mm,0.70mm> point at 0 120
\setplotarea x from 0 to 1, y from 0 to 100
\axis bottom label {Reduced Temperature} ticks 
	numbered from 0 to 1 by 0.2
	unlabeled short quantity 11 /
\axis left label {\lines {$\rm H_{c2, \parallel \hat{c}}$\cr (gauss)\cr}} ticks
	numbered from 0 to 100 by 20
	unlabeled short quantity 11 /
\put {b)} at 0.9 80
\putxerrorbar at 0.291975 60.50 with fuzz 0.06 %40 mK error for 1.54 K Tc
\putxerrorbar at 0.745679 22.2 with fuzz 0.012 %10 mK error for 1.54 K Tc
\putyerrorbar at 0.291975 60.50 with fuzz 2.80 %8b par. c
%\multiput {} at "/users/alison/hctdata/stI/8b/tangent/partang_8b.redt"
\multiput {$\bullet$} at "partang_8b.redt"
\multiput {.} at "lpartangtheor_8b.redt"
\multiput {$\circ$} at "newpartang.fit"
\endpicture
\caption[Comparison of WHH and linear fits to $\rm H_{c2}(T)$ data taken on a $
\rm T_c$ = 1.54 K sample.]{Comparison of WHH and linear fits to $\rm H_{c2}(T)$
data taken  on a $\rm T_c$  = 1.54 K sample.  a) ($\bullet$), data for $\rm
\vec{H} \perp \hat{c}$.  (.), linear fit with $\rm H_{c2}(0)$ = 748 g, $\rm
T_c$  =  1.52 K,  and $\cal R$   = 6.9e-3.  ($\circ$),   WHH fit  with $\rm
H_{c2}(0)$ =  518 gauss,  $\rm T_c$  = 1.53 K,  and  $\cal R$ =  1.6e-2. b)
($\bullet$), data with $\rm  \vec{H}  \parallel \hat{c}$.  (.),  linear fit
with  $\rm H_{c2}(0)$  = 85.8 gauss,  $\rm  T_c$ = 1.62  K, and  $\cal R$ =
1.6e-3.  ($\circ$), WHH fit with $\rm H_{c2}(0)$ = 59.8 gauss, $\rm T_c$ =
1.63 K, and $\cal R$ = 1.2e-2.}
\end{figure}
\end{document}

%\multiput {.} at "redtheorshiftpar.7a"
%\multiput {} at /users/alison/hctdata/stI/7a/redtshiftsumpar.7a
%\multiput {$\bullet$} at "redtshiftsumpar.7a"
%\multiput {.} at "redtheorshiftpar.7a"
%\multiput {} at "/users/alison/hctdata/stI/7a/redtshiftsumper.7a"
%\multiput {$\circ$} at "redtshiftsumper.7a"
%\multiput {.} at "redtheorshiftper.7a"
%\putyerrorbar at 0.287582 525.5 with fuzz 31.72 %7a perp. c
