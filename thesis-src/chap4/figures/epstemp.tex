\documentstyle[12pt]{report}
\input{/usr/local/lib/tex/macros/prepictex.tex}
\input{/usr/local/lib/tex/macros/pictex.tex}
\input{/usr/local/lib/tex/macros/postpictex.tex}
\pagestyle{empty}
\begin{document}
\begin{figure}
\beginpicture
\setcoordinatesystem units <0.8333mm,10mm>
\setplotarea x from 0 to 120, y from 0 to 10
\axis bottom label {$\theta$} ticks 
	numbered from 0 to 120 by 15
	unlabeled short quantity 25 /
\axis left label {$\rm H_{c2}(\theta)/H_{c2}(0^{\circ})$} ticks
	numbered from 0 to 10 by 2
	unlabeled short quantity 11 /
\put {$\rm H \perp \hat{c}$} at 90 2
%\multiput {$\circ$} at "/users/alison/angdata/scaled/lotang.8b.scaled
\multiput {$\circ$} at "lotang.8b.scaled"
%\multiput {$\bullet$} at /users/alison/angdata/scaled/folded.8b.scaled
\multiput {$\bullet$} at "folded.8b.scaled"
%\multiput {$\times$} at /users/alison/angdata/scaled/hitang.8b.scaled
\multiput {$\times$} at "hitang.8b.scaled"
\endpicture
\caption[Demonstration of temperature-dependent anisotropy parameter
$\epsilon$  in    $\rm C_4KHg$.]{Demonstration    of  temperature-dependent
anisotropy parameter  $\epsilon$   in $\rm  C_4KHg$.  1/$\epsilon$ is  $\rm
H_{c2}(90^{\circ})/H_{c2}(0^{\circ})$.  Data  is  for a $\rm  T_c$ = 1.54 K
$\rm   C_4KHg$  sample.  ($\circ$),   t  = 0.29.   ($\bullet$),  t  = 0.55.
($\times$), t = 0.76.  All $\rm  H_{c2}(0^{\circ})$ values  were determined
from the data, not the fits, so that this plot is model-independent.}
\label{epstemp}
\end{figure}
\end{document}
