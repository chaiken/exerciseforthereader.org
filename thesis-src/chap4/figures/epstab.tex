\documentstyle[12pt]{report}
\pagestyle{empty}
\begin{document}
\begin{table}
\caption[Comparison of the anisotropy parameter $\epsilon$ as obtained from
$\rm  H_{c2}(T)$   and  $\rm   H_{c2}(\theta)$  fits.]{Comparison    of the
anisotropy parameter  $\epsilon$ as  obtained from $\rm H_{c2}(T)$ and $\rm
H_{c2}(\theta)$ fits.  The $\rm H_{c2}(T)$ $\epsilon$ numbers were obtained
from the ratio  of  the    slopes.  The $\rm H_{c2}(\theta)$ numbers   were
obtained   from  fits  using  Eqns.~\ref{ldtheor}  (AGL) and  \ref{tinkham}
(Tinkham's  formula), with   type  I  superconductivity allowed  for  small
$\theta$.  In each case, the TF fits had lower  residuals than the AGL fits
(see Table~\ref{residsum}).   NA, TF  fit   not performed on this data;  ?,
information unavailable.}
\label{epsilontab}
\begin{center}
\begin{tabular}{|l|ccc|}
\hline
$\rm T_c$ (K) & 1/$\epsilon$ from $\rm H_{c2}(T)$ & 1/$\epsilon$, AGL $\rm
H_{c2}(\theta)$ & 1/$\epsilon$, TF $\rm H_{c2}(\theta)$  \\
\hline
& & & \\
0.72\cite{iye82} & ? & 11.5 at t=0.55 & ? \\
& & & \\
0.73\cite{iye82} & 9.7 & 11.3 at t=0.55 & NA \\
& & & \\
0.86\cite{iye82} & ? & 10.4 at t=0.47 & ? \\
& & & \\
0.95 & 8.9 & 10.0 at t=0.46 & NA \\
& & & \\
1.53 & 8.4 & 13.5 at t=0.29 & 14.8 at t=0.29 \\
& & 10 at t=0.76 & 7.7 at t= 0.76 \\
& & & \\
1.54 & 8.7 & 14.0 at t=0.29 & 16.0 at t=0.29\\
& & 15.5 at t=0.55  & 13.0 at t=0.55\\
& & 12.5 at t=0.76 & 12 at t=0.76\\
& & & \\
\hline
\end{tabular}
\end{center}
\end{table}
\end{document}
