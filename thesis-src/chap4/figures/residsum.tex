\documentstyle[12pt]{report}
\pagestyle{empty}
\begin{document}
\begin{table}
\label{residsum}
%\caption[Residuals for fits to $\rm H_{c2}(\theta)$ using the AGL formula
%and Tinkham's  formula, both  with and  without type I behavior.]{Residuals
%for  fits to  $\rm  H_{c2}(\theta)$  using the AGL   formula  and Tinkham's
%formula, both with and without type  I  behavior.  The residual index $\cal
%R$ is defined in Eqn.~\ref{residef}, the AGL formula is Eqn.~\ref{ldtheor},
%and  Tinkham's formula is  Eqn.~\ref{tinkham}.  Eqn.~\ref{typeItheta} shows
%how  each of these  formulae was modified to   account for  possible type I
%superconductivity.}
\begin{center}
\begin{tabular}{|l|cccccc|}
\hline
$\rm T_c$ & t$\rm \equiv T/T_c$ & Fit Type & $\cal R$ & $\rm H_{c2}(0^{\circ})$ (g)& 1/$\epsilon$ & $\rm H_c$ (g) \\
\hline
& & & & & & \\
1.54 &  0.76 & AGL &1.43 & 23.1 & 4.5 & 0 \\ 
& & & & & & \\
1.54 &  0.76 & AGL &1.11 & 14.5   & 12.5& 24.5 \\
& & & & & & \\
1.54 &  0.76 &  TF & 1.06 & 17.5   & 8.2 & 0 \\
& & & & & & \\
1.54 &  0.76 &  TF & 0.58 & 14.8   & 12& 24 \\
& & & & & & \\
\hline
& & & & & & \\
1.54 & 0.55 &AGL & 1.25 & 33 & 5.5 & 0 \\
& & & & & & \\
1.54 & 0.55 &AGL & 0.84 & 19 & 15.5  & 43 \\
& & & & & & \\
1.54 & 0.55 &TF  & 0.75 & 33 &  7  &  0 \\
& & & & & & \\
1.54 & 0.55 &TF  & 0.47 & 23 & 13  & 41 \\
& & & & & & \\
\hline
& & & & & & \\
1.54 & 0.29 &AGL & 1.18 & 47 & 9.5  & 0 \\
& & & & & & \\
1.54 & 0.29 &AGL & 0.39 & 35 & 14  & 65 \\
& & & & & & \\
1.54 & 0.29 &TF  & 0.83 & 50 & 9.5  & 0 \\
& & & & & & \\
1.54 & 0.29 &TF  & 0.26 & 35.8 & 16  & 65 \\
& & & & & & \\
\hline
\hline
& & & & & & \\
0.95 & 0.45 & AGL& 0.29 & 24 & 10 & 0 \\
& & & & & & \\
\hline
\hline
& & & & & & \\
0.73\cite{iye82} & 0.55 & AGL & 0.09 & 26 & 11.3 & 0 \\
& & & & & & \\
\hline
\end{tabular}
\end{center}
\end{table}
\end{document}
