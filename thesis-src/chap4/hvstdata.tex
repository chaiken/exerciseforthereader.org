\section{Experimental Results: Temperature Dependence of the Upper Critical Field}
\label{hvstdata}

%        The angular dependence of the upper critical field $\rm  H_{c2}$ at
%constant  temperature was  the   topic of  the previous   section.  In this
%section the focus is   the  conjugate  experiment, namely the   temperature
%dependence  of the critical field  at a fixed  angle.  

         Many of the experimental factors relevant  to the angular behavior
of the critical  field   will  also complicate the temperature  dependence.
Among  these are sample misalignment,  mosaic  spread,   and critical field
definition.   In addition   there are the    considerations of  temperature
measurement, stability, and equilibration. (See Section~\ref{electronics}.)
Nonetheless, the $\rm H_{c2}(T)$ results are a bit easier to interpret than
the $\rm H_{c2}(\theta)$ ones, if  only because the  functional form of the
data is  simpler.  The  approximately linear behavior   of the  temperature
dependence curves is  illustrated in  Figure~\ref{hc2temp},  which shows $\rm
H_{c2}(T)$ curves for four  different $\rm C_4KHg $  GIC's.  These data are
from  the  same samples whose  $\rm  H_{c2}(\theta)$   curves are shown  in
Figure~\ref{hc2theta}.

\begin{figure}
\vspace{15cm}
\caption[Critical field $\rm H_{c2}$ as a function of reduced temperature
for $\rm  C_4KHg$.]{Critical field $\rm  H_{c2}$  as a function  of reduced
temperature for $\rm C_4KHg$.  Dotted curves are least-squares line fits to
the data.   Fit parameters  are given in Table~\ref{linearparams}.  a) Data
for a $\rm C_4KHg$ with $\rm T_c$  = 0.95 K:  ($\circ$), $\rm \vec{H} \perp
\hat{c}$.   ($\bullet$), $\rm  \vec{H} \parallel  \hat{c}$
Data   for  a $\rm   T_c$   =  0.73   K   sample from Ref.~\cite{tanuma81}:
($\diamond$),  $\rm  \vec{H}   \perp \hat{c}$.   ($\times$),   $\rm \vec{H}
\parallel \hat{c}$.  b) Data for two $\rm C_4KHg$-GIC's with $\rm  T_c \,
\approx$ 1.5 K.  $\rm T_c$  =  1.53 K sample:  ($\circ$), $\rm \vec{H}
\perp \hat{c}$.  ($\bullet$), $\rm \vec{H} \parallel \hat{c}$. $\rm T_c$ = 1.54
K sample: $\diamond$, $\rm \vec{H} \perp \hat{c}$.   $\times$, $\rm \vec{H}
\parallel \hat{c}$.}
\label{hc2temp}
\end{figure}


        The fits in Figure~\ref{hc2temp} are all to the simple linear form\\

\begin{equation}
\label{eqn:linear}
\rm H_{c2}(t) \; = \; H_{c2}(0) \: (1 \, - \, t)
\end{equation}

\noindent where     t   is      the   reduced    temperature,   T/$\rm    T_c$.   The
least-squares-determined values of $\rm H_{c2}(0)$  and $\rm T_c$ for these
curves are given in Table~\ref{linearparams}.  The quality-of-fit parameter
$\cal R$ is calculated as in Eqn.~\ref{residef}, only here the error at the
$i$th point  $\sigma_i$ is  taken to be the same  as the experimental field
value, that is:\\

\[ \rm \sigma_i \; = \; H_{c2,i}^{exp} \; .
\]

\noindent  Of course the errors should really be smaller in magnitude than the 
data, but their  fractional size is thought to  be  roughly independent  of
temperature, the independent variable.  Letting the estimated error  be the
same  as the  data  is a   convenient  way of  implementing  this  standard
assumption,   which is called  statistical weighting.\cite{bevington69} The
field error bars shown on the plot are twice  the mean error per data point
in total length, where the mean error $\sigma$ is given by\\

\[ \sigma^2 \; = \; \sum_i \rm (H_i^{exp} \, - \, H_i^{theor})^2 / \nu
\]

\noindent where $\nu$ is the number of degrees of freedom.  This is a
standard way of defining error bars\cite{pugh66} which includes the effects
only of random errors, not systematic ones. The temperature  error bars are
based on an uncertainty of 40 mK at the low end, and 10 mK at the high end,
for reasons discussed in Section~\ref{electronics}.

\begin{table}
\caption[Parameters obtained from least-squares fits of a line to $\rm
C_4KHg $ $\rm H_{c2}(T)$ data.]{Parameters obtained from least-squares fits
of   Eqn.~\ref{eqn:linear}  to  $\rm  C_4KHg  $ $\rm  H_{c2}(T)$ data.  The
individual specimens  are labeled by the  value of  $\rm T_c$ obtained from
the zero-field  sweeps.  The other parameters,  including the ``fit''  $\rm
T_c$, are obtained from the least-squares linear fits to the critical field
data  that are  shown in Figure~\ref{hc2temp}.   The residual  $\cal  R$ is
calculated using Eqn.~\ref{residef}. NR denotes that the parameter was not
reported in the cited reference.}
\label{linearparams}
\begin{center}
\begin{tabular}{|c|ccccc|}
\hline
Exp. $\rm T_c$ (K) & $\perp$ or $\parallel$ & $\rm H_{c2}(0)$ (g) & Fit $\rm T_c$ (K) & $\rm \left| dH_{c2}/dT \right|$ (g/K) & {\cal R} $\times$ 100\\
\hline
& & & & & \\
0.72\cite{iye82} & $\perp$ & 715 & NR & 993 &  NR \\
                 & $\parallel$ & 62.4 & NR & 86.7 & NR\\
& & & & & \\
0.73\cite{iye82} & $\perp$ & 570 & 0.73 & 786 & 4.4 \\
                 & $\parallel$ & 58.5 & 0.73& 80.3 & 0.01 \\
& & & & & \\
0.86\cite{iye82} & $\perp$ & 654 & NR & 760.5 & NR \\
                 & $\parallel$ & 49.2 & NR & 57.2 & NR \\
& & & & & \\
0.95            & $\perp$ & 407 & 1.09 & 374 & 0.79 \\
                 & $\parallel$ & 45.9 & 1.04 & 44.0 & 0.13 \\
& & & & & \\
1.53 &           $\perp$   & 749 & 1.53 & 488 & 7.88 \\
                & $\parallel$ & 89.7 & 1.65 & 54.4 & 0.63 \\
& & & & & \\
1.54 &           $\perp$   & 748 & 1.52 & 493 & 0.69 \\
                 & $\parallel$ & 85.8 & 1.62 & 53.0 & 0.16 \\
& & & & & \\
\hline
\end{tabular}
\end{center}
\end{table}

        In  general,  the    data  in Figure~\ref{hc2temp}  are  quite well
described by Eqn.~\ref{eqn:linear}.    Ginzburg-Landau  models  predict   a
linear temperature-dependence of $\rm H_{c2}$ in  their region of validity,
so   this  result   is   not   surprising,   especially    since  Iye   and
Tanuma\cite{iye82} found  linear  temperature  dependence   for their  $\rm
C_4KHg$ samples.  One feature  of Table~\ref{linearparams} that merits some
discussion is that the $\rm T_c$ determined from the best  linear fit is as
much as 10\%  higher than that  determined from the zero-field  temperature
sweeps.  This disparity between the  different ways of measuring  $\rm T_c$
is a fairly   common occurrence  in  superconductivity  which is  generally
caused by curvature of the  $\rm  H_{c2}(T)$  data in the  region near $\rm
T_c$.\cite{orlando79}   \ Hohenberg   and Werthamer\cite{hohenberg67} found
that Fermi-surface anisotropy can cause positive  curvature of the critical
field near $\rm T_c$,  but here  negative curvature is  needed in order  to
explain the high extrapolated values of $\rm T_c$.

        For  the higher-$\rm  T_c$ specimens, the  $\rm T_c$  numbers found
from the $\rm \vec{H} \parallel \hat{c}$ fits  are  higher than those found
from the $\rm  \vec{H} \perp  \hat{c}$ fits.   Because of the discussion of
type I superconductivity  in the Section~\ref{typeI},  one  might wonder if
the $\rm  H_{c2, \parallel \hat{c}}(T)$ data  in these samples are actually
quadratic,   since a quadratic temperature   dependence   is  expected   for
measurements of the thermodynamic  critical   field.   A moment's   thought
indicates that  a  linear fit to a quadratic   function will have  a higher
intercept on the x-axis than a quadratic fit will.  (Se Figure~\ref{hparvst}.)  Therefore, if  the data
are truly quadratic, it is easy to see why a linear fit will give  a falsely
high $\rm T_c$.  The temperature dependence  data for the $\rm T_c \approx$
1.5 K  GIC's for  $\rm \vec{H} \parallel\hat{c}$  are  shown on an expanded
scale in Figure~\ref{hparvst}, along with the best quadratic and linear fits.
This   plot is  quite   similar to    Figure~\ref{hcfig}, where the  presumed
thermodynamic  critical  field  values  were   obtained  from fits  to $\rm
H_{c2}(\theta)$.  Here  the data  were  obtained  directly,  using
field sweeps at constant $\theta$ as a function of temperature.

\begin{figure}
\vspace{7.5in}
\caption[Critical fields with $\rm \vec{H} \parallel \hat{c}$ for  $\rm T_c
\approx 1.5$ K $\rm C_4KHg$ samples.]{Critical fields with $\rm \vec{H}
\parallel \hat{c}$ for $\rm T_c
\approx 1.5$ K $\rm C_4KHg$ samples.  $\Uparrow$ marks the value of $\rm T_c$
found using a zero-field temperature sweep.  a) ($\bullet$),  data for a $\rm
T_c$  = 1.53 K  sample; ($\diamond$),   a linear  fit to the  data  with $\rm
H_{c2}(0)$ = 89.7 Oe, $\rm T_c$ = 1.65 K and $\cal  R$ = 6.25e-3; (.), a
quadratic fit to the data with $\rm H_{c2}(0)$ = 64.0 Oe, $\rm T_c$ = 1.55 K
and $\cal R$ = 1.2e-2.  b) ($\bullet$), data for a $\rm T_c$ = 1.54 K sample;
($\diamond$), a linear fit to the data  with  $\rm  H_{c2}(0)$ = 85.8 Oe, $\rm
T_c$ = 1.62 K and $\cal R$ = 1.62e-3; (.), a quadratic  fit to the data
with $\rm H_{c2}(0)$ = 62.8 Oe, $\rm T_c$ = 1.51 K and $\cal R$ = 4.7e-2.}
\label{hparvst}
\end{figure}

        Figure~\ref{hparvst}  shows that using  a quadratic function to fit
the $\rm \vec{H} \parallel \hat{c}$ data does improve the agreement between
the   fit  $\rm  T_c$  and    that  obtained  from   a  zero-field   sweep.
Unfortunately, the quadratic function does not describe the data as well as
a  linear  function: for one sample the  residual for the quadratic  fit is
twice that of the linear fit, and for the other it  is three  times that of
the  linear fit.   Consultation of  standard tables\cite{bevington69} shows
that the  difference between these fits has   only about an even  chance  of
being  significant (for  about 15 data  points).   Therefore  the only safe
statement is that the question of type I superconductivity  in $\rm C_4KHg$
is still unresolved, since the fits to $\rm H_{c2, \parallel  \hat{c}}$ are
inconclusive.  As was mentioned in relation  to Figure~\ref{hcfig}, in $\rm
C_8K$ both  $\rm  H_c(T)$ and  $\rm H_{c2,\perp \hat{c}}(T)$  show positive
curvature,\cite{koike80} so perhaps in $\rm  C_4KHg$  it is not implausible
for both critical fields to show a linear temperature dependence.

         Some  lower-$\rm T_c$ samples,   such as the $\rm T_c$   = 0.95  K
specimen whose data  is shown in  Figure~\ref{hc2temp} (and others whose data
are not displayed) showed $\rm T_c$'s which were higher  than the zero-field
values  for both  field  orientations.     This  fact  suggests  that  some
experimental error is causing the discrepancies.  Because the  $\rm \vec{H}
\parallel   \hat{c}$   measurements  were  usually taken    going   down in
temperature,  and those for $\rm \vec{H}  \perp \hat{c}$   were generally taken
going up in temperature  (see Section~\ref{procedure}), it appears unlikely
that systematic temperature measurement problems are responsible.   At this
time the high values of $\rm  T_c$ obtained  from the $\rm H_{c2, \parallel
\hat{c}}$ fits  remain unexplained, but it is felt that some type of
measurement problem is the most likely cause.

        Another unexpected feature of  the  data for both  orientations  is
that  the linearity of $\rm H_{c2}(t)$   persists to unusually  low reduced
temperatures.   For  typical type II superconductors, Eqn.~\ref{eqn:linear}
holds only  for about 0.6 $<$ t  $<$  1.0, below   which  a saturation  in $\rm
H_{c2}(t)$ becomes noticeable.  This saturation is described quantitatively
by    the widely accepted   theory  of   Werthamer,  Helfand, and Hohenberg
(WHH)\cite{werthamer66} which   has been successful   in describing  a wide
variety of superconducting materials\cite{orlando79,fetter69}.  Anisotropic
materials which are described by the AGL model for their angular dependence
are expected to have a temperature dependence at constant angle describable
by the WHH theory.  The basic  equation of this  theory is one developed by
Maki and deGennes\cite{saintjames69,helfand66}:

\begin{equation}
\rm \ln \: t \; = \; \psi\left(\frac{1}{2}\right) \; - \; \psi\left(
\frac{1}{2} \; + \; \frac{D e H_{c2}}{2 \pi k_B T_c}  \right)
\label{makidegennes}
\end{equation}

\noindent where t is the reduced temperature, D is the diffusivity, and 
the digamma function $\rm \psi(x)$ is related to the gamma function by:

\[ \rm \psi (x) \; \equiv \; \frac{d}{dx} \ln \Gamma(x) \; .
\]

This equation is strictly applicable only to dirty superconductors or clean
superconductors near $\rm T_c$, but the  theory has  been extended to lower
temperatures       for       clean  superconductors       by    Helfand and
Werthamer\cite{helfand66}.    The
contribution of WHH\cite{werthamer66} was  to further extend  the theory to
incorporate effects due  to  Pauli  paramagnetism and spin-orbit  coupling.
The meaning and possible importance  of  these embellishments is  discussed
below in Section~\ref{spin-orbit}.  Simple two-parameter ($\rm T_c$, $\rm
\left.  dH_{c2}/dT \right|_{T_c}$) WHH  fits to  the $\rm C_4KHg$  data are
shown in  Figure~\ref{whhfit}, where they are compared  to the  best linear
fits, the same linear fits shown in Figure~\ref{hc2temp}.
        
\begin{figure}
\vspace{16cm}
\caption[Comparison of WHH and linear fits to $\rm H_{c2}(T)$ data taken on a $
\rm T_c$ = 1.54 K sample.]{Comparison of WHH and linear fits to $\rm H_{c2}(T)$
data taken  on a $\rm T_c$  = 1.54 K sample.  a) ($\bullet$), data for $\rm
\vec{H} \perp \hat{c}$.  (.), linear fit with $\rm H_{c2}(0)$ = 748 Oe, $\rm
T_c$  =  1.52 K,  and $\cal R$   = 6.9e-3.  ($\circ$),   WHH fit  with $\rm
H_{c2}(0)$ =  518 Oe,  $\rm T_c$  = 1.53 K,  and  $\cal R$ =  1.6e-2. b)
($\bullet$), data with $\rm  \vec{H}  \parallel \hat{c}$.  (.),  linear fit
with  $\rm H_{c2}(0)$  = 85.8 Oe,  $\rm  T_c$ = 1.62  K, and  $\cal R$ =
1.6e-3.  ($\circ$), WHH fit with $\rm H_{c2}(0)$ = 59.76 Oe, $\rm T_c$ =
1.63 K, and $\cal R$ = 1.2e-2.}
\label{whhfit}
\end{figure}

        As    might  be    expected,  the   Maki-deGennes   equation  gives
approximately linear behavior near $\rm T_c$.  Therefore the linear and WHH
fits  are  in   good  agreement   just below  $\rm   T_c$.    However,  the
low-temperature extension\cite{helfand66}   of the Maki-deGennes  formalism
produces the result

\[ \rm H_{c2}(0) \: \approx \: 0.7 \left.  \frac{dH_{c2}}{dT} \right|_{T_c}  T_c
\; .
\]

\noindent This formula is the quantitative expression of the saturation shown  in the
WHH curves of Figure~\ref{whhfit}, saturation which the data  does not appear
to exhibit.  The better agreement of the  linear fits than the WHH  fits is
confirmed by the residual  indices,  which are  significantly lower for the
linear fits.  

        A  more  impressive    demonstration of   linearity  is  shown   in
Figure~\ref{sumtemp}, which  gathers together the  data of 5 samples for both
orientations.  Here   the data are plotted  in  dimensionless units on both
axes; the reduced field $\rm h^*$ is defined by

\[ \rm h^* \; = \;  H_{c2}/ \left( \left. \frac{dH_{c2}}{dT}\right|_{T_c} T_c \right)
\]

\noindent  For these 143 data points, the residual index for the linear fit
is 3/4 that  of the WHH fit, indicative  of about a  90 \% probability that
the line describes the data better.  Figure~\ref{sumtemp} also shows that at
the   lowest reduced temperature  the  data   have already reached $\rm h^*
\approx$ 0.7, the zero-temperature value of  $\rm h^*$ calculated using the
Helfand-Werthamer\cite{helfand66}  formalism.  Therefore, while it would be
gratifying to measure $\rm H_{c2}(t)$ to lower reduced temperatures and see
even larger deviations from the WHH theory, the available data (down to t =
0.3) demonstrate convincingly that the observed deviations are real.

\begin{figure}
\vspace{5.5in}
\caption[Summary of $\rm H_{c2}$ data, both $\perp$ and $\parallel$ to
the c-axis.]{Summary of all $\rm H_{c2}$ data, both $\perp$ and $\parallel$
to the  c-axis.   The dimensionless quantities  plotted   are reduced field
($\rm h^*$) versus  reduced temperature  (t). ($\bullet$), 143  data points
taken on  5 different GIC's.   ($\circ$),  best 2-parameter WHH  fit to the
data with $\cal R$ = 1.7e-2.   (.), best linear fit  to the data with $\cal
R$ = 1.3e-2.  Both fits have $\rm \frac{dh^*}{dt}$ = -1 at t = 1.}
\label{sumtemp}
\end{figure}

        The conclusions  drawn  here are  perfectly consistent  with  those
stated by Iye and  Tanuma in their papers on  $\rm C_4KHg$.\cite{iye82} The
reason is  that their data extended  over a  smaller temperature range than
the MIT data.  At  the lowest  reduced temperature for which they  reported
measurements on $\rm C_4KHg$, t = 0.55, linear character is consistent with
both  the WHH and linear  fits, as   Figure~\ref{sumtemp} shows.  Therefore
extended linearity  was unobservable  in  Iye and  Tanuma's samples  in the
temperature interval in  which they performed measurements.  It  should  be
noted, though, that unusual $\rm H_{c2}(T)$  behavior may  not occur in the
lower-$\rm T_c$ samples.  Considering that the $\rm H_{c2}(\theta)$ for the
lower-$\rm T_c$ GIC's were well-described by the simple AGL model, it seems
reasonable that the lower-$\rm T_c$ $\rm H_{c2}(T)$ curves may agree well
with the Maki-deGennes equation.

        Examination of  Equation~\ref{makidegennes} shows  that  there is a
lot more information that can be extracted  from the $\rm  H_{c2}(T)$ data.
To   begin   with,  the      value   of   $\epsilon$,       the  anisotropy
parameter,\cite{morris72} can be calculated as

\[ \rm 1 / \epsilon \; = \; \left[ dH_{c2, \parallel \hat{c}}/dT \right] /
\left[ dH_{c2, \perp \hat{c}}/dT  \right]
\] 

\noindent and compared   with the   $\epsilon$ obtained   from    fits   to the  $\rm
H_{c2}(\theta)$     data.     The   resultant  numbers   are  displayed  in
Table~\ref{epstab}.

\begin{table}
\caption[Comparison of the anisotropy parameter $\epsilon$ as obtained from
$\rm  H_{c2}(T)$   and  $\rm   H_{c2}(\theta)$  fits.]{Comparison    of the
anisotropy parameter  $\epsilon$ as  obtained from $\rm H_{c2}(T)$ and $\rm
H_{c2}(\theta)$ fits.  The $\rm H_{c2}(T)$ $\epsilon$ numbers were obtained
from the ratio  of  the    slopes.  The $\rm H_{c2}(\theta)$ numbers   were
obtained   from  fits  using  Eqns.~\ref{ldtheor}  (AGL) and  \ref{tinkham}
(Tinkham's  formula), with   type  I  superconductivity allowed  for  small
$\theta$.  In each case, the TF fits had lower  residuals than the AGL fits
(see Table~\ref{residsum}).   NA indicates that a TF  fit was   not
performed on this data;  NR denotes 
 information that was not reported in the cited reference.}
\label{epstab}
\begin{center}
\begin{tabular}{|l|ccc|}
\hline
$\rm T_c$ & 1/$\epsilon$ from $\rm H_{c2}(T)$ & 1/$\epsilon$, AGL $\rm
H_{c2}(\theta)$ & 1/$\epsilon$, TF $\rm H_{c2}(\theta)$  \\
\hline
& & & \\
0.72\cite{iye82} & NR & 11.5 at t=0.55 & NR \\
& & & \\
0.73\cite{iye82} & 9.7 & 11.3 at t=0.55 & NA \\
& & & \\
0.86\cite{iye82} & NR & 10.4 at t=0.47 & NR \\
& & & \\
0.95 & 8.9 & 10.0 at t=0.46 & NA \\
& & & \\
1.53 & 8.4 & 13.5 at t=0.29 & 14.8 at t=0.29 \\
& & 10.0 at t=0.76 & 7.7 at t= 0.76 \\
& & & \\
1.54 & 8.7 & 14.0 at t=0.29 & 16.0 at t=0.29\\
& & 15.5 at t=0.55  & 13.0 at t=0.55\\
& & 12.5 at t=0.76 & 12.0 at t=0.76\\
& & & \\
\hline
\end{tabular}
\end{center}
\end{table}

        As  the table  shows,    agreement  between the   two   methods  of
determining $\epsilon$ is rather poor.  While $\epsilon$ as determined from
$\rm   H_{c2}(T)$  is consistently   about  9, that  determined   from $\rm
H_{c2}(\theta)$  ranges  from  8  to 16.    The   reason that the   angular
dependence's  $\epsilon$'s    are   higher   is  that   in   fitting   $\rm
H_{c2}(\theta)$ it has   been assumed  that  the number  measured  for $\rm
H_{c2}(0^{\circ})$ is actually $\rm H_c$,  and  that the real value of $\rm
H_{c2}(0^{\circ})$ is much lower.  As discussed in Section~\ref{typeI}, the
$\rm H_{c2}(\theta)$ data   cannot  be fit    without  making  use of  this
assumption.   In addition, the presence  of   type   I superconductivity is
supported by   the   $\rm  C_v$  measurements    done  by    Alexander
{\em et al.}\cite{alexander81}

        It  is  tempting to   conclude  that  the   values   for $\epsilon$
determined  from the angular dependence  are unreliable because of problems
with the fits.  The matter is not that simple though, since the variability
of $\epsilon$ can be demonstrated in a model-independent way using the $\rm
H_{c2}(\theta)$ data, without benefit of fits.  One way of doing this is to
plot $\rm H_{c2}(\theta)/H_{c2}(0^{\circ})$  versus $\theta$.  This scaling
forces the curves through the common point (0$^{\circ}$, 1), with the value
at  the peak being 1/$\epsilon$ at   the temperature  of measurement.    If
$\epsilon$ is  in fact constant with respect  to temperature, then the $\rm
H_{c2}(\theta)/H_{c2}(0^{\circ})$ versus $\theta$ curves should all overlay
one another except for random errors.

\begin{figure}
\vspace{5in}
\caption[Demonstration of temperature-dependent anisotropy parameter
$\epsilon$ in $\rm C_4KHg$.]{Demonstration of the temperature dependence of
the anisotropy parameter $\epsilon$ in $\rm  C_4KHg$, where 1/$\epsilon \rm
\equiv H_{c2}(90^{\circ})/H_{c2}(0^{\circ})$.   Data are for  a $\rm T_c$ =
1.54 K $\rm C_4KHg$ sample.  ($\circ$),  t = 0.29.   ($\bullet$), t = 0.55.
($\times$), t = 0.76.   All $\rm H_{c2}(0^{\circ})$ values were  determined
from the data, not the fits, so that this plot is model-independent.  Fits
to this data are shown in Figure~\ref{hctemp}.}
\label{epstemp}
\end{figure}

        That this is  not the case  is demonstrated  in Figure~\ref{epstemp}.
The t = 0.55 and t = 0.76 curves appear  to overlay for the most  part, but
the t =  0.29  curve clearly rises  above   the  other two  at   the  peak.
Figure~\ref{epstemp}  therefore  suggests  that  the variation in  $\epsilon$
between t = 0.29 and t = 0.55  is real, but that  any variation between t =
0.55 and t = 0.76 is at best  small.  From Table~\ref{epstab},  one can see
that the   magnitudes  of $\epsilon$  from   the best fits  (to  Tinkham's
formula)   support   this  conclusion,   although    the  verdict  of   the
higher-residual AGL fits is less clear.

        A  temperature-dependent $\epsilon$ implies a temperature-dependent
slope in at least one of the high-symmetry directions.  To be more precise,
an increase in the anisotropy at low reduced temperatures requires either a
downward deviation from linearity in $\rm H_{c2, \parallel \hat{c}}$  or an
upward deviation from linearity  in $\rm  H_{c2, \perp \hat{c}}$.  While it
is  certainly true that  neither  of these trends   is obvious in  the $\rm
H_{c2}(T)$ data, it also turns out that neither of these is ruled out.  The
reason is that  if the idea of type  I superconductivity in $\rm C_4KHg$ is
taken  seriously, then the  data  plotted in  Figure~\ref{hc2temp}b) are $\rm
H_c$  rather than  $\rm  H_{c2}$,  and   so make  no   statement  about any
hypothetical  curvature    of $\rm H_{c2,    \parallel   \hat{c}}(T)$.  Any
positive curvature of $\rm  H_{c2,  \perp  \hat{c}}$ would  be
lost in the noise of the data at low reduced temperatures, so that any kink
of the size predicted by the $\rm H_{c2}(\theta)$ would be unobservable.

        So what  are the best numbers  for  $\rm  \epsilon(T)$ from the sum
total of these  data  sets?  One  important   point to notice  is that  the
magnitude of the   anisotropy derived from the  linear  fits to  the   $\rm
H_{c2}(T)$ curves represents a sort of thermal average.  Thus  $\epsilon$ =
9 is probably  the mean value  over the range 0.3 $<$  t $<$   0.95.    The
evidence from  the angular dependence, which  measures the anisotropy  at a
specific  t, indicates  that the true $\epsilon$  is  higher at low reduced
temperatures and lower at  higher  reduced  temperatures.  Because the $\rm
H_{c2}(\theta)$ data is  much  less affected by  experimental  errors,  the
magnitudes of $\epsilon$ determined from the $\rm  H_{c2}(\theta)$ fits are
thought to be more reliable.

        Temperature-dependent   anisotropy has  been   observed  before  in
graphite intercalation compounds, specifically in  $\rm C_8KHg$ by  Iye and
Tanuma.  The increase in anisotropy from 17.6  at t = 0.81  to  21.6 at t =
0.23 is illustrated in Figure~\ref{stiiepstemp}.  The variation of $\epsilon$
shown there  is similar to  what is reported here  for $\rm C_4KHg$.   $\rm
C_8KHg$  is   the   only GIC  for  which    Iye and   Tanuma   showed  $\rm
H_{c2}(\theta)$ curves at more than one temperature.   This is unfortunate,
since it would be  interesting to  know whether a  variable $\epsilon$ is a
general property of GIC superconductors.

\begin{figure}
\vspace{5in}
\caption[Temperature-dependent anisotropy in $\rm
C_8KHg$.]{Temperature-dependent anisotropy in $\rm  C_8KHg$ is demonstrated
by a  plot of $\rm  H_{c2}(\theta)/H_{c2}(0^{\circ})$ versus $\theta$, just
as     in  Figure~\ref{epstemp}.    All  data    from   Iye  and    Tanuma,
Ref.~\cite{tanuma81} on a $\rm T_c$ = 1.94 K sample.  ($\times$), data at t
= 0.23.  Fit, ($\diamond$), with 1/$\epsilon$ = 17.6 and $\cal R$ = 6.8e-3.
($\bullet$), data at t = 0.81.  ($\circ$), fit with 1/$\epsilon$ = 21.6 and
$\cal R$ = 5.3e-3.}
\label{stiiepstemp}
\end{figure}

\begin{figure}
\vspace{5.5in}
\caption[Positive curvature of $\rm H_{c2}(T)$  in $\rm C_8RbHg$.]{Positive
curvature of $\rm H_{c2}(T)$ in $\rm C_8RbHg$.  Data are taken  from Iye and
Tanuma,    Ref.~\cite{iye82}.  ($\circ$),  $\rm  H_{c2,  \perp   \hat{c}}$.
($\bullet$),   $\rm H_{c2,  \parallel   \hat{c}}$.  Parameters for the line
fits: for $\rm \vec{H} \perp \hat{c}$, $\rm H_{c2}(0)$ =  3078 Oe, $\rm T_c$
= 1.36 K, and $\cal  R$ = 0.56;  for $\rm \vec{H} \parallel \hat{c}$,  $\rm
H_{c2}(0)$ = 89.0  Oe,  $\rm  T_c$  = 1.37 K,    and $\cal R$    =  3.02e-2.
Zero-field $\rm T_c$ for this sample was 1.4 K.\cite{iye82}}
\label{rbhct}
\end{figure}

        There  is     some  justification   for      speculating   that   a
temperature-dependent anisotropy  is    a common  property   of the  class.
According to  data shown  in Ref.~\cite{iye82},  all of the superconducting
GIC's (with the possible exception of $\rm C_4KHg$) show positive curvature
of their   upper critical     fields with  respect   to temperature  (d$\rm
H_{c2}^2$/dT$^2$ $>$  0).  The  largest  curvature seems  to occur  in $\rm
C_8RbHg$, as the data in Figure~\ref{rbhct} show.  Positive  curvature does
not guarantee a variable $\epsilon$, of  course, but the chances of exactly
the same curvature in  both critical  fields seems small.   In terms of the
theories described in  Section~\ref{models}, identical curvature in both of
the orientations would be a  coincidence  rather than a natural consequence
of the models (although identical curvature for both orientations  has been
reported for at   least one compound,  Fe$_{0.05}$TaS$_2$\cite{coleman83}).
These models are discussed in more detail in the section that follows.

        In the section that  follows,  reference will frequently be made to
theories  which predict positive   curvature  of   $\rm H_{c2}(T)$.   These
references should not be understood  as suggestions that positive curvature
is observed in  $\rm C_4KHg$, but  rather as expressions  of the philosophy
that whatever phenomena  are  the cause  of  the positive curvature in $\rm
C_8RbHg$ are most likely also to cause the extended  linearity seen in $\rm
C_4KHg$.  Theories which  can produce upward-curving critical field  curves
should easily be able to  produce straight ones through  adjustment of
parameters.  In these models,  after all, a straight  $\rm H_{c2}(T)$ means
merely  the  compensation  of the  forces  which  drive upward and downward
curvature.

        Before moving on  to the   interpretation of  these results,  let's
pause to summarize.  In $\rm C_4KHg$, the critical field both perpendicular
and parallel  to the graphite  c-axis shows enhanced linearity with respect
to the usual theory  for type II  superconductors.  The data   parallel  to
$\hat{c}$  are  only   marginally  consistent with   the quadratic behavior
expected for the  thermodynamic critical field  of type I  superconductors.
However, $\rm H_c(T)$ in $\rm C_8K$ is not well-fit by a quadratic, either.
Clem has shown   that superconducting  energy gap  anisotropy  can cause  a
slight deviation   from   the    BCS    temperature  dependence     of $\rm
H_{c}$.\cite{clem66} However, the deviation expected in Clem's theory is so
small that his  model  should not  be  considered  a  serious candidate  to
explain the deviations seen in GIC's.   The introduction of the possibility
of type I  superconductivity was suggested  by  the specific heat  data  of
Alexander {\em et   al.\/},  and   also by  the  poor quality  of the  $\rm
H_{c2}(\theta)$   fits  without it.   The  $\rm  H_{c2}(\theta)$  fits also
suggest that  the anisotropy parameter $\epsilon$ is temperature-dependent.
Both the temperature dependence of $\epsilon$ and the enhanced linearity of
$\rm  H_{c2}(T)$    are inexplicable      within  the simple    anisotropic
Ginzburg-Landau   theory,\cite{tilley65,lawrence71}  although basically the
data are well-described by this  model.  The AGL and other,  more detailed,
models are discussed at length in the next section.
