\section{Experimental Re\-sults: An\-gu\-lar De\-pen\-dence of the Upp\-er Critical Field}
\label{angdata}

	Two types of data were obtained from the critical field experiments
on  $\rm  C_4KHg_x$: $\rm H_{c2}$   versus the angle  $\theta$  at constant
temperature,  and  $\rm    H_{c2}$  versus  temperature   at constant angle
$\theta$.   These    two  varieties  of   curves  correspond   to different
cross-sections of the three-dimensional phase boundary, $\rm H_{c2}(\theta,
\, T)$, which separates the superconducting and normal phases.

	$\rm  H_{c2}(\theta)$ curves taken  at  about 0.44  K for  two $\rm
C_4KHg$      GIC's   are         plotted     in      Figure~\ref{hc2theta}.
Figure~\ref{hc2theta}a)  shows data taken  by Iye on a  specimen  with $\rm
T_c$ = 0.73 K, along with results from  an  MIT sample with  a $\rm T_c$ of
0.95 K.   Figure~\ref{hc2theta}b)  shows results from two MIT  samples with
$\rm T_c$'s of 1.53 and 1.54 K.  One  prominent feature of  the MIT data is
the rather large   scatter in the critical  field  values near $\theta$   =
90$^{\circ}$ ($\rm \vec{H} \perp \hat{c}$).  This scatter is  attributed to
error  in reading the angular  orientation  during  the experiments.   (See
Section~\ref{critf:exp} for an estimate  of  the measurement errors.)   The
scatter in  Iye's  data is much smaller,   probably due to a more  reliable
alignment method, perhaps  including a  servo-controlled gearing system for
rotating the sample.  The papers of Iye  {\em et  al.}  do not mention what
method of sample rotation was employed.

\begin{figure}
\vspace{7.5in}
\caption[Critical field $\rm H_{c2}$ as a function of the angle
$\theta$ for  $\rm C_4KHg$.]{Critical  field $\rm H_{c2}$  as a function of
the angle  $\theta$ for  4 $\rm  C_4KHg$ GIC's at T $\approx$  0.4 K.  Fits
(dotted lines) were  calculated using Equation~\ref{ldtheor}.   a) Data for
an MIT $\rm C_4KHg$ sample with  $\rm T_c$  =  0.95 K ($\circ$) and also
for a $\rm T_c$ = 0.73 K sample ($\bullet$) from Ref.~\cite{tanuma81}.  For
$\rm T_c$ = 0.95 K sample, $1/\epsilon$  = 10.0 and $\rm H_{c2}(0^{\circ})$
= 24 Oe with a residual $\cal R$ = 0.29.  For data  of Ref.~\cite{tanuma81},
$1/\epsilon$ = 11.3, $\rm H_{c2}(0^{\circ})$ = 26 Oe, and $\cal R$  = 0.090.
b)  Data for two $\rm  C_4KHg$ samples  with  $\rm  T_c \,  \approx$ 1.5 K.
($\circ$),  $\rm   T_c$  =   1.53  K  with   $1/\epsilon$  =    10.2, $\rm
H_{c2}(0^{\circ})$ = 46 Oe, and $\cal R$ = 0.73; $\bullet$, $\rm T_c$ = 1.54
K with $1/\epsilon$ = 9.5, $\rm H_{c2}(0^{\circ})$ = 47 Oe, and $\cal R$
= 1.18.}
\label{hc2theta}
\end{figure}

	The   dotted  lines   in Figure~\ref{hc2theta}   are  fits to    the
equation:\\

\begin{equation}
\rm H_{c2}(\theta) \; = \; \frac{H_{c2}(0^{\circ})}{\sqrt{\cos^2\theta \,
+ \, \epsilon^2 \sin^2\theta}}
\label{ldtheor}
\end{equation}

\noindent where $\theta$ is the angle defined in Figure~\ref{hc2def}a) and
$\epsilon$ is the critical field anisotropy parameter of Morris, Coleman
and Bhandari,\cite{morris72}, defined by:\\

\[ \rm \epsilon \; \equiv \; H_{c2,\parallel \hat{c}}/ H_{c2,\perp \hat{c}} \; \; .
\]

\noindent  The origin of the $\rm H_{c2}(\theta)$ formula\cite{morris72,kats69}
and the physical interpretation  of the $\epsilon$  parameter are discussed
in more detail in Sections~\ref{models} and \ref{tiltderiv}.    The residual  $\cal R$  parameter referred  to in
Figure~\ref{hc2theta} is here defined by:\\


\begin{equation}
{\cal R} \; = \; \sum_i \rm (H_i^{exp} \, - \, H_i^{theor})^2 /
\left( \sigma_i^2 \nu \right)
\label{residef}
\end{equation}


\noindent where the errors are estimated as:\\
\[ \rm  \sigma_i  \; = \; 0.1  H_i^{\rm exp} \left(1.0 \, + \, \sin\theta
\right) \; \; .
\]

\noindent and  $\nu$ is the  number of free  parameters $\equiv$ (the number
of  data points) -  (the number of  parameters used in  the  fit plus one).
This definition of the residual is similar  to that of the reduced $\chi^2$
in   standard  books  on  statistical   analysis of data.\cite{bevington69}
However, in the absence of  any knowledge of  the  actual  magnitude of the
errors in the critical field measurements, the residual parameter should be
thought of merely  as  a convenient figure-of-merit  for intercomparison of
various fits, rather than as an absolute measure of the  appropriateness of
Eqn.~\ref{ldtheor}.

	Qualitatively,  Equation~\ref{ldtheor} describes  the  experimental
points well.  Quick comparison of the fits with  the  points shows that the
fit is  better   for   the data   of   the  lower-$\rm  T_c$   samples   in
Figure~\ref{hc2theta}a)    than    for the    higher-$\rm  T_c$     ones in
Figure~\ref{hc2theta}b).   While the data   points  in a)   deviate  almost
randomly from the fit,  the data in  b) are systematically  higher than the
fits  in the angular  region around $\theta  \, = \,  0^{\circ}$.   One can
attempt    to  improve   the situation  by    raising  the  parameter  $\rm
H_{c2}(0^{\circ})$, but  since there are only two  parameters  to vary, the
net result is inevitably to worsen agreement in the wings or near the peak,
resulting in about the same value of $\cal R$.  The situation appears to be
similar for   some  of the  data    taken  by  Iye   and  Tanuma on   other
superconducting GIC's.  For example, Iye and Tanuma's $\rm C_8RbHg$ data in
Ref.~\cite{iye82}   showed  systematic   deviations   from  the    fits  to
Eqn.~\ref{ldtheor}.  Several reasons come to  mind as  explanations for the
systematic deviations seen here; some of these factors are due to extrinsic
experimental influences, but others are intrinsic to the physics of layered
superconductors, as discussed below.

\subsection{Extrinsic Influences on $\rm H_{c2}(\theta)$ in Anisotropic
Superconductors}
\label{extrinsic}

	 As was  discussed in  Section~\ref{procedure}, one of the possible
extrinsic causes  of the deviations in  $\rm  H_{c2}(\theta)$  is  that the
GIC's    a-axis  may not     have   been  mounted   perfectly   vertically.
Figure~\ref{misalign}a) shows  the properly aligned  case for  reference, and
Figure~\ref{misalign}b)  illustrates the case  where the  sample  is placed
inside the  inductance bridge with  its c-axis rotated  by an angle  $\phi$
from the horizontal.  In both a) and b), the laboratory  vertical is called
$\hat{z}$.  The  sample in-plane axes  are denoted by $\rm  \hat{a}$  and $\rm
\hat{a}^{\prime}$.  Thus in  a)  $\rm \hat{z}$ and
$\rm \hat{a}^{\prime}$ are parallel, whereas  in  b), the  misaligned case,
the angle between $\rm \hat{z}$  and $\rm \hat{a}^{\prime}$ is also $\phi$.
Since the   measurements  described here were  done  on HOPG, the  sample's
c-axis is  a well-defined crystallographic  direction, but the crystallites
are randomly oriented  in-plane,  and so the  directions $\rm  \hat{a}$ and
$\rm
\hat{a}^{\prime}$ are designated merely for convenience.

        The important  point to notice  here  is that in b) the $\rm
\hat{a}$   axis is still in the horizontal plane, even though the c-axis is
not.  Therefore rotation of the sample about $\rm \hat{z}$ will allow
correct measurement of $\rm H_{c2,\parallel \hat{a}}$ = $\rm H_{c2,\perp
\hat{c}}$, but the value measured for $\rm H_{c2, \parallel \hat{c}}$ will
be off be a factor  corresponding to $\phi$  degrees.  In fact, the minimum
in the measured $\rm H_{c2}(\theta)$ will come when $\rm \vec{H}$ is in the
plane  defined by the vector $\rm  \hat{c}$ and $\rm \hat{z}$.  Through use
of Eqn.~\ref{ldtheor}, one can see that the minimum measured critical field
for the misaligned case will be $\rm H_{c2}(0^{\circ})/\sqrt{\cos^2 \phi \,
+  \, \epsilon^2 \sin^2  \phi}$.  With a  bit more effort, by following the
original derivation  of  Morris,  Coleman, and  Bhandari\cite{morris72} the
formula for $\rm H_{c2}(\theta)$  for a specimen tilted by  an angle $\phi$
can be derived:\\

\begin{equation}
\label{tilthc2}
\rm H_{c2}^{eff}(\theta,\, \phi) \; = \; \frac{H_{c2, \parallel
\hat{c}}/\sqrt{\cos^2\phi \: + \: \epsilon^2\sin^2\phi}}{\sqrt{\cos^2\theta
\: + \: \epsilon^2\sin^2\theta / \sqrt{\cos^2\phi \: + \: \epsilon^2\sin^2\phi}}} \; .
\end{equation}

\noindent  The detailed derivation of this formula is outlined in 
Appendix~\ref{tiltderiv}.  If one  lets $\rm H_{c2,\parallel \hat{c}}^{eff}
\equiv    H_{c2,\parallel     \hat{c}}/\sqrt{\cos^2\phi   \:     +       \:
\epsilon^2\sin^2\phi}$    and    $\rm     \epsilon^{eff}  \;   \equiv    \;
\epsilon/\sqrt{\cos^2\phi \: + \:
\epsilon^2\sin^2\phi}$, then one recovers the form~\ref{ldtheor} only with
the   effective  quantities  replacing  the actual  ones.   The impact of a
non-zero  tilt angle is therefore  to increase the measured  value  of $\rm
H_{c2,\parallel  \hat{c}}$ but  not  to change  the measured  value of $\rm
H_{c2,\perp  \hat{c}}$,    therefore   reducing  the  apparent  anisotropy,
1/$\epsilon$  = $\rm  H_{c2,\perp\hat{c}}/H_{c2,\parallel \hat{c}}$.  Since
the experimental data  in  Figure~\ref{hc2theta}b)  is higher than  the fit
near $\rm \vec{H}\parallel\hat{c}$, sample tilt seems like a good candidate
for   explaining the  discrepancy.   However,    the impact of  tilt for  a
plausible  range of  $\phi$ is quite  small since  $\rm  H_{c2}(\theta)$ is
quite flat near $\theta \: = \: 0$.  The minute amount of distortion of the
curves that   occurs for  a believable  tilt    angle  ($\phi \:   \leq  \:
5^{\circ}$) is  demonstrated  in  Figure~\ref{tilteffect}.  After examining
this figure one  can conclude that the    effect of specimen tilt  on
$\rm H_{c2}(\theta)$ is probably unimportant.

\begin{figure}
% \vec{H} = H [cos(theta)cos(phi)\hat{c} +
%cos(theta)sin(phi)\hat{a^{\prime}} + sin(theta)\hat{a}]
%Then \vec{H}(theta=0) = H [cos(phi)\hat{c} + sin(phi)\hat{a^{\prime}}]
%and \vec{H}(theta=90) = H \hat{a}
\vspace{6in}
\caption[Why a
tilted  sample  affects the shape  of  $\rm H_{c2}(\theta)$.]{Why a  tilted
sample  affects the shape  of $\rm H_{c2}(\theta)$.  The notation $(\rm
\vec{H} \cdot \hat{x})\hat{x}$ signifies the projection of $\rm \vec{H}$
along $\rm \hat{x}$. a)  The aligned case.   Rotations of the sample around
the vertical $\rm  \hat{z}$ allow  $\theta$ to be  varied all  the way from
($\rm \vec{H}\parallel\hat{c}$) to ($\rm
\vec{H}\perp\hat{c}$)  ($\rm
\vec{H}\perp\hat{c}$  = $\rm  \vec{H}\parallel  \hat{a}$).   b) The
misaligned  case.   $\rm    H_{c2,\perp\hat{c}}$ can  still    be  measured
correctly,  but instead of  the true value  of $\rm H_{c2\parallel\hat{c}}$
one will get $\rm H_{c2,\parallel\hat{c}}/\sqrt{\cos^2\phi \, + \,
\epsilon^2\sin^2\phi}$.}
\label{misalign}
\end{figure}

\begin{figure}
\vspace{4.5in}
\caption[Effect of sample tilt on $\rm H_{c2}(\theta)$.]{The effect of
sample tilt on $\rm H_{c2}(\theta)$.  The three curves in this picture were
calculated using the parameters  $\rm H_{c2,\parallel \hat{c}}$ = 42 Oe,
anisotropy $\equiv 1/\epsilon$ = 15 and the following  values for the tilt
angle:  ($\circ$) $\phi$   =  0$^{\circ}$;  ($\bullet$)   $\phi$    =
10$^{\circ}$; ($\diamond$) $\phi$  = 40$^{\circ}$.  The $\phi$  = 0$^{\circ}$
curve  corresponds  to one  of  the  fits shown in Figure~\ref{hc2theta}b).
Note that the curves for $\phi$ = 10$^{\circ}$ and for $\phi$ = 0$^{\circ}$
are almost indistinguishable.}
\label{tilteffect}
\end{figure}

	Besides macroscopic tilt of the specimen, there  is another type of
misalignment that  can  influence the shape   of the  $\rm  H_{c2}(\theta)$
curves.  A microscopic type of misalignment is mosaic spread, which in this
context means the  half-width-at-half-maximum-intensity (HWHM) of   a GIC's
$(00\ell)$ peaks in a diffractometer rocking scan.   The mosaic spread is a
measure of the degree to which the c-axes of the crystallites in a piece of
HOPG are aligned.  Clearly if the planes of a GIC are not flat, the peak in
$\rm H_{c2}(\theta)$ will be somewhat  smeared  out.   Therefore, a  finite
amount of mosaic spread will  reduce the anisotropy  ratio by reducing $\rm
H_{c2}(90^{\circ}) \:  = \: H_{c2,\perp  \hat{c}}$.   The  effect of mosaic
spread  on the critical fields of  the GIC superconductors was estimated by
convolving the  formula in Equation~\ref{ldtheor} with a  gaussian function
to represent the probability of misalignment as  a function of misalignment
angle.  With a  gaussian distribution of  misaligned crystallites, for each
increment $\delta$ away from  perfect alignment, there  is a reduction by a
factor  1/$e$ in probability.  Therefore   only orientations  within  about
$\delta$ degrees of perfect alignment contribute significantly. The form of
the expression is:\\

\begin{equation}
\label{mostheta}
\rm Corrected \; H_{c2}(\theta) \; = \; \frac{\sum_{i} \exp{[-(\theta \, -
\, \theta_i)^2/(2 \delta^2)]} \: H_{c2}(\theta_i)}{\sum_{i} \exp{[-(\theta
\, - \, \theta_i)^2/(2\delta^2)]}}
\end{equation}

\noindent where $\theta$ is the angle for which the critical field is being
calculated,   $\rm  \theta_i$ is  the    dummy  variable (which runs  from
0$^{\circ}$ to 180$^{\circ}$), and $\delta$ is the mosaic spread in degrees.
The  results of the   calculations    are exemplified by   the  curves   in
Figure~\ref{mospread}.

\begin{figure}
\vspace{4.5in}
\caption[Effect of mosaic spread on $\rm H_{c2}(\theta)$.]{The effect of
mosaic    spread      on     $\rm   H_{c2}(\theta)$,     calculated   using
Equation~\ref{mostheta}.    The    same   parameters   were used    as   in
Figure~\ref{tilteffect}, except  that here the  mosaic spread, $\delta$, is
varied:   ($\circ$)  $\delta$ = 0$^{\circ}$;  ($\bullet$)  $\delta$ =
3$^{\circ}$; ($\diamond$) $\delta$ = 10$^{\circ}$.}
\label{mospread}
\end{figure}

	The most  appropriate HWHM  of the gaussian   would be the   mosaic
spread measured by $(00\ell)$ diffraction,  if known.   In the present work
the mosaic spread was unknown, and so $\delta$  was used  as a parameter in
the fits.  Because a non-zero mosaic  spread makes the $\rm H_{c2}(\theta)$
curves  broader,  it actually  worsened agreement  between   experiment and
theory, since the theoretical curves are already too broad in the  wings of
the peak.  This finding  does not, of course, indicate  that the GIC's used
in these experiments were perfect crystals; rather it is an indication that
the real cause of the unsatisfactory fits for the $\rm T_c$ = 1.5 K samples
has not yet been identified.

        The presence  of multiple crystalline  or disordered phases in $\rm
C_4KHg$ is another property that  could impact upon  the angular dependence
curves.  The differing properties of the two phases are discussed in detail
in Chapter~\ref{hydrog}.  As far as the  $\rm H_{c2}(\theta)$ curves go, it
suffices to say that since the phases may have different $\rm T_c$'s, they could
have different upper critical fields.   If these two phases are distributed
inhomogenenously in the GIC's, then at  the lowest temperatures, where both
of  them    are  superconducting,     their  presence could   distort  $\rm
H_{c2}(\theta)$.   However,  the temperature and   sample dependence of the
quality  of the  fits weigh against  this interpretation.   For  one thing,
Figure~\ref{hc2theta}b) shows that the  $\rm H_{c2}(\theta)$ curves are quite
reproducible  from sample to  sample,   an  unlikely  occurrence   if   the
distribution of phases is important.  Secondly, as mentioned  in connection
with Figure~\ref{thetatemp}, the quality of the fits improves with decreasing
temperature, the opposite of  what would be  expected if a  lower-$\rm T_c$
minority phase  were causing  the  deviations.  Thirdly, the  samples which
show  larger  deviations from   the   AGL  functional   form have  narrower
zero-field   transitions   than the lower-$\rm T_c$   specimens  which  are
well-fit by Eqn.~\ref{ldtheor}.  X-ray  and neutron diffraction experiments
confirm that  the higher-$\rm   T_c$ samples are  actually more  uniform   than the
lower-$\rm T_c$ ones.

	The last of the extrinsic factors that needs to  be considered as a
explanation of  the disagreement between the  $\rm H_{c2}(\theta)$ data and
fits to the data  is the  possibility  of bias influenced  by the method of
data reduction.  As was discussed  in Section~\ref{procedure}, the standard
working  definition of  $\rm H_{c2}$ (for  inductive transitions) was used;
this procedure gives $\rm H_{c2}$ as the intersection of a tangent drawn to
the most linear part of the field sweep with the level upper portion of the
trace.  The unavoidable question when encountering difficulties  in fitting
$\rm H_{c2}$ data is whether there is any reason  to expect agreement given
that the  theoretical definition  of $\rm H_{c2}$ is  so different from the
working one.  That is, there  is  no guarantee that the  field found by the
procedure described above  corresponds to the  highest field where vortices
can nucleate in the superconductor, which is  the theoretical definition of
$\rm  H_{c2}$.\cite{tinkham80,helfand66}  The possibility  of  bias  due to
analysis is particularly  troublesome   for  the study of $\rm  H_{c2}$  in
anisotropic  superconductors,  where the  transition shape can  be a strong
function of $\theta$, as Figure~\ref{hc2def}b) shows.

	The  only   response  that  an  experimentalist  can  make to  such
criticism is that the procedure defined above is as  good as any available,
and   that   it is   capable   of  producing data which    are  well-fit by
Equation~\ref{ldtheor}, as Figure~\ref{hc2theta}a) shows. In order to check
for   any bias  introduced  by    the  tangent method    of critical  field
determination used here, the alternative definition of  $\rm H_{c2}$ as the
90\% completion point  of the transition was  also tried  in analyzing some
$\rm    H_{c2}(\theta)$  data   sets.  Figure~\ref{ninety-tangent}    is  a
comparison  of   the curves obtained    using the  two  different  analysis
techniques.  As  this juxtaposition  shows,  slightly different  curves are
produced by   the   two  analysis methods.   However,   the agreement  with
Eqn.~\ref{ldtheor} is not  improved.   Therefore  it  is  thought that  the
unsatisfactory quality of the fits to the $\rm T_c$ = 1.5 K samples  is not
an artifact of the analysis.  In  a similar vein,  Decroux and Fischer have
noted  that $\rm H_{c2}$ results on  the  molybdenum chalcogenide compounds
(also called Chevrel phases) are not biased by the critical field definition
as long as the compounds do not contain magnetic ions.\cite{decroux82}

\begin{figure}
\vspace{4.5in}
\caption[Comparison of the effect of the two definitions of $\rm H_{c2}$ on
$\rm H_{c2}(\theta)$.]{Comparison  of the effect  of the two definitions of
$\rm  H_{c2}$   on $\rm H_{c2}(\theta)$.     $\circ$,  tangent  definition;
$\diamond$, 90\% definition.  Data are for a  $\rm T_c$ =  1.53 K sample at
T/T$\rm  _c$ =  0.29.  The 90\%   method  tends to produce  slightly higher
critical fields, but there is only a small difference between the shapes
of the curves for the two analysis methods.  Use of the 90\% definition
does not improve agreement with Eqn.~\ref{ldtheor}.}
\label{ninety-tangent}
\end{figure}

\subsection{Type I Superconductivity in $\rm C_4KHg$}
\label{typeI}
	The three factors  described   above, namely sample  tilt,   mosaic
spread, and critical field  definition, are  the extrinsic influences which
may affect the shape of $\rm H_{c2}(\theta)$.  None of these is  thought to
have a  large  effect  on  the  data.  There  are   also  more fundamental,
intrinsic explanations for  the disparity between data  and fit.   Probably
the most applicable of these to $\rm C_4KHg $  is the conceivable existence
of   type  I superconductivity  for  field   orientations near   $\theta$ =
0$^{\circ}$.   Type  I superconductivity  for  $0^{\circ}
\leq \theta \leq 25^{\circ}$ has been previously reported for $\rm C_8K$ by
several         groups\cite{kobayashi81a,kobayashi81,koike80}.         $\rm
C_{9.4}K$\cite{kobayashi81a,kobayashi81} and $\rm C_8Rb$\cite{kobayashi85},
on the other hand, are found to be type I for  all field  directions.   The
existence  of  type  I   character for   a    range of  angles   in layered
superconductors  was predicted even   before  its experimental discovery by
Kats.\cite{kats69}  For   these    reasons    the  question   of    type  I
superconductivity in $\rm C_4KHg$ should be approached with  an open mind,
despite  the type II  superconductivity reported for all field orientations
in Ref.~\cite{iye82}.

\begin{figure}
\vspace{10cm} %insert Figure 3b) from Koike 80 (p. 1113).
\caption[$\rm H_{c2}(\theta)$ for $\rm C_8K$.]{$\rm H_{c2}(\theta)$ for
$\rm C_8K$, from Ref.~\cite{koike80}.  The fields are labeled $\rm H_{c2}$
in the type II region and $\rm H_{c3}$ and $\rm H_c$ in the type I region.  }
\label{Khc2theta}
\end{figure}

	Iye and Tanuma concluded that  there is no type I superconductivity
in $\rm  C_4KHg$ because  they did  not observe supercooling in their field
sweeps   for any   angle   $\theta$.\cite{iye82} However,  the presence  of
supercooling  and the differential  paramagnetic effect  (DPE) are  not the
only criteria for  the  determination of  the  type  of  superconductivity.
Another approach to the question is to look at the Ginzburg-Landau $\kappa$
parameter since $\kappa$ = 1/$\sqrt{2}$ is the critical value dividing type
I  and type II superconductors.    $\kappa$ can be estimated using  specific
heat    and    critical    field   data through    use   of   the following
formulae:\cite{tinkham80}\\

\begin{equation}
\begin{array}{l}
\rm \kappa \equiv \frac{\lambda}{\xi} \; \approx \; \kappa_1 \equiv
\frac{H_{c2}}{\sqrt{2}H_{c}} \\
\\
\rm H_{c}(0) \; = \; \sqrt{2 \pi \gamma T_c^2}
\end{array}
\label{kappadef}
\end{equation}

\noindent where $\lambda$ is the magnetic field penetration depth, $\xi$ is the
coherence length,  $\rm  H_{c}(0)$  is  the zero-temperature  thermodynamic
critical field, and $\gamma$ is the normal-state linear electronic specific
heat coefficient.  The equality between $\kappa$ and $\kappa_1$  is good to
within about 20\%,\cite{eilenberger67} so we can use $\kappa_1$  as a rough
esimate of $\kappa$.   (In the discussion below,  what  is  referred  to as
$\kappa$ has actually been calculated using the definition  of $\kappa_1$.)
Because $\gamma$ has been  measured for $\rm C_4KHg$,\cite{alexander81} one
can estimate    $\kappa(0)$  by calculating   $\rm  H_{c}(0)$  and linearly
extrapolating  $\rm H_{c2}(T)$ to  zero temperature.  The  applicability of
this procedure to the  $\rm T_c$ =  1.5  K GIC's is  uncertain  because  no
superconducting  transition was observed down  to 0.8  K in  the samples on
which $\gamma$ was measured.   It  seems possible  that $\gamma$ could   be
different in  the $\rm T_c$ = 1.5  K and $\rm  T_c$ = 0.7  K  samples, even
though it is a property of the normal state.   Also,  $\gamma$ was given in
Ref.~\cite{alexander81} in molar units, so in order to convert to cgs units
one must calculate  the molar volume,  which  means that one  must assume a
structure for the compound.  Here a $(2  \times 2)R0^{\circ}$ in-plane unit
cell is  employed, since this  is the structure  which has  most frequently
been   observed     in    $\rm    C_4KHg$.\cite{kamitakahara84,K167}   (See
Figure~\ref{c8kstruct}.)  Despite these caveats, since no other measurement
of the specific heat is available, it is useful to see what can  be learned
by estimating  $\kappa$.    The results of this   calculation  are given in
Table~\ref{kappatable} for three  MIT $\rm C_4KHg$  samples and three  from
Ref.~\cite{iye82}.    Notice that the  magnitude of   $\kappa$ is generally
lower for the higher-$\rm T_c$  specimens.  The  implication is that type I
transitions are more likely to be observed in the higher-$\rm T_c$ GIC's.

\begin{table}
\caption[Calculated values of $\kappa$ for selected $\rm C_4KHg$
samples.]{Calculated values of  $\kappa$ for selected $\rm C_4KHg$ samples.
The  $\rm H_{c2}(0)$  numbers come  from  linear extrapolation  of the $\rm
H_{c2}(T)$ data (see Section~\ref{hvstdata}), and so are independent of the
$\rm H_{c2}(\theta)$ measurements.}
\label{kappatable}
\begin{center}
\begin{tabular}{|l|ccccc|}
\hline
& & & & & \\
Sample origin & $\rm T_c$ (K) & $\rm H_{c2,\perp\hat{c}}(0)$ (g) & $\rm H_{c2,\parallel\hat{c}}(0)$ (g) & $\rm \kappa_{\parallel\hat{c}}(0)$ & $\rm \kappa_{\perp\hat{c}}(0)$\\
& & & & & \\
\hline
MIT & 1.53 & 749 & 89 & 0.56 & 4.7\\
& & & & & \\
MIT & 1.54 & 748 & 86 & 0.54 & 4.7\\
& & & & & \\
MIT & 0.95 & 407 & 46 & 0.70 & 5.6\\
& & & & & \\
Ref.~\cite{iye82} & 0.72 & 715 & 62.4 & 0.81 & 9.3\\
& & & & & \\
Ref.~\cite{iye82} & 0.86 & 654 & 49.2 & 0.64 & 8.5\\
& & & & & \\
Ref.~\cite{iye82} & 0.73 & 575 & 58.8 & 0.77 & 7.5\\
& & & & & \\
\hline
\end{tabular}
\end{center}

\end{table}


	The meaning of $\kappa$ will be discussed in more detail below, but
for the moment it suffices to say that $\rm \kappa_{\parallel \hat{c}}$ tend
to  be  quite close to   $\rm  \kappa_{critical}$  =   0.707.   The primary
conclusion from  this calculation  is  that within  the  uncertainty of the
approximations made, the  type of the  superconductivity in $\rm  C_4KHg$ is
indeterminate, but that it could  very well be type I  for some samples for
$\theta$  close to zero.  This  is   especially true at higher temperatures
since, according to the two-fluid model:\cite{tinkham80}

\[
\kappa \, \propto \, \frac{1}{1 + t^2}
\]

\noindent where $t$ is the reduced temperature.  Thus it seems
quite likely that the presence of  type I superconductivity  is a factor in
explaining  the poor  fits obtained  to    $\rm  H_{c2}(\theta)$ data  with
Equation~\ref{ldtheor}, at least in some specimens at higher temperatures.

	Since  $\rm \kappa_{\parallel \hat{c}}$  decreases  with increasing
temperature, the  angular range in  which  type I  behavior   occurs should
increase as  t increases.   The variation of  the  angular range of  type I
character  is  observed in  TaN, the only  non-GIC  bulk material  known to
display  type   I   or   type    II      character depending    on    field
orientation.\cite{weber78}  Some  $\rm    H_{c2}(\theta)$    at    constant
temperature  curves for  TaN are shown  in Figure~\ref{TaN}.  In the bottom
trace, at T  = 1.65 K, TaN is  type II  for  all  angles, as  the excellent
agreement with the dashed fit indicates.  When the temperature is increased
to 1.8 K, the sample  has type I character near  the  [100] direction, with
the result  that the  data there deviate above  the dashed curve.  When the
temperature increase to 2.1  K, the measured critical  field for almost all
angles $\theta$ is $\rm H_{c}$, and when the temperature reaches 2.4 K, the
sample is type I for all orientations.  $\rm C_8K$ might also be type I for
all orientations for  temperatures sufficiently close  to $\rm T_c$.    The
change with temperature of the $\rm  H_{c2}(\theta)$  fits for $\rm C_4KHg$
is qualitatively similar,  although less  dramatic, since  type II behavior
persists for a much larger range of angles.  For the  higher-$\rm T_c$ $\rm
C_4KHg$ specimens,   the  $\rm H_{c2}(\theta)$  curves   are  probably most
comparable to the 1.8 K TaN trace in Figure~\ref{TaN}.


\begin{figure}
\vspace{18cm}
\caption[Type I and type II superconductivity in TaN.]{$\rm H_{c2}(\theta)$
curves for TaN showing  a transition from type I  to type II character as a
function of field direction.  From Ref.~\cite{weber78}.  TaN is the only  bulk  superconductor besides
$\rm C_8K$ (and possibly  $\rm C_4KHg$) known to  display this variability.
The  temperatures at  which  the curves were   taken  and the thermodynamic
critical fields are indicated.  Note that at 1.65 K, the sample is entirely
type II, but that at 2.4 K it is entirely type I.}
\label{TaN}
\end{figure}

	If the angular range of type I behavior increases with temperature,
then  the agreement between Eqn.~\ref{ldtheor}  and   the $\rm C_4KHg$ data
should deteriorate as the temperature  increases.  The validity of this  assertion
is demonstrated by Figure~\ref{thetatemp}, where the fit clearly worsens as
the temperature approaches $\rm T_c$.

\begin{figure}
%C_4KHg 8b data and LD fits with H_c = 0 at t = 0.3, 0.6, 0.8
\vspace{7.5in}
\caption[Anisotropic Ginzburg-Landau model fits   to $\rm
H_{c2}(\theta)$ data    for  a $\rm  C_4KHg$  sample   as  a    function of
temperature.]{Anisotropic  Ginzburg-Landau model fits  (dotted  curves)  to
$\rm H_{c2}(\theta)$ data  as a  function of  temperature.  All  fits  were
produced  with  the   parameters tilt   = 0$^{\circ}$  and  mosaic spread =
0$^{\circ}$.   a)  t   = 0.29, $\rm  H_{c2,\parallel\hat{c}}$   =   47  Oe,
anisotropy (1/$\epsilon$) =  9.5, and residual parameter $\cal   R$ = 1.18.
b) t = 0.57, $\rm H_{c2,
\parallel\hat{c}}$ = 33 Oe, anisotropy (1/$\epsilon$) = 5.5, and  $\cal R$ = 1.25.  c) t
= 0.78, $\rm H_{c2,\parallel\hat{c}}$ =  23.1 Oe, anisotropy (1/$\epsilon$)
= 4.5, and $\cal R$ = 1.43.}
\label{thetatemp}
\end{figure}

	If type I superconductivity is indeed present in $\rm C_4KHg$, this
fact should be accounted for in fitting the  data.  In Figure~\ref{hctemp},
new fits are shown in which the theoretical curve has the form:\\

\begin{equation}
\label{typeItheta}
\rm H_{c2}^{eff}(\theta) \; = \; \left\{ \begin{array}{lll}
				  \rm H_{c2}, & \rm H_{c2}(\theta) > H_c & {\rm
(type \;  II \; region)} \\ 
\rm H_c, & \rm H_{c2}(\theta) < H_c & {\rm (type \; I \; region)}
				  \end{array} \right.
\end{equation}

\noindent The idea behind using this functional form is that if $\rm  H_c$ is greater
than  $\rm  H_{c2}$,  then $\rm H_c$   will  presumably be measured  as the
critical field.  A quick comparison of Figs.~\ref{thetatemp} and
\ref{hctemp} shows that  inclusion of $\rm  H_c$  as a parameter dramatically
improves the  quality  of   the fits   at  all  temperatures.    This large
improvement is in contrast to the small (or  negative) impact on the quality of
fit of the mosaic spread and tilt parameters.  Residuals  for the fits with
and without type  I behavior are gathered  in   Table~\ref{residsum}.   The
improvement in the  residual index  is  not  accounted  for  simply by  the
addition of another free parameter since the number of  parameters is taken
into account in the definition of $\cal R$ (see Eqn.~\ref{residef}).

\begin{figure}
%C_4KHg 8b data and LD fits with H_c NOT = 0 at t = 0.3, 0.6, 0.8
\vspace{7.5in}
\caption[Anisotropic Ginzburg-Landau model fits   to $\rm
H_{c2}(\theta)$ data as a function of temperature,  taking into account the
possibility of  type  I behavior.]{Anisotropic  Ginzburg-Landau  model fits
(dotted curves) to $\rm H_{c2}(\theta)$ data as  a function of temperature,
taking into  account  the possibility of  type I  behavior.  All  fits were
produced   with the parameters  tilt  =  0$^{\circ}$  and mosaic   spread =
0$^{\circ}$.   a) t   =  0.29,  $\rm  H_{c2,\parallel\hat{c}}$ =    35  Oe,
anisotropy (1/$\epsilon$) = 14,  $\rm H_c$ = 65 Oe,  and residual parameter
$\cal R$ = 0.39.  b) t = 0.57, $\rm H_{c2, 
\parallel\hat{c}}$ =  19 Oe,  anisotropy (1/$\epsilon$)  = 15.5, $\rm
H_c$ =  43    Oe,   and   $\cal   R$ =    0.84.      c) t  =   0.78,   $\rm
H_{c2,\parallel\hat{c}}$ = 14.5 Oe, anisotropy (1/ $\epsilon$) = 12.5, $\rm
H_c$ = 24.5 Oe, and $\cal R$ = 1.11.}
\label{hctemp}
\end{figure}

	For these new fits, $\rm H_c$ was taken  as a free  parameter.  The
numbers  obtained for  the  thermodynamic critical   field from   the  $\rm
H_{c2}(\theta)$ fits for two samples are shown in Table~\ref{htherm_table},
and $\rm H_{c}(T)$ is plotted in Figure~\ref{hcfig}.   According to the BCS
theory, $\rm H_{c}(T)$ should be  quadratic, having the form $\rm  (1 \, -
\, t^2)$.   The fact that  the $\rm H_{c}$  values  found from  the angular
dependence are better fit by a straight line  is somewhat disturbing, since
$\rm  (1  \,  -  \, t)$  behavior  is  expected  for  $\rm  H_{c2}(T)$ (see
Section~\ref{hvstdata}).  However, $\rm   H_{c}(T)$ showed  {\em   positive
curvature\/} in  $\rm C_8K$ (see Figure~\ref{c6k}), where  signs of  type I
superconductivity (large hysteresis  and differential  paramagnetic effect)
were unmistakable.\cite{koike80} Since the  temperature dependence of  $\rm
H_{c2}(T)$ is   unusual in  $\rm  C_4KHg$, perhaps  it  is    not straining
credibility too much  to suggest that  $\rm H_{c}(T)$  could be  anomalous,
too, just as in $\rm C_8K$.

The obvious question  is whether  these
values of the critical field are consistent with those  calculated from the
specific  heat  measurements, also  given  in  the  table.  The  calculated
thermodynamic critical field tends to be about 1.8 times  the fit one.  The
most  likely  explanation for this   inconsistency   is that the  value  of
$\gamma$ quoted in Ref.~\cite{alexander81}  is  simply too high.   For  the
reasons mentioned above, a  higher value   of  $\gamma$ for  the  sample of
Ref.~\cite{alexander81} would not be very surprising.  

        Another  possible  explanation for the difference  in  magnitude of
$\rm H_c$ is that up to this point the demagnetization  factor $\cal D$ has
not been taken into account.  $\rm (1  \, - \, {\cal D})  H_c$ is the field
where normal regions  are first  formed.\cite{tinkham80,livingston69} Thus,
the ideal type  of   magnetization curve shown in  Figure~\ref{fig:meiss}a is
found only for ideal samples with $\cal D$ = 0; in real specimens the sharp
corner is  rounded.  The number,  size, and   shape of  the  normal regions
formed for  $\rm (1 - {\cal  D}) H_{c} <   H_{applied} < H_c$ depends
typically on sample size, shape, defects,  and  orientation with respect to
the applied  field.\cite{livingston69}   The primary effect  of  a non-zero
demagnetization factor then is to change the  shape of the  transition with
angle.  Because of  the method of   critical field determination used  here
(see Figure~\ref{hc2def}b), the possible influence  of demagnetization on the
transition      shape  cannot  be    completely  ruled    out.   Therefore,
demagnetization-induced distortion  could be  causing the measured critical
field to be substantially less than  the actual value of  $\rm  H_c$.   The
influence of demagnetization effects should be  particularly strong for the
conditions used in the study of  $\rm  C_4KHg$, since for  thin plates in a
transverse field (here $\rm \vec{H} \parallel  \hat{c}$)  $\cal D \approx $
1.0, meaning  that the samples  are effectively  always in the intermediate
state.\cite{livingston69}

\begin{table}
\caption[$\rm H_c$ values for $\rm C_4KHg$ obtained from fits to $\rm
H_{c2}(\theta)$ data.]{$\rm  H_c(t)$ values for $\rm C_4KHg$  obtained from
fits to  Eqn.~\ref{typeItheta}.   For  comparison, $\rm H_c(t)$  calculated
from the  specific  heat  data  for  a  $\rm T_c$ =  1.53  K sample is also
included.   The   samples'  $\rm   H_{c2}(\theta)$     data  are shown   in
Figs.~\ref{hc2theta}, \ref{thetatemp}, and \ref{hctemp}.}
\label{htherm_table}
\begin{center}
\begin{tabular}{|l|ccc|}
\hline
$\rm T_c$ (K) & $\rm H_c(t = 0.29)$ (g)& $\rm H_c(t = 0.55)$ (g)&  $\rm H_c(t =
0.76)$ (g) \\
\hline
1.53 ($\rm H_{c2}(\theta)$) & 65 & Not measured & 24 \\
1.54 ($\rm H_{c2}(\theta)$) & 65 & 41 & 24 \\
1.53 ($\rm C_v$) & 104.5 & 79.6 & 43.8 \\
\hline
\end{tabular}
\end{center}
\end{table}

        Ideally  it would be  possible to obtain the demagnetization factor
as a  function of angle  in order  to correct the $\rm  H_{c2}(\theta)$ and
$\rm H_{c}(T)$ curves.  Past experimentalists have gotten a value for $\cal
D$ by machining a material with known critical fields to the size and shape
of the  sample.\cite{denhoff82} Then  one gets  $\cal  D$ by comparing  the
measured critical fields of the machined specimen to  the known  values for
shapes with zero demagnetization.  Unfortunately this  procedure would have
to be carried out for each specimen.  If a reliable calculation of $\cal D$
were available, it would also allow a more  rigorous testing of the theory.
However,  since the demagnetization  is strongly  dependent  on the  sample
dimensions\cite{denhoff82} and because $\cal D$ could easily be affected by
such hard-to-quantify factors as exfoliation, the amount of effort required
does not seem worthwhile.  Any use of a value for the demagnetization would
also require the introduction of several new parameters.

	An ideal type I superconductor should be easy to distinguish from a
type II superconductor because of the discontinuity in its magnetization at
$\rm \left| \vec{H}  \right| \, = \,  H_c$.   (See Figure~\ref{fig:meiss}).
However, a non-zero demagnetization  factor can smear out the magnetization
curve,  making even type   I  transitions appear more  or  less continuous.
Despite the presence  of  demagnetization  effects,  it seems  sensible  to
examine the field sweeps to  see if there is any  evidence of discontinuity
under the conditions where type I behavior is suspected.

	  In  the    lower-$\rm T_c    \;  C_4KHg$    sam\-ples  (whose  $\rm
H_{c2}(\theta)$    curves   were  well-fit  by    Eqn.~\ref{ldtheor}),  the
experimental traces appear smooth for all angles, as Figure~\ref{transhape}a)
shows.  The  only major  difference  between $\theta \,  =  90^{\circ}$ and
$\theta \, = 0^{\circ}$ is that the high-angle transitions are considerably
broader, for reasons discussed in  Section~\ref{critf:exp}.  This result is
consistent with  the  finding  that these  samples  were type  II  for  all
orientations, and is consistent with Iye and Tanuma's findings.\cite{iye82}
	
	In  the $\rm T_c $  = 1.5 K   samples, the  field  sweeps for  $\rm
\vec{H} \perp \hat{c}$ look smooth, just as they do for the lower-$\rm T_c$ 
GIC's.  For $\rm \vec{H} \parallel \hat{c}$, on the other  hand, the curves
appear fairly continuous  on the upsweep,  but develop   a corner near  the
upper critical  field  on the downsweep  which   could be  identified  as a
discontinuity.  The arrow  in Figure~\ref{transhape}b) indicates the location
of the  corner.   While it  is  tempting   to identify  this   corner  as a
discontinuity stemming from a first-order transition, it  is a fairly small
feature.  Therefore, the evidence regarding type I versus type  II behavior
from the   field  sweeps  is  suggestive  but  ambiguous.   Low-temperature
magnetization measurements would  probably  be  easier to   interpret  than
susceptibility in  this  regard,   but unfortunately there was    no readily
available magnetometer that could be cooled to  the necessary $\approx  1$ K
range.

\begin{figure}
\vspace{15cm}
\caption[Comparison of field sweeps between type II and possible type I
transitions.]{Comparison of field sweeps between type II and  possible type
I transitions.  The vertical direction is the inductive  voltage, while the
horizontal direction  is  field.  All   traces  taken at  about  0.4 K.  a)
Transitions with $\rm \vec{H} \perp
\hat{c}$ and $\rm
\vec{H} \parallel \hat{c}$ for a $\rm T_c$ = 0.95 K sample.  For both
orientations the transitions appear smooth.  b) Transitions with $\rm \vec{H} \perp \hat{c}$
and $\rm
\vec{H} \parallel \hat{c}$ for a $\rm T_c$ = 1.5 K sample.  For $\rm
\vec{H} \perp \hat{c}$, the transition looks smooth, consistent with
its  expected type II  character.  For $\rm  \vec{H} \parallel \hat{c}$, on
the other hand, there is  a small discontinuity  in the susceptibility near
the upper critical field which is indicated by an arrow.   This feature was
seen consistently in $\rm T_c$ = 1.5 K samples.}
\label{transhape}
\end{figure}


\begin{figure}
\vspace{5in}
\caption[Thermodynamic critical fields obtained from $\rm H_{c2}(\theta)$
fits versus temperature.]{Thermodynamic critical  fields obtained from $\rm
H_{c2}(\theta)$ fits versus temperature for $\rm T_c$ = 1.5  K $\rm C_4KHg$
specimens.     The    numbers   plotted   here    are  the   same   as   in
Table~\ref{htherm_table}.  ($\bigtriangleup$), data for a $\rm T_c$  = 1.53 K
sample; ($\circ$),  data for a  $\rm T_c$ = 1.54  K sample;  ($\diamond$), a
linear fit to the data with $\rm H_c(0)$ = 85.2 g; ($\circ$), a quadratic fit
to the data  with $\rm H_c(0)$ = 66.5  g; ($\times$), $\rm H_c(t)$ calculated
using the specific heat data  of  Alexander {\em et  al.},\cite{alexander81}
which gives $\rm H_c(0)$ = 112 Oe.}
\label{hcfig}
\end{figure}

        Why   should  the lower-$\rm T_c$  specimens   be  type  II for all
orientations,   and  the  higher-$\rm  T_c$ ones    be   type   I  for some
orientations?   The   superficial answer  is  that  $\rm H_{c2}$  isn't  as
strongly   $\rm T_c$-dependent as might  be  expected, since  the  critical
fields of the $\rm T_c$ = 1.5 K samples are only slightly higher than those
of the $\rm T_c$ = 0.7 K samples.  (Refer to Table~\ref{kappatable} for the
relevant numbers.)  If $\rm H_c$ increases as $\rm T_c^2$  and $\rm H_{c2}$
increases only weakly with $\rm T_c$, the net result  must be a decrease in
$\kappa$ with increasing  $\rm T_c$.  The  deeper answer to the question is
that $\xi$ and $\lambda$, the lengths whose ratio determines  $\kappa$, are
both affected by changes in v$_F$ and  $\ell$,  the  Fermi velocity and the
mean-free-path.\cite{orlando79}   Therefore,   one  can invoke  differences
between the two sample types either in the Fermi  surface or in crystalline
perfection to  explain dissimilarity   of their   critical fields.   Of
course,  the same  mechanism should  ideally  also explain  their different
zero-field   transition  temperatures.  Since   the  lower-$\rm T_c$   gold
specimens  have  broader zero-field superconducting  transitions   than the
higher-$\rm T_c$ pink samples (see Table~\ref{tcfreq}), it seems natural to
think that the gold samples are  less well-ordered.  This line of reasoning
implies  that the   gold   specimens  have  a  lower  mean-free-path.   The
differences between the gold and  pink samples are  discussed  at length in
Chapter~\ref{hydrog}.  In the meantime, the implications of possible type I
superconductivity in $\rm C_4KHg$ deserve further consideration.

        Another good question about the $\rm  H_{c2}(\theta)$ curves is why
the  data are not perfectly flat  near  $\theta \,  = \, 0$ if the measured
critical field is $\rm H_c$ there.  Some angular  dependence of $\rm H_{c}$
is also present  in  the   TaN  data of Figure~\ref{TaN}.    As  was  first
suggested by   P.  Tedrow\cite{tedrow88}, the  angular  dependence  of  the
demagnetization factor could influence the shape of the field sweeps enough
to  cause a perceived   angular dependence  of  $\rm H_c$.   	Koike {\em et
al.}\cite{koike80} in their   study of $\rm   C_8K$  also noticed  a  small
angular  dependence   of    $\rm  H_c$     near     $\theta$ =    0    (see
Figure~\ref{Khc2theta}).  In addition,      they  measured   a     strongly
angle-dependent supercooling field which they thought must be $\rm H_{c3}$,
the  surface nucleation  field.     However, $\rm H_{c3}$ is not
angle-dependent in the sense indicated in Figure~\ref{Khc2theta}; it is
in fact equal to 1.69$\rm H_{c2}(theta)$, but here $\theta$ is the angle
between the crystal surface and the c-axis, not the angle between the
applied field and the c-axis.\cite{orlando88}  Thus $\rm H_{c3}(\theta)$
can only be measured by preparing different crystals with differently
oriented surfaces, not by turning the applied magnetic field.

        There   is   another   problem  with   the   identification  of the
angle-dependent supercoooling field with $\rm H_{c3}$.  The problem is that
$\rm H_{c3}$  is the surface  nucleation  field for a field   applied  {\em
parallel} to the surface,\cite{degennes66},  while $\rm  \vec{H}  \parallel
\hat{c}$ corresponds to the field {\em  perpendicular} to the  surface.  In
some anisotropic   Chevrel phases with  an   unusual grain structure,  $\rm
H_{c3}$ has been  seen for the   ``wrong'' orientation, but  for HOPG-based
GIC's   such  as   the ones  studied by   Koike   and coworkers,  no   such
microstructural   anomalies  are   anticipated.\cite{orlando88} Therefore  the
identification of the supercooling field $\rm \parallel
\hat{c}$ with $\rm  H_{c3}$ must be  mistaken.   It seems  likely that  the
angular    dependence     of    the   supercooling     field  reported   in
Ref.~\cite{koike80} is  a demagnetization-derived  effect, along   with the
slight angular dependence of $\rm H_c$ observed in $\rm C_8K$\cite{koike80}
and $\rm   C_8Rb$\cite{kobayashi85a}.    Certainly  the    change   in  the
demagnetization $\cal D$ (from 1 for  $\rm vec{H}$ perpendicular to a plate
to 0 for $\rm \vec{H}$ parallel  to a plate) is large  enough to cause this
apparent angular dependence.\cite{tedrow88,tinkham80}

\subsection{Type II/1 Superconductivity and $\rm C_4KHg $}
\label{typeII/1}

	 The possibility  of type I  superconductivity in  $\rm C_4KHg$ was
discussed above.  In this case a first-order transition is observed at $\rm
H_c$ rather  than a  second-order  one at $\rm H_{c2}$.   There is also the
possibility of  a first-order transition  at the lower critical field, $\rm
H_{c1}$, even in type II  superconductors.\cite{auer73} There are materials
which, as   the applied  field is   swept up, exhibit  first a  first-order
transition  at  $\rm H_{c1}$  and then  a second-order  transition at  $\rm
H_{c2}$,  as explained  below.   These superconductors  are  called  ``type
II/1'' to distinguish them from the more commonly encountered ``type II/2''
variety  which have second-order transitions both  at $\rm H_{c1}$ and $\rm
H_{c2}$.\cite{tinkham80} According to  Auer and Ullmaier, who studied  type
II/1 superconductivity in TaN  and NbN films, ``type II/1 superconductivity
is   a    general      phenomenon     for    all   low-$\kappa$     type-II
superconductors''.\cite{auer73}  If this statement  is correct, the numbers
in  Table~\ref{kappatable} suggest that type II/1  superconductivity should
occur for a  sizable range of angles in  $\rm C_4KHg$, especially  close to
$\rm T_c $.

	Type II/1   transitions occur in   materials which  have  attractive
interactions among vortices for large intervortex separations.   (Of course
the intervortex  interactions must be  repulsive at  small  separations, or
else the vortices will  coalesce  into macroscopic normal  regions, and the
material will be  type I rather than  type  II/1.)  When this  is the case,
there  will  be  an intervortex  separation  $d_0$  which    minimizes  the
interaction  energy.  If the vortex-interaction energy terms  dominate the
condensation energy in  the total free  energy, then once one vortex
has been nucleated, the sample can reduce its energy by admitting more flux
until the average vortex separation is  $d_0$.   The result is that in type
II/1 superconductors there is a discontinuity  in the magnetization at $\rm
H_{c1}$, which corresponds to a first-order transition.\cite{auer73}

	 Since type II/1 behavior  impacts upon the  beginning of the field
sweep and    not  the region  near   $\rm H_{c2}$,  it  cannot   affect the
measurement of  the  upper critical  field directly.  The influence of type
II/1 behavior on the measurement of $\rm  H_{c2}(\theta)$ must come through
its effect  on the shape   of  the field sweeps,  just  like  the  indirect
influence of  the demagnetization.  The  shape  of the transition  could be
significantly changed by a first-order transition,  and this could bias the
data analysis procedure enough to influence $\rm H_{c2}(\theta)$.  There is
no definite evidence for  type II/1 superconductivity  in $\rm C_4KHg$, but
its  presence      seems  likely      according  to     the   Auer-Ullmaier
analysis.\cite{auer73} The presence or absence  of a first-order transition
could undoubtedly be resolved  with  magnetization measurements   if a  1 K
SQUID magnetometer were available.

\subsection{Tinkham's Formula Fits to $\rm H_{c2}(\theta)$ in $\rm C_4KHg $}
\label{sec:tf}

	Another peculiarity of the $\rm  C_4KHg \; H_{c2}(\theta) $  curves
is that they  are well-fit  by Tinkham's  $\rm H_{c2}(\theta)$ formula  for
thin films:\cite{tinkham63}\\

\begin{equation}
\label{tinkham}
\rm \left| \frac{H_{c2}(\theta) \: sin(\theta)}{H_{c2}(90^{\circ})} \right| \; + \;
\left( \frac{H_{c2}(\theta) \: cos(\theta)}{H_{c2}(0^{\circ})} \right)^2 \;
= \; 1 \; .
\end{equation}

\noindent  The Tinkham formula fit is decent ($\cal R$ = 0.83 at t = 0.29
for a $\rm T_c$ = 1.54 K sample) even without allowing for type I behavior,
and is quite  good  ($\cal R$  = 0.26)   if type I  behavior  is taken into
account.  A summary of the  residuals for fits  to the $\rm
H_{c2}(\theta)$ data   is   given in Table~\ref{residsum},  and   a  direct
comparison    between  the     two    types   of  fits    is    shown    in
Figure~\ref{TINKvsAGL}a).  Figure~\ref{TINKvsAGL}b) shows the errors of the
two fits as a function of the angle $\theta$.  For the $\rm  T_c = 1.5  $ K
GIC's the Tinkham  formula fits have  residuals about  half of those of the
best  AGL       fits.     Consultation    of      standard     tables    on
statistics\cite{bevington69}  shows that  (for  39 degrees of freedom and a
3-parameter fit, the conditions of  Figure~\ref{TINKvsAGL}) there is  about
an 80\% probability that the Tinkham formula is a better description of the
data than the AGL model.  The data on the lower-$\rm T_c$ samples are still
best fit by the plain two-parameter AGL theory, just as was reported by Iye
and Tanuma.\cite{iye82}

	The reason that a  good  fit to Eqn.~\ref{tinkham} is unexpected is
that its derivation requires the assumption  that the superconducting order
parameter doesn't vary along the direction perpendicular  to the specimen's
surface.\cite{tinkham63} This assumption  seems reasonable for a film which
has a thickness  less than a superconducting  coherence length, but is hard
to believe when the sample is $10^5$ coherence  lengths in thickness.  Only
if  the sample   were so disordered  along  the c-axis that its  structural
coherence length was on the order of  the  superconducting coherence length
could   one envision that the   Tinkham formula should  apply.   However, the
metallic  nature  of   the  c-axis  conductivity\cite{fischer83}   and  the
reproducibility  of the $\rm  H_{c2}(\theta)$  results from  one  sample to
another argue strongly against this interpretation.

	Tinkham's formula has been used to  fit $\rm H_{c2}(\theta)$ curves
in  the artificially   structured   superlattice superconductors, such   as
Nb/Cu\cite{chun84} and Nb/Ta\cite{broussard88}.   However  in these systems
Eqn.~\ref{tinkham}  fits   only   below the   3D-2D   crossover  point (see
Chapter~\ref{othersys}),  and  $\rm C_4KHg$  is  clearly well  into  the 3D
regime,  where Eqn.~\ref{ldtheor}  is   supposed to  be applicable.  The 3D
nature of superconductivity in $\rm C_4KHg $ is unquestionable  because the
Klemm-Luther-Beasley $r$-parameter  is about  2000  at  0 K,  while 1  is the
critical   value    for the  dimensionality  crossover.\cite{klemm75}  (The
question of dimensionality  crossover as  it relates to GIC's  is discussed
further in Sections~\ref{hvstdata} and \ref{othersys}.)

\begin{table}
\caption[Residuals for fits to $\rm H_{c2}(\theta)$ using the AGL formula
and the Tinkham  formula, both  with and  without type I behavior.]{Residuals
for  fits to  $\rm  H_{c2}(\theta)$  using the AGL   formula  and the Tinkham
formula, both with and without type  I  behavior.  The residual index $\cal
R$ is defined in Eqn.~\ref{residef}, the AGL formula is Eqn.~\ref{ldtheor},
and the  Tinkham formula is  Eqn.~\ref{tinkham}.  Eqn.~\ref{typeItheta} shows
how  each of these  formulae was modified to   account for  possible type I
superconductivity.}
\label{residsum}
\begin{center}
\begin{tabular}{|l|cccccc|}
\hline
$\rm T_c$ & t$\rm \equiv T/T_c$ & Fit Type & $\cal R$ & $\rm H_{c2}(0^{\circ})$ (g)& 1/$\epsilon$ & $\rm H_c$ (g) \\
\hline
& & & & & & \\
1.54 &  0.76 & AGL &1.43 & 23.1 & 4.5 & 0 \\ 
& & & & & & \\
1.54 &  0.76 & AGL &1.11 & 14.5   & 12.5& 24.5 \\
& & & & & & \\
1.54 &  0.76 &  TF & 1.06 & 17.5   & 8.2 & 0 \\
& & & & & & \\
1.54 &  0.76 &  TF & 0.58 & 14.8   & 12& 24 \\
& & & & & & \\
\hline
& & & & & & \\
1.54 & 0.55 &AGL & 1.24 & 33 & 5.5 & 0 \\
& & & & & & \\
1.54 & 0.55 &AGL & 0.84 & 19 & 15.5  & 43 \\
& & & & & & \\
1.54 & 0.55 &TF  & 0.75 & 33 &  7  &  0 \\
& & & & & & \\
1.54 & 0.55 &TF  & 0.47 & 23 & 13  & 41 \\
& & & & & & \\
\hline
& & & & & & \\
1.54 & 0.29 &AGL & 1.18 & 47 & 9.5  & 0 \\
& & & & & & \\
1.54 & 0.29 &AGL & 0.39 & 35 & 14  & 65 \\
& & & & & & \\
1.54 & 0.29 &TF  & 0.83 & 50 & 9.5  & 0 \\
& & & & & & \\
1.54 & 0.29 &TF  & 0.26 & 35.8 & 16  & 65 \\
& & & & & & \\
\hline
\hline
& & & & & & \\
0.95 & 0.45 & AGL& 0.24 & 24 & 10 & 0 \\
& & & & & & \\
\hline
\hline
& & & & & & \\
0.73\cite{iye82} & 0.55 & AGL & 0.08 & 26 & 11.3 & 0 \\
& & & & & & \\
\hline
\end{tabular}
\end{center}
\end{table}

\begin{figure}
\vspace{5in}
\caption[Comparison of the Tinkham formula and AGL theory fits to $\rm
H_{c2}(\theta)$ data on  a $\rm T_c$ =  1.5 K $\rm C_4KHg$-GIC.]{Comparison
of the Tinkham formula and AGL theory fits to  $\rm H_{c2}(\theta)$ data on
a  $\rm T_c$  =  1.5  K  $\rm C_4KHg$-GIC.   $\bullet$,  data at  t = 0.55.
$\circ$, AGL fit with $\rm H_{c2}(0^{\circ})$ = 19 Oe, 1/$\epsilon$ = 15.5,
and a  residual  $\cal    R$  =  0.84.    $\diamond$,   TF fit  with   $\rm
H_{c2}(0^{\circ})$ = 23 Oe, 1/$\epsilon$ = 13, $\rm H_c$ = 41 Oe, and $\cal
R$ = 0.47. Below, a plot of  the errors of  each fit versus  $\theta$.  The
same symbols are used.}
\label{TINKvsAGL}
\end{figure}

	Despite the  statistically  significant improvement that  Tinkham's
formula gives over the AGL Eqn.~\ref{ldtheor}, it is  quite hard to justify
the use  of Tinkham's  formula  theoretically.   One obvious possibility is
that  the  superconductivity  measured  in  $\rm C_4KHg  $  is  not  a bulk
phenomenon, but is merely due to a thin layer on the surface.  The idea is not
that  there is  surface-nucleated superconductivity   at  the   edge of   a
homogeneous  bulk superconductor, but     that  there could  be   a  second
crystallographic phase   stable only near the   surface.  This explanation is
appealing  because it would help explain  the  mystery as to the difference
between the gold and pink phases of $\rm C_4KHg $: the gold lower-$\rm T_c$
phase which is   well-fit  by   Eqn.~\ref{ldtheor}  would be   due to  bulk
superconductivity, whereas the pink higher-$\rm T_c$ phase would be present
only on  the surface.  A surface phase  present only in  a thin layer could
easily mimic  two-dimensional  behavior, and would    be   expected to  fit
Tinkham's formula.

	 Superconductivity  in  a surface layer  has  already been shown to
give good  agreement with Tinkham's  formula.  In   lead  films doped  with
thallium, the ``thin film'' was a surface-nucleated layer about a coherence
length thick.\cite{tinkham64}  Nonetheless the ``film'' had critical fields
which agreed quite  well with the  $\rm H_{c2}(\theta)$ shape predicted  by
Eqn.~\ref{tinkham}.   Surface  superconductivity cannot be  responsible for
the  $\rm   H_{c2}(\theta)$ behavior  here   since it  can   give  only the
anisotropy ratio 1.69 = $\rm H_{c3}$/$\rm H_{c2}$.\cite{tinkham64}

	There is one major problem with the  idea of the  pink ($\rm T_c$ =
1.5 K) phase of $\rm  C_4KHg$ being present  only as a  surface film.  That
problem is that if thin films are responsible for the angular dependence of
$\rm H_{c2}$, they should give\cite{tinkham80}\\

\[ 
\rm H_{c2, \perp \hat{c}} \; = \; \frac{2 \sqrt{6} H_c \lambda_{\parallel \hat{c}}}{d}
\]

\noindent where d is the thickness of the film.  As discussed in
Section~\ref{otherintro}, thin films have  a parallel critical field with a
temperature dependence of the form $\rm H_{c2, \perp \hat{c}} \: \propto \:
(1 \, - \, t)^{1/2}$, a form which is in definite conflict  with the linear
behavior  found   for $\rm  H_{c2,   \perp \hat{c}}(t)$.  (The  temperature
dependence of $\rm H_{c2}$ at constant angle is discussed  in detail in the
Section~\ref{hvstdata}.)

        For a given  anisotropy, the  primary difference  between Tinkham's
formula and  Eqn.~\ref{ldtheor} is that  the  Tinkham's  formula curve  has
upward positive    curvature  for  all  angles,   whereas  the  AGL angular
dependence has an   inflection point in  the  wings of the  peak,  and  has
negative curvature at $\theta$ = 90$^{\circ}$.  The challenge in justifying
the application of Tinkham's formula to GIC's is to think  of a factor that
could  cause $\rm  H_{c2}(\theta)$ to  rise  more  steeply   than  the  AGL
dependence.  One aspect of the problem that is completely overlooked in the
derivation   of  Eqn.~\ref{ldtheor}  (and   in the derivation  of Tinkham's
formula, for  that matter)  is the  microscopic physics  of   the flux-line
lattice (FLL).  In an isotropic superconductor, in the absence  of defects,
the axis of symmetry of a vortex must lie along the applied  field.  If the
symmetry axis of the vortex were not along the applied field, the screening
currents  in the vortex would have  to be larger, which  would cost kinetic
energy.  Tilley,\cite{tilley65} one of  the  originators of the AGL  model,
calculated the  properties of  the flux-line lattice   (FLL) in anisotropic
superconductors.  He found that the energy  of  the FLL is  lowest when the
applied field is  along a crystallographic  symmetry  direction.   When the
applied field is  at an   arbitrary  angle  $\theta$   with respect to  the
symmetry axes,   the vortices  pay  a potential   energy  price  for  their
misorientation with respect to the crystalline axes.\cite{tilley65} Perhaps
when the applied  field is oriented only slightly  off  a  crystallographic
symmetry direction (such as $\rm
\vec{H}
\perp  \hat{c}$), the flux-line  lattice  might actually  minimize its total
energy (kinetic energy  from  screening currents  plus potential energy  of
misorientation)   by orienting the   vortices'  symmetry  axes   along  the
crystallographic symmetry  direction rather than along the   applied field.
Quantitative    calculations by    Kogan\cite{kogan81}    and   Kogan   and
Clem\cite{kogan81a} suggest that this rotation of the flux-line lattice may
be a common feature  of anisotropic superconductors.   It is  proposed that
rotation of the FLL to the crystallographic symmetry direction for $\theta$
near  90$^{\circ}$ could help explain  why the experimental data  rise more
sharply than the AGL model in the wings  of the  $\rm H_{c2}(\theta)$ peak.
Unfortunately,  coming  up with  a   test of this hypothesis is   not easy.
Magnetic     torque    measurements  might   be    able    to   give   some
information.\cite{hagen87}  Kogan  and Clem suggest neutron  scattering and
nuclear magnetic resonance tests.\cite{kogan81a}

	In summary,  Tinkham's formula provides  the best  fit to  the $\rm
H_{c2}(\theta)$ data for the  $\rm T_c$ =  1.5  K  specimens.  However, the
assumptions used in the  derivation of  the Tinkham formula\cite{tinkham63}
imply  a  temperature dependence of   $\rm H_{c2}$    which is strongly  in
conflict    with   experimental  data.     Therefore, solving    the   $\rm
H_{c2}(\theta)$ problems of  the $\rm C_4KHg$  data  with Tinkham's formula
creates new problems for  $\rm H_{c2}(t)$.   Figure~\ref{TINKvsAGL}b) shows
that the Tinkham's formula fit seems to be an improvement on the AGL theory
principally   in  the    wings,  where the    angular   dependence  of  the
demagnetization may play a role.  Therefore an unsatisfying  but reasonable
conclusion  is that the  agreement  with  Tinkham's formula is  fortuitous,
implying  that the formula  somehow mimics  the combination of  anisotropy,
tilt, type I behavior, demagnetization effects and  mosaic spread that were
actually    present  in the experiments.    Alternatively,  the microscopic
details of   the  flux-line  lattice  may  be   having an  effect  on  $\rm
H_{c2}(\theta)$.

\subsection{Microscopic Models of Su\-per\-con\-duct\-ivity and $\rm
H_{c2}(\theta)$ in $\rm C_4KHg$}
\label{gaptheta}

	In  the following  sections  on the temperature   dependence of the
upper critical field,  the  theories of choice  for explaining the data are
those  which include   anisotropic  gap or  multiband  effects.     Logical
consistency demands  that these effects also be  considered  in attempts to
explain the anomalies in $\rm H_{c2}(\theta)$.

	Many   theorists  have  calculated  the effects    of Fermi surface
anisotropy, gap anisotropy, or multi\-band con\-tri\-bu\-tions to su\-per\-con\-duct\-ivity
on $\rm  H_{c2}(T)$.\cite{youngner80,butler80,entel76,aljishi88}  Unfortunately,
most of  these theorists  have confined their  calculations  to one  or two
high-symmetry directions, so that a great deal  of  work remains to extract
$\rm H_{c2}(\theta)$ from their equations.

        The only model to explicitly consider the  case of a  general field
orientation is  that developed by  K. Takanaka.\cite{takanaka75} This model
incorporates  both  Fermi surface anisotropy  and  gap  anisotropy,  and is
discussed further   in  Section~\ref{models}.     Takanaka  gives  a   full
expression for $\rm H_{c2}(\theta, T)$ which could be used  to fit the $\rm
H_{c2}(\theta)$ data.  Takanaka's  model  was withdrawn, though, because of
the   unphysical divergence   of   $\rm   H_{c2}(t)$   below  about  t    =
0.7.\cite{ikebe80} While Youngner  and Klemm\cite{youngner80}  calculated a
corrected $\rm H_{c2}(T)$ that utilized some  of Takanaka's ideas, they did
not produce an improved $\rm H_{c2}(\theta)$ equation.

        Despite  the  unphysical  behavior    of   Takanaka's model at  low
temperatures,  one might consider  fitting it  to the experimental  data at
high reduced temperatures.  Takanaka does show  $\rm H_{c2}(\theta)$ curves
in Ref.~\cite{takanaka75}  for t  = 0.9 which  incorporate  gap  anisotropy
effects.  Toyota {\em et al.}\cite{toyota76} fit their $\rm NbSe_2$ data to
the  Takanaka  equation, but  it  is  not obvious  that  the Takanaka curve
describes   their  data  better than   the    AGL theory.  Furthermore,  if
anisotropy-induced anomalies are responsible for the deviations of the $\rm
C_4KHg \;  H_{c2}(\theta)$ data from  the AGL model,  one would  expect the
deviations  to     be       greatest   at    the      lowest        reduced
temperatures.\cite{youngner80}   Figure~\ref{thetatemp}  shows  that      the
deviations  are in fact greater at  higher reduced temperatures,  where the
simple  AGL model should  be  most  applicable.   The reason   is that  the
assumptions that go into Ginzburg-Landau theories are justifiable only near
$\rm T_c$,\cite{tinkham80} so if a more microscopic model  is needed to fit
the data, it should describe  low-temperature anomalies.  To put the matter
into common-sense terms, one would  expect anisotropy-derived deviations in
$\rm H_{c2}(\theta)$ in the same temperature range where  the anisotropy is
causing   anomalies in  $\rm    H_{c2}(t)$, {\em    i.e.}, at  low   reduced
temperatures.  The larger deviations from the  AGL theory at higher reduced
temperatures   is  therefore   in    conflict    with  expectations    from
anisotropic-gap and multiband  models, but consistent with type  I behavior
for small $\theta$, as discussed in Section~\ref{typeItheta}.

        The general problem with trying  to fit more  microscopic models to
the  $\rm H_{c2}(\theta)$  curves  is that  many parameters  (tilt,  mosaic
spread, $\rm H_c$, $\epsilon$, $\rm  H_{c2}(0^{\circ})$) have  already been
identified as being relevant.  5  parameters is already  a large number for
slightly noisy  curves  of  about 40  or  50 points.  Considering  that the
situation is  somewhat simpler with the  $\rm H_{c2}(T)$ data, it is hardly
surprising that not  many  detailed theories  of $\rm   H_{2}(\theta)$ have
appeared.

        Since the $\rm  H_{c2}(\theta)$ curves deviate  from the AGL theory
near $\rm T_c$ and the  $\rm H_{c2}(T)$  curves deviate from the AGL theory
near T=0, it does not seem too far-fetched to attribute different causes to
these anomalies.    With this   decoupling in  mind,   the  most convincing
explanation for the shape of the $\rm H_{c2}(\theta)$ curves in $\rm C_4KHg
$ would seem to be that based on type I superconductivity near $\rm \vec{H}
\parallel \hat{c}$.  This explanation is embodied by Eqn.~\ref{typeItheta},
which allows for  type I behavior.  The anomalies  in $\rm H_{c2}(T)$  are
discussed in the section that follows. 
