\subsection{Discussion}
\label{angdep:disc}


        Even if one had in hand  a workable model  for $\rm H_{c2}(\theta)$
which incorporated gap anisotropy and  multiband effects, the justification
for  fitting it to layered  superconductors' data would not  be clear.  The
difficulty lies in the fact that there are so many uncertainties associated
with  the data  collection and  analysis   for   $\rm H_{c2}(\theta)$.   In
particular, the  appropriateness of the  assumptions one makes  in choosing
the   experimental definition of $\rm  H_{c2}$   (discussed in reference to
Figure~\ref{ninety-tangent}) is    hard  to   judge.   The   possibility   of
introducing bias into critical  field  data through use of an inappropriate
definition of $\rm H_{c2}$ is  a  widespread  one for experimentalists  who
measure  parameters such as resistance  or  inductance which are indirectly
related to the energy gap.

One cannot   extract
information from data that is not  there, and  it appears entirely possible
that information about  the microscopic details  of a layered superconductor
like $\rm  C_4KHg $ is  simply not  contained in  the $\rm  H_{c2}(\theta)$
data.p
