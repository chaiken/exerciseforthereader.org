\section{Preparation of KHg- and CsBi-GIC's}
\label{synth}

        The  basis for understanding  the synthesis of   a ternary graphite
intercalation compound  comes from an examination  of the phase  diagram of
the relevant binary alloy.  The phase diagrams of the K-Hg and Cs-Bi binary
alloy systems  are shown in  Fig.~\ref{phasediag}.  There  are  two reasons
that the phase diagrams are important.  One is that the phase diagram gives
pertinent information about temperatures  and alloy  compositions necessary
for preparation of a given  GIC, as discussed  below.   For example, when a
particular heavy element has only limited solubility for the alkali metals,
as in  the case of  lead,\cite{hansen58}  then  that   heavy metal will  be
difficult, if not impossible, to intercalate in combination  with an alloy.
This is  the  case for  lead, as   has been  shown by  various  unsucessful
attempts  at  intercalation, both  here  at   MIT  and   by Lagrange    and
coworkers.\cite{lagrange85} Also,since many of the properties of  GIC's are
similar  to  those of  the  intercalant,\cite{I94} the phase diagram  is  a
useful source of information about the GIC.  For example, inelastic neutron
scattering studies of the  $\rm C_8K_{1-x}Rb_x$ system  have shown that the
elastic constants of the intercalant layer are minimum at  the same x where
the eutectic point of the binary phase diagram occurs.\cite{neumann84}

\begin{figure}
\vspace{8in}
\caption[Phase diagrams of the K-Hg and Cs-Bi binary alloys.]{Phase 
diagrams of the K-Hg and Cs-Bi binary alloys.\cite{hansen58}}
\label{phasediag}
\end{figure}

        According  to  Lagrange\cite{lagrange87},   the  stage of  a GIC is
determined only by the  stoichiometry of  the starting alloy.  This finding
is  illustrated by  Figure~\ref{alloychoice},  which shows   that  a  given
composition range for each of  the K-Hg and Cs-Bi systems  corresponds to a
unique stage in the final sample.  The current study confirms the idea that
alloy composition is the primary determinant  of  stage in the  ternary GIC's.
Good  agreement was also found between  the composition ranges indicated on
the phase diagrams for production of a given stage and the results reported
here.

\begin{figure}
\vspace{7in}
\caption[Range of compositions of the starting alloy which will produce a
given stage   GIC in the    K-Hg  and    Cs-Bi binary  systems.]{Range   of
compositions of the starting alloy which will produce a given  stage GIC in
the K-Hg and Cs-Bi binary systems.  From Ref.~\cite{lagrange87}. Upper, K-Hg
system.   In region  (1), $\rm  C_8K$  is produced; in   region  (2),  $\rm
C_4KHg$; in region  (3), $\rm C_8KHg$; in  region  (4), $\rm C_{12}KHg$; in
region (5), higher-stage  binary  compounds, or  no  reaction. Lower, Cs-Bi
system.  In region (1), $\rm C_8Cs$ is produced;  in (2), $\rm C_4CsBi$; in
region (3), no reaction.}
\label{alloychoice}
\end{figure}

        There were three fundamentally different methods used to synthesize
the specimens used   in these studies.  These  are   contact-intercalation,
sequential  vapor-phase     intercalation,   and single-step    vapor-phase
intercalation.  In the  contact-intercalation and  single-step  vapor-phase
intercalation  techniques,  the starting  alloy was   reacted  first in  an
evacuated pyrex tube,  and then exposed  to  the  graphite in   a  separate
furnace session.  An amount of alloy  several times the  quantity needed to
complete   the  reaction  stoichiometrically   was  always   used.   In the
sequential intercalation  method  used  for $\rm  C_8KHg$,  $\rm  C_8K$ was
prepared  by   reaction    of  elemental   potassium    with  graphite   at
200$^{\circ}$C, and  then a stoichiometric amount of   mercury  was reacted
with the $\rm C_8K$ at 290-300$^{\circ}$C.   This temperature was chosen to
be just above the liquidus line for the compound  $\rm KHg_2$ (see the K-Hg
phase diagram),   from which  $\rm   C_8KHg$  also  can  be prepared.   The
sequential-intercalation  method  was  first discovered   by El Makrini and
coworkers.\cite{elmakrini80a}

        The  principal  difficulty    with the   single-step    vapor-phase
intercalation technique was  the  inaccuracy  of measurement of the  alkali
metal.  The  volume  of  the alkali  metal  was  estimated by measuring the
volume of its glass container, thereby  assuming that the  metal filled the
container uniformly,   without   any cavities.  A   more accurate    way of
measuring the amount of alkali metal (and of  measuring mercury) is to have
a microbalance in  the glovebox where the  metals are handled.   This
technique  was   employed by   the  research group  at   the  University of
Kentucky.\cite{doll86}

        Contact intercalation,  in which the  graphite starting material is
actually in  contact with the liquid intercalant  inside an evacuated pyrex
ampoule, was used in the synthesis of  all CsBi-GIC's.   The starting alloy
used  in  these  reactions was   Cs$_5$Bi$_4$, and  the temperature of  the
intercalation was  about 480$^{\circ}$C to  500$^{\circ}$C,  just above the
liquidus line for the alloy  (see the phase diagram).   If the liquid metal
were not  placed in   contact with  the graphite, then   even if  all other
conditions were right, the final product was always $\rm C_8Cs$.\cite{E291}
This reaction was almost foolproof:  every   single  attempt at making  the
CsBi-GIC's  with the   contact   method resulted in   the intercalation  of
single-stage CsBi-GIC.  $\rm   C_4CsBi_{0.5}$ $\alpha$-phase  samples  were
violet, and  $\rm C_4CsBi_{1.0}$ $\beta$-phase  samples were blue-green, as
previously reported.   Thus  it  appeared that    the CsBi-GIC's are   less
sensitive to the details of preparation than the KHg-GIC's.

        The difficulty with  the  contact  reaction was that the GIC's were
often embedded in the hard  alloy metal when the  intercalation ampoule was
removed from the  furnace.   Removing the samples   from  the alloy without
fracturing them was often a tricky business.  An attempt was made to remove
the liquid alloy  from the sample surface by  tilting  the entire furnace a
few hours before  the intercalation ampoule was to  be  removed, but  these
attempts were unsuccessful.  The other problem with the CsBi intercalation,
which  probably is related  to  the  contact  reaction, is that macroscopic
inclusions of the starting alloy were often found  inside the samples. This
is discussed further below, and also in Section~\ref{csbitc}.

        The single-step   vapor-phase two-zone technique  can  be  used  to
synthesize both $\rm C_4KHg$ and $\rm C_8KHg$, as originally reported of El
Makrini {\em et  al.  \/}\cite{elmakrini80a} The modifier ``two-zone'' here
refers to  the fact that  the  graphite may be  kept  at  a slightly higher
temperature (T$\rm _{prep}$ + $\Delta$T) than the alloy ($\rm T_{prep}$) in
order to prevent condensation of metal on the GIC surface.\cite{I94} In the
$\rm  C_8KHg$  case,  KHg$_2$ alloy was placed  inside an evacuated ampoule
with graphite  and      the  two      were    reacted  isothermally      at
290$^{\circ}$-300$^{\circ}$C for three weeks.  The KHg$_2$ alloy was silver
in color,  and the  stage II GIC   was always   blue,  both   as previously
reported.\cite{J140}   In  the   $\rm  C_4KHg$     case, the starting alloys  ranged in
composition from K$_2$Hg$_3$  to K$_5$Hg$_4$.   The reaction
was  carried out  at temperatures between 200$^{\circ}$C and 260$^{\circ}$C
for a period  of two  weeks.   These  alloys were   gold in color, and  the
intercalation compounds ranged in  tint from  gold to copper to  pink.   No
systematic   effects were seen   as     a  function of  the starting  alloy
composition in  $\rm  C_4KHg$.  Most reactions  were  carried out  with the
alloy KHg.

        Temperature  differences $\Delta$T  between  the  graphite and  the
alloy for $\rm C_4KHg$ were varied between 0$^{\circ}$ and 10$^{\circ}$, as
opposed to the case for $\rm C_8KHg$, where only isothermal conditions were
employed.  In  the case of isothermal reactions,  the intercalation ampoule
was wrapped in several thick  layers  of aluminum foil  in order  to  guard
against   unintended temperature   gradients in  the     furnace.   If  the
intercalation ampoule was kept short, and a large  furnace was used,  it is
estimated that the temperature at the two ends of the ampoule could be kept
the same to within a degree or two.  Through use  of an ice-point reference
on the  furnace's temperature controller and  liberally applied  glass wool
insulation,  the  temperature   of  the  intercalation    ampoule  could be
controlled to within about $\pm$10$^{\circ}$C during the reaction.

        The outcome of the $\rm C_4KHg$ reaction is quite sensitive  to the
temperature of the furnace.  On  more than one occasion, a  tube containing
KHg alloy  plus  graphite which had  mysteriously  not reacted  at all  was
removed  from   the  furnace.  Half  the time  the  intercalation  could be
accomplished simply by cooking the tube for another  couple of weeks, under
apparently   identical conditions.   The  trickiness   of the  $\rm C_4KHg$
intercalation  has been noted    before by   Timp.\cite{J140}  It   is  not
surprising that the $\rm C_4KHg$ intercalation is more  difficult than that
of a compound with a simpler structure, like $\rm C_8K$, but it is somewhat
unexpected that  the $\rm  C_4KHg$ preparation  is much  more difficult  to
control than that of $\rm C_8KHg$.  Samples from two  $\rm  C_8KHg$ batches
are almost indistiguishable  from one another, even  if one is made through
sequential   intercalation,  and  the other    made  using  a   single-step
vapor-phase reaction.  The predictability  of the  $\rm C_8KHg$ reaction is
underlined    by the  reproducibility  of   its superconducting  transition
temperature, as measured   by several  research  groups.  These  $\rm  T_c$
values are  gathered in  Table~\ref{stiitc}.   In contrast, the  $\rm  T_c$
values reported for $\rm  C_4KHg$ and shown  in Table~\ref{stitc} exhibit a
great deal of scatter.

\begin{table}
\caption[$\rm T_c$ values reported for $\rm C_8KHg$.]{$\rm T_c$
values for  $\rm C_8KHg$ as reported by several research
groups.  There is almost universal agreement on $\rm T_c \: \approx$ 1.9 K.}
\label{stiitc}
\begin{center}
\begin{tabular}{||c|c|c||}
\hline
& & \\
$\rm T_c$ (K) & Method of Preparation & Ref.\\
& & \\
\hline
& & \\
1.70 & KHg$_2$  & \cite{koike81}\\
& & \\
1.82&  sequential  & \cite{iye83}\\
& & \\
1.84 & sequential  & \cite{iye83}\\
& & \\
1.848 & KHg$_2$ & \cite{delong82a} \\
& & \\
1.870 & sequential & \cite{delong82a} \\
& & \\
1.896 & KHg$_2$ & present work \\
& & \\
1.90 & KHg$_2$  & \cite{koike81}\\ 
& & \\
1.90 & sequential   & \cite{vogel81}\\
& & \\
1.90 & sequential  & \cite{pendrys81} \\
& & \\
1.90 & KHg$_2$  & \cite{tanuma81} \\
& & \\
1.93 & sequential  & \cite{alexander81} \\
& & \\
1.94 & KHg$_2$  & \cite{tanuma81} \\
& & \\
\hline
\end{tabular}
\end{center}
\end{table}

\begin{table}
\caption[$\rm T_c$ values reported for $\rm C_4KHg$.]{$\rm T_c$
values for  $\rm C_4KHg$ as reported by several research
groups.  There is a great deal of scatter in the values, which seem to fall
primarily 
into two groups, with most likely values being about 0.8 and 1.5 K.
$^{\dagger}$ indicates a broad superconducting transition.  $^{\ddagger}$
indicates a liquid-nitrogen quenched sample.\cite{delong82a}}
\label{stitc}
\begin{center}
\begin{tabular}{||c|c|c|c|c|c||}
\hline
& & & & & \\
$\rm T_c$ (K) & T$\rm _{prep}$ & $\Delta$T & Time & Color & Ref.\\
& & & & & \\
\hline
& & & & & \\
$<$1.5 & 200$^{\circ}$C  & 0?  & one week & copper & \cite{koike81}\\
$<$0.8 & 200$^{\circ}$C  & 0? & ?  &  copper & \cite{alexander81}\\
0.719$^{\dagger}$ & 200$^{\circ}$C  & 0 & two weeks & gold& present work\\	%mysample 5c
0.72 & 200$^{\circ}$C & 0? & several days & copper-pink & \cite{iye82}\\
0.726 & 200$^{\circ}$  & 6$^{\circ}$ & two weeks &gold & present work\\   %3b
0.73 & 200$^{\circ}$C & 0? & several days & copper-pink & \cite{iye82}\\
0.822$^{\dagger}$ & 200$^{\circ}$C & 0 & two weeks & gold & present work\\	%mysample 5d
0.86 & 200$^{\circ}$C & 0?  & several days & copper-pink & \cite{iye82}\\
0.875$^{\dagger}$ & 200$^{\circ}$C  & 10$^{\circ}$C  & two weeks & gold & present work \\	%6b
0.949$^{\dagger}$ & 200$^{\circ}$C  & 6$^{\circ}$C  & two weeks & gold & present work \\ %13c
1.09 & 200$^{\circ}$C & 10$^{\circ}$C & ? & gold & \cite{J140}\\
1.198 &234$^{\circ}$C   & 0$^{\circ}$C  & two weeks & copper & present work\\	%11b
1.317$^{\dagger}$ &260$^{\circ}$C  & 0$^{\circ}$C   & two weeks& copper & present work\\	%9
1.36$^{\dagger}$ & 200$^{\circ}$C$^{\ddagger}$ & 0$^{\circ}$C ? & two weeks
& ? & \cite{delong82} \\
1.459 &200$^{\circ}$C & 0$^{\circ}$C & two weeks &pink & From Timp. \\	%timp's sample
1.46$^{\dagger}$ & 200 $^{\circ}$C & 0$^{\circ}$C  & two weeks  & yellow/pink & present work \\	%5a
1.472$^{\dagger}$ &260$^{\circ}$C & 0$^{\circ}$C  & two weeks & copper & present work\\	%9b
1.508$^{\dagger}$ & 200$^{\circ}$C  & 0$^{\circ}$C  & two weeks & yellow/pink & present work \\   %5b
1.526 & 200$^{\circ}$C  & 0$^{\circ}$ & two weeks & pink & present work\\	%7a
1.528 & 200$^{\circ}$C  & 0$^{\circ}$C  & three weeks & pink & present work \\	%8c
1.536 & 200$^{\circ}$C & 0$^{\circ}$C  & three weeks & pink & present work \\	%8b
1.539 & 200$^{\circ}$C  & 0$^{\circ}$C  & three weeks & pink  & present work  \\	%8a
1.56 & 200$^{\circ}$C & 0$^{\circ}$C  & 14 days & pink & \cite{J140,timp83}\\
1.65 & 200$^{\circ}$C  & 5$^{\circ}$C & 14 days & red & \cite{J140,timp83}\\
& & & & & \\
\hline
\end{tabular}
\end{center}
\end{table}

        Why should the results  of  the stage II intercalation  be  so much
more  reproducible than those  of the stage I?   The answer seems to lie in
the multiphase behavior  of $\rm C_4KHg$,   as opposed to the  single-phase
behavior of  $\rm  C_8KHg$.   $\rm  C_8KHg$  has only  one  common in-plane
ordering, the (2$\times$2)R0$^{\circ}$ in-plane superlattice which has been
observed  in  neutron diffraction.\cite{kamitakahara84b}  This ordering  is
associated with an $\rm I_c$ = 13.56 \AA, the only repeat distance observed
in $\rm C_4KHg$.\cite{kamitakahara84b} $\rm C_4KHg$, on the other hand, has
two commonly observed phases.   These are  the $\alpha$ phase,  which has a
(2$\times$2)R0$^{\circ}$  in-plane ordering and  $\rm I_c$ = 10.24 \AA, and
the    $\beta$  phase, which   has   a   ($\sqrt{3}\times$2)R(30$^{\circ}$,
0$^{\circ}$)   in-plane    ordering     and    $\rm   I_c$        =   10.83
\AA.\cite{J140,kamitakahara84,K167}

        More details about the microstructure of the specimens used for the
superconductivity studies are given below.  The main point to be taken away
from the  discussion here  is that  the  synthesis of $\rm C_4KHg$  is very
sensitive to  the exact conditions used because  different  conditions will
produce a given combination of phases, as previously noted by Timp.\cite{J140}

        Even after several years of study of the question of the connection
between preparation conditions and the various phases, there are still many
uncertainties.   As an example,  notice in Table~\ref{stitc}  that the $\rm
T_c$ =  0.719 K and 0.822  K gold samples were  produced with nominally the
same conditions that usually give pink $\rm T_c$ = 1.5  K GIC's.  Also, the
$\rm T_c$ = 1.472 K and 1.317 K specimens came from the  same intercalation
ampoule.

        A few general statements about the synthesis of $\rm C_4KHg$ may be
made,   despite    some   uncertainties.   Intercalation  at   temperatures
significantly  higher  than 200$^{\circ}$C produces   samples  with   broad
superconducting  transitions that are more  or  less copper  in color.  The
intercalation temperature  of    350$^{\circ}$C used   by   Yang  {\em   et
al.\/}\cite{yang88}  would be  expected  to result  in specimens with quite
broad  superconducting   transitions.   Intentional  use of a   temperature
difference  ($\Delta$T)  on the   order of  10$^{\circ}$C does consistently
produce gold,  low-$\rm T_c$  specimens,  as originally noted  by  Timp and
coworkers.\cite{J140,timp83} Kim and coworkers  also observed that use of a
temperature difference  during intercalation increases the  final amount of
the $\rm I_c$ = 10.83 \AA\ $\beta$ phase.\cite{kim84}  The recent
experiments seemed to show that temperature  differences on  the  order of
5$^{\circ}$C were sufficient to produce  degradation in the superconducting
properties, in contrast to Timp's  claim that small temperature differences
enhance superconductivity.\cite{J140}

        The snag  in  this  neat  explanation is   that  gold samples  were
sometimes produced  using an isothermal reaction.  Whether  this was due to
poor control of the temperature  difference  between the alloy and graphite
is not known.  In addition, attempts to produce gold GIC's with $\rm T_c \:
\approx \: 0.7$K sometimes  resulted  in copper-colored  samples with  broad
transitions.  It seems likely that other poorly controlled factors were
also playing a role.

        One possible factor  is the overall time  allowed for the reaction.
In the  current   work,  and   also that of     Timp,\cite{timp83,J140} the
intercalation was  allowed to proceed for two  weeks or longer, whereas Iye
and collaborators\cite{iye82}  only  allowed  intercalation  to occur   for
``several days''.   Yang and coworkers also allowed  only  ``several days''
for their reactions.\cite{yang88}  The   Univ.  of  Nancy group  does   not
specifically state how long they allowed for intercalation of their samples
(which  were  used in   the specific heat  studies  of Alexander  {\em  et.
al\/}\cite{alexander81}). The reaction  time could be an important variable
if, for example, the higher-$\rm T_c$ phase is the equilibrium state of the
system, but its formation is slower than that of the lower-$\rm T_c$ phase.
If this were  the  case,  leaving the ampoule  in the furnace  longer might
result  in a  higher-$\rm T_c$.   Bendriss-Rerhrhaye of the   University of
Nancy notes  that in the CsBi-GIC's  the pure $\beta$  phase is produced if
the reaction is allowed to  proceed for  two weeks,  but that a  mixture of
$\alpha$ and $\beta$ is found  after only one week.\cite{bendriss86}In this
interpretation,  if for  some reason   a particular  batch of graphite  was
harder to  intercalate, KHg  intercalation compounds  formed  from it might
still be  in   the gold,  lower-$\rm T_c$  phase   after the usual reaction
period, even with $\Delta$T = 0.  Alternatively, a particular intercalation
might take longer if the KHg alloy used had  been  contaminated  by air and
had a refractory oxide coating.

        Another factor   in the intercalation  of $\rm  C_4KHg$   which has
received  little attention  in the literature    is  how the  intercalation
ampoule was transferred into and  out  of the furnace.  In the  experiments
reported here, the furnace  was always cooled  down  to about 70$^{\circ}$C
before  the ampoules were  removed.   Sometimes, though,  the furnaces were
cooled down overnight.  At the University  of Kentucky, it was customary to
remove ampoules  while they  were  still  hot.\cite{doll86} Sometimes   the
ampoules were even quenched in liquid nitrogen when  they were removed from
the  oven.\cite{delong82a,heinz83} It  is  not   known what  practice   the
Japanese,\cite{iye82}       French,\cite{elmakrini80a}   and       previous
MIT\cite{timp83} investigators followed  in this regard.    If the low-$\rm
T_c$ phase is  the equilibrium phase  at   higher  temperatures,  and the
higher-$\rm T_c$ phase is the equilibrium phase  at room temperature, then
removing the intercalation ampoule from the furnace while  hot might quench
in the high-temperature, low-$\rm T_c$ phase.

         Only recently did the MIT investigators  become aware of different
practices  at  different  institutions    regarding the removal     of  the
intercalation ampoule  from the  furnace.  Unfortunately, little  attention
was paid to this  point during the years  of  sample  synthesis recorded by
Table~\ref{stitc}.  Recently it was decided to study the effect of removing
the intercalation ampoules from the furnace hot,  but the result was sample
exfoliation in each of about 5 batches of $\rm C_4KHg$ thus  prepared.   It
is not understood why this exfoliation occurred, since it was  not a common
consequence of hot sample extraction at the Univ. of Kentucky.\cite{doll86}
At any rate, the results of this study were inconclusive.

        Annealing studies of $\rm  C_4KHg$  seem merited in  light  of  the
possibility that the higher-$\rm T_c$ GIC's  are an equilibrium  phase, and
the lower-$\rm T_c$ ones are metastable.  Particularly if the intercalation
reaction has  been hindered by slow kinetics,  it seems that putting a gold
low-$\rm  T_c$  sample   back   into   the    furnace   might  improve  its
superconducting  properties.  However, as    reported  by Lagrange  {\em et
al.\/}, the  thermal stability of $\rm  C_4KHg$  is very  low, even  in the
presence of excess amalgam, so that attempts at annealing usually result in
deintercalation.\cite{lagrange80a} The  difficulty of  thermally annealling
$\rm  C_4KHg$   is not surprising   in light  of   the exfoliation problems
reported above.

        A third factor in sample preparation which might well be thought to
influence $\rm T_c$ is the  quality of the  graphite host material.  In the
stage I  binary compound $\rm C_8K$,  $\rm  T_c$ was   found to be strongly
correlated with the type of graphite used.  That is, a $\rm  T_c$ of 0.15 K
appears       to      be        average           for        HOPG      host
material,\cite{koike80,kobayashi81,kaneiwa82} but  $\rm  T_c$ = 0.13 is the
highest     $\rm      T_c$           observed        for      Grafoil-based
samples,\cite{kobayashi79,kobayashi81a,sano80}  and  0.08 K was reported to
$\rm C_8K$ powder.\cite{koike80} In  the current  series of experiments, no
such correlation between graphite type and $\rm T_c$ was found; kish, HOPG,
and Madagascar graphite flakes could  all be transformed into gold  or pink
samples, depending  on the intercalation  conditions  used.  Defects in the
graphite could  be responsible for slight variations  in  $\rm T_c$ in $\rm
C_4KHg$, say between 1.5  and 1.55 K.  This would  imply that the  absolute
size  of the effect  of the graphite host  on $\rm  T_c$ (about  50  mK) is
constant between $\rm C_8K$  and $\rm C_4KHg$, rather  than  the percentage
change in $\rm T_c$ being constant (50  mK is  about 30\%  of $\rm T_c$ for
$\rm C_8K$, but only  about 3\% of $\rm  T_c$ of $\rm  C_4KHg$).  Certainly
the impact of the starting  graphite is not as  great for $\rm  C_4KHg$  or
$\rm C_8KHg$  as for $\rm C_8K$.

        The  primary conclusion  from  the  study of the synthesis  of $\rm
C_4KHg$ is that higher  intercalation  temperatures  tend to lead  to lower
$\rm T_c$'s.  Elevating the graphite temperature above 200$^{\circ}$C seems
to suppress $\rm T_c$, whether the alloy temperature is also raised or not.
However, raising the temperature of both the alloy  and  the graphite tends
to give  copper-colored samples with broad  superconducting transitions and
$\rm T_c \: \approx$ 1.0 K, while  raising only the alloy temperature tends
to give gold samples with narrower transitions and $\rm T_c \; \approx 0.8$
K.   These conclusions are  in  general agreement   with those of  Timp and
colleagues.\cite{J140,timp83} However,  the  synthesis of  $\rm  C_4KHg$ is
still not  entirely understood, since the  results of intercalation are not
altogether  predictable.  This is believed to  be due  to the importance of
factors whose importance is difficult to evaluate.  Among the possibilities
are the duration of  the intercalation reaction, hot  intercalation ampoule
extraction, and alloy cleanliness.

        Reproducibility problems with  the $\rm T_c$  of $\rm  C_4KHg$  are
undoubtedly due to the  several in-plane phases  it can contain.  The  wide
variation found  for $\rm C_4KHg$  contrasts with  the ease  of reproducing
results for $\rm C_8KHg$, which has only one in-plane phase.  One naturally
wonders  whether  the difficulty  of  reproducing superconductivity  in the
CsBi-GIC's is also  tied in  with the  question of multiple  phases.   This
question  is  discussed further  in Section~\ref{csbitc}.  The section that
follows, \ref{charac} treats the normal-state characterization  of the GIC's  used in the
superconductivity experiments.   The  relationship  between   the  in-plane
phases and superconductivity is the topic of Chapter~\ref{hydrog}.
