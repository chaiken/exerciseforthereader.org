\documentstyle[12pt]{report}
\input{/usr/local/lib/tex/macros/prepictex.tex}
\input{/usr/local/lib/tex/macros/pictex.tex}
\input{/usr/local/lib/tex/macros/postpictex.tex}
\pagestyle{empty}
\begin{document}
\begin{figure}
\beginpicture
%7a x-rays
\setcoordinatesystem units <0.3794cm,0.8cm>
\setplotarea x from 2 to 33, y from 0 to 10
\axis bottom label {2$\theta$} ticks 
	numbered from 2 to 32 by 6
	unlabeled short quantity 32 /
\put {a)} at 4 9
%13c x-rays
\setcoordinatesystem units <0.4497cm,0.8cm> point at 0 12 
\setplotarea x from 2 to 32, y from 0 to 10
\axis bottom label {2$\theta$} ticks 
	numbered from 2 to 32 by 6
	unlabeled short quantity 31 /
\put {b)} at 4 9
\endpicture
\caption[$(00\ell)$ x-ray scans for pink and gold $\rm C_4KHg$.]{$(00\ell)$
x-ray   scans for    pink and  gold   $\rm C_4KHg$.   The large peak   near
22$^{\circ}$ in each scan is from the copper sample holder.  The broad hump
from about 6$^{\circ}$ to  14$^{\circ}$ is due to the  glass tube  that the
sample holder is  in.  a) $\rm I_c$ =  $(10.22 \pm 0.03)$ \AA pink  sample.
b) $\rm I_c$ = $(10.18 \pm 0.03)$ \AA gold sample.}
\label{stixr}
\end{figure}
\end{document}
