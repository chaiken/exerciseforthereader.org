\documentstyle[12pt]{report}
\input{/usr/local/lib/tex/macros/prepictex.tex}
\input{/usr/local/lib/tex/macros/pictex.tex}
\input{/usr/local/lib/tex/macros/postpictex.tex}
\pagestyle{empty}
\begin{document}
\begin{figure}
\beginpicture
%6b x-rays, no H
\setcoordinatesystem units <0.6653cm,0.8cm>
\setplotarea x from 2 to 26, y from 0 to 10
\axis bottom label {2$\theta$} ticks 
	numbered from 2 to 26 by 6
	unlabeled short quantity 25 /
%6b x-rays, with H
\setcoordinatesystem units <0.6653cm,0.8cm> point at 0 12 
\setplotarea x from 2 to 26, y from 0 to 10
\axis bottom label {2$\theta$} ticks 
	numbered from 2 to 26 by 6
	unlabeled short quantity 25 /
\endpicture
\caption[$(00\ell)$ x-ray scans before and after hydrogenation for a gold
$\rm C_4KHg$ sample.]{$(00\ell)$ x-ray scans before and after hydrogenation
for a gold $\rm C_4KHg$ sample.  The  broad  hump from about 6$^{\circ}$ to
14$^{\circ}$ is due to the glass tube that the sample  is in.  Before: $\rm
T_c$ = 0.84  K gold $\rm C_4KHg$ sample.   $\rm  I_c$  = $(10.24 \pm 0.03)$
\AA.  After:  Same sample as in ``Before''  picture, only after exposure to
200 torr hydrogen  gas.  $\rm T_c$  = 1.535 K.   $I_c$ = $(10.24 \pm  0.03)$
\AA.}
\label{hydxrd}
\end{figure}
\end{document}
