\section{Introduction}
\label{prepintro}

        Sample preparation and characterization is quite an important topic
for the ternary  graphite-based superconductors.  A  thorough discussion of
specimen  synthesis  and  handling  is obligatory  not  only because it  is
customary, but because the details of these techniques appear to impact the
results    of  the experiments   discussed   in the   following chapters in
non-trivial ways.  Reports in the  literature on superconducting GIC's have
often been confusing or even contradictory for reasons possibly  related to
differences between the samples prepared  by the  various research  groups.
Because  of  the inadequate information  about   sample preparation in many
papers  (as in, ``The samples  were prepared using the  technique of Ref. X
\ldots''), one is often left  wondering whether in-house experiments  and
those by other groups were  performed on  comparable materials.  It  is the
aim of this  chapter to give enough  information about the samples  used in
the current  experiments so that anyone  who reproduces  these efforts will
not have to ask a similar question.

        The leaders in the field of ternary GIC discovery and synthesis are
the group at the University  of  Nancy.    Under  the leadership of  Profs.
H\'erold and Lagrange,  a team  of  students and  staff has  synthesized an
impressive array of ternary  GIC superconductors.   The first reported were
the   MHg-GIC's,     where M   stands    for    one the     heavy    alkali
metals.\cite{lagrange80}   Soon  after  the  same   authors   described the
preparation  of   the MTl-GIC's,\cite{lagrange80,elmakrini80a} and recently
they      announced      the  discovery    of     the  large    class    of
MBi-GIC's.\cite{lagrange85} All of these materials have been reported to be
superconducting.    Lagrange\cite{lagrange87}   has   recently reviewed the
lessons learned about the intercalation chemistry of  this entire  class of
compounds.

        It should be noted   that the  Nancy  group has  also   synthesized
ternary GIC's whose  superconductivity  seems  probable, but has not   been
experimentally verified.   Besides   the alkali-metal/heavy-metal ternaries
mentioned above,  whose  intercalant  sandwich  consists of  multiple metal
layers, there are   also ternaries  with  monolayer intercalant sandwiches.
Among these candidates are  solid solutions of known  GIC  superconductors,
such $\rm C_8K_{1-x}Rb_x$,\cite{billaud74} and others such as sodium-barium
GIC's\cite{billaud74a,billaud78} whose superconductivity seems quite likely
in  terms of existing   models.\cite{M143,fischer85}  These materials  have
received    a  lot of    attention  from  a   structural   characterization
standpoint,\cite{chow83,chow87a} but  few cryogenic  transport studies  have been
performed  on  them.  The reader will find more information on these
compounds in a recent review by Solin and Zabel.\cite{solin88}
There   is   also   a  whole   new    class     of
alkali-metal/antimony compounds  whose  low-temperature\cite{essaddek88,elmakrini88}  properties
are unknown.   Despite the  fundamental interest  in these materials, they
will  not be  discussed  further  here,    since  determination  of   their
superconductivity is left for future investigators.

        The superconductivity of the alkali-metal/heavy-metal ternary GIC's
is the focus of the remainder of this  work.  Sample preparation issues are
particularly  important  for   superconductivity  experiments  because they
presumably are the key to understanding why such a wide range of transition
temperatures   has  been  reported  for some    superconducting GIC's.  For
example, $\rm T_c$  for $\rm  C_4KHg$  has been  variously reported in  the
range from 0.7\cite{iye82}  to 1.6 K,\cite{N128}  while $\rm  T_c$ for $\rm
C_4CsBi_{0.5}$ has been  quoted from 4.05  K\cite{lagrange85} to  $\leq$0.5
K.\cite{E291} In the next few  sections an  attempt will  be made to relate
sample    preparation  conditions    to   reproducibility  problems    with
superconductivity experiments in GIC's.

