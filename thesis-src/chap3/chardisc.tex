\section{Discussion and Conclusions}
\label{chardisc}

        Besides the characterization measurements that were discussed in this
chapter,  several  additional experiments were  attempted that were not  as
successful.    Among   those attempted   are    optical   transmission  and
resistivity.

        Optical experiments  are an   obvious possibility considering  that
different $\rm T_c$'s correspond to different sample colors.  The apparatus
necessary to perform reflectivity measurements  is not readily available at
MIT,   so   it  was   decided   to  try  transmission measurements instead.
Transmission experiments on  $\rm C_8K$ and  $\rm C_{24}K$ were  previously
performed by Zhang and Eklund at the  University of Kentucky.\cite{zhang87}
The chief difficulty in optical transmission work with metals  is getting a
thin enough sheet that a measurable amount of light  can actually be passed
through.  In fact, preparing even a transparent  sheet of HOPG proved to be
tough,  let alone intercalating it in  a intact condition.  The lesson from
these efforts is that reflectivity  measurements would be much practical to
perform if a suitable apparatus could be located.

        Another  experiment for which a  need   is felt is  low-temperature
resistivity.  Surprisingly, in-plane resistivity measurements have not been
reported below 100 K on $\rm  C_4KHg$,\cite{elmakrini80} even though c-axis
resistivity    data  has      been   published   down    to   liquid-helium
temperature.\cite{fischer83} The possibility  of  a charge-density wave  in
low-$\rm T_c$  $\rm   C_4KHg$ samples\cite{delong83}   provides  plenty  of
motivation  for performing this  experiment.  In addition, more information
about normal-state  transport would be  helpful  in  order  to compare  the
critical field data to existing theories.\cite{orlando79}

        One attempt was made  to  measure the in-plane resistivity  of $\rm
C_4KHg$   from  room  temperature  to   liquid-helium   temperature.   Four
silver-paint  leads were placed  on the sample  inside a glovebag,  and the
sample was sealed inside a glass-tube with epoxy, a  technique developed by
previous students.\cite{Z260} Despite the fact  that the leads did not pull
off the a-face of the sample, and despite  the fact  that the sample showed
no  signs of discoloration  or deintercalation, a resistivity  about 10$^4$
times    the     previously   reported     value      of         20    $\mu
\Omega$-cm\cite{elmakrini80b} was measured at room temperature.   The leads
all appeared to  have the same resistivity,  so  no obvious  cause for  the
failure of this  experiment could  be identified.  Perhaps  silver paint is
simply too reactive to use  with $\rm C_4KHg$.  The  best  idea would be to
repeat this experiment using another contact medium, such as the gold paste
that has been successfully used with $\rm C_8K$.\cite{koike80}

        Both the reflectivity and resistivity experiments still  need to be
performed on $\rm  C_4KHg$.   However,  a lot has   been learned about  the
samples used in  the   superconductivity experiments  from characterization
efforts that  were successful.  From the x-ray experiments, it
was discovered that the low-$\rm  T_c$ and higher-$\rm  T_c$ specimens have
approximately  the same structure.    Good   agreement  was found with  the
assertion  of Timp\cite{J140} and Kim {\em   et  al.\/}\cite{kim84}  that a
temperature difference  during  intercalation  produces less-ordered  GIC's
with  a greater fraction  of $\beta$ phase.   It is worth noting that every
specimen  that showed any  $\rm C_4KHg$  peaks  in x-rays  was found  to be
superconducting with $\rm T_c \: \geq$ 0.7 K.  
        
        The neutron  scattering    experiments   showed   that   the  x-ray
diffraction experiments   were  somewhat  misleading,  in  that  x-ray data
sometimes indicated the presence of a single phase where neutrons indicated
two phases.  This finding may have important implications for understanding
the hydrogenation experiments.  Analysis of neutron diffraction data showed
slight differences between gold and  pink samples whose significance is not
immediately   clear.   Good general  agreement  was   found with   the previous
structural   analysis   of $\rm  C_4KHg$.\cite{yang84}

            Raman  scattering  experiments  on  $\rm   C_4KHg$   showed  no
difference between the pink and gold phases with experimental  error.  This
is  despite the  observation of  zone-folded phases  indicative of in-plane
ordering by Timp.\cite{N128} The lack  of  zone-folded peaks  in the recent
Raman studies could be due to surface disorder stemming  from laser-induced
heating or slight accidental exposure to air.

        Various measurements were also  performed to find  out the chemical
composition of the  specimens used for  the low-temperature and diffraction
experiments.  The proposal  that the low $\rm  T_c$ of  gold  $\rm  C_4KHg$
samples is caused by Hg deficiency\cite{H242} does not seem to hold up.  In
fact, support is found for  the idea that  the low-$\rm  T_c$  phase has  a
higher mercury/potassium   ratio,\cite{yang84} although other   factors are
probably also important.

        The  CsBi-GIC's  used in  the low-temperature  studies appear to be
very   similar  to  those made  in   other  laboratories.   The  x-ray  and
electron\cite{speck88z} diffraction    patterns,   color,    and   chemical
composition from RBS  all agree with the  results reported  by Lagrange and
coworkers,\cite{lagrange87}   the  original    discoverers  of   these  new
materials.   $\rm   T_c$  measurements  on   CsBi-GIC's are   described  in
Section~\ref{csbitc}.

        In  general  the current  study   of  the synthesis  of  KHg-   and
CsBi-GIC's     has       confirmed      the     findings     of    previous
investigators.\cite{lagrange87,J140,yang84}   Extensive    characterization
efforts  have  allowed the identification of  some correlations between the
structural and chemical properties of a GIC and its superconductivity.  The
superconductivity measurements are   described  in detail in the  following
chapters.
