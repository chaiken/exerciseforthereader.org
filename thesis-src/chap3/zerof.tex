\subsection{Zero-Field $\rm T_c$ Measurements in $\rm C_4KHg$}
\label{zerof}

        The  preparation for the  zero-field  $\rm T_c$ measurements was as
follows.  After a  batch of KHg-GIC's  was removed from  the furnace in its
intercalation  ampoule,  the  individual  GIC's were transferred  inside an
inert-gas-filled glovebox to 4mm   Pyrex tubes.  When  the  specimens  were
intended for  critical  field studies, they were  mounted in special sample
holders before placement in  the glass tubes  (see Section~\ref{mounting}).
Once the specimens  were  inside  the glass tubes,  the tubes were attached  to a
stopcock via  a Cajon quick-connect vacuum fitting.    They were then taken
out of the glovebox and attached to a  pumping station, where they could be
evacuated to  pressures on the  order of 10$^{-5}$   torr with a  diffusion
pump.   Once the ampoules were evacuated,  they were  filled with about 200
torr of He gas which had been passed through a liquid-nitrogen cold trap to
remove water.  This gas  was intended  to  provide  thermal contact of  the
sample   to  the  glass  tube,   thus   making   the cryogenic  temperature
measurements more accurate.  The He-filled tubes were then  sealed off with
a gas/oxygen torch.

        Once the GIC's were  in individual  tubes, they  were characterized
for staging fidelity with $(00\ell)$ x-rays (see Section~\ref{xrd}).  Those
that  were well-staged   and uniform in   color  were taken to the National
Magnet Laboratory  for  zero-field $\rm T_c$   and
critical field measurements.  The zero-field $\rm T_c$ measurements
were performed using the same cryogenics, electronics, and thermometry that
were employed in the critical field studies.  This apparatus is described
in Section~\ref{critf:exp}.

        $\rm  T_c$   was determined  by measuring     the voltage  from  an
inductance bridge as a function of  temperature  from 4.2  K down to 0.4 K.
For  materials  in which a higher  $\rm T_c$ than  4.2 K  seemed plausible,
higher temperatures were achieved  by bleeding  warm He gas into the liquid
helium   cryostat.  The  transition   was  easily  recognized   as  a large
reproducible decrease in  the inductive voltage.  Signal-to-noise ratios of
1000 were not uncommon.

        $\rm  T_c$ was  defined   to  be the  intersection of  a line drawn
tangent to the most linear part of the transition with the level upper part
of the transition.  This definition, which  is a standard one for inductive
measurements,\cite{iye82}   is   illustrated  by  Figure~\ref{tcdef}.   The
transition width, $\rm \Delta T_c$, was defined  as the interval between the
10\% and 90\% completion temperatures of the transitions.  The
dimensionless figure of merit used to characterize sample quality is $\rm
\Delta T_c/T_c$.  This ratio is a $\rm T_c$-independent measure of how
narrow a specimen's superconducting transition  is.  $\rm \Delta  T_c/ T_c$
ranged from  about 0.2 for poorly ordered  samples  to about  10$^{-2}$ for
well-ordered specimens.

\begin{figure}
\vspace{12cm}
\caption[Experimental definition of $\rm T_c$.]{Experimental definition of
$\rm T_c$.}
\label{tcdef}
\end{figure}

        $\rm T_c$ and  $\rm  \Delta T_c/T_c$ are  reported for a  number of
samples in Table~\ref{tcfreq}.   $\rm \Delta  T_c/T_c$ mostly  increases
with decreasing $\rm T_c$, which suggests  that the low-$\rm T_c$ GIC's are
less  homogeneous than  the higher-$\rm  T_c$  ones.   The  observation  of
broader transitions in  lower-$\rm T_c$ samples  is in keeping with the idea
that the lower-$\rm T_c$ specimens  contain  both the $\alpha$  and $\beta$
phases, while the higher-$\rm  T_c$ specimens contain  only the $\alpha$ phase.

          If the $\beta$ phase is responsible for  $\rm T_c$ suppression in
$\rm  C_4KHg$,  one might expect  that  samples  might  sometimes  have two
separate  superconducting transitions, the first  at 1.5 K  and a second at
0.8 K.   Yet two separate transitions  have never  been observed in  a $\rm
C_4KHg$ sample, although very broad  transitions have sometimes been  seen.
An interpretation for the lack of  two  transitions is suggested by the TEM
work performed by Timp\cite{K167}.  The TEM micrographs consistently showed
that  the in-plane phases were  intermixed on a  microscopic scale.  If the
regions  of $\beta$ phase are  small  and are surrounded by $\alpha$-phase,
they  may   well  be  proximity-coupled to    the   $\alpha$-phase regions.
Proximity coupling between the  lower- and higher-$\rm  T_c$ regions of the
specimen can give  a single transition, just  as  is observed.  Caution  is
necessary  when discussing the   intermixture   of phases,  though, since a
high-enough degree of admixture   along  the  c-axis would give   broadened
$(00\ell)$   x-ray   lineshapes.\cite{hendricks42}   As  the    spectra  in
Section~\ref{xrd} show, these broadened lineshapes have not been observed.
        
\begin{table}
\caption[Transition temperatures $\rm T_c$ and reduced widths $\rm \Delta
T_c/T_c$ for a sampling of $\rm C_4KHg$ specimens.]{Transition temperatures $\rm T_c$ and reduced widths $\rm \Delta
T_c/T_c$ for a sampling of $\rm C_4KHg$ specimens.}
\label{tcfreq}
\begin{center}
\begin{tabular}{|ccc|}
\hline
& & \\
$\rm T_c$ (K) & $\rm \Delta T_c/ T_c$ & color \\
& & \\
\hline
& & \\
0.719 & 0.20 & gold \\	% 5c
& & \\
0.726 & 0.20 & gold \\  %3b
& & \\
0.822 & 0.17 & gold \\	% 5d
& & \\
0.875 & 0.073 & gold \\ %6b
& & \\
0.949 & 0.20   & gold \\  %13c
& & \\
1.198 & 0.21 & copper \\	%11b
& & \\
1.317 & 0.138  & pink \\	%9c
& & \\
1.443 & 0.075  & pink; from Timp \\ %timp's sample
& & \\
1.46 & 0.096 & pink \\	%5a
& & \\
1.472 & 0.03 & pink \\	%9b
& & \\
1.508 & 0.1 & pink \\  %5b
& & \\
1.526 & 0.042 & pink \\	%7a
& & \\
1.528 & 0.047 & pink \\	%8c
& & \\
1.536 & 0.04 & pink \\	%8b
& & \\
1.539 & 0.02 & pink \\	%8a
& & \\
\hline
\end{tabular}
\end{center}
\end{table}

        A sceptic might claim that the color of the $\rm C_4KHg$ samples is
a  property only of  the surface layers,  and is not  really  linked to the
superconductivity.  This  is not an unreasonable  idea since the $(\sqrt{3}
\times \sqrt{3})$R30$^{\circ}$ superlattice has been tentatively identified
as a  surface phase  in Section~\ref{ramdata}.  However,  the gold color is
strongly correlated with   the  presence of   the  $\beta$ phase    in  the
$(00\ell)$ x-rays.  Furthermore, the color is as  good  a predictor of $\rm
T_c$    as  the x-rays   are,   as Table~\ref{tcfreq}  shows.  Lagrange and
coworkers have also reported the existence of the gold color in association
with the $\beta$ phase.\cite{lagrange83} They report that repeated cleaving
of a ``gold'' sample will expose both pink and gold surfaces in turn.

        From these considerations  and the evidence  in Table~\ref{tcfreq},
it is  clear that gold  samples  have lower  $\rm   T_c$'s than pink  ones.
Furthermore, this $\rm T_c$ depression seems to  be linked  to the presence
of  the $\beta$ phase.   The possible  effect  of  the   two phases of $\rm
C_4KHg$   on  superconductivity   measurements  is   discussed  further  in
Chapter~\ref{hydrog}.  The differences in  the critical  fields of the  two
types of $\rm C_4KHg$ are detailed in Chapter~\ref{critf}.


