\subsection{Raman Characterization of $\rm C_4KHg$}
\label{ramdata}

        The motivation for the use of Raman  scattering comes from previous
Raman   studies  of  the  different phases  of $\rm  C_4KHg$   by  Timp and
coworkers.\cite{N128} An extensive  transmission electron  microscopy (TEM)
and Raman study   by Timp {\em   et al.\/}  found  zone-folded Raman  modes
associated with in-plane ordering.  The identification  of Raman peaks with
zone-folded modes  can give a clue  to the in-plane  ordering of  GIC's, as
explained below.

        Zone-folded  modes  are zone-center   (q  =  0) phonons   found  in
superlatices.   These  modes originate from the folding   of the
graphite bands into the zone-center that occurs in superlattice phases that
have  a  larger real-space  lattice constant  than  graphite,  and  hence a
smaller Brillouin zone.  Only q=0 modes  participate in Raman processes due
to the requirement of wave-vector conservation and  the very small momentum
of a photon compared to that of a typical phonon.   In principle,\cite{F120}
and in practice,\cite{N128} the comparison of the  positions of zone-folded
Raman  peaks    with theoretical models   allows  identification   of   the
corresponding in-plane superlattice.

        A Raman study of $\rm C_4KHg$ was performed by  Timp {\em et al.\/}
in Ref.~\cite{N128}.  The essence  of this paper  is that the  pink samples
are identified primarily with a $(2
\times   2)$R0$^{\circ}$  in-plane  superlattice,  while  red  samples  are
identified  primarily with   a  $(\sqrt{3}  \times  \sqrt{3})$R30$^{\circ}$
ordering.\cite{N128}  Figure~\ref{c8kstruct}  shows  the  atomic  positions
corresponding to these  in-plane  arrangements.  The  gold samples  show no
evidence  of  zone-folded peaks at   all.\cite{N128} Both the  pink and red
phases were found to contain as a minority constituent a  $(\sqrt{3} \times
2)$R(30$^{\circ}$,  0$^{\circ}$).    The   observation of the   $(2  \times
2)$R0$^{\circ}$  and  $(\sqrt{3}  \times   2)$R(30$^{\circ}$,  0$^{\circ}$)
phases is in good  agreement with neutron diffraction,\cite{kamitakahara84}
x-ray   diffraction,\cite{elmakrini80}   and  TEM\cite{K167}   experiments.
Kamitakahara\cite{kamitakahara84}   found     using    $(10\ell)$   neutron
diffraction spectra  that   the $(2 \times 2)$R0$^{\circ}$  and  $(\sqrt{3}
\times 2)$R(30$^{\circ}$, 0$^{\circ}$) phases correspond to the $\rm I_c$ =
10.24  \AA\ and $\rm  I_c$   = 10.83 \AA\  repeat  distances.   The  x-ray,
neutron, TEM and  Raman  experiments  together unequivocally identify   the
10.24  \AA\ $\alpha$-phase  with  the ($2  \times 2$)R0$^{\circ}$  in-plane
structure, and the 10.83 \AA\ $(\beta)$-phase with the ($\sqrt{3}
\times 2$)R(30$^{\circ}$, 0$^{\circ}$) in-plane structure.

        The $(\sqrt{3}  \times \sqrt{3})$R30$^{\circ}$ phase has  only been
ob\-served  in  the TEM and  Raman\cite{K167} experiments, though.  TEM shows
that  the   $(\sqrt{3}  \times   \sqrt{3})$R30$^{\circ}$  phase  appears to be
associated with an anomalous $(9.4 \pm 0.01)$ \AA\ repeat distance which is
not seen   in  x-ray or  neutron  diffraction   experiments,  even by Timp.
Timp\cite{K167}  also finds  a   $(\sqrt{3}  \times \sqrt{3})$R30$^{\circ}$
phase in $\rm C_8KHg$  associated with  an anomalous $(12.8  \pm 0.1)$ \AA\
repeat distance.  The usual lattice constant for $\rm  C_8KHg$ is $\rm I_c$
= 13.6 \AA.\cite{kamitakahara84b,K167} Because TEM and Raman scattering are
more surface-sensitive than  x-ray and neutron diffraction, the observation
of the $(\sqrt{3} \times \sqrt{3})$R30$^{\circ}$ structure only in  the TEM
and Raman experiments suggests that this phase might be found only near the
sample surface.  Although the evidence for the  existence of the $(\sqrt{3}
\times \sqrt{3})$R30$^{\circ}$ phase is  compelling,  none of this evidence
proves that this structure is found in bulk KHg-GIC's.

        New Raman spectra were taken by Dr.  G. Doll on pink and  gold $\rm
C_4KHg$ samples at room temperature.  The GIC's were HOPG-based and about 3
mm  by 3mm by  1  mm in  size.   An argon   ion  laser's 4880 \AA\ line was
focussed to a spot on the surface.  Care was taken to limit the laser power
to less than 10 mW  to avoid damaging  the samples.  Spectra taken  on $\rm
T_c$   =  1.53  K   and  a   $\rm T_c$ =  0.719  K   GIC's   are  shown  in
Figure~\ref{ramfig}.  The  spectra were taken over a  much wider   range of
frequencies than is indicated in the Figure, but the  only peak observed in
either sample is the one displayed.  This peak corresponds to the E$_{2g2}$
mode of graphite.\cite{N128} The frequency of this  peak in the new work is
about   10    cm$^{-1}$ higher than    in  the   experiments   of Timp  and
coworkers.\cite{N128}  Raman spectra were taken  on 6 other specimens whose
data is  not   shown, but no    zone-folded peaks were   observed.    X-ray
diffraction scans taken  after the Raman spectra showed  no degradation  in
staging.

\begin{figure}
\vspace{7.5in}
\caption[Raman spectra on $\rm T_c$ = 0.719 K and  $\rm T_c$ = 1.5 K $\rm
C_4KHg$.]{Raman  spectra on gold  and pink $\rm C_4KHg$.  a)  Spectrum of a
$\rm T_c$ = 0.719 K gold sample with a single $\rm I_c$ value  = 10.14 \AA.
The peak frequency is 1597.1 cm$^{-1}$ and the HWHM is  13.2 cm$^{-1}$.  b)
Spectrum of a $\rm T_c$ = 1.53 K pink sample with  a single $\rm I_c$ value
= 10.22 \AA.  The peak  frequency is 1593.9  cm$^{-1}$ and the HWHM is 15.5
cm$^{-1}$.}
\label{ramfig}
\end{figure}

        It was  hoped that the  observation of  zone-folded Raman peaks  in
$\rm C_4KHg$ would  allow  a systematic study  of the relationship  between
in-plane   structure    and superconductivity.      As    was   shown    in
Section~\ref{xrd},  the c-axis structure seemed  to provide  no information
about the superconductivity since the $(00\ell)$ scans of low-$\rm T_c$ and
high-$\rm T_c$ specimens were identical.  Unfortunately  the new results on
$\rm C_4KHg$ Raman  spectra also show  the low-$\rm T_c$ and high-$\rm T_c$
samples to be identical, in that  no  zone-folded  peaks were  seen  in any
spectrum.  The lack of  zone-folded   peaks would seem   to  imply in-plane
disorder at  least at  the  surface  in these  samples.   Since  $\rm  T_c$
measurements were taken on the same samples as the Raman spectra, and since
there was no sample transfer in-between the  $\rm  T_c$ measurement and the
Raman experiment, one can conclude that the Raman experiments are much more
sensitive to the in-plane order than the superconductivity  is.  The reason
may well be that, as  mentioned, Raman is a fairly  surface-sensitive probe
of    ordering, while  the $\rm    T_c$   measurements are   sensitive   to
superconductivity anywhere in  a sample's cross-section  (see Figure~\ref{scingfract}).  Thus any exposure
to  air that might  have occurred  during  the sample  handling might  have
affected only the surface,  which  would  show up in the  Raman but not the
superconductivity.

        Hydrogenated $\rm C_4KHg$ specimens also showed only one peak in
Raman scattering, a Lorentzian E$_{2g2}$ line at about 1598 cm$^{-1}$.  The
spectra of the hydrogenated samples were basically indistinguishable from
those of the pink and gold samples.  A low-temperature Raman study of these
different types of $\rm C_4KHg$ might provide more information.

        Like  the  $(00\ell)$  x-ray data,  the  Raman data show  almost no
difference between the lower-$\rm  T_c$  and higher-$\rm T_c$ $\rm  C_4KHg$
specimens.  Therefore they can provide little insight into the relationship
between the in-plane   structure and superconductivity.  In  the  hopes  of
answering these questions it was decided to  try neutron diffraction. Since
neutrons are quite sensitive  to  hydrogen, neutron diffraction data  taken
before and after hydrogenation might offer important clues to the puzzle.
