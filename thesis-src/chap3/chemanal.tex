\subsection{Chemical Analysis of KHg- and CsBi-GIC's}
\label{chemanal}

        The determination  of  the stoichiometry of  graphite intercalation
compounds is an important aspect  of their characterization.  The reason is
that     intercalation  is a  slow    chemical  process,   particularly for
higher-stage and   ternary  compounds.  Even   if an intercalation reaction
proceeds for several weeks at an elevated temperature, the final product (a
GIC)  may  not  have reached   its   equilibrium  state.  The  hypothetical
metastable final state  of the GIC  might be a mixed-stage  compound, which
wouldn't  be so bad, since mixed-stage   samples are easily detectable with
$(00\ell)$ x-rays.    Mixed-stage samples  can be  separated  out  from the
well-staged ones and discarded.

        A  much   more  insidious   situation arises   if  the  compound is
well-staged but is off the equilibrium stoichiometry.  For one thing, there
is  the  question of what the  equilibrium stoichiometry is, a longstanding
problem with acceptor GIC's.\cite{C159} In principle, stoichiometries can be
determined from the x-ray diffraction pattern, as was done with the neutron
scattering data.  However, to have real  confidence in  the stoichiometries
found from fits  to $(00\ell)$  integrated intensities, one needs  either a
large number  of  peaks, or extensive  knowledge of  the  structure  of the
material in question.  Especially in  the case  of  multiphase systems like
the KHg-  and  CsBi-GIC's, an independent     check of  the  stoichiometries
determined from diffraction data is highly desirable.

        One  type of  non-destructive chemical analysis  that has been used
with  great success  on  GIC's is   Rutherford backscattering  spectrometry
(RBS).  RBS provides information  on composition only in  about the top one
micron of a specimen,\cite{S175} but it has the  advantage of doing so in a
structure-independent way.  Also, since RBS gives  depth resolution as well
as species identification, any alloy adsorbed on the surface can readily be
differentiated from the bulk.  RBS measurements were performed on KHg-GIC's
by Salamanca-Riba {\em et al.\/}, who found stoichiometries of  $\rm C_{3.0
\pm 1.5}KHg_{0.77 \pm  0.08}$ for stage  1, $\rm C_{7 \pm 1.4}KHg_{0.72 \pm
0.07}$ for stage 2, and  $\rm C_{13 \pm  1.3}KHg_{0.62 \pm 0.06}$ for stage
3.  The low ratio of carbon atoms to potassium atoms found in this study is
somewhat surprising, but the margin of error is so  large  that no definite
conclusions can be drawn.  However, the tentative C/K ratio of 3.0 supports
the existence of the ($\rm \sqrt{3}  \times \sqrt{3}$)R30$^{\circ}$ phase,
for  which 3.0  is  the   stoichiometric C/K ratio.   For the ($2    \times
2$)R0$^{\circ}$  and   ($\sqrt{3} \times  2$)R(30$^{\circ}$,   0$^{\circ}$)
phases, a ratio of four is anticipated.  Once again, the ($\rm \sqrt{3}
\times \sqrt{3}$)R30$^{\circ}$ phase seems to be turning up in a
surface-sensitive experiment; this phase had previously been seen only in
Raman scattering and TEM studies.\cite{N128}  Because of the nature of the
measurements in which the ($\rm \sqrt{3}
\times \sqrt{3}$)R30$^{\circ}$ phase has been seen, one might wonder if
its appearance is related to deintercalation due to  sample heating.

        The better-bracketed  parameter found by   RBS is the   Hg/K ratio,
which appears to be about 0.7.  The stoichiometries found for the KHg-GIC's
in several  different experiments are summarized  in Table~\ref{stoichtab}.
As is evident from the Table, a Hg/K  ratio of 0.7  is less than any of the
values established from the x-ray or neutron diffraction experiments.  This
Hg/K ratio was only determined on one GIC of  each stage,\cite{S175} so its
magnitude may be affected by sample dependence.  However, the agreement for
all  three samples of different stage  is impressive.   In  fact, one might
wonder if the constant Hg/K ratio with such a  widely  varying C/K ratio is
not a  sign that another property of  the surface rather  than  the bulk is
being  measured.   This  possibility   was put forward   by the  authors of
Ref.~\cite{S175},  who noted that  surface depletion  of mercury from  $\rm
C_4KHg$ under heating had   been   observed using  x-rays.\cite{A157}   The
explanation of  the tendency toward surface  depletion of mercury  is  that
since it  intercalates after the  potassium does,\cite{A157,elmakrini80} it
must  also  deintercalate first.   This  idea is believable   also  because
mercury has a high vapor pressure and is rather  chemically inert, with the
result  that its gradual  surface depletion  is a common phenomenon in many
types of materials.  For example, surface mercury depletion is a problem in
HgCdTe wafers used in infrared detectors.\cite{hgcdte83}

        RBS measurements  have also   been performed on  the CsBi-GIC's  in
collaboration with Dr. J. Steinbeck.\cite{E291} The experimental conditions
were similar to  those  of  Ref.~\cite{S175}, except that  the  GIC's  were
cleaved before measurement to remove the alloy adhered on the surface.  For
an $\alpha$-phase  stage 1 CsBi-GIC  sample, the result of this  experiment
was  a  stoichiometry $\rm  C_{(2.5  \pm  0.65)}CsBi_{0.59   \pm 0.03}$, in
ex\-cell\-ent  agree\-ment with  the  pre\-vi\-ously  re\-ported Bi/Cs ra\-tio  of   $\rm
C_4CsBi_{0.55}$.\cite{lagrange87,yang87} The accord between the Cs/Bi ratio
found here and that  determined by x-rays\cite{lagrange87,yang87} increases
one's confidence in the Hg/K ratio found from RBS, although  there is not a
problem with volatility  in   the   CsBi-GIC's  the  way that  there  is in
mercury-containing materials.  The problem in  synthesizing the  CsBi-GIC's
is signalled by the unexpectedly low C/Cs ratio, which is  an indication of
the tendency  to have macroscopic alloy inclusions  in the samples, even when
the  surfaces  are  cleaned of any  adhered  metal.   These  inclusions are
discussed further in Section~\ref{csbitc}.

\begin{table}
\caption[Stoichiometries of the KHg-GIC's as determined from various
experiments.]{Stoichiometries  of the KHg-GIC's  as determined from various
experiments.  ND = not determined by this experiment;  NR = not reported by
the authors of the cited reference.}
\label{stoichtab}
\begin{center}
\begin{tabular}{|c|ccccc|}
\hline
& & & & \\
Compound & C/K ratio & Hg/K ratio & Experiment & Ref.\\
& & & & \\
\hline
& & & & \\
gold  & ND & $(1.09 \pm 0.07)$ & diffraction  & this work\\
& & & & \\
pink  & ND & $(0.96 \pm 0.03)$ &  diffraction  & this work \\
& & & & \\
majority  & NR & 1.0 & diffraction  & \cite{yang84}\\
& & & & \\
minority  & NR & 1.3 &  diffraction  & \cite{yang84} \\
& & & & \\
majority  & 1.0 & 1.0 &  diffraction  & \cite{elmakrini80}\\
& & & & \\
majority  & ($3.0 \pm 1.5$) & ($0.77 \pm 0.08$) & RBS &  \cite{S175}\\
& & & & \\
$\rm T_c$ = 0.73 K  & 4.74 & 1.2 & chem. anal. & this work \\
& & & & \\
$\rm T_c$ = 1.3 K   & 3.33 & 0.8 & chem. anal. & this work\\
& & & & \\
$\rm T_c$ = 1.54  & 4.3 & 1.05 & chem. anal. & this work \\
& & & & \\
\hline
\end{tabular}
\end{center}
\end{table}

        Another  non-destructive  measurement that    can be   performed on
air-stable samples is weight uptake.   If the mass  of the host graphite is
determined before intercalation,  and the stoichiometric  ratio  of the two
intercalant  metals is known from another  measurement, such  as  RBS, then
after intercalation the extent to  which the reaction  was completed can be
estimated.  For $\rm C_4CsBi_{0.55}$,  a simple calculation  shows that the
expected weight uptake  is 5.2 times  the  graphite mass.   Measured weight
uptakes were usually  on the order  of 5.5 to 5.6  times the graphite mass,
even after careful cleaning of the  surface.  This excess weight is another
indication of the inclusions indicated  by the  RBS and TEM  studies.  (TEM
observations of the inclusions are discussed in Section~\ref{csbitc}.)

        While  it  would   be   highly  desirable  to perform weight-uptake
measurements on the   KHg-GIC's, this  has not been   done  because of  the
compounds'  high reactivity  in air.  An ideal  set-up would  be to  have a
microbalance inside a dry-box for routine weight-uptake  measurements after
intercalation,   a  practice    carried   out    at  the  University     of
Kentucky.\cite{doll86}   Weight uptake  data   on   $\rm  C_4KHg$ would  be
particularly interesting to  compare  to measurements of the areal fraction
of superconductivity from inductive studies  of  the zero-field transition.
(See  Section~\ref{critf:exp} for an  explanation of how the areal
fraction of superconductivity is determined.)

        The next-best-thing to weight uptake and RBS  for the determination
of  stoichiometry is  wet chemical  analysis.   Wet  chemical analysis  was
performed on  $\rm C_4KHg$ GIC's  by Dr.  W.   Correia  of  the  Center for
Materials Science  Chemical Analysis Facility.   The  basic procedure is to
dissolve the specimen  in   a   solution and  precipitate out  the  various
constituents  separately  for weighing.  The  contribution   of each of the
elements  in the  sample  can   be  determined   this way.     While  these
measurements  can  be fairly  accurate, and  have the advantage  of  giving
information about the  bulk of the specimen, they  also  have the  distinct
disadvantage of  destroying  the sample.    Therefore, while  wet  chemical
analysis is useful,  it has  been performed on  only a  few specimens.  The
results are included in Table~\ref{stoichtab}. 

        Drawing conclusions from the data in the Table is difficult.  There
does not seem to  be a strong connection  between Hg content and $\rm T_c$,
contrary  to much    speculation  that  Hg-deficient   samples  have  lower
transition temperatures.\cite{H242}  The  lowest $\rm T_c$  sample actually
has the  highest Hg/K ratio  of  1.2.  The observation that gold lower-$\rm
T_c$ specimens have  more $\beta$ phase than higher-$\rm  T_c$ specimens is
therefore      consistent   with    the      work  of    Yang    {\em    et
al.\/},\cite{yang84,yang88} who found the $\beta$ phase of $\rm C_4KHg$ has
a higher Hg  content  (of  1.3).  The  $\rm T_c$  =  1.3  K  sample has   a
composition that agrees well with the RBS measurements, while the $\rm T_c$
=  1.54  K   sample agrees  in   composition   with  the neutron scattering
measurements.  All  the specimens appear to  be  close to the expected C/Hg
ratio   of  4.  A  systematic study   of   any possible   relation  between
stoichiometry and  superconductivity  still needs to  be performed on fully
characterized samples.  The available evidence  does not make a strong case
for the importance of variable stoichiometry since most of the samples have
chemical   formulae    close    to    that    originally reported   by   El
Makrini.\cite{elmakrini80}

        Of course the most important characterization that needs to be done
on  superconducting materials  is    the   measurement  of  the  zero-field
superconducting transition temperature, called $\rm T_c$  here.  Zero-field
$\rm T_c$ experiments are discussed in the next section.
