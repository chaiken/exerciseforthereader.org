\subsection{Neutron Diffraction Studies of $\rm C_4KHg$}
\label{neutrons}

        The original intention of the neutron scattering experiments was to
do a  full structure determination of  the different phases of $\rm C_4KHg$
through Rietvelt analysis of  a powder diffraction pattern.  Unfortunately,
the  neutron absorption cross-section of mercury  is so high that  taking a
powder pattern turns  out to be impractical.\cite{neumann88}

        The next idea was to look at $(hk0)$ in-plane diffraction rings for
the  two  types  of $\rm C_4KHg$  samples.    It was   anticipated that the
lower-$\rm T_c$ gold samples might show less in-plane  ordering, or perhaps
simply a different intercalant arrangement, and that this arrangement would
change during {\em in  situ \/} hydrogen  exposure.   Because neutrons  are
sensitive to  hydrogen, unlike x-rays  or electrons, this experiment could
even allow the   determination of the hydrogen positions,   should they  be
ordered.  Both  the  $\alpha$ and $\beta$  phases  were seen  using neutron
diffraction   by Kim  and   coworkers\cite{kim84}  and  Kamitakahara    and
coworkers,\cite{kamitakahara84b}  so the   prospects for success   of  this
experiment seemed quite good.

        In  order to have  sufficient intensity to see the  desired $(hk0)$
peaks within a reasonable amount of data collection time, large samples are
required.  Initial attempts    to  synthesize HOPG  samples  of   the  size
(2$\times$2$\times$0.05)   cm$^3$  were   unsuccessful  because  of  severe
inhomogeneities, much  greater  than those  found in smaller samples.   The
intercalation compounds formed after 4 or 5 weeks in the furnace were black
and still 0.05 cm thick on one end, and obviously gold  (or pink) and quite
thick on the other end.  Despite the fact that these  samples (5 batches of
them) were somewhat  intercalated, they showed  no $(00\ell)$ x-ray  peaks,
and so were deemed unacceptable  for neutron  diffraction.   Interestingly,
small HOPG pieces and Madagascar graphite flakes in the  same intercalation
ampoule  formed well-ordered $\rm C_4KHg$.   Presumably  the big pieces did
not intercalate uniformly because of small temperature gradients across the
graphite.  In order to prevent thermal gradients from developing, the large
HOPG pieces in the second set of  batches  were packed in glass wool inside
the  intercalation ampoule.   These batches were   left in  the furnace for
about six weeks.  The specimens from the  second set of batches looked more
uniformly intercalated than those from the initial attempts.

        The samples were  mounted in special   gas-loading  cans compatible
with cryostats of the type designed at the  Institut  Laue-Langevin.  These
cryostats are similar  to  normal liquid-helium ones, such  as   the Janus
Varitemp  dewar, except that  they have aluminum  ``windows'' at the bottom
for the admission of neutrons.  The  gas-loading cans were  designed by Dr.
D.    Neumann  of the National  Bureau  of  Standards   (NBS)  to allow the
admission of up to ten atmospheres gauge pressure of  gas, more than enough
for  the  hydrogenation  experiment.    (The details  of  the hydrogenation
experiment are  discussed in    Chapter~\ref{hydrog}.)  The   samples  were
affixed   with aluminum   foil or  vacuum grease    inside thin rectangular
aluminum chambers  at the bottom  of the  cans.   The   chambers had indium
o-ring seals.  The rest  of the  gas-loading can consisted of a   series of
valves and 1/8" stainless-steel tubing connected to the sample chamber with
standard Cajon VCR fittings.

        After the GIC's were mounted in the cans,  they were transported to
NBS in Gaithersburg,  Maryland, where  they were   examined  using the BT-4
triple-axis neutron diffractometer.   Thermal  neutrons of wavelength 1.528
\AA\ were  monochromated  by   reflection   off an HOPG   polycrystal   and
collimated  with standard neutron  optics.  The diffracted beam was counted
by a $^3$He detector which was interfaced to a PDP-11 lab computer.

        Unfortunately on  the  first trip  to  NBS,  no large $\rm  C_4KHg$
samples  had  been produced which had $(00\ell)$  x-rays.  Therefore, small
HOPG-based  specimens   which  had     been  used  in  the  low-temperature
measurements were used  instead.   On the second   trip to NBS, there  were
problems with  the reactor, and so no  data was taken  due  to the  lack of
neutrons.  By the time neutrons were available (about 3 weeks), the samples
had gone bad, probably due  to slow leaks  through the indium  seals.  Thus
the only data that was taken was on small (3mm $\times$  3mm $\times$ 1 mm)
GIC's, which prevented the observation of $(hk0)$ rings or inelastic phonon
peaks.  The neutron diffraction data consists of $(00\ell)$ scans on a pink
sample, a gold sample, a gold sample with hydrogen, and a gold  sample with
deuterium.  In addition, rocking scans were performed to measure the mosaic
spread of each specimen.  The  mosaic spread  of these HOPG-based specimens
varied between  1.7$^{\circ}$ and 2.7$^{\circ}$,  which are  fairly typical
values for GIC's.\cite{neumann88} $(00\ell)$ x-ray and  neutron spectra for
the unhydrogenated gold sample are shown in Figure~\ref{14cneut}.

\begin{figure}
\vspace{20cm}
\caption[X-ray and neutron diffraction spectra of a gold $\rm C_4KHg$
specimen.]{X-ray and neutron diffraction $(00\ell)$ spectra of  a gold $\rm
C_4KHg$  specimen.   The $\beta$-phase peaks are  marked with $\downarrow$.
a) Only the $\rm I_c$ = 10.24 \AA\ $\alpha$ phase is  clearly visible using
x-rays.  The small bump to the left of the $(002)$ may be the $\beta$-phase
$(002)$ peak.  b) With neutrons, both the $\rm  I_c$ =  10.24 \AA\ $\alpha$
phase  and  the $\rm I_c$  =  10.83 \AA\   $\beta$ phase show  well-defined
peaks.}
\label{14cneut}
\end{figure}

        The  neutron  diffraction pattern  shown in  Figure~\ref{14cneut}b)
implies that the $\beta$-phase  constitutes   about 10\% of the   specimen.
This is somewhat surprising since the x-ray  data in Figure~\ref{14cneut}b)
shows only one  set of peaks.  A   careful scrutiny of  the x-ray  spectrum
allows one to identify a candidate for the $\beta$-phase $(002)$  peak, but
nonetheless one would guess from the x-ray data  that  the $\beta$-phase is
present   only in a   very small amount.  The  unexpected  presence of  the
$\beta$-phase  in this sample invites the  question, Do   all  gold samples
contain some  $\beta$ phase?  It   seems  that  x-rays  are insufficient to
answer this question.  Partly this  is due  to the scattering of the x-rays
by the sample-encapsulating tube, but it is  also partly due to the limited
skin depth of x-rays.  Timp  made  a  related  comment  in reference to his
magnetoresistance measurements: ``$\ldots$the conventional $(00\ell)$ x-ray
diffraction-characterization technique  is  an  inadequate  measure of  the
homogeneity   of the  staging  order for    SdH  measurements, because  the
magnetoresistance measurement for  homogeneous sytems  is sensitive to  the
bulk  properties of  the   compound   while   molybdenum $K_{\alpha}$ x-ray
radiation has an extinction depth  of only  $\approx35 \: \mu$m (in stage-1
KHg-GIC's).''\cite{W179}  The  lack of  correlation between  the $(00\ell)$
x-rays and $\rm T_c$ suggests that  Timp's comments about magnetoresistance
apply equally well  to superconductivity.  Thermal neutrons,  on  the other
hand, have a penetration depth of about  one centimeter, which is more than
enough to see the  whole thickness  of a GIC.\cite{neumann88} The idea that
all gold stage I  KHg-GIC's contain the  $\beta$ phase may be important for
understanding    the     hydrogenation  experiments,     as   discussed  in
Chapter~\ref{hydrog}.

        An  attempt was made   to add   hydrogen to the   gold sample whose
spectrum is shown in Figure~\ref{14cneut}.  Unfortunately the  peaks of the
sample  could   not be found   after its  transfer  from a glass  tube to a
gas-loading can.  This transfer was probably unsuccessful because there was
no glovebox available at NBS, and the  transfer is too time-consuming for a
glovebag.   Instead  of  the {\em  in   situ\/}  hydrogen addition,  a gold
specimen which had  already  been exposed to hydrogen was  studied  without
transfer to a gas-loading can.  The $(00\ell)$ data from this sample, whose
$\rm T_c$ was 0.84 K before  hydrogenation and 1.54 K  afterward, are shown
in Figure~\ref{6bneut}a).  The  $(00\ell)$ peaks  of an unhydrogenated pink
$\rm    C_4KHg$   specimen   are      presented     for   comparison     in
Figure~\ref{6bneut}b).  These spectra are  believed to be the first neutron
diffraction scans on  $\rm  C_4KHg$ which  show only  the   $\alpha$ phase;
previous experiments by Yang\cite{yang84}  and Kim\cite{kim84} showed  both
$\alpha$ and $\beta$ phase peaks, just as Figure~\ref{14cneut}b) does.  The
spectrum of the deuterated sample is not shown here because these data were
marred  by problems with sample motion  during the scan,  while   the other
neutron scans were unaffected by sample movement.

\begin{figure}
\vspace{20cm}
\caption[Neutron diffraction spectra of a gold  $\rm C_4KHg$ specimen with
hydrogen and a pink  sample without hydrogen.]{Neutron  diffraction spectra
of a gold $\rm C_4KHg$ specimen with hydrogen  and a pink  specimen without
hydrogen.  Note the lack of $\beta$ phase in either sample.   a) Spectrum of
a gold sample whose $\rm T_c$ was 0.84  K  before  hydrogenation and 1.54 K
afterward.  b) Spectrum of a pink $\rm C_4KHg$ sample.}
\label{6bneut}
\end{figure}


        The  two  spectra shown  in  Figure~\ref{6bneut}  do not  look very
different from one another.  In fact, none of the three neutron diffraction
patterns shown  here look very  different  from one another except for  the
presence  of the $\beta$-phase peaks  in the  gold sample without hydrogen.
In order to be able to intercompare the spectra in a more quantitative way,
the integrated  intensities of the  diffraction   patterns  were  fit to  a
5-layer model\cite{lagrange83a} of the unit   cell,  using a  gradient-least
squares  optimization  algorithm     to  obtain   the   residual-minimizing
parameters.  Only the majority  $\alpha$-phase intensities were fit because
the number of peaks in the $\beta$ phase was too small.

        The  method used to  fit the  integrated intensities is  a standard
one, and is similar to that used by Yang and  coworkers  to fit their $\rm
C_4KHg$ neutron  data.\cite{yang84}   Essentially  the   method  is the
following:  the data are  converted   to  relative  structure factors  $\rm
F^{exp}$ by dividing the integrated intensities through by the intensity of
the highest peak.  The experimental relative structure factors are then
compared to the calculated ones, which are obtained from\\

\begin{equation}
\label{stfact}
\rm F^{cal}_{(00\ell)} \; = \; \sum_i A_i N_i \exp{ \left[ -B_i \left( \sin \theta_{\ell}/ \lambda \right)^2 \right]} \: \cos \left( \frac{2 \pi z_i \ell}{I_c} \right)
\end{equation}

\noindent where the sum is taken  over the 6 layers of the model. Thus the
subscript i stands for the properties of the element in layer i.  A$\rm _i$
is the neutron scattering  factor  of the element in layer  i, N$\rm _i$ is
its stoichiometry (with  respect  to carbon  $\equiv$ 4), B$\rm _i$ is  its
Debye-Waller factor, $\theta_{\ell}$ is the diffraction angle, $\lambda$ is
the wavelength of the neutrons, and z$\rm _i$ is the c-axis position of the
ith layer with respect to the center  of the sandwich.   The definitions of
the various layer positions are illustrated in Figure~\ref{zeedef}.

\begin{figure}
\vspace{12cm}
%Figure from Yang's 1984 MRS paper
\caption[Definition of the distances used in the fits to the neutron
diffraction data.]{Definition of the distances $\rm z_i$ used  in  the fits
to  the neutron diffraction  data (from Ref.~\cite{yang84}).  The distances
are  measured from the center of  the sandwich, halfway between the mercury
layers.  $\Delta$z, the Hg layer splitting, is twice the distance of the Hg
layers from z = 0.}
\label{zeedef}
\end{figure}


        Once  the theoretical structure  factors  were calculated  using  a
given set of  parameters  and  Eqn.~\ref{stfact}, the  quality  of fit  was
quantified  using a standard  expression for  the residual  parameter $\cal
R$.\cite{hamilton65}The  definition  of  $\cal  R$  used  here  weights the
individual structure factors  by  their experimental uncertainties, and  is
considered  superior.\cite{ali88}  The $\cal R$ used here is called the
normalized standard deviation and is given by:\cite{hamilton65}

\begin{equation}
\label{hamilton}
 \cal R \rm \; \equiv \; \left[  \frac{\sum_{\ell} w_{\ell} \left( F_{\ell}^{exp} \: - \: |F_{\ell}^{calc}| \right)^2}{\sum_{\ell} w_{\ell} \left(F_{\ell}^{exp} \right)^2} \right]^{1/2}
\end{equation}


\noindent Here $\rm F_{\ell}^{exp}$ is the observed structure factor, $\rm
F_{\ell}^{calc}$ is the calculated structure  factor, and $\rm w_{\ell}$ is
the experimental  uncertainty in $\rm  F_{\ell}^{exp}$.   This experimental
uncertainty   comes  from instrumental    factors  and  the height   of the
background   in the  spectrum.    The   definition   of  the  residual   in
Equation~\ref{hamilton} is  different     from  that  used   by  Yang   and
coworkers.\cite{yang84}  Therefore the new  residuals reported  here should
not be compared with those of  Ref.~\cite{yang84}, but should only  be used
for  intercomparison.   A  gradient least-squares routine\cite{bevington69}
was used  to optimize  the parameters by  minimizing the magnitude of $\cal
R$.

        Most of the parameters used to calculate the structure factors were
known beforehand.  The neutron scattering lengths A$\rm _i$ can be found in
any standard   nuclear data  reference  book,\cite{mughabghab73}   and  the
Debye-Waller    factors were   taken    from   the   work   of   Yang   and
coworkers.\cite{yang84} These  Debye-Waller factors  were $\rm B_C$  = 1.66
\AA$^2$, $\rm B_K$ = 0.78
\AA$^2$, and $\rm B_{Hg}$ = 4.75 \AA$^2$. $\rm I_c$ was found from the peak
positions in the  $(00\ell)$ scans, and so was  not used as a parameter  in
fitting the integrated intensities.   The potassium atoms scatter  neutrons
so  weakly  compared to  carbon  and   mercury  that  the K positions   and
stoichiometry have only a  small   effect on   the fit.  Therefore  the K/C
relative stoichiometry was  fixed at 0.25.  The remaining  parameters  were
z$_K$, the position of the potassium layer, $\Delta$z,  the position of the
mercury layer, and N$\rm _{Hg}$, the stoichiometry of the mercury layer.

        In  Table~\ref{neutnum},  the values  of these parameters  obtained
from fits to  the diffraction patterns  shown in Figures~\ref{14cneut}  and
\ref{6bneut}b)  are  compared  with   the  numbers found  by  Yang  {\em   et
al.\/}\cite{yang84} The   residual index  found for the    fit to the  pink
sample's data is  0.036, while that found for  the gold sample is 0.044. 

\begin{table}
\caption[Parameters obtained from fits to the $\rm C_4KHg$ neutron
diffraction data.]{Parameters  obtained  from  fits   to  the $\rm  C_4KHg$
neutron diffraction  data.   $\rm N_{Hg}$  is the number  of Hg atoms per 4
carbon atoms.  Additional parameters for the  hydrogenated sample: hydrogen
position = 3.9 \AA;  hydrogen/carbon ratio  = 1.0.  The parameters reported
for the hydrogenated sample are not well constrained  since the residual in
this case is considerably higher than for the other diffraction patterns.}
\label{neutnum}
\begin{center}
\begin{tabular}{|ccccc|}
\hline
& & & & \\
Type & $\rm I_c$ (\AA) & $\rm N_{Hg}$  & $\rm z_K$ (\AA) & $\Delta$z (\AA) \\
& & & & \\
\hline
& & & & \\
gold, majority  & 10.24 & 1.09 & 1.997 & 0.184 \\
& & & & \\
pink, majority  & 10.24 & 0.96 & 2.287 & 0.0 \\
& & & & \\
hydrogenated, majority  & 10.24 & 1.3 & 2.4 & 0.24 \\ 
& & & & \\
majority\cite{yang84} & 10.24 & 1.0 & 2.1 & 0.0 \\
& & & & \\
minority\cite{yang84} & 10.83 & 1.3 & 2.8 & 0.0 \\
& & & & \\
majority\cite{elmakrini80} & 10.16 & 1.0 & 2.34 & 0.25 \\
& & & & \\
\hline
\end{tabular}
\end{center}
\end{table}

        For  the  hydrogenated sample, there   were other  free  parameters
besides those  listed in the table.   These  were the   hydrogen  position,
stoichiometry, and Debye-Waller factor.  A ballpark starting  value for the
H Debye-Waller factor,  4.58  \AA$^2$,  was  estimated  from  the inelastic
neutron-scattering data taken on $\rm  C_4KH_x$  by Kamitakahara, Doll, and
Eklund,\cite{solin88}        using    a     procedure     suggested      by
Kamitakahara.\cite{kamitakahara88} (This procedure is briefly  described in
Appendix~\ref{hydrogdw}.)  Unfortunately,   even   with all these  free  parameters,
attempts to  fit these data with  reasonable  numbers produced  unacceptably
high residual values, on the order of 0.2.  Here  ``reasonable'' is defined
very liberally to  mean that the  stoichiometries and atomic distances were
all  required  to  be positive.  To obtain even   $\cal  R$ =  0.2,  it was
necessary to use a hydrogen stoichiometry of 1.0 and a hydrogen position of
3.9 \AA, closer to the graphite plane than the potassium plane.  Therefore,
the values obtained from  the fit to the  hydrogen diffraction  pattern are
included only for completeness, and are not to be taken too seriously.

        The failure of the 6-layer  model to fit  the hydrogen data is  not
easily understood.  It does not seem likely that  hydrogen causes a drastic
rearrangement  of the  atomic positions along  the  c-axis considering that
$\rm  I_c$ is  unchanged  by hydrogenation within experimental  error  (see
Figure~\ref{hydxrd}).  The best course  would probably be to fit $(00\ell)$
x-rays to determine the positions and stoichiometries  of  the non-hydrogen
layers.  These other layers' parameters could then  be fixed in fitting the
neutron diffraction pattern,  where only the hydrogen   parameters would be
varied.  The   available   x-ray    diffractometer  cannot  provide  enough
$(00\ell)$ x-ray peaks to determine all the K and Hg parameters, though.

        It  was  decided  to synthesize  hydrogenated   single-crystal $\rm
C_4KHg$ specimens so that their entire $(hkl)$ spectrum could  be imaged in
one scan with an x-ray  precession camera.\cite{speck88z}  Then there would
be enough peaks to determine at  least the K and Hg  stoichiometries, which
could then  be used as input to  the fits for  the hydrogenated samples. In
the one attempt made at a precession camera study of an unhydrogenated $\rm
C_4KHg$ single crystal, promising results were obtained.  This work was not
completed due to time constraints, but it deserves a more sustained effort.

        At this point it is most fruitful to  return to a discussion of the
lessons of   Table~\ref{neutnum}.    The first  major   point   is the good
agreement between  the   parameters obtained  for the  pink   $\rm  C_4KHg$
specimen and those found by  Yang {\em et al.\/}\cite{yang84} These samples
appear to have been basically identical.   The gold  sample appears to have
been slightly more mercury-rich than the pink specimen  and Yang's majority
phase.  The parameters for the gold  sample also indicate  that its central
mercury layer was  slightly split, but that  its potassium plane was closer
to the center of the sandwich.  Considering  that the differences among the
parameters of the  various samples are  small,  and that each type of  $\rm
C_4KHg$ has been measured only once, any conclusions drawn from the numbers
in this table must be considered quite tentative.

        In order to get a better feel for  the real-space structure implied
by  the    numbers in   the    table,    the   structure factors    can  be
Fourier-tranformed to yield some information  about  the nuclear positions.
The analytic form of the Fourier transform is

\begin{equation}
\label{denscalc}
\rm \rho(z) \; = \; \sum_l \: F^{exp}_{(00\ell)} \: \cos \left( \frac{2\pi
\ell z}{I_c} \right)
\end{equation}

\noindent where $\rm \rho(z)$ is the nuclear scattering intensity as a
function  of position along the   c-axis.\cite{yang84} In order to  do this
calculation,  one must  find out  the  correct  signs  for the experimental
structure factors  from the  calculated  ones,  since    the ratios of  the
integrated  intensities are necessarily  positive.    The  density map that
results     from  this    construction     is     called  a       Patterson
function.\cite{speck88z}

        Eleven peaks  is rather  few  to  perform  such a  transform  since
aliasing will occur on length scales of the order $\rm I_c$/11$\approx$ 0.9
\AA.  Therefore, if a good fit to the data has  been obtained, a reasonable
procedure is to use calculated structure factors to extend the experimental
structure factors up to higher $\ell$.  This extension was  performed up to
$\ell$ = 50  for  the pink and gold  samples.   Extending  the data of  the
hydrogenated sample seemed  to be a questionable idea  considering the poor
quality  of the  fit.   Patterson functions  were  also  calculated by  the
authors of Ref.~\cite{yang84}, so comparison with their transforms can also
be made.  Such comparisons are shown in Figure~\ref{rhofig}.

\begin{figure}
\vspace{19cm}
\caption[Real-space structure of  $\rm C_4KHg$ along the
c-axis.]{Real-space structure of the majority  phase of $\rm  C_4KHg$ along
the c-axis as calculated from the Fourier transform of the extended neutron
diffraction data.  All plots were  scaled to a  carbon peak height  of 1.0.
a) Plot of nuclear scattering  intensity versus distance along the graphite
c-axis  for  a pink sample ($\ast$)  and  a gold   sample  ($\circ$).  b) A
similar plot comparing the structure  of the  MIT pink sample ($\ast$) to a
structure ($\circ$) calculated from fit parameters reported by Yang {\em et
al.\/}\cite{yang84}}
\label{rhofig}
\end{figure}


        While the Patterson plots shown in  Figure~\ref{rhofig} do not give
any  new information, they   do  serve   to  deliver   the information   in
Table~\ref{neutnum} in more easily  digestible form.  There do  seem  to be
small differences between  the pink and gold  samples that  are larger than
the  difference between  the  pink sample and   Yang's  parameters for  the
majority phase. Since these measurements have  only been performed on a
few specimens, it is necessary to be very cautious.   The apparent shift in
the potassium peak position between  the pink specimen  and Yang's specimen
is  meaningless because the  potassium layer contributes  so  little to the
overall scattering intensity.

        The only distinction  between the higher-$\rm  T_c$ pink GIC's  and
the lower-$\rm T_c$ gold ones that can be  made with confidence is that the
gold samples are more likely to contain the minority  $\beta$ phase.   This
assertion is well-founded because it  corroborates  an observation  already
made with $(00\ell)$  x-rays.  Because $\beta$-phase  peaks  were  observed
with neutrons in a specimen which appeared  single-phase from x-rays, it is
tempting to  conclude that all  gold lower-$\rm  T_c$  $\rm C_4KHg$ samples
contain the $\beta$ phase.  The  paper of Yang  and coauthors suggests some
structural   differences  between  the  majority  and minority phases.  The
possible  consequences for  superconductivity  of  the slightly higher $\rm
I_c$  and    Hg  content   of  the  $\beta$     phase  are   discussed   in
Chapter~\ref{hydrog}.

        There are other ways to measure the stoichiometry of compounds that
provide   an  important  check  on  the parameters   obtained from fits  to
diffraction data.  Among  these are Rutherford backscattering  spectrometry
(RBS), weight uptake measurements, and wet chemical analysis.  The results
of these experiments on GIC's are discussed in the next section.
