\section{Normal-State Char\-ac\-ter\-iza\-tion of KHg- and CsBi-GIC's}
\label{charac}

\subsection{Characterization with $(00\ell)$ X-Rays}
\label{xrd}

        The  most common  method  for staging   determination  in GIC's  is
$(00\ell)$ x-ray diffraction.\cite{I94} This technique  was also applied to
the KHg- and   CsBi-GIC's,  although  more for determination of   the phase
($\alpha$ and/or  $\beta$)  than for identification  of stage.  Mixed-stage
KHg or CsBi specimens were never at any time produced in the course of this
study, although a  given  stage  might  contain the admixture  of   several
different phases.  The lack of stage admixture is  not  surprising for $\rm
C_4KHg$, which is formed from the intercalation of K and Hg vapor into $\rm
C_8K$,\cite{A157,elmakrini80} but is somewhat surprising for  the stage I
$\rm C_4CsBi_{0.5}$ and  $\rm C_4CsBi_{1.0}$,  which are formed from higher
stage CsBi-GIC's.\cite{bendriss86b}

        The  presence of   the  peaks of   only one phase  in  a $(00\ell)$
diffraction  was the most  common  occurrence.   However, gold $\rm C_4KHg$
samples sometimes  contained  the $\rm I_c$  = 10.83 \AA\  $\beta$ minority
phase as well as the $\rm  I_c$ = 10.24  \AA\ $\alpha$ majority phase.  The
x-rays scans of many gold samples contained  only the majority-phase peaks,
though.  Kim {\em  et al.\/} note  that their experiments indicate that use
of  a   small  temperature  difference  $\Delta$T  increases the  amount of
$\beta$-phase.  This  finding  which is consistent  with Table~\ref{stitc},
which  shows that  gold samples   are typically intercalated   with a small
$\Delta$T.      Predominantly    $\beta$-phase       specimens    are never
produced.\cite{lagrange83} Occasionally $\rm C_8K$  ($\rm I_c$ =  5.35 \AA)
was also observed  in   the diffraction scans   of  gold-colored   stage I.
Samples which contained $\rm C_8K$ were never  used in further experiments,
though.

        The occasional presence of $\rm C_8K$  in gold but not pink samples
is intriguing in light of the  hypotheses presented  in Section~\ref{synth}
about  the synthesis of the different  types of $\rm  C_4KHg$.   One of the
ideas put forward in Section~\ref{synth}  to explain the occurrence of gold
samples  was that  they  were  the product of  an  incomplete intercalation
reaction.   As   mentioned  above,  $\rm   C_4KHg$   is   formed   from the
intercalation of  K   and Hg   vapor  into  $\rm     C_8K$,  which    forms
first.\cite{timp83,elmakrini80} If the gold  samples are the  result  of an
incomplete reaction,  one would expect  to see   $\rm C_8K$ peaks   in gold
specimens, but not in the completely reacted pink specimens.  As noted above,
this is exactly what is observed.

        The  $(00\ell)$ scans  shown  here   were taken  using   molybdenum
K$_{\alpha}$  radiation with   a wavelength  of  0.708    \AA.   Molybdenum
radiation was chosen because it  passes relatively unattenuated through the
glass tubes in which the samples were encapsulated.  The radiation from the
Mo anode passed through a 1.0$^{\circ}$ horizontal collimation  slit on its
way to the   GIC, and a ``High-Resolution''  Soller  slit and 0.1$^{\circ}$
vertical  collimation slit on  its way  to the detector,   which was of the
extrinsic silicon type.  The diffractometer used is part of  the Center for
Materials Science  X-Ray Facility   at MIT, and is  of standard design.

        Typical $(00\ell)$ scans for $\rm C_4KHg$ pink ($\rm T_c$ = 1.53 K)
and  gold ($\rm  T_c$ = 0.95 K) specimens  are shown in Figure~\ref{stixr}.
These plots show intensity versus the angle  2$\theta$ between the incident
and diffracted beams.    The diffractograms  for both the   pink  and  gold
samples show   peaks  from  the   copper  sample  holders  used    for  the
low-temperature  measurements  (discussed in   Section~\ref{mounting}) near
2$\theta$ = 22$^{\circ}$ and broad humps from the glass tube that surrounds
the sample from about 6$^{\circ}$ to  about 14$^{\circ}$.  These  scans are
quite  similar to  those  shown by Timp\cite{timp83}  and  Lagrange {\em et
al.\/}\cite{lagrange80b}.   The magnitude of  $\rm I_c$ was  estimated from
using Bragg's  Law using the separation between  the  most intense  pair of
peaks, the (002)  and  (004).  The average value of  $\rm  I_c$ obtained in
these measurements, 10.20
\AA\ ,  is    in   accord   with the    $(10.12  \pm   0.03)$   \AA\   found by
Timp\cite{timp83},  the   10.24   \AA\    found by     Univ.   of    Kentucky
researchers,\cite{yang88}     and the     10.16    \AA\      reported      by
Lagrange.\cite{lagrange80}  However, Lagrange shows  the intensity   of the
(004) peak  higher   than  the  intensity   of  the  (002),   contrary   to
Ref.~\cite{timp83} and   Figure~\ref{stixr}.  This  reversal  of the  usual
intensity ratios was seen for some  of the samples used  in this study, but
it appeared that the effect was due to improper  centering of  the specimen
on the goniometer.

\begin{figure}
\vspace{20cm}
\caption[$(00\ell)$ x-ray scans for pink and gold $\rm C_4KHg$.]{$(00\ell)$
x-ray   scans for    pink and  gold   $\rm C_4KHg$.   The large peak   near
22$^{\circ}$ in each scan is from the copper sample holder.  The broad hump
from about 6$^{\circ}$ to  14$^{\circ}$ is due to the  glass tube  that the
sample holder is  in.  a) $\rm I_c$ =  $(10.22 \pm 0.03)$ \AA\  pink
sample.  $\rm T_c$ = 1.53 K.  b) $\rm I_c$ = $(10.18 \pm 0.03)$ \AA\  gold
sample.  $\rm T_c$ = 0.95 K.}
\label{stixr}
\end{figure}

        $(00\ell)$  x-ray diffraction was also  performed before and  after
hydrogenation on both pink and gold specimens.  Hydrogenation  appeared not
to affect $\rm    I_c$, as  Figure~\ref{hydxrd}   shows.   The scan   after
hydrogenation  has a larger (004)  peak than (002),  while the scan  before
hydrogenation has  a  larger (002) peak.  As  noted  before, this  reversal
appears to be an alignment-dependent  effect.  Other diffraction data taken
after hydrogenation data on this sample show the (002) peak higher.   There
does not seem to be any reproducible difference between the $(00\ell)$
patterns before and after hydrogenation.

\begin{figure}
\vspace{20cm}
\caption[$(00\ell)$ x-ray scans be\-fore and after hyd\-rog\-en\-ation for a gold
$\rm C_4KHg$ sample.]{$(00\ell)$ x-ray scans be\-fore and after hyd\-rog\-en\-ation for a gold
$\rm C_4KHg$ sample.  The  broad  hump from about 6$^{\circ}$ to
14$^{\circ}$ is due to the glass tube that the sample  is in.  Before: $\rm
T_c$ = 0.84  K gold $\rm C_4KHg$ sample.   $\rm  I_c$  = $(10.24 \pm 0.03)$
\AA.  After:  Same sample as in ``Before''  picture, only after exposure to
200 torr hydrogen  gas.  $\rm T_c$  = 1.535 K.   $I_c$ = $(10.24 \pm  0.03)$
\AA.}
\label{hydxrd}
\end{figure}

        $(00\ell)$   scans     for  two      CsBi-GIC's   are    shown   in
Figure~\ref{csbixray}.   a)  shows  the x-rays   of a   $\rm C_4CsBi_{0.5}$
$\alpha$-phase specimen with $\rm I_c$ = 10.61 \AA, and  b) shows a similar
scan taken  by  Bendriss-Rerhrhaye.\cite{bendriss86} These scans  look different
because   the spectrum of  Bendriss   is taken   with $\theta$
increasing from right to left, and the MIT spectrum is taken with 2$\theta$
increasing from left to right.   $\alpha + \beta$-phase specimens were also
synthesized which had an additional set of peaks corresponding to $\rm I_c$
=  11.43 \AA.  The MIT values  for $\rm I_c$  compare favorably   with the
$(10.61 \pm 0.02)$ \AA\ for $\alpha$-phase and $(11.48 \pm  0.02)$ \AA\ for
$\beta$-phase reported by  McRae {\em et  al.\/}\cite{mcrae86} There is not
any obvious difference  between the  two sets of $(00\ell)$'s  despite  the
fact that  $\rm T_c$  =  4.05 K  is reported for  the University  of  Nancy
samples and  $\rm T_c$  $<$ 0.5  K for the  MIT  specimens.   (There are more
details about the superconductivity of CsBi-GIC's in Chapter~\ref{csbi}.)

\begin{figure}
\vspace{19cm}
\caption[$(00\ell)$ x-ray scans for $\alpha$-phase  stage 1 CsBi-GIC's.]{$(00\ell)$ x-ray scans for $\alpha$-phase  stage 1 CsBi-GIC's.  The broad hump
from about 6$^{\circ}$ to 14$^{\circ}$  is due to  the glass tube  that the
sample holder is in.  a) $\rm I_c$ = $(10.61 \pm 0.03)$ \AA\ $\alpha$-phase
sample.    b)      Similar        $(00\ell)$      scan     taken         by
Bendriss-Rerhrhaye.\cite{bendriss86}   In  b), $\theta$  is increasing from
right to left, whereas in a) 2$\theta$ is increasing from left to right.}
\label{csbixray}
\end{figure}

        In  both  the  case  of  the  KHg-GIC's  and  the  CsBi-GIC's,  the
$(00\ell)$  scans show essentially  no difference between low-$\rm T_c$ and
high-$\rm T_c$ samples.  In an attempt to  find some structural information
that would  correlate with  $\rm  T_c$,  Raman  scattering experiments were
performed.   The  results of these  experiments are  described  in the next
section.
