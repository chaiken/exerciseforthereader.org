\begin{center}
\begin{large}
SUPERCONDUCTING PROPERTIES OF TERNARY GRAPHITE INTERCALATION COMPOUNDS\\
\medskip
by\\
\medskip
ALISON CHAIKEN\\
\end{large}
\end{center}

\noindent Submitted to the Department of Physics on \today, in partial\\
fulfillment of the requirements for  the degree of Doctor  of Philosophy in
Physics.\\

\section*{Abstract}
\label{abstract}

An  extensive study  of   the  superconducting properties  of  the  ternary
graphite   intercalation   compounds has   been  carried out,  with special
emphasis on the potassium-mercury (KHg) and cesium-bismuth  (CsBi) systems.
The specimens were synthesized using a variety of conditions, and an effort
was  made to correlate the  preparation conditions with the superconducting
properties.  Superconducting transition temperatures  $\rm T_c$ between 0.7
and  1.5 K  are found  for   $\rm  C_4KHg$, while  no superconductivity was
observed in  the compound $\rm C_4CsBi_{x}$,  contrary to previous reports.
For $\rm C_4KHg$  lower transition temperatures and broader superconducting
transitions are associated with the presence of the minority $\beta$ phase.
Neutron and  x-ray  diffraction analysis  show no  significant   difference
between the majority $\alpha$-phase regions  in the lower-  and higher-$\rm
T_c$ specimens.  The increase of $\rm T_c$ in the KHg-GIC's with increasing
stage  (and  therefore    decreasing    intercalant  concentration)  is  in
contradiction  to  expectations   from  theories  of    the superconducting
proximity effect.  The  trend of $\rm T_c$  in the KHg-GIC's is interpreted
as evidence  for   the participation of  both graphitic   and   intercalant
electrons in  the  superconductivity.  The   upper   critical field   phase
boundary $\rm H_{c2}(\theta,  \: T)$ was   measured on  both  types of $\rm
C_4KHg$ specimens.  The  probable observation  of type I  superconductivity
for some field orientations is reported for the $\rm T_c$ = 1.5  K samples.
The  critical field  data  are   analyzed in  terms   of  the   anisotropic
Ginzburg-Landau   model,  and  evidence for  extended  linearity  of   $\rm
H_{c2}(T)$ and temperature-dependent anisotropy  is presented.  Comparisons
to more detailed  models  of anisotropic superconductivity  are  made where
appropriate.   Hydrogenation  experiments on $\rm  C_4KHg$ show  a dramatic
increase in $\rm T_c$, quite similar to a  $\rm T_c$ enhancement seen under
applied pressure.  The effect of hydrogenation is  tentatively explained to
be the suppression of a charge-density wave state.  This assignment is made
by  analogy  to  the transition   metal dichalcogenides, which  have   many
features in  common with the  superconducting GIC's.  Possible  reasons for
the lack of superconductivity in the CsBi compounds  are discussed, with an
emphasis on implications for the other superconducting GIC's.

\begin{list}{}{\setlength{\rightmargin}{\leftmargin}}
\item[Thesis Supervisor:]  Mildred S. Dresselhaus
\item[Title:] Institute Professor
\end{list}
