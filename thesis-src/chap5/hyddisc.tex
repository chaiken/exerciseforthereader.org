\section{Possible Explanations for the Hydrogen-induced $\rm T_c$ Enhancement}
\label{hyddisc}

        Any model for the effect of hydrogen on $\rm C_4KHg$  must not only
account  for   the $\rm T_c$   increase but also  be  consistent with other
relevant experiments.  Before conjecturing  what the cause  of the  rise in
$\rm T_c$ might be, it is useful to review some of these other experiments.

        Most   researchers who   have worked with the  KHg-GIC's   have not
measured   the superconducting transition  temperature.    This is entirely
understandable considering  the time-consuming and expensive nature  of the
$^3$He experiments.  Unfortunately, the lack of $\rm  T_c$ characterization
in most papers on $\rm C_4KHg$  limits their helpfulness  for understanding
the sample dependence  of $\rm T_c$  and the  hydrogen-induced enhancement.
Unless the type  of $\rm C_4KHg$ sample used  by another group is known, it
is hard to know how their data fit into the overall picture.

        Ideally, the most helpful type of information on $\rm C_4KHg$ would
be measurements  of   a  physical  property (besides $\rm  T_c$)   that was
different in pink and unhydrogenated gold samples, but the same in pink and
hydrogenated gold  samples.  Despite many  attempts to find such a property
(described in  Section~\ref{charac}),  no  such  measurement  is  currently
known.

        One    set  of   data  does  have   special significance    for the
interpretation  of the hydrogenation  experiments: the measurements of $\rm
T_c$ as a function of applied pressure in the KHg-GIC's performed by DeLong
and                        collaborators.\cite{delong82,delong82a,delong83}
Figure~\ref{stitcvsP}a) shows the superconducting transition  as a function
of pressure for a $\rm C_4KHg$ specimen which had $\rm  T_c \: \approx$ 1.3
K and $\rm  \Delta T_c$/$\rm  T_c \, \approx$ 0.31  at room pressure.   The
dramatic sharpening  of  the transition (final  $\rm  \Delta T_c$/$\rm T_c$
$\approx$ 2e-2) and shift  of $\rm T_c$ to  approximately  1.5  K under the
very  small applied pressure  of  0.8 kbar  are remarkably similar  to  the
effect of hydrogenation.  The smallness of an applied  pressure of 0.8 kbar
can  best be appreciated  by noting that $\rm dT_c/dP$  in $\rm  C_8KHg$ is
about -6.5$\times 10  ^{-5}$ K/bar,\cite{delong82a}  so that  0.8  kbar  of
applied pressure shifts $\rm T_c$ in $\rm C_8KHg$ by only  about 50 mK.  In
both the hydrogenation and pressure  experiments,  a minute perturbation to
the sample radically  narrows the superconducting transition  and increases
$\rm T_c$.

\begin{figure}
\vspace{20cm}
\caption[Pressure dependence of $\rm T_c$ in KHg-GIC's]{Pressure
dependence  of  $\rm T_c$  in KHg-GIC's.    From  Ref.~\cite{delong83}.  a)
Pressure-induced  transition narrowing  in  $\rm  C_4KHg$.  Notice that the
application  of a small pressure, 0.8  kbar, increases $\rm  T_c$ to 1.5 K,
while application  of  further pressure decreases  $\rm    T_c$  at a  rate
dT$\rm_c$/dP = -5$\times10^{-5}$ K/bar. b)  Monotonic decline of $\rm  T_c$
with   pressure in $\rm C_8KHg$.  $\rm  dT_c/dP \,  = \, -6.5\times10^{-5}$
K/bar.}
\label{stitcvsP}
\end{figure}


        The large sensitivity of the  superconductivity  in $\rm C_4KHg$ to
small  pressures  is  particularly interesting  to  contrast with  the slow
monotonic      decrease of  $\rm       T_c$  with    pressure    in    $\rm
C_8KHg$.\cite{delong82a} The slow  decline of $\rm  T_c$ with pressure that
is  observed     in $\rm   C_8KHg$    is typical    of nearly-free-electron
metals.\cite{delong82a}  Nearly-free-electron   character  is a  reasonable
model        for      the    KHg-GIC's       since          band  structure
calculations\cite{senbetu85,holzwarth88} show   substantial intercalant $s$
and  $p$ character  at the  Fermi  level.  Therefore  the behavior  of $\rm
C_8KHg$ could be considered  conventional, whereas the transition narrowing
observed in  $\rm  C_4KHg$ is quite   anomalous.    At pressures above  the
initial discontinuity in $\rm T_c$, the pressure dependence of $\rm C_4KHg$
also  becomes   conventional,   with $\rm   dT_c/dP$   = -5$\times 10^{-5}$
K/bar.\cite{delong83}

        As pointed  out by  DeLong and Eklund,\cite{delong82a,delong83} the
$\rm T_c(P)$ experiments on the  KHg-GIC's are reminiscent of  those on the
transition metal dichalcogenides.   In  Section~\ref{otherhyd}  the  strong
$\rm  T_c$ increase with low  pressure of NbSe$_2$ was  contrasted with the
weak  $\rm T_c$ increase  with pressure of NbS$_2$.\cite{smith72} The large
low-pressure magnitude $\rm  dT_c/dP$ of $\rm NbSe_2$  is attributed to the
suppression of a  CDW.  Once the CDW of  $\rm NbSe_2$ has been destroyed by
pressure, its $\rm dT_c/dP$  is  almost the  same  as that  of  the non-CDW
compound NbS$_2$.\cite{smith72}  The CDW material NbSe$_3$ and  its non-CDW
relative TaSe$_3$ also show $\rm   dT_c/dP$   behavior similar  to    the
NbSe$_2$/NbS$_2$ pair.

        DeLong and Eklund have proposed that $\rm C_4KHg$  and $\rm C_8KHg$
might be such a CDW/non-CDW matched pair.\cite{delong83} According to this model, the
initially broad transition of $\rm C_4KHg$ is due to the  presence  in some
of the sample  of a  CDW state that is gapping  part  of the Fermi surface.
The portion of the sample that supported a CDW would have  a depressed $\rm
T_c$, whereas the non-CDW part of the sample would have $\rm  T_c$ = 1.5 K.
When a small amount of pressure destroys  the CDW,  the  whole material has
the intrinsic $\rm T_c$ = 1.5 K transition.

        As an alternative theory, DeLong  and Eklund\cite{delong83} proposed
that the  pressure drives  an ordering  transition in $\rm   C_4KHg$.  They
mention the improvement of long-range order in the intercalant layers or an
improvement  in    stacking fidelity as   possible    ordering transitions.
However,  there  are  good  reasons  why   an order-disorder  transition is
unlikely  to produce the  behavior seen in Figure~\ref{stitcvsP}a).  First,
$\rm T_c$   in  most  superconductors   is not  usually as    sensitive  to
crystallographic  order   as the  hydrogenation  and  pressure   data would
suggest.  There  are many examples of  disordered superconductors with $\rm
T_c$'s close to those of single crystals.  Other superconducting properties
such  as   the  critical fields and   currents are much   more sensitive to
crystalline  order  than  $\rm T_c$.   (These  transport  properties   depend
directly on the mean-free-path, as discussed in Section~\ref{models}).

        The other reason that the disorder-order  hypothesis seems unlikely
is  much   more  fundamental.  This   second argument,   which is based  on
thermodynamic considerations, is due to Clarke and Uher.\cite{clarke84} The
Clarke-Uher argument is based on  the observation that  the change of shape
of the  superconducting  transition   is almost  entirely  reversible.   As
Figure~\ref{stitcvsP} shows, when applied pressure  is released from a $\rm
C_4KHg$  specimen that had  been  pressurized up to   8 kbar, the  sample's
superconducting transition returns  almost exactly to its  original  shape.
The small amount of deviation between  the original and final room-pressure
transitions can probably   be attributed to  plastic   deformation  of  the
sample.  If the effect of pressure were genuinely to force a disorder-order
transition, then one would expect  the material to remain ordered  when the
pressure is removed.   The reason is  that the  part of  the Helmholtz free
energy (F = E $-$ TS) which favors the formation of a  disordered phase is the
entropic term.   The  entropic   contribution  to   the   free  energy   is
proportional to temperature, which   is the  formal  justification  for the
observation that higher temperatures encourage the formation  of disordered
phases.   Metastable disordered phases do exist  at low  temperatures since
their  rate of transformation is  suppressed due  to   the lack  of thermal
energy.  However, should  disordered  material be transformed   to  ordered
material  at   low  temperature by  the   application  of  pressure,  basic
thermodynamics suggests that the materials should remain ordered  since the
entropic forces driving disorder are effectively zero at 1 K.  

        These   thermodynamic considerations suggest that a  non-hysteretic
low-temperature transformation   must be of   the  order-order rather  than
disorder-order variety.   Order-order  transitions  can be   driven by  the
energy term  in the free-energy,   and so do not always   require thermally
assisted  growth.  In metallurgy  a  transformation which is  not thermally
assisted is called displacive or martensitic.\cite{christian81} One type of
martensitic  transition  is  the  charge-density  wave  transition  already
discussed in connection with the TMDC experiments.\cite{fuller81}

        There is additional evidence  to support the  identification of the
hydrogen- and pressure-induced transformation as an order-order transition.
If the difference between the low-$\rm T_c$ and $\rm T_c$  = 1.5 K material
were merely the degree of disorder, then it should be  impossible to make a
low-$\rm T_c$ sample with  a sharp transition.  Yet  Table~\ref{hydtctable}
shows that one gold specimen had $\rm  \Delta  T_c$/$\rm T_c$ only 7$\times
10^{-2}$, comparable to the width of the better $\rm T_c$ = 1.5  K samples.
It   is true that   the  lower $\rm   T_c$  samples tend   to  have broader
transitions, but this is probably  just an indication that  the  lower-$\rm
T_c$ phase is harder to grow.  If the $\beta$ phase is metastable, it seems
sensible that it would be harder to  grow in a  well-ordered condition than
the putatively stable $\alpha$ phase.

        The  importance of the   $\rm T_c$-versus-pressure   experiments is
therefore twofold.   One contribution of  the  pressure experiments is   to
reinforce the evidence from the hydrogenation experiments that the low-$\rm
T_c$ phase  can be  destroyed by   very  small perturbations.   The   vital
contribution of the  pressure experiment is that  the reversibility  of the
transformation shows    it  to  be   an  order-order  transition,  not  the
disorder-order    transformation   also    suggested      by   DeLong   and
Eklund.\cite{delong83}

        The  next  logical  question  is  why the  low-temperature ordering
should  be a CDW  transition    rather than  an ordinary structural   phase
transition.  $\rm C_8K$ has a structural phase transition near 13 kbar to a
$\rm      T_c$    =     1.5       K       phase,     as   discussed      in
Section~\ref{critfdisc}.\cite{delong83,avdeev87}  The   high-pressure phase
has a $\sqrt{3} \times \sqrt{3}$R30 $^{\circ}$ structure.\cite{clarke84} It
seems reasonable to ask whether the low-$\rm T_c$  material in $\rm C_4KHg$
could not also undergo a structural phase transition  to a higher-$\rm T_c$
phase.  As remarked in  Section~\ref{neutrons}, there is evidence from  the
neutron  scattering experiments  that  all lower-$\rm T_c$ samples  contain
both the $\alpha$ and $\beta$ phases of  $\rm C_4KHg$.  Therefore one might
well hypothesize that the effect  of  hydrogen  and  pressure is simply  to
transform the $\beta$ phase material into  $\alpha$ phase.  This hypothesis
explains  all the superconductivity data  quite well, but  it runs afoul of
the neutron diffraction  data.   In a neutron-diffraction  study,   Kim and
coworkers found no evidence for  a change in the  relative abundance of the
$\alpha$  and $\beta$ phases up  to 13.8 kbar.\cite{kim84} The  fraction of
$\beta$ phase in the neutron diffraction sample was 2\%  both  at 1 atm and
13.8 kbar.  Kim {\em et el.\/}  also  saw no evidence from  the diffraction
patterns   for ordering of   the sample or   for  changes  in  the relative
intensities  of the peaks  up    to 13.8  kbar.\cite{kim84}   This  neutron
diffraction  study effectively rules out any  structural  phase  transition
explanation for  the  data of DeLong  and  Eklund.\cite{delong83} Since  it
appears highly probable that hydrogenation has the same effect as pressure,
the data of   Kim and coworkers   appear  to rule  out a  structural  phase
transition explanation for the hydrogen experiment as well.

        The primary objection  to the hypothesis of  CDW formation in  $\rm
C_4KHg$  has to    be why a  CDW    has not  been observed  in   any of the
aforementioned diffraction experiments.    An answer  to  this question has
been   put  forward by Wilson,   DiSalvo,  and  Mahajan,   the discovers of
charge-density waves   in  the   TMDC's.\cite{wilson74}  After the periodic
lattice distortion associated with CDW formation was discovered in TaSe$_2$
using electron diffraction, an intensive search  for superlattice lines was
made with x-ray and neutron diffraction.  Even  when researchers knew where
these lines  were, observing them with  x-rays  was still quite  difficult,
requiring special equipment and long  exposure times.\cite{wilson75} Wilson
{\em et al.\/} explain that the observation of CDW's is made much easier by
the  dynamical diffraction that occurs  with electron  beams, but  not with
photon or neutron beams.\cite{wilson75}

        Timp\cite{K167} performed extensive  electron microscopy studies on
$\rm C_4KHg$  samples,  but he  never reports any  observations  below room
temperature.  Because CDW's  are    easily observed  only   with   electron
diffraction\cite{wilson75} or   nowadays,   with   a   scanning  tunnelling
microscope,\cite{coleman85}  there  seems  to have   been   no experimental
opportunity to  see  a possible CDW in  $\rm C_4KHg$.  An estimate can
be made of  the hypothetical  CDW transition temperature   in $\rm  C_4KHg$
using Eqn.~\ref{bilbro}. To calculate $\rm T_{CDW}$, let the intrinsic $\rm
T_c$ of $\rm C_4KHg$ $\equiv \: \rm  T_{c0}$ be 1.5  K.  Let the suppressed
$\rm T_c$ be 0.8 K.  The fraction of the  Fermi surface which is removed by
a CDW transition, $\rm  N_1/N$, can be  estimated by using  the BCS formula
for $\rm T_c$, assuming the same Debye temperature and same BCS interaction
parameter V for both transition temperatures.   The  result is that removal
of about 11\% of the Fermi surface will account for the  observed $\rm T_c$
depression in  the gold $\rm C_4KHg$  phase.  Plugging  this  fraction into
Eqn.~\ref{bilbro} gives $\rm T_{CDW}$ = 243 K.  This is  a {\em very} rough
estimate  for   $\rm   T_{CDW}$ since Eqn.~\ref{bilbro}  is   exponentially
dependent   on   the  fraction  of  FS   removed   by   the CDW transition.
Nonetheless, this estimate gives one hope that a CDW transition may  yet be
observed in $\rm C_4KHg$.

        Before moving on to further discussion of the CDW hypothesis, it is
worth discussing  a  few   alternative ideas.   The    possibilities  of  a
disorder-order transition or an order-order structural phase transition can
be  safely  eliminated, for   reasons  discussed  above.  The optic-phonons
explanation of $\rm T_c$ enhancement by  hydrogen that is applicable to the
transition metals is not promising for $\rm C_4KHg$.  The  reason  that the
optic-phonons picture seems unsuitable here  is  that it cannot explain  why
after hydrogenation all samples have the same $\rm T_c$, 1.5 K.

    One more suitable possibility  might be  an  explanation based   on the
electronic  properties of $\rm  C_4KHg$  rather than the  lattice modes  or
structure.  Roth   and coworkers\cite{H242}  proposed  a model  in  which a
reduction in carrier density  due to hydrogenation lowered the  Fermi level
until it  resided in a density-of-states  maximum.  The idea was  that this
maximum density-of-states would  correspond to a $\rm  T_c$  of 1.5 K,  and
that all subsequent perturbations would move the Fermi level away  from the
maximum and   consequently  depress   $\rm T_c$.   Roth  {\em    et  al.\/}
hypothesized that mercury vacancies might be  responsible for  a lower $\rm
T_c$ in some samples   since the $\rm T_c$   of  KHg alloys   monotonically
increases with increasing mercury content.\cite{roberts76}

        Since  this electronic-structure-based proposal  was made, evidence
has accumulated that mercury stoichiometry  does not  differ  significantly
among higher- and lower-$\rm T_c$ $\rm C_4KHg$ samples.  This evidence was reviewed
in Section~\ref{chemanal}.  One might wonder whether the coexistence of the
$\alpha$ and  $\beta$ phases in  the lower-$\rm T_c$ $\rm C_4KHg$ specimens
might not shift the Fermi level  enough to  lower the density-of-states and
suppress $\rm T_c$.   This possibility is not  out of the question, but the
similarity of the pressure and  hydrogenation data between  the  TMDC's and
the GIC's is  powerful  evidence for the  CDW hypothesis.  Furthermore,  the
very small  amount of  pressure  or hydrogen needed  to  increase $\rm T_c$
strongly hints that a phase transition is involved.

        Up to this point there has been no justification for the claim that
$\rm C_4KHg$ is the type of solid expected to undergo a charge-density wave
transition.  This is no accident  considering that the CDW phenomenon, like
superconductivity, is  a subtle   collective  effect that  depends on   the
details of  band structure,  lattice  modes,  and electron-phonon coupling.
The best argument for a charge-density wave in  $\rm C_4KHg$ would  seem to
be  that its  Fermi  surface, as  calculated by Holzwarth\cite{holzwarth88}
(see  Figure~\ref{holzfs}),  is  quite   similar to  that of   the  TMDC's.
According to  Holzwarth,  a CDW transition in  $\rm C_4KHg$ seems possible,
but    the   degree  of  nesting   is   very   sensitive to  the   proposed
splitting\cite{elmakrini80}    of      the  mercury    layers    in    $\rm
C_4KHg$.\cite{holzwarth88a} The similarity  of the $\rm C_8K$ Fermi surface
to that of  TaS$_2$ was  previously noted by Inoshita,\cite{inoshita77} who
found  a possible  Fermi   surface nesting  wavevector.   The  hypothetical
nesting wavevector for $\rm C_8K$ is shown in Figure~\ref{nesting}.

\begin{figure}
\vspace{12cm}
\caption[Possible Fermi surface nesting wave vector in $\rm C_8K$]{Possible
Fermi     surface     nesting  wave    vector   in    $\rm    C_8K$.   From
Ref.~\cite{inoshita77}.  The  horizontal cross-section  of  the  FS  in the
$\Gamma$-K-M plane is shown.  The arrow indicates the proposed nesting wave
vector near the M point.}
\label{nesting}
\end{figure}
%Figure 6c from Inoshita

        If a CDW does occur in $\rm C_4KHg$,  it should be visible in other
experiments besides electron diffraction.  For  example, a discontinuity in
resistivity   or susceptibility might   occur.   Published resistivity  and
susceptibility data  do   show  interesting  anomalies, but at    different
temperatures.\cite{elmakrini80b}  The  basal-plane  resistivity undergoes a
change in slope at about 200 K in  both  $\rm C_4KHg$ and $\rm C_4RbHg$, as
illustrated in Figure~\ref{stitransdata}a).  The magnetic susceptibility is
flat at 200 K but  shows   a  small anomaly  at  about 50 K.   Notice  that
resistivity  and susceptibility anomalies are  not observed for the stage 2
specimens.   It  would   be  informative   to   repeat these  measurements,
especially on GIC's whose $\rm T_c$'s have been measured.

\begin{figure}
\vspace{20cm}
\caption[Temperature dependence of the resistivity and susceptibility in
the alkali-metal mercury  GIC's]{Temperature dependence of  the resistivity
and    susceptibility in     the   alkali-metal     mercury  GIC's.    From
Ref.~\cite{elmakrini80b}.   a) Temperature dependence of  the resisitivity.
Curves (1) and (2) are for $\rm  C_4RbHg$; (3) is  for $\rm C_4KHg$; (4) is
for $\rm C_8RbHg$; and (5) is for $\rm  C_8KHg$.  b) Temperature dependence
of the susceptibility.  Curves (1) and (2) are for $\rm C_4KHg$; (3) is for
$\rm  C_8RbHg$;  (4) is for  $\rm C_4K_{0.5}Rb_{0.5}Hg$;  (5) is  for  $\rm
C_4RbHg$, and (6) is for $\rm C_8KHg$.}
\label{stitransdata}
\end{figure}

        A summary of the CDW hypothesis for $\rm C_4KHg$  is in order.  The
basic  picture is that  proposed by DeLong and Eklund.\cite{delong83}  This
picture relies heavily on analogies to the much-studied transition metal
dichalcogenides.   In  almost any reasonable   model,  the main  difference
between the pink and gold phases of $\rm  C_4KHg$ is that  the gold samples
contain the $\beta$ phase. Specific to the model described here is the idea
that the $\beta$ phase is unstable to the formation of a  CDW which opens a
gap on  some of the  Fermi surface at  a temperature well above  $\rm T_c$.
The removal of some of the FS lowers $\rm T_c$ in the gold samples from the
intrinsic  value  of 1.5  K  to about  0.8  K.   Hydrogenation and pressure
suppress the CDW, presumably by destroying  the FS nesting.  Destruction of
the CDW  close   its gap,  and restores $\rm  T_c$  to 1.5  K.   Removal of
pressure allows reformation of the CDW and a return  of $\rm T_c$ to 0.8 K.
This model is  by no means proven, but  it explains  all the available data
and is  reasonable because  of the similarity of  the superconducting GIC's
and TMDC's.

