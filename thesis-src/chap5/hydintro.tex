\section{Introduction}
\label{hydintro}

        Frequent reference  has   been made  in Chapters~\ref{samprep}  and
\ref{critf} to differences between the measured properties of gold and pink
specimens of $\rm C_4KHg$.  In Table~\ref{pink-gold},  a summary appears of
the    results  of previous chapters   regarding   these  two sample types.
Although several distinctions between the pink  and gold $\rm  C_4KHg$ have
been pointed out,  so far  there has been no attempt  at  an  explanation.  Such  an
understanding  is  clearly    very   important  for  an   understanding  of
superconductivity in  GIC's  since  the   two types  of  $\rm C_4KHg$  have
different superconducting properties.

\begin{table}
\begin{center}
\caption[A summary of the known differences  between the pink and gold $\rm
C_4KHg$]{A summary of the known differences between the  pink and gold $\rm
C_4KHg$.  The numbers here are  representative of  a typical sample  of its
type, although some  variation  was  observed from sample  to sample.  $\rm
\lambda_{ep}$        is      McMillan's     electron-phonon        coupling
parameter.\cite{mcmillan68}  $\kappa$ is   the   Ginzburg-Landau  parameter
discussed in Section~\ref{angdata}; $\kappa
\, < 1/\sqrt{2}$ indicates type I superconductivity. The density of states,
N(0), shown here was calculated in Section~\ref{critfdisc} from the $\rm
H_{c2}(\theta)$ data. NA means not applicable.}
\label{pink-gold}
\begin{tabular}{||c|c|c|c||}
\hline
& & & \\ Property & Experiment & Pink & Gold \\ & & & \\
\hline
& & & \\
Primary $\rm I_c$ &  diffraction\cite{kamitakahara84,lagrange83}& 10.24 \AA\ & 10.24 \\
& & & \\
Secondary $\rm I_c$ & diffraction\cite{kamitakahara84,lagrange83}& NA  &10.83 \AA \\
& & & \\
Primary structure & neutrons\cite{kamitakahara84};TEM\cite{J140} & $(2 \times 2)$R0$^{\circ}$ &  $(2 \times 2)$R0$^{\circ}$  \\
& & & \\
Secondary  structure & neutrons\cite{kamitakahara84};TEM\cite{J140} &NA & $(\sqrt{3} \times 2)$R(30$^{\circ}$, 0$^{\circ}$) \\
& & & \\
Stoichiometry & chem. anal. & C$_{4.3}$KHg$_{1.1}$ & C$_{4.7}$KHg$_{1.2}$ \\
& & & \\
$\theta_D$ & $\rm C_P$\cite{alexander81} & unknown & 269 K\cite{alexander81} \\
& & & \\
$\rm \lambda_{ep}$ & $\rm C_P$\cite{alexander81} + $\rm T_c$ &  unknown & 0.38\cite{iye82} \\
& & & \\
N(0) & $\rm C_P$\cite{alexander81} & unknown & 0.094/(eV $\cdot$ atom)$^{dagger}$\\
& & & \\
$\rm T_c$ & inductive & 1.5 K & 0.8 K \\
& & & \\
$\rm \kappa_{\parallel\hat{c}}(0)$ & critical field  & 0.55 & 0.75 \\
& & & \\
$\rm \kappa_{\perp\hat{c}}(0)$& critical field  & 4.7 & 8.5 \\
\hline
\end{tabular}
\end{center}
\end{table}

          The  most noticeable aspect  of Table~\ref{pink-gold} is the lack
of knowledge  for the  pink   phase  about    the properties   ($\Theta_D$,
$\rm \lambda_{ep}$, N(0))   that  are most  directly   connected   to $\rm T_c$, the
zero-field superconducting transition temperature.  Clearly a specific heat
experiment on a $\rm T_c$ = 1.5 K $\rm C_4KHg$  sample is highly desirable.
(The specific heat experiment of Alexander {\em et  al.\/} did not indicate
superconductivity in $\rm C_4KHg$ down to 0.8 K.\cite{alexander81}  The
samples used were said to be ``pink-copper'' in color.\cite{alexander81,lagrange80a})

        The other block to progress is the  impossibility of preparing pure
$\beta$-phase ($\rm I_c$ =  10.83 \AA)  samples.\cite{lagrange83}  From the
information in the table it is hard to discern whether  the presence of the
$\beta$ phase is by itself sufficient to depress $\rm T_c$  from 1.5 to 0.8
K.  The gold samples  might also be  more disordered in-plane than the pink
ones, for example.\cite{N128}  Also, there is  no proof that  all  low-$\rm
T_c$ samples  contain   the  $\beta$  phase,  although  neutron  scattering
suggests that this might be the case.  (See Section~\ref{neutrons}.)

        With just the information in the table on hand, it  is hard to make
a  judgement about the reason   for the lower-$\rm  T_c$ in  the gold  $\rm
C_4KHg$  samples.  More  experimental  results  are  needed to  confidently
identify  a cause.  Optical  transmission and resistivity measurements were
attempted, as  discussed in  Section~\ref{chardisc}, but these  experiments
were unsuccessful.  Hydrogenation  experiments are an appealing alternative
because of the interesting effects of hydrogen chemisorption in the closely
related   binary  compound  $\rm C_8K$.\cite{Z260,doll87}    In particular,
hydrogen  chemisorption  in   $\rm  C_8K$    raises $\rm T_c$     from 0.15
K\cite{koike80} to 0.22 K.\cite{kaneiwa82}  Dramatic increases in $\rm T_c$
due  to  hydrogenation are also  observed  in  many of the transition metal
dichalcogenides,\cite{friend79}  which  have similar  upper  critical field
curves to the  ternary  GIC's.   Hydrogen can  also   cause a  $\rm    T_c$
enhancement    in     the  elemental  superconductors     Pd,    Al,    and
Th\cite{stritzker78}.  The  mechanism  of  the hydrogen-related $\rm   T_c$
enhancement in these materials is discussed below.

        %the pink phase takes longer to make than the gold phase; anything that slows
%down the intercalation reaction results in the presence of gold phase
%Lagrange and coworkers have previously noted that the $\alpha$ phase of
%$\rm C_4RbTl_{1.5}$ is metastable, and will turn itself into 
