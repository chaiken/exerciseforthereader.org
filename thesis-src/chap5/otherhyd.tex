\section{Hydrogenation and Superconductivity in Other Systems}
\label{otherhyd}

\subsection{Hydrogenation of the Elemental Superconductors}

        Superconductivity in  metal-hydrogen  systems has been  reviewed by
Stritzker and W\"uhl.\cite{stritzker78} Only a few highlights are mentioned
here.
        
        In order to appreciate the effect of hydrogenation on
superconductivity, one must understand the role of lattice vibrations.
Everyone knows from the BCS equation\cite{bcs}

\[ \rm T_c \; = \; \Theta_D \exp{\left(-\frac{1}{N(0)V}\right)}
\]

\noindent that increasing the density of states at the Fermi level
increases $\rm  T_c$.   This  expectation is  borne  out   by  the detailed
calculations of McMillan,   which treat the electron-phonon  coupling  more
exactly.\cite{mcmillan68} However,  the BCS equation  also gives   that the
impression that raising the Debye  temperature  by  stiffening the acoustic
phonons will raise $\rm T_c$,  which is usually  false.   In general, it is
``soft'' phonons   rather than stiff  ones which   enhance  $\rm  T_c$, the
best-known   example    of     this   rule-of-thumb    being   the      A15
superconductors.\cite{bilbro76} In the special case of hydrogenated metals,
the  relationship  between  the   electron-phonon coupling   parameter $\rm
\lambda_{ep}$ and the  lattice frequencies reduces  to an especially simple
form because of the large mass  difference between hydrogen  and the metal.
This relationship is:\cite{gupta84}

\begin{equation}
\label{lambdacalc}
\rm \lambda_{ep} \, \approx \, \lambda_{metal} \, + \, \lambda_{H} \; = \; \frac{\eta_{metal}}{M_{metal} \Omega_D^2} \, + \, \frac{\eta_H}{M_{H} \langle \omega_{optic}^2 \rangle}
\end{equation}

\noindent where $\eta$ is an electron-phonon matrix
element.\cite{mcmillan68} The typical  phonon  frequency   $\rm \omega_{optic}$
might be thought of as proportional to the Einstein temperature  $\rm T_E$.
This equation, in  conjunction with   the McMillan  formula for  $\rm  T_c$
(Eqn.~\ref{mcmillan}),  shows that  raising     the   Debye   and  Einstein
temperatures should in general lower $\rm T_c$.

        In  most  of the  elemental  superconductors,  hydrogen gas acts to
depress   $\rm    T_c$.\cite{stritzker78} Hydrogen generally  weakens   the
coupling  of the carriers     to   acoustic  phonons,  so   unless  it  can
overcompensate with strong electron coupling  to  optic phonons, $\rm  T_c$
will  decrease.\cite{gupta81} Much of  the motivation   for   the  study of
hydrogenated  superconductors came  from the discovery of superconductivity
in    PdH.  Elemental palladium   is   not  itself  superconducting, so the
observation    of 8.8   K superconductivity     in    PdH   aroused   great
interest.\cite{stritzker78} Just  as  surprising is the  higher $\rm T_c$ =
10.7 K found in the deuterated compound PdD.  The $\rm T_c$  enhancement in
PdH has been attributed primarily  to the suppression of pair-breaking spin
fluctuations that are seen in  Pd.\cite{stritzker78} A secondary cause (and
the reason for the inverse isotope effect) is the presence of low-frequency
hydrogen (or  deuterium)  optic modes  which  increase  the electron-phonon
coupling.\cite{economou81,gupta81} The $\rm T_c$ enhancement in thorium and
aluminum  is   likewise explained as  the   result of  low-frequency  optic
modes.\cite{economou81,gupta81}

        In general it  seems  that the presence  of metallic hydrogen bands
seems  to  favor superconductivity and the    presence  of localized  ionic
hydrogen seems to suppress it.  When the hydrogen bands have  some metallic
character,  conduction electrons spend  more time  in  the  vicinity of the
hydrogen ions and thus the electron-phonon coupling enhancement tends to be
greater.\cite{economou81} These  two  roles for hydrogen may be  called the
``Proton   Model'' and     the   ``Anion Model.''\cite{switendick78}   This
categorization is in  reality  an  oversimplification: in   most transition
metals   the      hydrogen   bands   have   some   resemblance     to  both
pictures.\cite{switendick78}


\subsection{Superconductivity in the Po\-tas\-sium-Hy\-dro\-gen GIC's}
\label{khcompounds}

        The effect of hydrogen on superconductivity in $\rm C_8K$ must be a
balance  of opposing tendencies  since hydrogen can   either raise or lower
$\rm T_c$.  From the discussion above concerning the transition metals, one
might anticipate that the opposing factors might be the decrease in carrier
density and  increase  in  electron-phonon  coupling  that   are  generally
associated with hydrogenation of metals.\cite{switendick78} The $\rm T_c$'s
of hydrogenated K-GIC's are collected in Table~\ref{khtctable}.

\begin{table}
\begin{center}
\caption[Superconducting transition temperatures of the
KH-GIC's]{Superconducting transition temperatures of the KH-GIC's  and $\rm
C_8K$.  Two methods of preparation are possible: direct intercalation of KH
powder into graphite,\cite{enoki88a}  and intercalation of K  into graphite
to   form  $\rm C_8K$,  followed  by hydrogen  chemisorption.\cite{enoki85}
Single-phase specimens are difficult  to  obtain  because of  slow hydrogen
sorption kinetics.\cite{suzuki85b,enoki85}}
\vspace{0.5in}
\label{khtctable}
\begin{tabular}{|cccc|}
\hline
& & & \\
Compound (stage) & Method of Preparation & $\rm T_c$ (K) & Ref. \\
& & & \\
\hline
& & & \\
$\rm C_8KH_{2/3}$ (II)& chemisorption & $<$0.052 & \cite{sano80}\\
& & & \\
$\rm C_4KH_x$ (I)& KH powder & $<$0.070 & \cite{suzuki85b}\\
& & & \\
$\rm C_8K$ (I) & NA & 0.150 & \cite{koike80}\\
& & & \\
$\rm C_8KH_{0.19}$ (53\% I + 47\% II)& chemisorption& 0.22 & \cite{kaneiwa82}\\
& & & \\
$\rm C_8KH_x$ + $\rm C_{24}K$ (II)& KH powder & 0.31 & \cite{suzuki85b}\\
& & & \\
\hline
\end{tabular}
\end{center}
\end{table}

        The situation  in  the KH-GIC's is complicated  by the  presence of
multiple phases, just as in the KHg-GIC's.  In  the KH-GIC's, both $\alpha$
($\rm I_c$ = 8.43-8.53 \AA) and $\beta$ ($\rm I_c$ = 9.13  \AA)  phases can
coexist   in  a  given  sample.\cite{I217,suzuki85b}   The structure    and
stoichiometry of these phases is uncertain.\cite{miyajima88}  To complicate
matters further, staging disorder is also a common feature of the KH-GIC's,
unlike the KHg-GIC's, which are nearly always single-stage.  One reason for
the staging disorder in  the chemisorbed compounds  is the complex behavior
as  a  function of   hydrogen stoichiometry.   Starting  with  a $\rm C_8K$
sample, hydrogen goes into interstitial sites in $\rm C_8K$ up  to x = 0.1.
For x $>$ 0.1, the stage II compound, also denoted $\rm C_8KH_x$, begins to
form.  Stage I and stage II coexist until  x = 0.67, when  the last stage I
material  transforms  to  stage  II.\cite{lagrange74}  Other   reasons  for
disorder are the large number of  steps that  go into  the reaction and the
fairly complicated final structure.

        Despite  all this complexity,  Enoki  {\em et  al.\/} have  done an
admirable job of making sense out of  the numbers in Table~\ref{khtctable}.
Their  interpretation,  which is based  on specific  heat  and conductivity
measurements, is summarized  by Figure~\ref{khC_v} below.    The decline in
$\gamma$ with increasing x shows that H is lowering the carrier density and
must have at   least a partially ionic  character.   The lower density   of
states  in the  KH-GIC's  as  compared  to the   K-GIC's is   also  seen in
Shubnikov-deHaas\cite{Z260}  and   optical   measurements.\cite{doll87} The
sequential increase  and decrease in  $\theta_D$ and  $\rm  T_E$ shows that
hydrogen first stiffens and then softens both the  acoustic and optic modes
in $\rm C_8K$.

\begin{figure}
\vspace{17cm}
\caption[Hydrogen stoichiometry dependence of thermodynamic quantities in
$\rm C_8KH_x$ and $\rm C_8RbH_x$]{Hydrogen stoichiometry dependence  of the
superconducting    transition temperature $\rm     T_c$,  Debye temperature
$\theta_D$, the Einstein  temperature $\rm   T_E$, and the linear  specific
heat coefficient $\gamma$   in $\rm C_8KH_x$  and $\rm  C_8RbH_x$.     From
Ref.~\cite{enoki85}.   The  label F(x)/F(0) indicates  that  each  of   the
quantities is plotted normalized to 1.0 at x = 0.}
\label{khC_v}
\end{figure}
%FIgure 8 from enoki85

        The data shown in Figure~\ref{khC_v} can be analyzed to produce the
numbers    listed   in  Table~\ref{khcvtable},  which   is   adapted   from
Ref.~\cite{enoki85}.  The  conclusions one draws  from  examination of this
table are similar to those from study of the transition metal hydrides, but
with some important differences.  As noted in Ref.~\cite{enoki85}, the main
effect of hydrogen is that as a function of x  the electron-phonon coupling
parameter increases and the density of states at the Fermi level decreases.
The $\rm  T_c$  enhancement with hydrogenation  is  therefore accounted for
mostly  by a   larger $\lambda$.   The   increase  in $\rm \lambda_{ep}$ with x   is
overwhelmed at large x by the decrease in carrier density that is reflected
by the fall in the DOS.

\begin{table}
\begin{center}
\caption[Parameters rel\-e\-vant to su\-per\-con\-duct\-ivity in the KH- and
RbH-GIC's]{Parameters  relevant  to   superconductivity  in  the   KH-  and
RbH-GIC's.  Adapted  from Ref.~\cite{enoki85}.  The density of   states has
been  corrected  for the  electron-phonon   coupling.  $^{\dagger}$ means a
calculated parameter; ? means not measured.}
\vspace{0.5in}
\label{khcvtable}
\begin{tabular}{|c|ccccc|}
\hline
& & & & & \\
Compound & $\rm T_c$ (K) & $\Theta_D$ (K) & $\rm T_E$ (K) & N(0)
(states/eV$\cdot$C-atom) & $\rm \lambda_{ep}$\\
& & & & & \\
\hline
& & & & & \\
$\rm C_8K$ & 0.15 & 393.5& 82.2 & 0.31 & 0.30 \\
& & & & & \\
$\rm C_8Rb$  & 0.026 & 245.4 & 60.3 & 0.28 & 0.27 \\
& & & & & \\
$\rm C_8KH_{0.05}$ & 0.19$^{\dagger}$ & 403& 90.7 & 0.29 & 0.31 \\
& & & & & \\
$\rm C_8RbH_{0.05}$ & ? & 319.1 & 56.2 & ? & ? \\
& & & & & \\
$\rm C_8KH_{0.65}$  & $<$0.05 & 350& 58.1 & 0.10 & 0.87\\
& & & & & \\
& & & & & \\
\hline
\end{tabular}
\end{center}
\end{table}

        None of the previous statements is  at  all surprising in  light of
the results  of hydrogenation  on the  transition metals.  What is a little
bit hard to account for is the mechanism by which $\lambda_{ep}$ increases.
According   to  Equation~\ref{lambdacalc},  the   hydrogen-induced rise  in
$\Theta_D$  and $\rm T_E$ in the  KH-GIC's should  lower rather than  raise
$\lambda_{ep}$.  The source of the $\rm  T_c$ increase in  the KH-GIC's may
be the  optic  modes associated with  the  hydrogen atoms, just   as in the
hydrogenated   transition metals.\cite{enoki88}   The  very slight metallic
character of H in KH-GIC's\cite{miyajima88,enoki87} could contribute to the
enhancement of the   electron-phonon coupling.  Evidence  for the  hydrogen
hole band near   the  Fermi level comes from   electron   spin   resonance,
thermopower, and conductivity  measurements.\cite{miyajima88,enoki87}   The
schematic density of states of $\rm C_8K$ before and after hydrogenation is
shown  in   Figure~\ref{khdos}.   The  reasonableness  of   the  optic-mode
explanation for the hydrogen-induced $\rm T_c$ increase is  hard to judge,
but further experiments are underway.

\begin{figure}
\vspace{16cm}
\caption[Schematic density-of-states for $\rm C_8K$ before and after
hydrogenation]{Schematic density-of-states for  a) $\rm  C_8K$ and b)  $\rm
C_8KH_{0.55}$.  From Ref.~\cite{miyajima88}.  Note the very small  hole band
near $\rm E_F$ in b).}
\label{khdos}
\end{figure}

        Superconductivity in the KH-GIC's is not completely understood, but
there are several generalizations that can be made.  Firstly, hydrogenation
tends   to decrease the   density of states at  the  Fermi level by putting
carriers into low-lying hydrogen bands.   Secondly,  hydrogenation strongly
increases the electron-phonon coupling, although how  exactly it does so is
not clear.  Whether the optic phonons associated  with hydrogen are crucial
for the KH-GIC's, as they are in the transition metals, is not yet certain.

        Besides the  KH-GIC's,   another group of  materials  with  much in
common  with  the KHg-GIC's is  the transition metal  dichalcogenides.  The
effect of  hydrogenation on superconductivity   in the TMDC's  is discussed
below.
 

\subsection{Hydrogenation of the Transition Metal Dichalcogenide Superconductors}
\label{hydtmdc}

        The impact  of   hydrogen on  superconductivity   in the transition
metals bears some  resemblance to the effect  hydrogen has on the KH-GIC's.
To be convinced of this, compare Figure~\ref{tmdcHfig}a) with the $\rm T_c$
versus x plot  in Figure~\ref{khC_v}.  In  both figures $\rm T_c$  at first
rises with increasing H content, reaching its maximum near x $\approx$ 0.1,
and then falls to an unmeasurably low level at high x.  ($\rm T_c \, <$ 0.5
K for H$_{0.87}$TaS$_2$.\cite{murphy75}.   This data point is  not shown in
Figure~\ref{tmdcHfig}.)  $\rm T_c$  also  increases slightly   for NbSe$_2$
when a small amount of hydrogen is added.\cite{murphy75}

\begin{figure}
\vspace{18cm}
\caption[$\rm T_c$ increase in TaS$_2$ induced by hydrogenation and
pressure]{$\rm T_c$ increase in TaS$_2$ induced by a) hydrogenation and b)
pressure.  a) From Ref.~\cite{murphy75}. The error bars represent the
transition width, while the circles are the volume \% superconducting.
This experiment was performed on a powder sample. At a hydrogen
concentration ofu.87, $\rm T_c \, <$ 0.5 K (not shown).  b) From
Ref.~\cite{friend79}.  $\rm T_{CDW}$ is the CDW onset temperature, while
$\rm T_c$ is the usual superconducting transition temperature.  4H$\rm
_{b}$ and 2H are TaS$_2$ polytypes with different crystal structures.}
\label{tmdcHfig}
\end{figure}

        The resemblance between the $\rm C_8KH_{x}$ and the TMDC data turns
out to be mostly coincidence since the causes are probably quite different.
The  hydrogen-induced  $\rm  T_c$ enhancement  in  the TMDC's  TaS$_2$  and
NbSe$_2$ is now  known to be  due to suppression of  a charge-density  wave
transition        that      occurs       in          the     unhydrogenated
materials.\cite{murphy75,friend79}  The charge-density  wave  can  also  be
suppressed (and consequently $\rm T_c$ can  be increased) by intercalation,
pressure,  or dopants  beside  hydrogen.\cite{friend79}  CDW destruction by
intercalation was briefly mentioned in Section~\ref{tmdc}.  CDW suppression
by pressure  is  illustrated  in Figure~\ref{tmdcHfig}b) for  TaS$_2$.  CDW
suppression  by pressure  is  a general phenomenon  for  the  CDW's  of the
layered TMDC's.\cite{friend79} CDW destruction by impurities is exemplified
in the system  Nb$_{1-x}$Ti$_x$Se$_3$.  NbSe$_3$ has  $\rm T_c$  $<$ 50 mK,
whereas  the compound  with  x  =  0.001 has   $\rm  T_c \,  \approx$   2.1
K.\cite{fuller81}  The  extreme sensitivity of   the CDW to impurities  has
understandably  resulted  in   great    reproducibility     problems   with
superconductivity in the TMDC's.\cite{fuller81}  

        The  CDW-based explanation for $\rm T_c$  enhancement in the TMDC's
is fairly well-established because of the  observation of a  CDW in several
different    experiments.    For   example,  Figure~\ref{tase2resist}  show
discontinuities    in   the   resistivity   of  TaSe$_2$     due  to a  CDW
transition.\cite{wilson74}     There    can    be  no   doubt    that these
high-temperature resistivity anomalies are due to CDW  formation  since the
associated periodic lattice  distortion   has been directly  observed using
TEM, neutron scattering, and x-ray diffraction.\cite{friend79,wilson74}

\begin{figure}
\vspace{15cm}
\caption[Resistivity discontinuities in TaSe$_2$ associated with CDW
formation]{In-plane resistivity discontinuities in TaSe$_2$ associated with
CDW formation.  From Ref.~\cite{wilson74}.   1T- and 2H- refer to different
polytypes (crystal  structures).  The CDW  transitions occur at  473  K  in
1T-TaSe$_2$ and   at 117   K in   2H-TaSe$_2$, respectively.  Notice   that
1T-TaSe$_2$  has   a higher  resistivity   below  its  transition,  whereas
the resistivity of 2H-TaSe$_2$ decreases at its transition.}
\label{tase2resist}
\end{figure}

        Except for 2H-NbS$_2$, the CDW phenomenon occurs in all polytypes
of all members of the class MX$_2$, where M = V, Nb, or Ta, and X = S or
Se.\cite{friend79} Of these materials, only 2H-NbS$_2$ does not show a
large $\rm T_c$ enhancement with pressure.  Since NbS$_2$ is also the only
material which does not have a CDW transition, its comparatively small $\rm
dT_c/dP$ ( only 5$\times 10^{-6}$ K/bar) is evidence that CDW suppression
is the cause of the $\rm dT_c/dP$ in the other MX$_2$
materials.\cite{friend79,smith72} A host of experiments show that
perturbations (such as pressure, hydrogen, and impurities) tend to rapidly
increase $\rm T_c$ up to the point where the CDW is suppressed, whereas
they affect $\rm T_c$ only slowly above the CDW transition.  For example,
NbSe$_2$ has a large $\rm dT_c/dP$ (4.95 $\times 10 ^{-5}$ K/bar) for
pressures below 35 kbar, the pressure where the CDW is completely
suppressed.  Above 35 kbar, the slope is only $\rm dT_c/dP \, = \, 2.8
\times 10^{-6}$ K/bar,   even smaller than  that in  NbS$_2$.\cite{smith72}
Obviously this is no coincidence.

           The thrust of many of the superconductivity experiments on the
TMDC's is that there is a competition between the superconducting and CDW
transitions, so that preventing one condensed state encourages the other
one.  This line of reasoning says that once the CDW is completely
suppressed by a perturbation, a material reaches its intrinsic magnitude of
$\rm T_c$.\cite{fuller81} By ``intrinsic'', one means the $\rm T_c$ value
that would be expected from the usual theories given $\Theta_D$,
$\lambda_{ep}$, etc.  Further application of the perturbation may then
either increase or decrease $\rm T_c$, depending on the properties of the
individual superconductor.

        The reason that CDW's and superconductivity compete with one
another is quite simple.  In the BCS theory of superconductivity,\cite{bcs}
the condensation energy of the superconducting state is $\rm 1/2 \: N(0)
\Delta_s^2$, where $\rm \Delta_s$ is the superconducting energy gap.  In a
similar mean-field  theory  of   the  charge-density wave   transition, the
condensation energy is also proportional to $\rm N(0) \Delta_{CDW}^2$ (with
some additional multiplicative factors),\cite{rice73} where $\rm
\Delta_{CDW}$ is now  the CDW  energy gap.  The  juxtaposition of these two
expressions  for  the  condensation energy   immediately shows  why the two
phases tend  not to coexist: they both  need to create a gap  at the  Fermi
surface.   Once the Fermi  surface has a  gap  above it,  any further phase
transition driven  by   an  instability  of  the  Fermi  sea  is completely
suppressed.

        In real metals, a CDW transition tends to gap only a portion of the
Fermi  surface, leaving some  condensation energy available  for   use in a
superconducting  transition.  Thus a  high-temperature CDW transition tends
merely to lower $\rm T_c$ and not to  altogether prevent superconductivity.
Bilbro and  McMillan  found  a relationship between the   amount   that the
superconducting transition  temperature is lowered by  a higher-temperature
CDW phase  transition and  the amount of  Fermi surface which is removed in
that transition.\cite{bilbro76} The relationship is:

\begin{equation}
\label{bilbro}
\rm T_{CDW} \; = \; \left[ \frac{T_{c0}}{T_c} \right]^{N/N_1} \: T_c
\end{equation}
        
\noindent where $\rm T_c$ is the CDW-suppressed superconducting transition
temperature, $\rm  T_{c0}$ is  the intrinsic  transition  temperature, $\rm
T_{CDW}$ is the CDW onset  temperature, and  N$_1$/N is the fraction of the
density-of-states  removed  in the CDW  transition.   Fuller, Chaikin,  and
Ong\cite{fuller81} obtained N$_1$/N as a function of pressure P in NbSe$_3$
by measuring the size of the resistivity discontinuity at $\rm T_{CDW}(P)$.
The result is $\rm N_1/N$ = (0.6 - 0.18p), where p is the pressure in kbar.
Fuller and coauthors\cite{fuller81}  used these numbers in conjunction with
Eqn.~\ref{bilbro} to fit their $\rm T_c(P)$ data, and  found good agreement
at    low   pressures.  

        This  agreement   is convincing evidence   that  CDW-suppression of
superconductivity is indeed  due  to the competition of the  two  condensed
phases for the Fermi surface.  Any perturbation of the materials that tends
to  suppress the CDW,   such as pressure,  hydrogenation, or intercalation,
will therefore  also tend to  increase $\rm T_c$.   The enhancement of $\rm
T_c$ by hydrogenation in  the  TMDC's obviously is  due to a very different
mechanism than that which  has been put forward  for the transition  metals
and the alkali-metal  GIC's.  Since  hydrogen   first  increases and   then
depresses $\rm T_c$  in $\rm C_8K$, one  might wonder  whether  CDW's might
also play a role.  This  possibility  is discussed below in connection with
the  results   on the  KHg-GIC's.   Before   relating the  outcome   of the
hydrogenation experiments on the KHg-GIC's, the details  of the experiments
are briefly described.



        %CDW hard to see with x-rays or neutrons; dynamical scattering helps
%CDW hard to see with x-rays even after its existence was known
%x-rays intensity of CDW only about 10-2 to 10-4 compared to normal peaks
