\section{Conclusions}
\label{hydconcl}

        The  hydrogen-induced $\rm T_c$  enhancement that   occurs  in $\rm
C_4KHg$ is a  dramatic, reproducible  phenomenon.  Hydrogenation data looks
much like that from  experiments with applied pressure,  so it  is believed
that  these effects  have  a  common    origin.  The charge-density    wave
explanation of the difference between the pink  and gold samples also seems
consistent   with  the   hydrogenation and  pressure   experiments.   Other
explanations, such as  a disorder-order transition, or  a  structural phase
transition, are not consistent with the all the available data.

        Since there  is no  direct experimental evidence  for a CDW in $\rm
C_4KHg$, there are many experimental  tests that should still be performed.
Temperature-dependent  electron  microscopy   and   transport  measurements
($\chi$(T) and $\rho$(T)) have already been mentioned.  Other possibilities
are specific heat  experiments to directly  measure N(0) and $\rm \lambda_{ep}$, and
NMR or  ESR measurements to try to  learn more about the chemical  state of
the absorbed hydrogen.  In addition, studies of $\rm T_c$ as a  function of
H pressure are  desirable.  Some of these  measurements may be  carried out
with existing well-characterized samples by Prof.   T.  Enoki of  the Tokyo
Institute of Technology.  No doubt more surprises are in store.

        Like the KHg-GIC's,  the  CsBi-GIC's have several  in-plane phases.
The existence of superconductivity in the CsBi-GIC's has been controversial
due to reproducibility problems.\cite{lagrange87}  The  various $\rm T_c$'s
and multiple phases of the  CsBi-GIC's are obviously of  great interest for
comparison with the results described in this chapter.
