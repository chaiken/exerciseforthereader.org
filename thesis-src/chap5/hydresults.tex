\section{Effect of Hydrogenation on Superconductivity in the KHg-GIC's}
\label{hydresults}

        $\rm  T_c$     measurements    on     KHg    are   summarized    in
Table~\ref{hydtctable}.   The hydrogenation had  very  little effect on the
superconductivity  of   pink  specimens,  which  initially   had  a  narrow
transition and $\rm T_c \:  \approx$ 1.5 K.  The transition  of a pink $\rm
C_4KHg$    sample  before   and   after   hydrogenation    is   shown    in
Figure~\ref{hydeffect}b).  On the  other hand, hydrogenation  had two major
effects   on the gold  samples.    The  first    effect was  to  narrow the
superconducting  transition, decreasing $\rm   \Delta T_c$/$\rm  T_c$ by as
much as a factor of two.  In  addition, $\rm T_c$  in the hydrogenated gold
samples  was about 1.5 K, just  as  in the pink specimens.  Superconducting
transitions before and after hydrogenation  for a gold  sample are shown in
Figure~\ref{hydeffect}a).  The  effect of hydrogen on  a   specimen with an
initially  broad  transition but  a  fairly  high  $\rm  T_c$ is  shown  in
Figure~\ref{hydeffect}c).   Notice    that    the 3    transitions   before
hydrogenation  are quite different from one  another, whereas afterward the
three all have $\rm T_c \: \approx$ 1.5 K and $\rm \Delta T_c$/$\rm T_c$ on
the order of 10$^{-2}$.

        A  valid  question about  these   experiments is  whether  a  small
fraction of  the sample volume  might account for the  changes seen in  the
superconducting transition.   This concern  is especially important because
the inductive  measurements are sensitive to a  transition  in as little as
2\% of  the typical  sample  volume (see Section~\ref{electronics}).  Also,
the small size of the hydrogen uptake might lead one to suspect that only a
tiny portion of the sample was hydrogenated.  Because of the sensitivity of
the transition height to  the sample's  position within the secondary coil,
and  the  lack of   information about the sample's exact   dimensions after
intercalation, it is not possible to say with any certainty what the actual
areal  fraction  of superconductivity   before and after  hydrogenation is.
However,  the superconducting transitions  measured   both before and after
hydrogenation were always at least  2  $\mu$V high, a considerable fraction
of the 5 $\mu$V expected for the typical sample cross-sectional area.  (See
the discussion of  Figure~\ref{scingfract} for more  details.)  Most of the
transitions were on  the  order  of 4 to  8 $\mu$V  high.  The  $\rm T_c$'s
quoted  in Table~\ref{hydtctable} can  be  said with  certainty to
correspond to  at  least 40\% of  the  areal  fraction,  and probably  they
represent true bulk superconductivity.  Therefore the hydrogen-induced $\rm
T_c$ enhancement in $\rm C_4KHg$ is undoubtedly a bulk effect.

\begin{table}
\begin{center}
\caption[Effect of hydrogenation on $\rm T_c$ in KHg-GIC's]{Effect of
hydrogenation on $\rm T_c$ in KHg-GIC's.  All the  $\rm T_c$'s are for $\rm
C_4KHg$,  except  for  the blue  1.879  K sample,  which was $\rm  C_8KHg$.
$^{\dagger}$    indicates the   second   hydrogenation   of a    previously
hydrogenated  sample.   $^{\ast}$   indicates that deuterium    rather than
hydrogen was added.  $^{\S}$ indicates that the remeasurement of a
transition one year after hydrogenation.}
\vspace{0.5in}
\label{hydtctable}
\begin{tabular}{|ccccc|}
\hline
& & & & \\
Initial $\rm T_c$ (K) & Initial $\Delta T_c$/$\rm T_c$ & Initial color &
Final $\rm T_c$ (K) & Final $\rm \Delta T_c$/$\rm T_c$ \\
& & & & \\
\hline
& & & & \\
0.719 & 0.20 & gold & 0.941 & 0.21\\ 	% 5c with hydrogen
& & & & \\
0.875 & 7.3$\times 10^{-2}$ & gold & 1.519 & 1.2$\times 10^{-2}$\\ %6b with hydrogen
& & & & \\
0.875 & 7.3$\times 10^{-2}$ & gold & 1.535$^{\S}$	& 7.8$\times 10{-3}$\\ %6b after one year
& & & & \\
1.317 & 0.14& copper & 1.497 & 6.5$\times 10^{-2}$\\	%9c with hydrogen
& & & & \\
1.472$^{\ast}$ & 3.0$\times 10^{-2}$ & copper & 1.562 & 2.4$\times 10^{-2}$\\ %9b with deuterium
& & & & \\
1.497$^{\dagger}$ & 6.5$\times 10^{-2}$ & violet$^{\dagger}$ & 1.491 & 6.2$\times 10^{-2}$\\ %9c
& & & & \\
1.528 & 4.7$\times 10^{-2}$ & pink & 1.515 & 2.2$\times 10^{-2}$ \\ %8c with hydrogen
& & & & \\
1.879 & 8.5$\times 10^{-2}$ & blue & $<$1.2 & NA \\ %$\rm C_8KHg$ 4b
& & & & \\
\hline
\end{tabular}
\end{center}
\end{table}

\begin{figure}
\vspace{20cm}
\caption[Superconducting transitions before and after hydrogenation in $\rm
C_4KHg$]{Superconducting  transitions before  and  after  hydrogenation  in
three  types  of  $\rm C_4KHg$  samples.   a)  A  gold  sample.   $\rm T_c$
increases from 0.88 K to 1.54 K,  and  $\rm \Delta T_c$/$\rm T_c$ decreases
from 7.3$\times 10^{-2}$ to 7.8$\times 10^{-2}$.  b) A pink sample.  $\rm
T_c$  is  almost  constant; $\rm   \Delta  T_c$/$\rm   T_c$ decreases  from
4.7$\times 10^{-2}$ to 2.2$\times  10^{-2}$.  c) A copper-colored sample.
$\rm T_c$ increase   from 1.32 K  to 1.50  K;  $\rm  \Delta  T_c$/$\rm T_c$
decreases from 0.138 to 6.47$\times 10^{-2}$.}
\label{hydeffect}
\end{figure}

        Another  issue related to sample  homogeneity  is  the question  of
multiple superconducting transitions.  It is  reasonable to ask whether the
hydrogenated $\rm T_c$  = 1.54  K  specimen  (whose transition  appears  in
Figure~\ref{hydeffect}a))  did not still  have  a transition at 0.88 K, its
pre-hydrogenation  $\rm T_c$.  The  answer is no, within  the limits of the
sensitivity of  the apparatus.  A second, lower-temperature  transition was
always checked for and never found in samples whose $\rm T_c$ was increased
by hydrogenation.  The lack  of  a second transition is additional evidence
that hydrogen increased the bulk $\rm T_c$.

        A related hypothesis is that the transition narrowing is merely due
to suppression of the superconductivity of the lower-$\rm T_c$ phase.  This
assertion cannot be  correct since $\rm T_c$  enhancement was also  seen in
GIC's with initially narrow transitions.

        Because   of  the slow kinetics   of  hydrogen uptake in HOPG-based
GIC's,  one might  wonder whether $\rm  T_c$ would  increase   above 1.5  K
through further hydrogen additions.  As is shown in Table~\ref{hydtctable},
a second 5-minute  hydrogen exposure had  almost no effect  on $\rm T_c$ or
$\rm \Delta T_c$/$\rm T_c$.   As  another check  of time-dependent effects,
the superconducting transition of one hydrogenated GIC was remeasured after
a wait of one year.  The $\rm  T_c$ increased only  from 1.52 K  to 1.54 K,
which  is within the experimental  uncertainty.  The $(00\ell)$ x-rays also
did not show any change over this period.

        In order   to get more  clues  about the  origin  of the  $\rm T_c$
enhancement,  one copper-colored  $\rm  C_4KHg$  was  exposed  to deuterium
rather than  hydrogen.  Deuterium  appears    to have the  same effect   as
hydrogen, as is shown in Table~\ref{hydtctable}.   The lack of  any isotope
effect associated  with  the  hydrogenation suggests that   the  $\rm  T_c$
enhancement is not due to optic phonons, the mechanism  responsible for the
$\rm T_c$ enhancement in the transition metals.\cite{economou81}

        Table~\ref{hydtctable}  also shows that  one $\rm  C_4KHg$ specimen
showed only a small $\rm T_c$ increase as  a  result of hydrogenation, from
0.72 K to 0.94 K.  The  reason why the  final  $\rm T_c$ of this sample was
not 1.5 K is not known.  The fact that this sample had the lowest $\rm T_c$
of   any  of  the  measured  samples suggests  that  it may  have  been too
disordered to improve substantially upon hydrogenation.   Or alternatively,
the hydrogen uptake  rate in this specimen may have been especially slow.

        In light of the results on $\rm C_4KHg$, the  apparent  decrease of
$\rm T_c$ in $\rm  C_8KHg$ by hydrogen addition   is quite surprising.   It
would be a mistake  to make very much   of this result  since hydrogen  was
added to  only one  stage   2  sample.  The  hydrogenation of  $\rm C_8KHg$
deserves further investigation.

        The  data   in   Table~\ref{hydtctable}   tend  to   underline  the
differences  between the  pink  and gold   $\rm  C_4KHg$  GIC's  that  were
summarized in Table~\ref{pink-gold}.   These  experiments show  that adding
hydrogen to  a gold sample seems to  change its  superconducting properties
into  those of a pink   sample.   Hydrogenation  of  a sample with  a broad
transition seems to turn the lower-$\rm  T_c$ material that forms  the foot
of the transition into $\rm T_c$ = 1.5 K material.  Hydrogen appear to have
no effect on regions that initially have $\rm  T_c$ = 1.5 K.   The striking
aspect of these results is thus not that hydrogen raises $\rm T_c$ but that
hydrogen appears to be transforming  one type of  $\rm C_4KHg$ into another
since the end state appears to be always the same.

        By what means does hydrogen transform the lower-$\rm  T_c$ material
into $\rm  T_c$ = 1.5  K  material?  And what is  the  essential difference
between the lower and higher-$\rm T_c$ material?  The next section contains
some speculative attempts at answers to these questions.
