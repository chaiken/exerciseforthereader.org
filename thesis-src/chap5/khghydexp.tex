\section{Experimental Meth\-ods: Hy\-dro\-gen\-ation Ex\-per\-i\-ments}
\label{khghyd:exp}

	The hydrogen--doped stage 1 KHg--GIC's were prepared in a two--step
process,     analogous     with      the     preparation     of    $    \rm
C_{8}KH_{x}$.\cite{lagrange78} First, several batches of stage 1 KHg--GIC's
were  prepared    by    the usual two--zone       method,  as described  in
Section~\ref{synth}.   These   specimens  where characterized by $(00\ell)$
x-ray  diffraction  and their  zero-field superconducting   transition  was
measured,  as described  in Section~\ref{zerof}.  Raman scattering  spectra
were also recorded on some samples, as described in Section~\ref{ramdata}.

	In order to perform the hydrogen doping, the same samples that were
characterized  without  hydrogen were transferred under a  vacuum  of about
10$^{-5}$  torr  to  a  new  ampoule.   Hydrogen  gas was  purified through
diffusion  through a Pd-Ag tube   inside a  Resource Systems   Model  RSD-1
hydrogen purifier.  (200$\pm$2) torr of this gas was admitted to the sample
tube through  a glass break-seal.  After  about 5  minutes of exposure, 200
torr of He gas was also bled  into the  tube for thermal contact during the
low-temperature measurements.  The sample tube was then sealed off from the
vacuum system with a gas torch.

        After  the   hydrogen had  been admitted to   the sample tube,  the
pressure of  the gas   was monitored  using a  Setra Model  300D capacitive
manometer, although in the early experiments a mercury-column manometer was
used.  No  change in hydrogen pressure  was determined within  experimental
error;  that is, the  time  rate of change of   pressure was about the same
whether or  not the glass tube contained  a KHg-GIC sample.  Thus the   any slight
change in the  pressure  during exposure is attributed to  thermal drift of
the sensor.

        The apparently small hydrogen uptake of the HOPG-based KHg-GIC's is
consistent  with the very slow  hydrogen sorption that occurs in HOPG-based
K-GIC's.\cite{guerard83} For example, the $\rm C_8KH_{0.19}$ specimens used
in  the  superconductivity studies of  Kaneiwa {\em  et al.\/} were reacted
with hydrogen gas for 105 days.\cite{kaneiwa82} The slow  speed of hydrogen
diffusion into HOPG-based GIC's is attributed to the low defect  density of
HOPG.  If  $\rm C_4KHg$ has  more defects than $\rm  C_8K$, it might absorb
hydrogen  more quickly.  Nonetheless, the hydrogen  uptake  of $\rm C_4KHg$
must be quite small.

        It would be interesting to   try longer hydrogen exposure times  or
higher hydrogen  pressures  to see  if  more hydrogen is   absorbed.   High
hydrogen pressures were  avoided in these  experiments for safety  reasons,
but  experiments with higher hydrogen pressures  are planned  at the  Tokyo
Institute of Technology in the near future.\cite{enoki88} Hydrogen exposure
times longer than 5 minutes  were not used because  of  fear that air leaks
into the apparatus could  degrade the  sample quality.  Samples attached to the
vacuum system  without hydrogen gas addition tend  to  begin to  lose their
surface  luster   after  about 5   minutes.  Making    KHg-GIC's out of   a
less-ordered host material  (such   as  Grafoil) might also permit   larger
hydrogen absorption.\cite{guerard83}

        Despite the  fact that bulk absorption  of   hydrogen must be quite
small, the  effect of  hydrogen on  the $\rm   C_4KHg$ samples   was easily
visible.   Isothermally prepared  GIC's,  which were initially pink, became
blue during hydrogen exposure,  and then turned  a dark violet.  Gold GIC's
became blue--violet  indefinitely  afterward.  The samples'  color did  not
change noticeably after a period of about 12 hours.  A $\rm  C_8KHg$ sample
exposed to hydrogen showed no color change at all.  These color changes may
well occur only on the surface,  and so should not  be taken too seriously.
As  discussed in  Section~\ref{xrd},  post-hydrogenation $(00\ell )$  scans
showed that the samples'  repeat distances  were  unchanged   upon hydrogen
doping.  No change in the lattice constant was anticipated due to the small
amount of hydrogen that  was absorbed.  Hydrogen  absorption also seemed to
have   little  impact   on the  Raman  spectrum.  However,   the  effect of
hydrogenation on  the  superconductivity was more   dramatic,  as   will be
described in the next section.



        %mention long-term stability of hydrogenated compounds (> 1 year)
