\setcounter{chapter}{6}
\chapter{Conclusions and Prospects}
\pagestyle{headings}
\markright{Conclusions}
\label{final}

\section{Summary of New Results}
\label{newres}

        The  primary focus  of  this   thesis    has been to     understand
superconductivity  in  the KHg-GIC's.  There  are  several  aspects to  the
problem.   One is  to understand  the  stage-dependence of $\rm  T_c$.   As
discussed  in Section~\ref{models} and  \ref{concl}, the  increase  of $\rm
T_c$ with increasing stage can be  viewed as evidence  of an important role
for graphitic electrons in  GIC superconductivity.   If graphitic electrons
also participate in  superconductivity,  then the apparent increase of $\rm
T_c$ with increasing stage in the KHg-GIC's is not so surprising.

        The most unusual aspect of GIC superconductivity is the anisotropy
 of  the upper critical field, $\rm H_{c2}$.  This anisotropy  can
also be understood in terms of participation by graphitic electrons in the
superconducting state.\cite{M143}  $\rm H_{c2}(\theta, T)$  data for $\rm
C_4KHg$ have been interpreted in terms of the anisotropic Ginzburg-Landau
model (AGL).\cite{kats69,morris72,tilley65}  Quantitative agreement was
found for gold specimens with $\rm T_c$ about 0.8 K, in accord with the
reports of Iye and Tanuma.\cite{iye82}

        For pink specimens with $\rm T_c$ about  1.5 K,  corrections to the
AGL model had to be included in order to achieve a good fit.  The  AGL $\rm
H_{c2}(\theta)$ formula agreed  well with the pink  $\rm C_4KHg$  data once
allowance was made for type I  superconductivity.  Type I superconductivity
occurs in a small temperature-dependent range of applied-field orientations
near  $\rm  \vec{H}   \parallel \hat{c}$.  The    occurrence   of  type   I
superconductivity in pink $\rm C_4KHg$ is in accord with calculations based
on the specific heat data of Alexander {\em et al.\/}\cite{alexander81} The
values of  the thermodynamic critical field $\rm  H_{c}$ obtained from fits
to the $\rm H_{c2}(\theta)$ data imply a smaller density of states  at $\rm
E_F$,   N(0),  than that   measured  by Alexander.\cite{alexander81}   This
discrepancy   concerning the magnitude   of  N(0) could  simply  be  due to
demagnetization  effects on the $\rm H_{c2}(\theta)$   measurements,  which
could reduce the apparent value of $\rm H_{c}$.  On the other  hand, if the
magnitude of N(0)  obtained from  the  angular dependence  data is correct,
this smaller  value could help to explain  why $\rm T_c$  is lower  in $\rm
C_4KHg$ than in $\rm C_8KHg$.

        Extended linearity of  $\rm  H_{c2}(T)$ was  also  observed in $\rm
C_4KHg$.  The temperature dependence   of $\rm  H_{c2}$  in $\rm   C_4KHg$ (and  in other
superconducting GIC's\cite{koike80,iye82}) is reminiscent of critical field
data   on  the layered   TMDC's.\cite{ikebe80,dalrymple84}  Because  of the
evidence that both intercalant  and graphitic electrons participate  in GIC
superconductivity, multiband  models of  $\rm H_{c2}$  would  appear  to be
appropriate.\cite{aljishi88,entel76}  The nearly cylindrical   shape of the
graphitic piece   of   a  GIC's   Fermi   surface\cite{I94}  suggests  that
anisotropic Fermi-surface models might also  be good candidates to describe
$\rm  H_{c2}$    of superconducting   GIC's.\cite{butler80,youngner80} More
quantitative information about the Fermi surface of the GIC's is necessary
before    these models can   be  tested,   although   some calculations are
available.\cite{holzwarth88}

        Pink and gold $\rm C_4KHg$ specimens were found  to differ in their
superconducting transition  temperature   and critical field   behavior, as
mentioned   above.  The  only   detectable distinction in the  normal-state
properties of these compounds is that the gold specimens often (and perhaps
always)  contain some of   the higher-$\rm  I_c$   $\beta$ phase.  The pink
samples, on the other  hand, are  composed solely  of the  lower-$\rm  I_c$
$\alpha$ phase.  The relative amounts of the two phases  can be controlled,
within limits, by fine-tuning the intercalation conditions.

        The  depression of $\rm T_c$ by  a small fraction  of  the minority
$\beta$ phase is  unusual.  In  general, only magnetic   impurities are  so
potent in lowering $\rm   T_c$.\cite{tinkham80}  Surprisingly,  adding a  small
amount of hydrogen or applying a minute hydrostatic pressure\cite{delong83}
restores   $\rm  T_c$  to   about  1.5 K.   A   $\rm   T_c$ enhancement  by
hydrogenation   and  applied  pressure is   also  seen   in several of  the
transition    metal  dichalcogenides.\cite{friend79}  These  TMDC's support
charge-density waves,\cite{wilson75} so the increase  of $\rm T_c$ with the
application of a small perturbation is attributed to the suppression of the
CDW state.\cite{friend79} The suppression of  a CDW by hydrogen and applied
pressure also    seems reasonable  in   $\rm  C_4KHg$.\cite{delong83}   The
similarity of the Fermi surfaces  of  the TMDC's and  superconducting GIC's
lends additional credence to the CDW hypothesis.\cite{inoshita77}

        The highly variable range of  $\rm T_c$ in  $\rm C_4KHg$ is matched
by the CsBi-GIC's.  For $\rm C_4CsBi_{0.5}$, $\rm T_c$ has been reported to
be   as        high    as     4      K,\cite{mcrae85}        while    other
investigators\cite{E291,stang88} have found no superconductivity down to 50
mK.  Extensive efforts to reproduce the higher $\rm T_c$ value have been
unsuccessful.\cite{lagrange85a}  

        The  apparent  lack  of  superconductivity in  the   CsBi-GIC's  is
puzzling considering that  the alloy CsBi$_2$ is  superconducting with $\rm
T_c$   =  4.75  K.\cite{mcrae85}  The high    resistivity anisotropy of the
MBi-GIC's (M = a heavy  alkali metal) may offer an  important  clue.  McRae
{\em et al.\/}\cite{mcrae88} note that $\rm \rho_c / \rho_a$ is much higher
in the   MBi-GIC's  than in other  donor  compounds.  They interpret  their
MBi-GIC   c-axis resistivity   data in  terms   of the  hopping  conduction
mechanism  proposed  by   Sugihara.\cite{E343,M325}    Perhaps  c-axis band
conduction in $s$-like intercalant states is necessary for the existence of
superconductivity in  GIC's.   Similar    ideas  have been     discussed by
Al-Jishi.\cite{M143}  In  light  of   the  hypothetical CDW  state  in $\rm
C_4KHg$, the possibility  of a CDW should  also be taken seriously for  the
MBi-GIC's.

        These  considerations  may also apply to  the acceptor GIC's, which
also tend  to  have hopping conduction  along the c-axis.\cite{mcrae88} The
acceptor GIC's are discussed in more detail below.

\section{Conclusions}
\label{concl}

        One of the recurring themes in the GIC superconductivity literature
has been  the question of whether the  graphite or alkali  metal layers are
superconducting in $\rm  C_8K$.\cite{iye82,takada82}   The  origin  of  the
notion  that {\em  either \/}  the  graphitic or alkali   electrons must be
responsible for the superconductivity is a bit hard to fathom.   After all,
no one ever discusses the superconductivity of the transition  metal layers
of the transition metal dichalcogenides.  The case  that both graphitic and
intercalant electrons contribute to GIC  superconductivity\cite{M143} would
seem to be even clearer than the equivalent TMDC question since neither the
alkali metal   nor  the graphite   is separately superconducting.   In  the
MHg-GIC's, the starting alloys are superconducting by themselves.  However,
the   potassium  and mercury planes   cannot be  solely responsible for the
superconductivity of the  compound,   since  otherwise  $\rm T_c$ would  be
expected  to decline monotonically    with increasing stage.  In  fact,  as
discussed in Section~\ref{models}, $\rm   T_c$ of  the   KHg-GIC's has  the
opposite trend from that predicted by the proximity effect.

        Frequent   reference  has   been  made  to  the   properties of the
superconducting transition metal  dichalcogenides in this  work, especially
to  NbSe$_2$.  In  Chapter~\ref{othersys},   a   progression  in increasing
anisotropy was  proposed that  began with   bulk Nb, followed  by NbSe$_2$,
alkali-intercalated NbSe$_2$,  and ended  by  organic-molecule intercalated
TMDC's.   When this  project first began,  there was a natural  tendency to
compare the alkali-intercalated  GIC's with  the alkali-intercalated TMDC's
and  the  acceptor  GIC's   with   the  organic-intercalated GIC's.    This
comparison is a bit misleading, though.  The point that has been overlooked
in this juxataposition is the rather obvious fact that NbSe$_2$  is already
superconducting.  Because  the properties of the  TMDC's  are  not strongly
modified  by   alkali intercalation,\cite{woollam76} in  that   case  it is
perhaps appropriate   to speak of   the   superconductivity of the NbSe$_2$
sandwiches.

        What is the superconducting sandwich in the case of the GIC's?  The
line  of reasoning     presented   here   suggests   that    perhaps    the
carbon-intercalant-carbon sandwich is superconducting; or perhaps even more
carbon  planes are involved.   If superconductivity is  a joint property of
both carbon and intercalant planes,  perhaps the higher  $\rm T_c$ of  $\rm
C_8KHg$ than $\rm  C_4KHg$ is not the great  surprise that it  is sometimes
made out to be.  There is no reason that adding more  carbon layers between
the KHg trilayers  should depress $\rm  T_c$ if the  carbon layers are also
superconducting.  Those sceptics  who insist   that  mercury must  dominate
superconductivity in  the KHg-GIC's should  reflect on  the  fact that  the
critical fields of $\rm C_6K$\cite{avdeev87}  are quite similar to those of
$\rm C_4KHg$.  $\rm C_6K$ also has a $\rm T_c$ of 1.5 K.\cite{avdeev87}

        Iye and Tanuma projected that a 3D-2D crossover might occur for the
stage 3 KHg-GIC, where they estimated that $\rm \xi_{\parallel \hat{c}} \:
\approx \: I_c$.\cite{iye82}  One can easily see that this estimate should
not   be    appropriate   if  the  carbon  planes    also  participate   in
superconductivity.   If the   carbon   planes that  are    adjacent  to the
intercalant  layers  are also superconducting, then  it is their separation
that  should determine the  coupling  dimensionality.  Presumably at a high
enough stage the superconducting  packages  would be expected to  decouple,
but it  is   not clear what  the  correct  length  scale to  describe  this
decoupling would be.

        Will  a 3D-2D coupling-dimensionality  crossover  ever  be  seen in
GIC's?  The analogy with the TMDC's helps  to clarify this  question.  When
one  takes  a  superconducting TMDC  and  intercalates it   with an organic
molecule,  the    molecules    act  as   electron  donors.\cite{thompson72}
Nonetheless,  the organic  molecules  partially  decouple the TMDC  layers,
raise the   resistivity   anisotropy,\cite{thompson72} and     allow     2D
superconductivity   at low       temperatures.\cite{coleman83}    Molecular
intercalants into GIC's act as electron acceptors, not donors.\cite{I94} As
in the TMDCIC's, molecular intercalants in GIC's act  to decouple the layers
and raise the  resistivity  anisotropy.\cite{mcrae88} The analogy with  the
TMDCIC's leads one to suppose that acceptor should be 2D superconductors at
a sufficiently low temperature.  However, GIC's with molecular intercalants
appear  not to be   superconducting.\cite{erice:scy}  

        The  lack of superconductivity  in  the acceptor GIC's  is not that
surprising if one believes  that the cooperation  of carbon and intercalant
electrons is required.  The intercalant   bands  are  usually the   major
contribution to the c-axis conductivity, so the high resistivity anisotropy
in these materials implies a low carrier density  at the Fermi level in the
intercalant bands.   In fact,  recent work, both experimental\cite{mcrae88}
and theoretical,\cite{E343,M325} suggests   that   hopping  conduction   is
responsible  for c-axis transport  in acceptor GIC's,  at least  for stages
greater than one.  Perhaps band conduction along the c-axis  is a necessary
condition  for superconductivity in  GIC's.   That band conduction is not a
sufficient condition for superconductivity is  shown by the case   of  $\rm
C_6Li$,   which   has   the     lowest resistivity anisotropy     of    any
GIC,\cite{mcrae88},  and  yet is not superconducting.\cite{erice:scy}  This
problem deserves further investigation.

        If one believes  following Al-Jishi\cite{M143} that the presence of
intercalant electrons at $\rm E_F$ is necessary for superconductivity, then
one possible implication may the  non-existence of 2D superconductivity  in
GIC's.  In other  words,  when one  decouples  the  metallic  planes   with
insulating  intercalants  in the   TMDC's,  one  gets  2D superconductivity
because the  individual  TMDC planes  are themselves   superconducting.  In
contrast,  when one decouples the metallic  carbon  planes  with insulating
intercalants  in the  GIC's,  one   destroys superconductivity along   with
three-dimensionality.    The      Al-Jishi    model     may   imply    that
three-dimensionality is needed for superconductivity in GIC's.  These ideas
lead back to the fundamental question of the nature of superconductivity in
GIC's.

\section{Suggestions for Future Research}
\label{future}


        Some progress has been made toward answering most of  the questions
that were asked in the first chapter of this  work.  The one  question that
continues  to  be  elusive   is  the most basic  one,   why is $\rm   C_8K$
superconducting?   In the coming  years experimentalists could help  to pin
matters down by  extending   the study  of GIC  superconductivity  to  such
compounds  as $\rm C_8K_{(1-x)}Rb_x$  and $\rm  C_6Ba$ since, according  to
available     theories,\cite{M143,takada82}   these  compounds    should be
superconducting.    The  $\rm   C_8K_{(1-x)}Rb_x$ system  is   particularly
interesting since several phonon modes  soften dramatically for x $\approx$
2/3.\cite{neumann84,solin84a} The data in Figure~\ref{softmode}  illustrate
the  softening   of  the acoustic    modes   discovered  by  Neumann    and
coworkers.\cite{neumann84} These authors   attribute   the anomaly    to  a
composition-dependent   charge transfer.   All  other things   begin equal,
conventional theories of  electron-phonon coupling would  tend to predict a
maximum of $\rm T_c$  versus x  for the $\rm C_8K_{(1-x)}Rb_x$  system near
this minimum  in the phonon frequencies.  Yet  the $\rm  T_c$  for x  = 1.0
($\rm C_8Rb$) is only 26 mK,\cite{kobayashi85}  compared to the $\rm T_c$ =
150 mK for x  = 0.0 ($\rm C_8K$).\cite{koike80}  Investigation of $\rm T_c$
as a function of stoichiometry in this system is clearly  an important test
of theories of GIC superconductivity.

\begin{figure}
\vspace{15cm}
\caption[Soft phonon modes in the $\rm C_8K_{(1-x)}Rb_x$ system.]{a) Softening
of the elastic  constant $\rm C_{33}$ as  a function of  composition in the
$\rm  C_8K_{(1-x)}Rb_x$  system.  From Ref.~\cite{neumann84}.  The  elastic
constant was  obtained from a fit  to  the acoustic  branch  of  the phonon
system.   The phonons  were  observed using   inelastic neutron scattering.
Similar softening of the  M-point optic modes  has  been seen  using  Raman
scattering.\cite{solin84a}  b)   $\rm   T_c$   versus    x   in   the  $\rm
C_8K_(1-x)Rb_x$   system.   Only  the    endpoint compounds      have  been
characterized.}
\label{softmode}
\end{figure}


        The  high-pressure experiments, which show $\rm  T_c$ of $\rm C_4K$
to be as high  as  2 K,\cite{avdeev86}  also offer an  important clue.  The
structure  of  $\rm C_4K$ is currently  uncertain,   but it is  believed to
include   a   double intercalant  layer  of   K    atoms,  much  like  $\rm
C_4KHg$.\cite{avdeev87} These  experiments offer  another fundamental test of
the theories of GIC superconductivity.

        Despite the  progress reported here in   studies of  the KHg-GIC's,
some   questions about these materials  remain.   In   general, it would be
advisable  to  repeat  many of   the experiments  that have  already   been
performed  (resistivity, susceptibility, specific heat)  on specimens whose
$\rm T_c$   has been  measured.    Comparative  studies of   gold  and pink
specimens are  of  particular interest because  of the possibility  of  the
detection of a  charge-density  wave.   More work on  hydrogenation of $\rm
C_4KHg$ is  also needed;  studies as a function of  hydrogen uptake are  of
particular interest.
        
        In order to interpret these experiments a lot more theoretical work
is  needed, particularly  band-structure   calculations for  input  to  the
microscopic    critical field  models.     As  always   in  the  field   of
superconductivity, experiment and  theory have  a  stimulating partnership.
In  the coming  years   our   understanding of  both low-temperature    and
high-temperature superconductivity will continue to increase.
