\section{Mag\-ne\-to\-trans\-port Mea\-sure\-ments on CsBi-GIC's}
\label{csbisdh}

	The original idea in undertaking the study of the CsBi-GIC's was to
perform  a detailed  examination of   their  superconductivity.  Since,  as
described in the previous section, no superconductivity  could be found, it
was decided to pursue experiments on the normal state which  would  help to
explain the lack of a  transition.  One place  to look for  clues about the
lack of  superconductivity is the  materials' Fermi surface (FS), since only the
electrons  with energies near  $\rm E_F$ participate  in superconductivity.
Shubnikov-de Haas  (SdH)  measurements have in  the   past   been useful in
measuring    the     Fermi     surface    properties    of     metals    in
general\cite{ashcroft76} and GIC's  in particular.\cite{M98,Z260}  The  SdH
measurements   performed    here     at    MIT   are      complementary  to
optical,\cite{yang88} magnetic resonance,\cite{stang88a} and zero-field
transport\cite{mareche86} performed elsewhere.

\subsection{Magnetoresistance Measurements: Experimental Considerations}
\label{sdh:exp}

	For  the SdH experiments,   the   samples were mounted  as  for the
resistivity   measurements   described   in  section~\ref{csbitc}.  The  dc
transverse magnetoresistance was measured in fields  up to 23 tesla  and at
temperatures from 4.2 to 1.2 K using currents from 0.1 to 10 mA.  A typical
1.2 K  data   set for  a   stage  1  $\alpha$-phase CsBi-GIC is   shown  in
Figure~\ref{mr}.  The  resistance of the GIC approximately  doubled  at 23 T
from its zero-field  value.  The approximately quadratic, downward-curving,
behavior at low field is characteristic of saturation  due to closed orbits
in the planes perpendicular to the magnetic field.\cite{Z260}  

	This  non-oscillatory portion of  the data  was  fit and subtracted
off.  The data were Fourier transformed using a program  written by Dr.  M.
Shayegan.  This algorithm uses a Hamming window  to eliminate  aliasing due
to the truncation of  the  field  sweeps at  23  T.  Experience  with  this
program showed that nonetheless  peaks in Fourier intensity at  frequencies
less than about 30 T were very susceptible to the details of the background
fit.  For  this  reason, it is  thought  that  the  lowest-frequency  peaks
observed in the Fourier transforms  of  Ref.~\cite{Z260}  (labeled $\alpha$
there)   are probably  spurious ones,  due  to  suboptimal background fits.
Removal  of these $\alpha$  frequencies   from Table II in Ref.~\cite{Z260}
improves  the  agreement between the Dresselhaus-Leung  model\cite{F60} and
the data.   A similar comment  could  be made   with respect to Table I  in
Ref.~\cite{W179}.    Fortunately,   the  quantitative    data   analysis in
Ref.~\cite{Z260,W179}  relies on the   highest  SdH  frequency and not  the
lowest ones, and so is unaffected by the  spurious low frequencies.  In the
analysis reported  here, the background  was  carefully fit  and subtracted
before the data was fed to the Fourier transform program.

\begin{figure}
\center
\vskip 9.cm 
\caption[Magnetoresistance of a stage 1 $\alpha$-phase CsBi-GIC, $\rm
C_4CsBi_{0.6}$]{The  transverse magnetoresistance of  a $\rm C_4CsBi_{0.6}$
(stage 1, $\alpha$-phase) sample  at 1.2 K with  a current  of  1 mA.   The
current is applied in the graphite planes; the magnetic field is  along the
graphite c-axis.}
\label{mr}
\end{figure}

\subsection{Results}
\label{sdh:results}
	
	Similar	Shubnikov-de   Haas  oscillations  were  seen  in  all pure
$\alpha$-phase samples studied.   Only very weak  oscillations were seen in
mixed-phase $\alpha + \beta$  samples, probably  due to  a smaller value of
$(\omega_c \tau)$ caused  by in-plane  disorder.  A representative  Fourier
transform   calculated  from the   $\alpha$-phase   SdH  data  is  shown in
Figure~\ref{sdh:calc}a).  This  scan shows   the frequency which   was seen
consistently, at $(1120 \pm 110)$ T.

	In order to verify the  authenticity of the highest  frequency, the
one used  in the subsequent data  analysis,  various small portions of  the
data  were   Fourier  transformed   separately, to see   if the  1120 T
requency was an artifact of the analysis.  However, it  was found that this
frequency appeared strongly no matter what field range of data was used for
the Fourier transform.  In addition, in order to  get some  intuition as to
what the 1120  T frequency should  look like  experimentally,  a calculated
version of  the trace was compared  to the  data using  the fit parameters.
The calculation  used  the accepted    form of the temperature  and   field
dependence of the amplitude of SdH oscillations:\cite{Z260}\\

\begin{equation}
\label{amplit}
\rm A(T, H) \; = \; T \, \exp \frac{- 2 \pi^2 k_B (T \, + \, T_D)}{\hbar \omega_c}
\end{equation}

\noindent where A is the amplitude of the oscillations,  $\rm T_D$ is the
Dingle temperature, and $\rm \omega_c$ is  the cyclotron frequency, through
which  A is   dependent  on the  magnetic  field.    The Dingle temperature
characterizes the degree of disorder in a solid through:\cite{Z260}\\

\begin{displaymath}
\rm T_D \; = \; \frac{\hbar}{\pi k_B \tau_{col}}
\end{displaymath}

\noindent where $\rm \tau_{col}$ is the transport collision time.  An
unusually large value of the Dingle temperature of about 5 K had to be used
to produce the  calculated  trace shown in   Figure~\ref{sdh:calc}b).   The
collision time in the $\rm C_4CsBi_{0.6}$ samples is probably short  due to
the scattering   from  the  metal   inclusions  discussed    previously  in
Section~\ref{csbitc}.  Confidence   in the  validity   of  the  frequencies
obtained  from the Fourier transforms  is  increased  because  of the  good
qualitative agreement between the  calculated curve and  the data shown  in
Figure~\ref{sdh:calc}b).

\begin{figure}
\center
\vskip 9.cm 
\caption[Fourier transform of SdH data; simulated SdH trace]{a) Fourier transform of the data in Figure~\ref{mr}.  Most
reproducible  frequency is $(1120  \pm 110)$ T,  although other frequencies
sometimes occur.  b)  Comparison of the  data  from Figure~\ref{mr}  with a
simulated trace   calculated  using  Equation~\ref{amplit}   and parameters
obtained from the Fourier transform of the  data.   The calculated curve is
offset from the data for clarity.}
\label{sdh:calc}
\end{figure}

	Once the maximum  SdH frequency of  1120 T has been determined, one
can use  the dilute-limit  model as a  first approach to understanding  the
data.  This model assumes  that the primary  effect of intercalation is  to
increase the size of the graphitic $\pi$-derived ellipsoids at the  K point
of the Brillouin zone, leaving the shape of the ellipsoids unmodified.  The
validity of this assumption for  stage I  GIC's could be criticized because
of  the high  intercalant   densities  involved.  However,  the   excellent
agreement obtained by Timp {\em  et al.}\cite{W179} with  other more direct
measurements  of   Fermi   energies  and    charge transfers  gives    some
justification  for  the  model's  use.   [These   values  are  collected in
Table~\ref{electronicparams} below.]

	Proceeding then cautiously in  the spirit of  the Dresselhaus-Leung
(DL) model\cite{F60}, one can  relate  the compounds'  Fermi energy  to its
maximal FS cross-section,  $\rm {\cal A}_{max}^e$, and   thus to
the largest observed SdH frequency, $\rm \nu_{max}$.   The expressions that
describe this relationship, derived by DL, are:\\

\begin{equation}
\label{sdh:E_F}
\rm {\cal A}_{max}^e \; = \; \frac{4 \pi}{3a_0^2 \gamma_0^2 (1 \, + \, 2\gamma_4/\gamma_0)^2}(E_3 \, - \, E_F)(E_2 \, - \, E_F)\\
= \frac{2\pi {\mit e}\nu_{max}}{\hbar {\mit c}}
\end{equation}

\noindent Here the E's are parameters of the DL model, the $\gamma$'s are
graphite band parameters,  and $\rm a_0$  is  the graphite in-plane lattice
constant.  Equation~\ref{sdh:E_F}  can  be used  to  solve for   $\rm E_F$,
giving  a value  of 0.93  eV.   This value seems low  for a  stage  1 donor
compound, as Table~\ref{electronicparams} shows.   Yet, within the accuracy
of   the  approximations made in  the   use  of Equation~\ref{sdh:E_F}, the
agreement  of  this value with the 1.3  eV obtained by  Yang and colleagues
from reflectivity  measurements\cite{yang88} is  acceptable.   The 1.3   eV
number   is  itself  based  on  the  ``mirror bands''  approximation, whose
quantitative correctness for donor compounds is not well-established.

	With knowledge  of  the  in-plane  structure,  one  can   also $\rm
C_4CsBi_{0.6}$ also  estimate the  charge transfer  per   carbon atom, $\rm
f_C$.  $\rm  f_C$  is approximately equal  to the  number of electrons  per
carbon atom in a GIC since the carrier density  in pristine graphite  is so
low.\cite{I94} The number of electrons per carbon  atom is the ratio of the
number of electrons  per unit cell to  the number of  carbon atoms per unit
cell.  The number of electrons per unit cell is the ratio  of the FS volume
to the Brillouin zone (BZ) volume:\\

e$^-$ / C-atom = (e$^-$ /unit cell) / (C-atoms / unit cell)\\

\noindent The volume of a piece of $\pi$-electron FS can be estimated in the
 spirit of the DL model as the size of a cylinder with a
circular cross-section of area $\rm  {\cal A}_{max}^e$ and height $\rm 2\pi
/ I_c$:\\

\[ \rm V_{cyl} \; = \; \frac{2 \pi}{I_c} {\cal A}_{max}^e
\]

\noindent  The total volume of the FS is then the number  of cylinders per
BZ  (six) times the  volume per cylinder.  Note  that the number  of carbon
atoms  per unit cell and the  volume  of   the  BZ  depend  on the in-plane
structure of  the  intercalant.  At  the   time that Ref.~\cite{E291}   was
published, the authors did not know of any work  on the in-plane structural
work on   the  CsBi-GIC's, and   so calculated $\rm    f_C$ using  the $\rm
(2X2)R0^{\circ}$, since this structure is common  in donor GIC's.\cite{I94}
A new number for $\rm f_C$ has been calculated now that a specific in-plane
structure has been observed in these materials.

	Bendriss-Rerhrhaye in her  thesis  proposes  a  $\rm  (3\surd13   X
8)R(15^{\circ},0^{\circ})$   rhombic       unit     cell     for       $\rm
C_4CsBi_{0.6}$.\cite{bendriss86}  This   proposed   structure  is shown  in
Figure~\ref{csbistruct}.   The  BZ  associated  with such a unit  cell will
obviously be  much smaller  than the graphitic  one, but this  is partially
compensated for by the large number of carbon atoms (192) in the unit cell.
Using  the 1120 T  SdH frequency  to calculate ${\cal  A}_{max}^e$ gives an
$\rm  f_C$   of -0.026 electrons per  carbon   atom.  This value  is  about
two-thirds of that obtained from  reflectivity measurements by Yang {\em et
al.}\cite{yang88},   just as the $\rm    E_F$ obtained from   SdH is  about
two-thirds of  that  from reflectivity.  [See Table~\ref{electronicparams}]
The $\rm f_C$ number of Yang and colleagues is obtained directly from their
Fermi     energy     using   the   Blinowski-Rigaux    (BR)   tight-binding
model.\cite{blinowski80} Therefore it is dependent on both the mirror-bands
approximation (used to get $\rm E_F$) and the assumptions of  the BR model,
which, like the DL model, uses graphitic  parameters  to describe the $\pi$
bands of the   intercalation compound.    Taking  all  approximations  into
account, the $\rm E_F$ and $\rm f_C$  numbers of  the two references should
be about equally reliable.  The lower value  of the Fermi energy calculated
from the SdH data could be due to a high-frequency oscillation that was not
observed.  A high-frequency oscillation appears to  have been missed in the
compound $\rm  C_4KH_x$.\cite{doll87} A frequency  of about 1700 T would be
needed for perfect agreement of the CsBi SdH data with the data of Yang.

\begin{figure}
\center
\vskip 9.cm 
\caption[Proposed structure of $\rm C_4CsBi_{0.6}$]{$\rm  (3\surd13 X 8)R(15^{\circ},0^{\circ})$ in-plane unit cell
proposed for $\rm C_4CsBi_{0.6}$ by A. Bendriss-Rerhrhaye.\cite{bendriss86}.}
\label{csbistruct}
\end{figure}

\begin{table}
\caption[Electronic parameters of the superconducting GIC's.]{Electronic 
parameters for selected superconducting GIC's.  The stage and phase of each
compound are listed in parentheses.
  The
compounds are listed in order of increasing superconducting transition
temperature. $\omega_p$ is the unscreened plasma frequency.}
\label{electronicparams}
\begin{center}
\begin{tabular}{|lc|cccc|}
\hline
Compound & $\rm T_c$ (K) & $\rm E_F (eV)$ &  $\omega_p$ (eV) & $\rm f_C$ \\
\hline
$\rm C_8Cs$ (1) & $<$0.06\cite{tanuma81} & 1.9\cite{lagues84}&-& -0.05\cite{takahashi86,dicenzo86}\\
$\rm C_{24}Cs$ (2) & $\approx$0 &-&-& -0.040\cite{fretigny85}\\
$\rm C_{24}K$ (2) & $<$0.011\cite{koike80} & 1.42\cite{doll87} & 4.2\cite{doll87} & -0.042\cite{doll87,I63,preil84}\\
& &-& 0.85\cite{preil84,preil83} &-& \\
$\rm C_4KH_{0.8}$  (1) & $<$0.070$^{\ast}$\cite{suzuki85b} &
$\approx$1.58$^{\ast}$\cite{doll87} & 3.31$^{\ast}$\cite{doll87} & $\approx$-0.052$^{\ast}$\cite{doll87}\\
&-& 0.89$^{\ast}$\cite{O353} & - & -0.02$^{\ast}$\cite{Z260}\\
$\rm C_8KH_{2/3}$ (2) & $<$0.052$^{\ddag}$\cite{sano80} & - & -  \\
$\rm C_8KH_{0.8}$ (2) & - & 1.13$^{\ast}$\cite{doll87} & 3.26$^{\ast}$\cite{doll87} & -0.026$^{\ddag}$\cite{doll87} \\
&-& 1.15$^{\ast}$\cite{Z260} &-& -0.033$^{\ast}$\cite{Z260}\\
$\rm C_4CsBi_{0.6}$ (1$\alpha$) & $<$0.050\cite{stang88} & 1.3\cite{yang88} & 2.6\cite{yang88} & -0.04\cite{yang88}\\
& $<$0.5\cite{E291} & 0.93\cite{E291} &-& -0.027$^*$\\
$\rm C_4CsBi_{1.0}$ (1$\beta$) & $<$0.4\cite{E291} & 0.95\cite{yang88} & 2.5\cite{yang88} & -0.02\cite{yang88}\\
$\rm C_8K$ (1) & 0.15\cite{kaneiwa82} & 1.6\cite{doll87} & 4.5\cite{doll87} & -0.075\cite{tanuma78}\\
&-& 1.4\cite{tanuma78} & 4.65\cite{fischer85} & -0.125\cite{preil84}\\
$\rm C_8KH_{0.05}$ (1) & 0.19$^{\dagger\ddag}$\cite{enoki85} & - & - & -\\
$\rm C_8KH_{0.19}$ (1+2) & 0.22$^{\S\ddag}$\cite{kaneiwa82} & - & - & -\\
$\rm C_8KH_{0.8}$ (2) & 0.31$^{\ast}$\cite{suzuki85b} & - & - & -\\
$\rm C_4KHg$ (1) & 1.5\cite{W179} & 1.53\cite{W179} &  & -0.063\cite{W179}\\
& 0.73\cite{tanuma81} & 1.6\cite{yang88} & 5.1\cite{yang88} & -0.06\cite{yang88}\\
&-& 1.47\cite{preil84} & 5.0\cite{fischer85} & -0.073\cite{conard81}\\
$\rm C_8KHg$ (2) & 1.9\cite{pendrys81} & 1.2 & 4.7\cite{yang88} & -0.03\cite{yang88} \\
&-& 1.0\cite{W179}  &  & -0.049\cite{W179} \\
&-& 0.87\cite{preil84} &-& -0.0417\cite{preil84}\\
\hline
\end{tabular}
\end{center}
$^{\dagger}$Calculated through interpolation in Ref.~\cite{enoki85}.\\
$^{\S}$Superconductivity is attributed only to the stage 1  phase $\rm
C_8KH_{0.1}$ in Ref.~\cite{enoki85}.\\
$^{\ast}$Made from KH powder.\\
$^{\ddag}$Chemisorbed from $\rm C_8K$ starting material.\\
$^*$Calculated here from data in Ref.~\cite{E291}.
\end{table}

