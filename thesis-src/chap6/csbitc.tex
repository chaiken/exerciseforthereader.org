% -*- mode: latex -*-
\section{The Quest for Superconductivity in CsBi-GIC's}
\label{csbitc}

	Shortly  after     the   announcement    of   the  discovery     of
superconductivity  in the   MBi-GIC's, an effort  was  undertaken at MIT to
reproduce  the results reported   by  the   University of Nancy  group.  As
described     in Section~\ref{synth}, the  samples   used  in  the  MIT
experiments were prepared using conditions  close to those  used in  Nancy.
The x-ray diffraction patterns,   TEM micrographs, and sample colors   were
found  to       be   similar  to     those   previously    reported    [see
Chapter~\ref{charac}].

\subsection{Experimental Method}
\label{csbi:exp}

	The techniques used in the $\rm T_c$ measurements on the CsBi-GIC's
were nearly the   same as those  used  for the   KHg-GIC's,  described   in
Section~\ref{khghyd:exp}.  There  were several differences,   however.  One
was that  the  CsBi  samples were  routinely  cleaved before the  $\rm T_c$
measurement so that any transition observed would be due to the bulk of the
sample,   and  not  a spurious effect  arising from  alloy  adsorbed on the
surface.   Lagrange and coworkers do not  report whether they cleaved their
samples  or not.  KHg-GIC's very rarely  had any adsorbed alloy since their
intercalation  is accomplished through a  vapor-phase reaction, in contrast
to the contact reaction necessary to intercalate the MBi-GIC's.  Therefore,
KHg samples were not generally cleaved before $\rm T_c$ measurements.

	Because   cleaving is  a  difficult   operation to perform   in the
glovebox on small samples, the CsBi-GIC's were cleaved in air.  Handling in
air was also necessary to  perform the weight-uptake measurements described
in  Section~\ref{chemanal}.   Because  of     the  compounds'  excellent
stability   in air,\cite{lagrange87} the   brief exposures to  air were not
thought to  influence the superconductivity  experiments.  The  work of the
Nancy group does not state whether their samples were exposed to air before
the low-temperature experiments.

	The CsBi-GIC's were place in glass ampoules for the low-temperature
experiments. They were not fixed in metal sample-holders as the KHg-GIC's
were [See Section~\ref{mounting}], though, because no critical field
measurements were performed on them.  The inductive $\rm T_c $ measurements
were performed on the CsBi-GIC's in the same manner as on the KHg-GIC's
[See Section~\ref{critf:exp}].  The temperature range of the inductive
measurements was from 4.2 K to about 0.4 K, just as for the  KHg
experiments.

	The air stability of the CsBi compounds allowed resistive $\rm T_c$
measurements to  be performed  easily on  them.   Contacts  were  made with
silver paint on  the edges  of the GIC  in a standard  four-probe geometry.
The samples  were then encapsulated in  a cell filled with helium   gas for
good  thermal  contact at  low temperatures.   Both  zero-field temperature
sweeps and fixed-temperature field sweeps  were performed on CsBi $\alpha$-
and $\alpha + \beta$-phase samples.  The temperature range of the resistive
measurements  was from liquid  helium temperature (4.2 K)  to  about 1.1 K.
Current densities used were on the order of 10$^{-1}$-10$^{-2}$ A/cm$^{2}$.
The results did not depend on the current density employed.  The values of
the normal-state resistivity obtained from the temperature sweeps were
estimated to be uncertain by about 25\% due to remanent fields in the
Bitter magnets used for the Shubnikov-de Haas measurements.

\subsection{Results}
\label{csbi:tcres}

	The unanticipated outcome of the $\rm T_c$ measurements was that no
superconductivity could  be  detected in  either  $\alpha$-phase   or mixed
$\alpha + \beta$-phase samples.\cite{E291} By using Equation~\ref{vbridge},
it was determined that the superconducting effective area that would give a
signal-to-noise ratio of one corresponded to about 2\%  of  the area of the
CsBi  polycrystals.  Therefore, from the inductive  measurements it  can be
said with   some confidence that   no more  than 2\%  of  these  samples is
superconducting within the temperature range from 4.2 to 0.5 K.

	The temperature region  above  liquid helium temperature  was  only
searched briefly for a transition at a $\rm T_c$ higher  than that reported
by Lagrange {\em et al.}  It is safe to  conclude that a higher-temperature
transition  is not present, though, since  an applied magnetic  field of 23
Tesla did not produce any kind of superconducting transition.  (The highest
critical field known is well under 1 T.\cite{iye82})

	In addition to the null result from the inductive measurements, no
interesting features in the resistivity were detected at low temperature.
The compounds appeared to exhibit typical metallic behavior ($\rm d\rho/dT
> 0$).  The measured resistivity at 4.2 K was $(8 \pm 2) \mu\Omega$-cm in
one stage 1 $\alpha$-phase sample, and $(30 \pm 7) \mu\Omega$-cm in a stage
1u $\alpha +
\beta$ sample.  These values are somewhat higher than the 5 K resistivity values quoted
by     Lagrange  {\em et   al.},   who measured   their resisitivity values
inductively.\cite{mcrae85} An  applied magnetic field produced  only traces
consistent    with    magnetoresistive behavior,      nothing     like    a
superconducting-normal transition.  [See discussion below.]

	The  MIT superconductivity  measurements are obviously  in conflict
with  those described previously\cite{lagrange85},  as are those of several
other          groups      who          also        failed    to       find
superconductivity.\cite{yang88,stang88}   In  the  most  extensive set   of
measurements  to  date,  Stang and  coworkers in  Berlin   failed  to  find
superconductivity in stage 1  $\alpha$-phase  CsBi-GIC's down to 50 mK even
though  they had sensitivity to  a transition in about  0.1\% of the sample
volume.\cite{stang88} Even   the   Nancy   group    has had  difficulty  in
reproducing the $\rm T_c$ values they reported.\cite{lagrange85a} A summary
of  all $\rm  T_c$ measurements on  the MBi-GIC's  to date  is reported  in
Table~\ref{mbi:tc}.

\begin{table}
\caption[Superconducting transition temperatures of the MBi-GIC's and
related alloys.]{Superconducting transition  temperatures of  the MBi-GIC's
and  related   alloys.   For Ref.\cite{E291},   the  volume fraction column
actually contains the  effective  areal  fraction of superconductivity,  as
explained in Section~\ref{critf:exp}. NR=Not reported. NA=Not applicable.}
\label{mbi:tc}
\begin{center}
\begin{tabular}{|lc|cccc|}
\hline
Compound & Stage/Phase & $\rm T_c$ (K) & Scy. Vol. \%  & Alloy & Reference \\
\hline\\
$\rm C_4CsBi_{0.5}$ & 1 ($\alpha$) & 4.05 & NA$^*$ & NR & \cite{mcrae85}\\
$\rm C_4CsBi_{0.5}$ & 1 ($\alpha$) & $<$1.5$^{\dagger}$ & NR & $\rm Cs_5Bi_4$ & \cite{yang88}\\
$\rm C_4CsBi_{0.5}$ & 1 ($\alpha$) & $<$0.5 & $<$2 & $\rm Cs_5Bi_4$& \cite{E291}\\
$\rm C_4CsBi_{0.5}$ & 1 ($\alpha$) & $<$0.05 & $<$0.1 & $\rm Cs_{64}Bi_{36}$& \cite{stang88}\\
$\rm C_4CsBi_{x}$ & 1 ($\alpha + \beta$) & $<$1.2 & $<$2 & $\rm Cs_5Bi_4$ & \cite{E291}\\
$\rm C_4CsBi_{x}$ & 1 ($\alpha + \beta$) & 4.55 & 0.13 & $\rm CsBi$& \cite{stang88}\\
$\rm C_4CsBi_{x}$ & 1 ($\alpha + \beta$) & 4.69 & 0.12 & $\rm CsBi$ & \cite{stang88}\\
$\rm C_4CsBi_{x}$ & 1 ($\alpha + \beta$) & 4.69 & 0.24 & $\rm Cs_{48}Bi_{52}$ & \cite{stang88}\\
$\rm C_4CsBi_{x}$ & 1 ($\alpha + \beta$) & 4.69 & 0.8 & $\rm Cs_{48}Bi_{52}$ & \cite{stang88}\\
$\rm C_4CsBi_{x}$ & 1 ($\alpha + \beta$) & $<$0.05 & $<$0.1 & $\rm Cs_{48}Bi_{52}$ & \cite{stang88}\\
$\rm C_4CsBi_{1.0}$ & 1 ($\beta$) & 2.3 & NA$^*$ & NR &\cite{mcrae85}\\
$\rm C_4CsBi_{1.0}$ & 1 ($\beta$) & $<$1.5$^{\dagger}$ & NR & CsBi & \cite{yang88}\\
$\rm C_4CsBi_{x}$ & (1+2) ($\alpha + \beta$) & 4.72 & 1.9 & $\rm Cs_{47}Bi_{53}$ & \cite{stang88}\\
$\rm C_8CsBi_{1.0}$ & 2 ($\beta$) & 2.7 & NA$^*$ & NR & \cite{mcrae85}\\
$\rm C_8CsBi_{1.0}$ & 2 ($\beta$) & 3.45 & NR & NR & \cite{bendriss86}\\
$\rm C_8CsBi_{1.0}$ & 2 ($\beta$) & 4.73 & 5 & $\rm Cs_{39}Bi_{61}$ & \cite{stang88}\\
$\rm C_8CsBi_{1.0}$ & 2 ($\beta$) & $<$1.5$^{\dagger}$ & NR & $\rm Cs_5Bi_4$ & \cite{yang88}\\
$\rm C_8CsBi_{1.0}$ & 3 ($\beta$) & 4.71 & 4 & $\rm Cs_{39}Bi_{61}$ & \cite{stang88}\\
$\rm C_8CsBi_{1.0}$ & 3 ($\beta$) & $<$0.05 & $<$0.1 & $\rm Cs_{39}Bi_{61}$ & \cite{stang88}\\
$\rm C_8CsBi_{1.0}$ & 3 ($\beta$) & 4.72 & 1 & $\rm Cs_{39}Bi_{61}$ & \cite{stang88}\\
$\rm C_8RbBi_{0.5}$ & 2 ($\alpha$) & 1.25-1.5 & NA$^*$ & NR & \cite{mcrae85}\\
KBi & NA &  3.6 & NA & NA & \cite{bendriss86}\\
$\rm KBi_2$ & NA & 3.58 & NA & NA & \cite{mcrae85}\\
$\rm RbBi_2$ & NA & 4.25 & NA & NA & \cite{mcrae85}\\
$\rm CsBi_2$ & 4.75 & NA & NA & NA &\cite{mcrae85}\\
\hline
\end{tabular}
\end{center}
$^*$Resistive measurements.  Volume fraction of superconductivity not determined.\\
$^{\dagger}$Also checked up to 7.0 kbar applied pressure.\\
\end{table}
%\end{minipage}

	The   irreproducibility   of the  $\rm  T_c$   measurements in  the
CsBi-GIC's would   be more surprising  were  it not for the  wide  range of
transition    temperatures    found      for      other     superconducting
GIC's.\cite{erice:scy} In $\rm C_4KHg$, as was discussed previously, fairly
subtle  variations in  intercalation conditions  have  a   large effect  on
superconductivity.  The wide  range of $\rm  T_c$ values reported  for $\rm
C_4KHg$ is  thought to be due  to the  presence of the  $\beta$  phase [see
Section~\ref{hydintro}]    in low-$\rm    T_c$    samples.   One  possible
explanation for the lack  of  superconductivity in  the non-superconducting
CsBi-GIC's is therefore that they  possess minority phases  which  suppress
superconductivity.       Alternatively,         as         Lagrange     has
suggested,\cite{lagrange87} it  seems plausible that the  differences found
in the  superconducting    behavior in  the    CsBi  experiments   could be
attributable  to  small  differences  in    in-plane   ordering  or   in
stoichiometry.   This type of  sensitivity to ordering and/or stoichiometry
has also been considered as an explanation for the wide  range of $\rm T_c$
values reported for $\rm C_4KHg$.\cite{Z234}

	On the other hand, there is another simple line  of reasoning which
explains  the  discrepancy among  the various  sets   of  experiments.  The
alternative     explanation,   also      discussed   by     Stang       and
coworkers,\cite{stang88} is  that the  superconducting transitions reported
by Lagrange  {\em et al.}  were  actually transitions of inclusions of  the
alloy  $\rm CsBi_2$.   This possibility   is suggested  by the  presence of
inclusions in  the CsBi-GIC's which could  not   be  completely  removed by
repeated  cleaving.  The inclusions   were imaged during  TEM studies which
showed that they appeared throughout the samples prepared  both here at MIT
and at the University of Kentucky.\cite{speck88z}   An electron micrograph of
an inclusion is shown in Figure~\ref{inclusion}.

\begin{figure}
\vspace{15cm}
\caption[Electron micrograph of intercalant inclusions in a CsBi-GIC]{An
electron   micrograph showing   intercalant  inclusions (bright regions)   in  a   $\rm
C_4CsBi_{x}$  $\alpha + \beta$-phase polycrystal  grown  here at  MIT.  The
magnification for  this  micrograph is indicated by  the  100 nm scale bar.
[Micrograph prepared by J. Speck, MIT.]}
\label{inclusion}
\end{figure}


	In the case of the  MIT samples, the inclusions presumably  had the
approximate  composition   of  the  starting  alloy,   $\rm  Cs_5Bi_4$,   a
composition which is not superconducting.  As would be expected, therefore,
no inclusion-derived superconducting signal was seen in the MIT CsBi-GIC's.
However,  as  shown by  the  phase diagram  in   Figure~\ref{phasediag}, $\rm
Cs_5Bi_4$ is  adjacent  to $\rm CsBi_2$, a phase  which is superconducting.
The implication is that  a starting alloy  with  a composition between $\rm
Cs_5Bi_4$   and $\rm   CsBi_2$  (such  as   CsBi) should be  a  mixture  of
superconducting  and  non-superconducting material.   One  would  therefore
expect   that  a GIC  made  from such a   starting  alloy  would  have some
superconducting   inclusions.    Inclusion-derived   superconductivity  was
apparently seen by  Stang {\em et  al.}.\cite{stang88}  They observed  that
superconductivity  was  present only   in  small  volume  fractions  in the
CsBi-GIC's, consistent with  inclusion-derived transitions.  Further, Stang
and coworkers found  that  $\rm T_c$ was  about 4.7 K almost independent of
stage and phase.  [See Table~\ref{mbi:tc}.]   The only  processing variable
which seemed to impact $\rm T_c$ was the starting alloy  composition, which
presumably determines what fraction of the inclusions  are superconducting.
A plot of $\rm T_c$  versus  starting  alloy composition  is  displayed  in
Figure~\ref{csbi:alloytc}.

\begin{figure}
%\beginpicture
%\setcoordinatesystem units <50mm,20mm>
%\setplotarea x from 0 to 2, y from 0 to 5
%\linethickness=1pt
%\setdashes
%\putrule from 0.25 0 to 0.25 2.7
%\putrule from 0.25 3.3 to 0.25 5
%\putrule from 0.6667 0 to 0.6667 2.5
%\putrule from 0.6667 3.5 to 0.6667 5
%\putrule from 0.8 0 to 0.8 0.25
%\putrule from 0.8 0.7 to 0.8 0.95
%\putrule from 0.8 1.7 to 0.8 2.5 
%\putrule from 0.8 3.5 to 0.8 5
%\setsolid
%\axis bottom label {Bi/Cs ratio of starting alloy} ticks 
%	numbered from 0 to 2 by 0.5
%	in unlabeled short quantity 9 /
%\axis left label {\lines {$\rm T_c$ (K)}} ticks
%	numbered from 0 to 5 by 1
%	in unlabeled short quantity 11 /
%\axis right ticks
%	in unlabeled long quantity 6 
%	in unlabeled short quantity 11 /
%\axis top ticks
%	in unlabeled long quantity 5 
%	in unlabeled short quantity 9 /
%\multiput {$\bigotimes$} at 0.8 0.5  0.8 1.2 / %mit.plot
%\multiput {$\bigtriangleup$} at  0.8 1.5  1.0 1.5 / %yang.plot
%\multiput {$\bigcirc$} at 0.563 0.05 %stang.plot
%1.0 4.55 
%1.0 4.69 
%1.08 4.69 
%1.08 4.69 
%1.08 0.05 
%1.56 4.73 
%1.56 4.71
%1.56 0.05 
%1.56 4.72 /
%\multiput {X} at 0.8 0  2.0 4.72 / %alloy data
%\multiput {$\downarrow$} [t] at 0.8 0.5  0.8 1.2 0.563 0.05  0.8 1.5  1.0 1.5
%1.08 0.05 1.56 0.05 /
%\begin{small}
%\put {\lines {\underline{Cs$_3$Bi}\cr \underline{+ Cs}\cr}} at 0.15 3
%\put {\lines {\underline{Cs$_3$Bi}\cr \underline{+ Cs$_3$Bi$_2$}\cr}} at 0.45 3
%\put {\lines {\underline{Cs$_3$Bi$_2$}\cr +\cr \underline{Cs$_5$Bi$_4$}\cr}} at 0.75 3
%\put {\lines {\underline{Cs$_5$Bi$_4$}\cr \underline{+ CsBi$_2$}\cr}} at 1.3 3
%\end{small}
%\endpicture
\vspace{15cm}
\caption[Dependence of $\rm T_c$ on intercalant composition for MBi-GIC's ]{A plot of superconducting transition temperature $\rm T_c$ for
$\rm C_{4}CsBi_{x}$ versus  starting alloy Bi/Cs ratio.  $\bigotimes$, MIT
data from  Ref.~\cite{E291}; $\bigtriangleup$, University  of Kentucky data
from Ref.~\cite{yang88}; and  $\bigcirc$, Freie  Universit\"at Berlin  data
from Ref.~\cite{stang88}.   The X are  alloy (not GIC) data from the
CRC  Handbook.  The  presence  of  $\downarrow$  means that the   nearest point
represents an upper bound on $\rm T_c$.   Data from the University of Nancy
is not  included because precise starting  alloy compositions are not given
for their samples.}
\label{csbi:alloytc}
\end{figure}

	The implication of the  MIT and Berlin work\cite{stang88}   is that
superconductivity  in   contact-intercalated    GIC's   can be  mimicked by
inclusions of a superconducting intercalant phase.  In this interpretation,
the superconductivity observed  by Lagrange and  colleagues in  resistivity
measurements was   due to a network  of  $\rm CsBi_2$ through  the  sample,
either adsorbed on the surface  or interconnected through the bulk.   (Bulk
interconnections could perhaps be provided by Daumas-Herold domains.)  This
superconducting alloy would completely short out any normal-state transport
due  to the GIC itself.   In  this interpretation,  the University of Nancy
low-temperature   resistivity measurements do    not  measure   a  property
intrinsic to the GIC, but to the adsorbed alloy.  Then  the reason that the
MIT measurements  give a higher  value of $\rho_a$ at  4.2 K than the Nancy
measurements is that  the MIT measurements  give $\rm \rho_{a, GIC}$  while
the Nancy ones give $\rm \rho_{a, alloy}$.  Interpretation of these results
would be  more  certain  if it   were known  exactly what   starting  alloy
compositions the French used and whether they  cleaved their samples before
performing the transport measurements.

	In the end, one cannot  make  a firm statement   as to whether  the
CsBi-GIC's are  superconducting or  not.  Circumstantial  evidence strongly
indicates that  the  answer is no,   but considering   the long  history of
variability in the $\rm T_c$'s of GIC's, it is wise to  be cautious.  After
the discovery  of superconductivity in  $\rm  C_8K$ at  0.55 K  in  1965, a
transition was not observed again until 1978 at a much lower temperature of
0.139 K,\cite{koike78} with several unsuccessful attempts to  reproduce the
original experiment  in intervening  years.\cite{poitrenaud70}  Considering
that it is  not possible to   tell whether the  non-superconducting samples
made outside Nancy are similar in every detail to those the superconductors
synthesized there,  an authoritative statement ruling out superconductivity
in  MBi-GIC's cannot  be  made.  This is   especially true for  the KBi-and
RbBi-GIC's, which have not yet been intensively studied outside Nancy.  The
best one can do  at this time  is  say  that  available evidence   does not
support the identification of the MBi-GIC's as superconductors.  A possible
explanation why these compounds are not superconducting is discussed in Section~\ref{csbidisc}.
