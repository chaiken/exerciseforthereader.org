% -*- mode: latex -*-
\section{Introduction}
\label{csbi-intro}
	 Several unresolved questions pertinent to the superconductivity of
the  KHg-GIC's have been discussed at  length in  previous chapters.  Among
the most intriguing  of these is why $\rm  C_8KHg$ has a higher  $\rm  T_c$
than $\rm C_4KHg$, how the superconducting properties of the various phases
of $\rm  C_4KHg$ differ, and  whether two-dimensional superconductivity can
be observed   in GIC's.  These  questions are  difficult to approach   on a
first-principles basis because  of  the large unit  cell and   complex band
structure of the ternary  compounds.  As  a  result,   much of the  current
understanding has been derived from	 qualitative comparisons of ternary
GIC data with that from different  materials, principally the  binary GIC's
and TMDC's discussed in Chapter~\ref{othersys}.

	When Lagrange  and his colleagues    at the  University  of   Nancy
announced   the  synthesis of   the MBi-BIC's  (M   =  K,  Rb,   or  Cs) in
1985,\cite{lagrange85},   they    opened up  a  whole  new  arena   for the
exploration of these issues.  The new ternary GIC's  were not only reported
to be  superconducting, but they also possessed  many of the   more unusual
features of the  KHg-GIC's.  Specifically, Lagrange {\em  et al.} said that
the CsBi-GIC's show a higher $\rm T_c$ for stage II than  stage I, the same
surprising    stage    dependence    that      is     exhibited   by    the
MHg-GIC's.\cite{mcrae86}  Secondly,  for each alkali metal  and  each stage
they  reported two  different  phases  of  the  MBi-GIC's, with  the phases
distinguished  by  different sandwich thicknesses.\cite{lagrange85,mcrae86}
These phases  were   said  to have    different superconducting  transition
temperatures,\cite{mcrae86}  perhaps   similar  to  the  situation   in the
KHg-GIC's  [see Chapter~\ref{hydrog}].   The announcement  of the MBi-GIC's
also gave new life  to the search for  two-dimensional superconductivity in
GIC's since  the rubidium  compounds  could  be prepared  in  stages  up to
seven.\cite{lagrange85}

	The discovery of the MBi-GIC's spurred a great deal of experimental
activity,   in large part  because   of  the   high $\rm  T_c$  of about  4
K\cite{lagrange85} reported  for the  CsBi  $\alpha$-phase  material.   The
results of further studies on the superconductivity  of these materials are
reported below.
