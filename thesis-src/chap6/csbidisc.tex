\section{Discussion}
\label{csbidisc}

	The  most  obvious conclusion to  be extracted from the numbers  in
Table~\ref{electronicparams}  is  that the bismuth   in the  CsBi-GIC's  is
acting as an acceptor of electrons, similar to behavior of the  heavy metal
in other trilayer ternary GIC's.   This conclusion is drawn  due to the lower
$\rm  E_F$  and $\rm f_C$   of the bismuth  GIC's compared  to the   binary
Cs-GIC's, and also due to the fairly high $\rm  E_{2g2}$  Raman frequencies
displayed by the  CsBi-GIC's.\cite{yang88} According to Refs.~\cite{chan87}
and
\cite{pietronero78}, a higher Raman  frequency  for donor compounds is indicative of a  smaller
in-plane  carbon-carbon   distance  and   thereby a   smaller   $\rm  f_C$.
Therefore, all evidence points to less cesium charge  per carbon atom being
resident in the graphite layers than in $\rm C_8Cs$  and possibly even $\rm
C_{24}Cs$.

	The  pic\-ture then  is   of  pos\-i\-tive\-ly charged   ces\-ium  lay\-ers and
neg\-a\-tive\-ly  charged bis\-muth  and  carbon layers.   Since it  is   generally
accepted that the superconducting MHg-GIC's and  MTl-GIC's also have layers
of alternating charge\cite{herold81},  this insight would  appear  to be of
little help in understanding the apparent  lack of superconductivity in the
CsBi-GIC's.  It is hard to see why back transfer of charge from  the carbon
layers would reduce $\rm  T_c$ in the MBi compounds  when it seems to raise
$\rm T_c$ in the MTl- and MHg-GIC's.

	A closer look  at Table~\ref{electronicparams} shows that,  roughly
speaking, GIC's with $\rm \mid f_C \mid$ less than about 0.042 (= 1/24) are
not  found   to   be superconducting.    An  accumulation  of  experimental
evidence\cite{fretigny85,doll87,I63,preil84}  supports  the conclusion that
in the stage 2 alkali metal GIC's that the  alkali metal  is fully ionized,
and in fact it  seems reasonable to  suppose that in  any ternary  compound
where $\rm \mid  f_C \mid <  0.04 $ the  alkali metal  atom will  be  fully
ionized.  The thinking  behind this is that  the graphitic $\pi$ bands have
such a high affinity for the  alkali-metal s-electron  that the only reason
that  $\rm f_C$  can fall  very low  is for  the heavy  metal  acceptor  to
outcompete the $\pi$-band for the s-charge.  Therefore a low $\rm f_C$ in a
ternary compound  implies  that $\rm f_M$ =   +1.  The  data accumulated in
Table~\ref{electronicparams} therefore  suggests  that superconductivity in
ternary GIC's  is suppressed by an  empty  s band, a conclusion  in keeping
with the Al Jishi model\cite{M143} of  superconductivity in binary graphite
intercalation   compounds.  This model  suggests  that s-band  occupancy is
necessary  for  superconductivity in  GIC's.   

	The next  natural question to  arise is that  of why physically the
s-band   occupation   should be  imperative    for superconductivity.  More
intuition on this question can be developed by examining quantities closely
linked to  s-band  occupancy, namely   the   c-axis   resistivity and   the
resistivity anisotropy.  Table~\ref{resistanis}, taken directly from the
work of McRae and Mar\^ech\'e\cite{mcrae88}, shows that the compounds
identified above as having an empty s-band also have a high $\rho_c$ and a
high resistivity anisotropy $\equiv \rho_c/\rho_a$.  High values for the
anisotropy and $\rho_c$ indicate an almost two-dimensional band structure,
often with hopping conduction along $\hat{c}$.\cite{R200,E343}

\begin{table}[p]
\caption[Resistivity anisotropy in selected GIC's]{Two tables prepared by McRae and Mar\^ech\'e\cite{mcrae88} which
list the c-axis resistivity and resistivity anisotropy of many GIC's, both
donors and acceptors.   $A\equiv \rm \rho_c/rho_a$.
  Correct sources for these numbers are given in Ref.~\cite{mcrae88}.}
\label{resistanis}
\end{table}

	The chain of  reasoning developed  in this section   seems to imply
that   two-dimensionality     of    band  structure    tends  to   suppress
superconductivity in GIC's.  This is a conclusion which has been previously
reached by leading superconductivity experts\cite{matthias71,hannay65} when
they turned their attention to the  problem  of superconductivity in GIC's.
The reasons why two-dimensional superconductivity might be  expected in the
closely related  transition metal dichalcogenides   intercalation compounds
and not in the GIC's is discussed in Section~\ref{tmdc}.

	The apparent lack of superconductivity in the  MBi-GIC's is a great
disappointment. Many experimentalists wanted  to take advantage  of the new
materials' increased air stability  and  reported  higher  $\rm   T_c$'s to
extend the scope  of  experiments that  could reasonably   be  performed on
superconducting  GIC's.  Despite  the  letdown,  attempts   to  answer  the
question  of  why the MBi-GIC's  are  not superconducting  have led to some
insight about the properties of the  MHg-GIC's, which are the  topic of the
rest  of  this   thesis.    Finally, it  should   be  mentioned   that  the
irreproducibility   of  transition  temperatures  in other    intercalation
compounds  leaves   one with the   hope that  superconductivity  may yet be
confirmed in the MBi-GIC's.
