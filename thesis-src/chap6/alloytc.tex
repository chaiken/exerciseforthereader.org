\documentstyle[12pt]{report}
\input{/usr/local/lib/tex/macros/prepictex.tex}
\input{/usr/local/lib/tex/macros/pictex.tex}
\input{/usr/local/lib/tex/macros/postpictex.tex}
\pagestyle{empty}
\begin{document}
\begin{figure}
\center
\beginpicture
\setcoordinatesystem units <50mm,20mm>
\setplotarea x from 0 to 2, y from 0 to 5
\linethickness=1pt
\setdashes
\putrule from 0.25 0 to 0.25 2.7
\putrule from 0.25 3.3 to 0.25 5
\putrule from 0.6667 0 to 0.6667 2.5
\putrule from 0.6667 3.5 to 0.6667 5
\putrule from 0.8 0 to 0.8 0.25
\putrule from 0.8 0.7 to 0.8 0.95
\putrule from 0.8 1.7 to 0.8 2.5 
\putrule from 0.8 3.5 to 0.8 5
\setsolid
\axis bottom label {Bi/Cs ratio of starting alloy} ticks 
	numbered from 0 to 2 by 0.5
	in unlabeled short quantity 9 /
\axis left label {\lines {$\rm T_c$ (K)}} ticks
	numbered from 0 to 5 by 1
	in unlabeled short quantity 11 /
\axis right ticks
	in unlabeled long quantity 6 
	in unlabeled short quantity 11 /
\axis top ticks
	in unlabeled long quantity 5 
	in unlabeled short quantity 9 /
\multiput {$\bigotimes$} at 0.8 0.5  0.8 1.2 / %mit.plot
\multiput {$\bigtriangleup$} at  0.8 1.5  1.0 1.5 / %yang.plot
\multiput {$\bigcirc$} at 0.563 0.05 %stang.plot
1.0 4.55 
1.0 4.69 
1.08 4.69 
1.08 4.69 
1.08 0.05 
1.56 4.73 
1.56 4.71
1.56 0.05 
1.56 4.72 /
\multiput {X} at 0.8 0  2.0 4.72 / %alloy data
\multiput {$\downarrow$} [t] at 0.8 0.5  0.8 1.2 0.563 0.05  0.8 1.5  1.0 1.5
1.08 0.05 1.56 0.05 /
\begin{small}
\put {\lines {\underline{Cs$_3$Bi}\cr \underline{+ Cs}\cr}} at 0.15 3
\put {\lines {\underline{Cs$_3$Bi}\cr \underline{+ Cs$_3$Bi$_2$}\cr}} at 0.45 3
\put {\lines {\underline{Cs$_3$Bi$_2$}\cr +\cr \underline{Cs$_5$Bi$_4$}\cr}} at 0.75 3
\put {\lines {\underline{Cs$_5$Bi$_4$}\cr \underline{+ CsBi$_2$}\cr}} at 1.3 3
\end{small}
\endpicture
\caption[Dependence of $\rm T_c$ on intercalant composition for MBi-GIC's ]{A plot of superconducting transition temperature $\rm T_c$ for
$\rm C_{4n}CsBi_{x}$ versus  starting alloy Bi/Cs ratio.  $\bigotimes$, MIT
data from  Ref.~\cite{E291}; $\bigtriangleup$, University  of Kentucky data
from Ref.~\cite{yang88}; and  $\bigcirc$, Freie  Universit\"at Berlin  data
from Ref.~\cite{stang88}.   The X are  alloy (not GIC) data from the
CRC  Handbook.  The  presence  of  $\downarrow$  means that the   nearest point
represents an upper bound on $\rm T_c$.   Data from the University of Nancy
is not  included because precise starting  alloy compositions are not given
for their samples.}
\label{csbi:alloytc}
\end{figure}
\end{document}
